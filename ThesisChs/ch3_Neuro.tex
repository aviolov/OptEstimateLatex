\chapter{Mathematical Models in Neuroscience}
\label{ch:neuro_background}

We now describe in detail the basic mathematical model of a neuron that will be
used often in the sequel. 

An introduction to mathematical models in neuroscience is given in Gerstner and
Kistler, \cite{Gerstner2002}, also available online, while the book Stochastic
Methods in Neuroscience, \cite{Laing2009} gives a nice overview of several
current research applications of stochastic techniques to neuroscience.


Neurons relay information by means of voltage spikes - sudden sharp increases in
voltage. Although many details remain unclear, the information content is
thought to be contained in the length of the time-interval between these spikes.
In the simplest case, this can be thought of as a rate - the average number of
spikes per time interval, but more complicated coding schemes are hypothesized to
exist. 

Many experiments allow for manipulating an individual neural cell. A natural
goal then is to make a cell produce a given spike train. This may arise, for
example, in brain-machine interfaces or in artificial prosthetics. 

\section{Problem Formulation}
The most basic representative model for a neuron is the noisy
leaky-integrate-and-fire model:
\begin{equation}
\begin{gathered}
dX_s = \left(\a(t) - \frac{(X_s - \m}{\tc} \right) \intd{s} + \b \intd{W_s},
\\
X(0) = 0,
\\
X(\ts) = \xth \implies  
\begin{cases}
X(\ts^+) = 0 &  
\end{cases}
\end{gathered}
\label{eq:X_evolution_uo}
\end{equation}


