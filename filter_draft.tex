%%This is a very basic article template.
%%There is just one section and two subsections.
\documentclass{article}

\usepackage{amsmath} 
\usepackage{amscd}
\usepackage{amssymb}
\usepackage{amsfonts}
\usepackage{amsthm}
\usepackage{amsfonts}
\usepackage{amsthm}

\usepackage{circuitikz}
\usepackage{pgf}
\usepackage{tikz}
\usetikzlibrary{arrows,snakes,backgrounds}
% \usetikz
\usepackage{subfig}

\usepackage[super]{nth}
% \usepackage{appendix}
% \usepackage{listings}
% \usepackage{color}
\usepackage{ulem}
\usepackage{hyperref}
%\usepackage{url}
\usepackage{cancel}
\usepackage{cleveref}
 
\usepackage{aviolov_style}
\usepackage{local_style} 


\begin{document}


\title{Filtering in 1-D SDEs}
 \author{Alexandre Iolov, Susanne
Ditlevsen, Andr\'e Longtin  \\ $<$\href{mailto:aiolo040@uottawa.ca}
		{aiolo040 at uottawa dot ca}$>$, alongtin at uottawa dot ca}

\date{\today}

\maketitle

\abstract{Infer an unknown, but assumed smooth signal driving an observed
SDE.}

\tableofcontents

\section{SDE Filtering}
Having observed the realization of an SDE we want to know what kind of
stimulation 'drove' it. 

Consider the generic parametrized and perturbed SDE system 
\begin{equation}
dX = U(x, u(t); \th) \intd{t} + \sqrt{2D} dW
\label{eq:SDE_evolution} 
\end{equation}
and, for illustration case, specialize \cref{eq:SDE_evolution} to the OU
system:
\begin{equation}
dX = (\underbrace{-\b(X) + u(t)}_{U(x,\a; \th)}) \intd{t} +
\underbrace{\s}_{\sqrt{2D}} dW
\label{eq:OU_evolution} 
\end{equation}
where we know $\b, \s$, but do not know the perturbation: $u(t)$. We want to get
an estimate of $u(t)$ given only observations of $X$.  


\subsection{Some basic concepts/notation}
% We will call $f$ the probability transition density associated with $X_t$, 
% \begin{equation}
% f(x,t| x_0 ;\th, \a(\cdot)) dx = \Prob[ X_t \in x + dx | x_0, \a, \th]
% \end{equation} 
% 
% For a fixed parameter set, $\th$, and control policy, $\a(\cdot)$, $f$ is the
% solution to a Fokker-Planck equation:
% \begin{equation}
% \di_t f = \L_{\th, \a(\cdot)}[f] 
% \label{eq:fokker_planck_forward_density}
% \end{equation}
% where the differential operator $\L$ is given by:
% $$
% \L_{\th, \a(\cdot)}[f] = D \cdot \di_x^2 [f] - \di_x[U(x,\a(x,t); \th) \cdot f]
% $$
% We will call $\L$ the forward operator and its adjoint, $\Lstar$, the backward
% operator.

If the process $X$ is observed continuously, its log 'likelihood' $l$
can be obtained directly via the Radon-Nikodym theorem (\cite{Phillips2009}
eq. 15 e.g.) as

\begin{equation}
l(X|u) = \int_0^T \frac{U(X,u; \th )}{\s^2} dX - \frac{1}{2}\int_0^T
\frac{U^2(X,u; \th )}{\s^2	}dt 
\label{eq:log_likelihood_Girsanov}
\end{equation}
\section{Specifying the problem via the log-likelihood of the cts. trajectory}
In general the problem of determining $u(t)$ given knowledge of $X_t$  
can be tackled in several ways. One, following
\cite{Rogers2013},\cite{Phillips2009} but also work of Susanne++ (Anders'
dissertation) is to determine $u$ as the solution to the cts.
log-likelihood.

$$
u(t) = \argmax_u l(X|u)
$$

In fact, if we assume that $u(t) = \int \dot{u}dt$ i.e that $u$ is some $C^1$
function for example, we have a classical Calculus of Variations problem (I
follow Rogers here, b/c I'm a little uncertain on some of the derivations, I
need to recheck them, but it seems right)  

Basically, Rogers re-writes the log-likelihood as:
\begin{equation}
l(X|u) = - \frac{1}{2} 
\int_0^T \frac{\big( \dot{x} - U(x,u; \th )\big)^2}{\s}dt
\label{eq:log_likelihood_C2}
\end{equation}

where $\dot{x}$ is obtained by smoothing a fine, but discrete skeleton of $X_t$
with a $C^2$ spline, such that the time-derivative exists (of course the
idealized stochastic process $X$ is not cts. differentiable)

Note that in this case, $\dot{x}$ is actually given data to the problem and the
goal is to minimize \cref{eq:log_likelihood_C2} wrt. $u$.

if we drop irrelevant constants and let 
$$
\psi(u, \dot{u}, t) = \frac{\big( \dot{x} - U(x,u; \th )\big)^2}{\s}$$ be the
'action', then we have the standard Calculus of Variations problem:

$$
u = \argmin_{u \in C^1} \bigg \{ \int_0^T \psi(u, \dot{u} , t) \intd{t} \bigg\}
$$

In principle this will result in a pair of BVPs (ODEs) arising from the
Euler-Lagrange equations those can be solved using any Optimal Control software
(there are many). One can also avoid the Calculus of Variations (indirect
methods) and go for direct (numerical Optimal Control) methods. The important
thing is that one is solving ODEs, not PDEs!!!

For my reminder, the EL equations look like ($p = \dot{u}$)
\begin{align}
\di_u \psi = \di_t \di_p \psi = 0!!!
\end{align}

Ok, here we have a problem, as in since the derivative $\dot{u}$ does not enter
the equation, we have that it does not matter what $\dot{u}$ is, ok, this
obvious, we just set 

$$
\di_u \psi = 0  \implies \big( \dot{x} - U(x,u; \th ) =0 
$$
and we are done\ldots Well, that's the Maximum Likelihood Approach to
finding $u$ \ldots

It would be more interesting if $u$ was another Stochastic Process, then we are
really looking into Anders' dissertation. (for the Fitzhugh-Nagumo system)


\bibliographystyle{plain} 
\bibliography{library,local}

\end{document}