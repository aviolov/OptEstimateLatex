%%This is a very basic article template.
%%There is just one section and two subsections.
\documentclass{report}

\usepackage{amsmath}
\usepackage{amscd}
\usepackage{amssymb}
\usepackage{amsfonts}
\usepackage{amsthm}
\usepackage{amsfonts}
\usepackage{amsthm}

\usepackage{circuitikz}
\usepackage{pgf}
\usepackage{tikz}
\usetikzlibrary{arrows,snakes,backgrounds}
% \usetikz
\usepackage{subfig}

\usepackage[super]{nth}
\usepackage{appendix}
\usepackage{listings}
% \usepackage{color}

\usepackage{algpseudocode}
\usepackage{algorithm}

\usepackage{hyperref}
%\usepackage{url}

\usepackage{cancel}
\usepackage{slashbox}
\usepackage{cleveref}

\usepackage{aviolov_style} 
\usepackage{local_style}

\newtheorem{thm}{Theorem}[section]
\newtheorem{lemma}{Theorem}[thm]
% \theoremstyle{definition}
\newtheorem{ex}{Example}[thm]
\newtheorem{defn}{Definition}[thm]

\includeonly{chapters/ch2_Math}
% \includeonly{chapters/ch2_Math,chapters/ch6_OptDesign}
% \includeonly{chapters/ch6_OptDesign}
%%
\begin{document}

\pagenumbering{roman}
\pagestyle{empty}
\begin{titlepage}
\centering
\vspace*{1in}
\begin{Large}\bfseries
Parameter Estimation, Optimal Control and Optimal Design in Stochastic Neural
Models
\par
\end{Large}
\vspace{1.5in}
\begin{large}\bfseries
Alexandre V. Iolov\par
\end{large}
\vfill
A Thesis submitted for the degree of Doctor of Philosophy
\par
\vspace{0.5in}
Department of Mathematics
\par
University of Ottawa
\par
\vspace{0.5in}
%TODO: date: September 2015
\today
\par
\vspace{0.5in}
\includegraphics[width=0.15\textwidth]{./UOlogoBW.jpg}
\par
\end{titlepage}

\pagestyle{plain}
\abstract{This thesis poses and solves estimation and control problems
in computational neuroscience, in particular problems dealing with the
stochastic nature of neural systems. The main tool used is the description of
the system by a Fokker-Planck partial differential equation for evolution of
probability densities. 

The thesis deals with three problems in escalating degree of mathematical
sophistication and computational difficulty

The thesis focuses on developing computational schemes in order to solve the
problems. The schemes are tested for a wide range of parameters to demonstrate
their robustness}

% TODO: Acknowledgements
% \chapter*{Acknowledgements}
% I would like to thank my supervisor Andre Longtin, Sussane Ditlevsen at KU,
% colleagues at Ottawa and Copenhagen, parents, \ldots and Kirsten, and Daniel
% and Bombi
 
\tableofcontents 
\cleardoublepage

%TODO (caption[] for each table): 
\listoftables
\clearpage 
%TODO list of figures in captions
\listoffigures

\pagenumbering{arabic}
\pagestyle{headings} 


\chapter{Introduction}
This thesis deals with stochastic problems in parameter estimation and
optimal control arising in computational neuroscience. 

Neuroscience is the study of how information is processed by living beings. Its
main building block is a single neuron cell. A neural cell maintains a certain
membrane potential, $v(t)$, an electrochemical gradient between its interior and
exterior, which is the key mechanism in which information is processed and
transmitted. For our purposed we will assume that $v$ is essentially uniform
throughout the neuron - that is we will idealize the neuron as a single point in
space. In all neural systems, information is processed and transmitted by the
sharp, transient changes in the membrane potential called spikes (TODO: see a
figure). Currently, the dominant theory is that the exact shape of the potential
excursion is irrelevant, but that all information is contained entirely in the
time occurrence of the spike or equivalently the information is in
the duration of the interval between two subsequent spikes.

There is a rich history on modelling the dynamics of the membrane potential
starting from the Nobel-prize winning work of Hodgkin and Huxley in the 50's.
Their model, still widely considered as a benchmark for the dynamics of the
membrane potential, is a 4-dimensional ordinary differential equation (ODE),
where one of the states is the voltage itself and the other three are phenomenological
equations describing the behaviour of ion channels responcible for the
generation of the non-linear excursion - the spike itself. There has been many
reductions of the Hodgkin and Huxley model to a smaller dimension, including the
Fitzhugh-Nagumo model and the Morris-Lecar model, which still retain the basic
excursion non-linearity of the spiking mechanism. A even-more drastic
simplification is to linearize the dynamics and then impose the non-linear
voltage excursion of the neural spike artificially. This leads to the leaky
integrate-and-fire (LIF) model, discussed already by Lapicque in 1907, which
forms the foundation of all the models used in our work.

Neural cell response, especially {\sl in vivo}, is stochastic. Given the same
stimulus the resulting inter-spike intervals will not be the same. There are
several reasons and explanations for why that is. The simplest one is that in
addition to the stimulus the neuron is bombarded by extraneous, random
stimulation, for example from other neurons. To account for this random
behaviour the dynamics of the membrane potential, $v(t)$, are modelled as
following a stochastic differential equation (SDE). SDEs are a generalization
of ODEs, which allow for a stochastic force-term. 

This thesis deals with two main topics - estimation of single-cell neural models
and control of neural dynamics. Mathematically, this amounts to estimation of
parameters in SDEs and control of SDEs. A complication arising from the
practical aspects of neuroscience is that often experimentally the exact value
of the voltage is difficult to observe and it is only the spikes that are
distinctly observable. Mathematically, this implies that our estimation
and control algorithms will often be based on observation of {\sl first
passage} times of the system rather than detailed trajectories. This makes the problem
significantly more challenging and requires some non-standard
algorithms. At the same time, the study of First-Passage Times (FPTs) is a
classic part of the Stochastic Analysis and there are many known
techniques and results that we can build on.

The three problems that are discussed in this thesis can be summarized as
such: in the first problem, we estimate parameters from spike observations; 
in the second, we discuss how to control a neuron to spike at some pre-selected
time; and in the third problem, we combine both approaches whereas we
control the system in order to best estimate its parameters.

The thesis is structured as follows: First, in \cref{ch:math_background} we
introduce the mathematical background from stochastic processes, parameter
estimation and optimal control used in the rest of the thesis; in
\cref{sec:math_models_in_neuroscience} we discuss the standard stochastic models
of a single neuron. The next three chapters,
\cref{ch:estimate,ch:spike_control,ch:optimal_design} form the novel portion of
this thesis where we discuss each of the three problems introduced above. These
three chapters have been adapted from journal papers either accepted for
publication or currently in review. \Cref{ch:estimate} has been published in
TODO: (cite it); \cref{ch:spike_control} has been submitted in TODO: (cite it);
and \cref{ch:optimal_design} is being prepared for submission at \ldots.
Although the journal articles form the foundation of the chapters, each chapter
has been rewritten to achieve a uniform notation and to fit into the overall
flow of the thesis.

In the Conclusion, \cref{ch:conclusion}, we summarize the main findings and
provide a brief outlook.


\cleardoublepage
\chapter{Mathematical Background}
\label{ch:math_background}
<<<<<<< HEAD
=======

\usepackage{amsfonts}
\usepackage{mathrsfs}

>>>>>>> 6d1ee3c9eb52b6bed66343a6488d0f9a4ca3aef0
% DEFINITIONS:
\def \Prob 	  {{ \mathbbmtt{P}  }} %the expectation operator
\def \th 	  {{ \theta}}
\def \FI 		{{\Phi}}
\def \KL     {{ K\!L}}

\def \adot {{ \dot{\alpha} }}
\def \Udot {{ \dot{U}}}
\def \In	{{ i_n}}
\def \vt 	{{ v_{\textrm{th} } }}
\def \Ihat  {{ \hat{I}  }}
\def \p 	{{ \phi  }}

\def \G		{{ \bar{G}  }} %{{ \reflectbox{G} }}
\def \Gest  {{ \hat{\G} }}
\def \Gtilde		{{ \tilde{\G} }}
\def \D		{{ \reflectbox{D} }}
\def \dphi {{ \delta \phi }}

\def \L {{ \Lambda }}
\def \P {{ \Phi }}
\def \n {{\nu}}
\def \Fx {{ F_x}}
\def \Fxx {{ F_{xx} }}
\def \Ft {{ F_t }}
\def \xest {{ \hat{x}_t}}
\def \muncond {{ {m}_x}}
\def \mcond {{ {m}_x^c}}

\def \f {{\rho}}
\def \F {{\Phi}}
\def \Fn {{\mathcal{F}}}

\def \abg		{{\a,\b,\g }} 
\def \aest      {{ \hat{\a} }}
\def \best      {{ \hat{\b} }}
\def \gest      {{ \hat{\g} }}
\def \estabg	{{\aest, \best, \gest}}
\def \abgest	{{\estabg }}

\def \sAlg {{ \mathcal{A} }}
\def \N {{ \mathcal{N} }}
\def \Udomain {{ \mathcal{U} }}
\def \Umax {{ u_{\textrm{max}} }}
\def \umax	{{ u_{\textrm{max}} }}
\def \amin	{{ \a_{\textrm{min}} }}
\def \amax	{{ \a_{\textrm{max}} }}
\def \astar {{ \a^* }}
\def \xth	{{ x_{th} }} 
\def \yth	{{ y_{th} }}
\def \xmin	{{ x_{-} }}
\def \xmax	{{ x_{+} }}
\def \xmid  {{ x_{mid} }}

% \def \x {{ \boldsymbol{x} }}
% \def \u {{ \boldsymbol{u} }}
% \def \p {{ \boldsymbol{p} }}
% \def \q {{ \boldsymbol{Q} }}
% \def \f {{ \boldsymbol{f} }}
\def \tf {{ t_f }}
\def \tc {{ \tau_{c} }}
\def \lc {{ \lambda_{c} }}
\def \ts {{ t_{\textrm{sp} } }}
\def \tn {{ t_n }}
\def \tns {{ \{t_n \} }}
\def \T {{ T^*  }}

\def \free {{\textrm{free} }}


\def \Ttwo {{ \hat{T}_{(2)} }}
\def \Ttwol {{ \hat{T}^{\lambda}_{(2)} }}
\def \Tone {{ \hat{T}_{(1)} }}
\def \Ti {{ \hat{T}_{(i)} }}

\def \Normal {{ \mathcal{N} }}
\def \L {{ \mathcal{L} }}
\def \Lstar {{ \L^* }}
\def \H {{ \mathcal{H} }}
\def \dx {{ \delta\! x}}
\def \da {{ \delta\! \a}}
\def \df {{ \delta\! f}}

% COMP EXAM:
\def \Lonepm {{\mathbb{L}^1}}
\def \Ltwopm {{\mathbb{L}^2}}
\def \Fil 	{{\mathcal{F}}}

\def \x {{ \vec{x} }}
\def \X {{ \vec{X} }}
<<<<<<< HEAD
% parameterized densities / adjoints:
\def \ft {{ f_\th}}
\def \pt {{ p_\th}}
\def \dft {{ \delta f_\th}}
\def \wt {{ w_\th }}

\def \aopt {{\a_{opt} }}
=======

>>>>>>> 6d1ee3c9eb52b6bed66343a6488d0f9a4ca3aef0

Here we collect a list of mathematical tools that are used in the thesis. We
first point the reader to the general references by Oksendal for SDEs
\cite{Oksendal2007} and by Fleming and Rishel for Optimal Control
\cite{Fleming1975}. Another very readable introduction to the field of both SDEs
and Optimal Control are the online notes of Professor L. Evans
\cite{Evansa,Evansb}. We have also used Jacobs as our main tutorial on
first-passage times for SDEs \cite{Jacobs}.
 
\section{Stochastic Differential Equations}
\label{sec:SDEs}
In view of our ultimate goals, we will restrict ourselves to  stochastic
processes whose sample paths are continuous, and more specifically to SDEs
driven by Brownian motion. We will not provide proofs of results in this
section, but only state the results with a view towards establishing the
notation for the rest of the thesis. We will assume that the reader is familiar
with the following concepts:
\begin{itemize} 
  \item a probability space $\{\O, \sAlg, P\}$, where $\O$ denotes the probability space, $\sAlg$ a sigma-algebra, and $P$ a probability measure
  \item a continuous-time stochastic process, $X_t$
  \item a filtration $\Fil_t$, in particular the filtration generated by a
  stochastic process.
\end{itemize}

\subsection{The Wiener Process and the It\^o Integral}

The Wiener Process is the fundamental building block of the stochastic calculus. 
It is often called Brownian Motion and we will denote it $\{W_t\}_{t\geq 0}$. 
\begin{defn}Wiener Process, $W_t$:
\begin{enumerate}
  \item $W_0 = 0$.
  \item $W_t - W_s \sim N(0, |t-s|)$ , i.e. the process increments are
  normally distributed with mean 0 and variance $|t-s|$.
  \item $\forall \{t_i\}_1^N, \quad \{W_{t_i} - W_{t_{i-1}} \}_2^N \sim$
  independent, i.e.\ the process has independent increments.
\end{enumerate}
\end{defn}
The fact that the finite incremental distributions suffice to specify a unique
continuous-time stochastic process is known as Kolmogorov's extension theorem.
Now, we collect a few more relevant properties of the process $W_t$ and its
sample paths:
\begin{enumerate}
  \item The sample paths of $W_t$
are almost surely (a.s.) continuous and are in fact Holder continuous for any
exponent $\g < 1/2$
\item The sample paths of $W_t$ are nowhere differentiable
\item $W_t$ is a Markov process: $\Prob[W_t \in B | \s(W_{s' \leq s })] =
\Prob[W_t \in B \,| \,W_s]$, for any Borel set, $B \subset \R$, where $\s(W_{s'
\leq s })$ is the filtration generated by the $W_t$.
\end{enumerate}

We now turn to defining stochastic integrals based on the Wiener Process.
 
\begin{defn} Progressively Measurable Functions:

Let $\Fil_t$ be the filtration generated by the Wiener Process.

Let $X_t$ be a stochastic process which is $\Fil_t$-measurable $\forall t$ and
which is jointly measurable in $(t,\o)$. We call such an $X_t$
\emph{progressively measurable}.
\end{defn}

\begin{defn} $\Ltwopm, \Lonepm$

We define $\Ltwopm[0,T]$ as the space of all progressively measurable
$X$ such that
\begin{equation*}
\Exp \left[ \int_{[0,T]} X_t^2 \intd{t} \right] < \infty
\end{equation*}

Similarly, we define $\Lonepm[0,T]$ as the space of all progressively measurable
$X$ such that
\begin{equation*}
\Exp \left[ \int_{[0,T]} |X_t| \intd{t} \right] < \infty
\end{equation*} 
\end{defn}

$\Ltwopm$ will be the class of functions for which the It\^o integral is
well-defined. 
\begin{defn} It\^o Integral:
\label{defn:ito_integral}

Let $P^n := {a = t^n_1 \ldots t^n_{m_n} = b}$ be a partition of the interval
$[a,b] \subset [0, \infty)$. Let $|P_n| = \sup_i|t_i - t_{i-1}|$. 
Let $ \lim_{n \rightarrow \infty} |P_n| \rightarrow 0 $ and consider an $ X_t
\in \Ltwopm[a,b]$,

then
\begin{equation}
\int_{[a,b]} X_t \intd{W} := \lim_{n \rightarrow \infty}  
\sum_{i=1}^{m_n} X_{t_i}\left( W(t_{i+1}) - W(t_{i})\right)
\end{equation}

\end{defn}

To be precise, our definition is actually a theorem, and the real
definition is one that uses step functions and passes to the limit.  Also we
will write the limits of integration $\int_0^T \cdot  \intd{W}$ or $\int_{[0,T]}
\cdot  \intd{W}$ interchangeably. 

Again, we state without proof a few interesting properties of $\int X_t
\intd{W}$. \begin{thm} It\^o Integral Properties

\begin{enumerate}
  \item $\Exp[\int_0^T X_t \intd{W} ] = 0$ 
  \item $\Exp[\left(\int_0^T X_t \intd{W}\right)^2 ] = \Exp[\int_0^T X_t^2
  \intd{t}]$
  \item $I(t) = \int_0^t X_t \intd{W} $ is a martingale 
  \item $I(t) = \int_0^t X_t \intd{W} $ has continuous sample paths.
\end{enumerate}
\end{thm}

\subsection{It\^o SDEs and It\^o's Lemma}
\begin{defn}[It\^o SDE] Let
$B(x,t):\R\times [0,T]\ra \R$, $G(x,t):\R\times [0,T]\ra \R,$ be given
functions. We say that a stochastic process $X$ satisfies:
\begin{equation}
dX =B dt + G dW
\end{equation} 
over $[0,T]$ if
\begin{enumerate}
  \item $X$ is progressively measurable wrt. $\Fil_t$
  \item $B(X,t) \in \Lonepm[0,T]$
  \item $G(X,t) \in \Ltwopm[0,T]$ 
  \item $X_t = X_0 + \int_0^t B(X_s, s) \intd{s} + \int_0^t G(X_s, s) \intd{W_s}$.
\end{enumerate}
\end{defn}  

We are now ready to present the celebrated It\^o Lemma which is the chain-rule of
stochastic calculus.

\begin{thm}[It\^o Lemma]
Suppose $X$ satisfies the stochastic differential $dX_t =
B(X,t)dt + G(X,t) dW$ as above and take $v(x,t) \in C^{2,1}[ \R \times [0,T]]$,
that is $v$ is twice-continuously differentiable wrt.\ its first argument and
once wrt.\ its second argument.

Set $Y(t) = v(X,t)$
\\
then
$$
dY =  \left( \di_t v + \di_x v \cdot B + \di^2_x v \cdot
\frac{G^2}2 \right)
\intd{t} + \left(   \di_x v\cdot G  \right)\intd{W}
$$ 
\end{thm} 
 

As an example we will discuss the Ornstein-Uhlenbeck process, which is the
basis for the models we will face later on.
\begin{ex}[O-U Process] Let $X_t$ follow:
\begin{equation}
dX = \left( \frac {\m -X_t}{\tc} \right) dt + \b dW
\label{eq:OU_equation_generic}
\end{equation}
with an initial condition, $X_0$, which may be an arbitrary distribution
independent of the Wiener Process, $W$.
We can solve this as follows: 
\begin{align*}
dX =& \left( \frac {\m -X_t}{\tc} \right)  dt + \b dW
\\
dX + \frac {X_t}{\tc} dt=&  \frac\m\tc dt + \b dW
\\
e^{t/\tc} dX + e^{t/\tc}\frac {X_t}{\tc} dt
=& e^{t/\tc}\frac\m\tc dt + \b e^{t/\tc} dW
\\
Xe^{t/\tc} - X_0 
=& \int  e^{t/\tc}\frac\m\tc dt +  \int \b e^{t/\tc} dW
\\
X_t =& e^{-t/\tc} X_0 + \m(1-e^{-t/\tc}) +  \frac{\sqrt{\tc}\b
e^{-t/\tc}}{\sqrt{2}} W(e^{2t/\tc}-1)
\end{align*}
which means that $X$ forgets its initial conditions exponentially fast and
converges to a normal random variable with mean $\m$ and variance
$\tfrac{\tc \b^2}{2}$.
\end{ex}

At the end of this sub-section, we state It\^o's Lemma for a multidimensional
state.
 
% \begin{thm}[It\^o Lemma for $dW \in  \R^{m}$] Suppose $X$ satisfies the
% stochastic differential $dX_t = F dt + G dW$, where $ F \in \R, G \in
% \R^{1\times m}$ and $dW = \left(dW^{(i)}\right) \in  \R^{m}$ is a vector of
% independent Wiener Processes. Take $u(x,t) \in C^{2,1}[ \R \times [0,T]]$. Let
% $D :=\left[{\sum_k G_{k} G_{k} } \right] \in \R$. Set $Y(t) = u(X,t)$. Then $$
% dY =  \left( \di_t u +  \di_{x} u \cdot F + \di^2_{x} u \cdot D \right)
% \intd{t} + \left(  \di_{x} u  \cdot  G \right) dW $$ \end{thm}

\begin{thm}[It\^o Lemma for $X \in R^n$, $dW \in  \R^{m}$]
Suppose $X$ satisfies the stochastic differential $dX_t = B(X,t) dt + G(X,t)
dW$, where $B : [R^n \times [0,T] \ra  \R^n]$, $G: [R^n \times [0,T]\in
\R^{n\times m}$ and $dW = \left(dW^{(i)}\right) \in  \R^{ m}$ is a vector of independent Wiener Processes. Take $u(x,t) \in C^{2,1}[ \R^n
\times [0,T]]$. Let $D_{ij} := \tfrac{1}{2} \left[{\sum_k G_{ik} G_{jk} }
\right]_{ij}$. Set $Y(t) = u(X(t),t)$. Then
$$
dY =  \left( \di_t u + \sum_i \di_{x_i} u \cdot B_i + 
\sum_{i,j} \di^2_{x_i x_j} u \cdot D_{ij} \right)
\intd{t} +
 \left(  \sum_{ij} \di_{x_i} u  \cdot G_{ij} dW^{(j)} 
\right)$$
\label{thm:ito_lemma}
\end{thm} 

\subsection{Fokker-Planck and Kolmogorov's Backward Equations}
The Fokker-Planck equation and Kolmogorov's Backward equation describe the
forward (resp. backward)  evolution of the probability density of $X_t$.

For the rest of this section we will work in $ \R^n$, i.e. $X$ will be an $n-$dimensional vector that satisfies the stochastic differential
\begin{equation}
dX_t = B(X,t) dt + G(X,t) dW,
\label{eq:generic_Ito_SDE_Rn}
\end{equation}
using the notation defined in Theorem \ref{thm:ito_lemma}. 

Let 
\begin{equation}
 \f(x,t| y,s) \intd{x} =  \Prob[X_t \in \intd{x} | X_s = y]
 \label{eq:transition_prob_defn} 
 \end{equation}
be the transition probability density. Then $\f$ satisfies:
\begin{equation}
\di_t \f(x,t| y,s)= -\sum_i \di_{x_i} \left[ B_i(x,t) \f(x,t| y,s) \right] 
+ 
\sum_{i,j}  \di^2_{x_i x_j} \left[ D_{ij}(x,t) \f(x,t| y,s) \right].
\label{eq:generic_FokkerPlanck}
\end{equation} 
This is called the \emph{Fokker-Planck} or \emph{Forward Kolmogorov} equation.
It can be seen as an equation describing the conservation of probability. To this end define the
probability current $\phi \in \R^n$ and its $i$th component $\phi_i \in \R$
as:
\begin{equation}
\phi_i =   B_i(x,t) \f(x,t| y,s) 
- \sum_{j}  \di_{x_j} \left[ D_{ij}(x,t) \f(x,t| y,s)\right].
\label{eq:FokkerPlanck_prob_flux}
\end{equation}
Then the Fokker-Planck equation, \cref{eq:generic_FokkerPlanck}, can be written as:
\begin{equation}
\di_t \f= -\grad \cdot \phi,
\label{eq:Fokker_planck_as_prob_flux}
\end{equation}
which just says that the change in probability is the
difference between the flow in and the flow out. 

Initial conditions for $\f$ are given by the distribution of $X_s$. 
Boundary conditions (BCs) depend
on the domain of $X$ and what happens to $X$ once it hits its boundary. If the domain
is all of $\R^n$ then we only insist that $\lim_{|x| \ra \infty} \f = 0$. If the
domain has boundaries, then there are two common scenarios which we will also
encounter in the sequel:
\begin{enumerate}  
  \item absorbing BCs
  \item reflecting BCs.
\end{enumerate}
At an absorbing boundary, the particle $X$ is removed and $\f=0$ there.
In this situation, we will not have conservation of probability and the integral
of $\f$ over $X$'s domain will monotonically decrease.

At a reflecting boundary, the particle $X$ bounces back into its domain and we
will have that the probability current $\p\cdot n = 0$ where $n$ is the outward
normal at the reflecting boundary.

Now consider $\f$ as a function of the initial values $y,s$, holding the
terminal values $x,t$ fixed. Then
\begin{equation}
-\di_s \f(x,t|y,s)= \sum_i  \left[B_i(y,s) \cdot \di_{y_i}\f(x,t|y,s) \right] 
+ \sum_{i,j}   \left[ D_{ij}(y,s) \di^2_{y_i y_j}\f(x,t|y,s) \right]
\label{eq:FP_backward_pde}
\end{equation} 
This is the \emph{Backward Kolmogorov} equation for $\f$. Note the minus sign
in front $\di_s \f$. A mnemonic for the signs of the Backward vs. Forward
equation is that as $t$ increases in the forward case, $\f$ diffuses and so $\di_t$ and $\di^2_{x}$
have the same sign; in the backward case, as $s$ increases, that is as $s$ approaches $t$, $\f$
anti-diffuses and the two partial derivatives have opposite signs. 
  
The differential operator on the right-hand side of
\cref{eq:FP_backward_pde} occurs often in the study of SDEs and has its own
name.
\begin{defn}[Generator of an SDE] The Generator $A$ of $X$ is defined by:
$$
A[\psi(x)] = \lim_{t \searrow 0^+} \frac{\Exp[\psi(X_t)]  - \psi(x)}{t} ;  \quad X_0 =
x \in \R^n
$$
\end{defn} 
\begin{lemma} For an It\^o SDE as in \cref{eq:generic_Ito_SDE_Rn} 
$$
A[\psi(x)] = \sum_i B_i(x,t) \cdot \di_{x_i} \psi + \sum_{i,j} D_{ij}(x,t)
\di^2_{x_i x_j}\psi $$
\end{lemma}


\subsection{First-Hitting Times}
Most of the problems in this thesis are related to {\sl first-hitting times}.

\begin{defn}
Let $\Fil(t)$ be some filtration, a random variable $\t$ is called a
\emph{stopping time} if 
$$
\{\o : T(\o) \leq t\} \in \Fil_t \, \forall t  
$$
\end{defn}
The colloquial way of describing stopping times is that at any time we know
whether $T$ has occurred or not. For a counterexample, the time that a Wiener
Process achieves its maximum over some interval is not a stopping time, since at any
given time, we do not know if the maximum has occurred or not. The most common
example of a stopping time is the first hitting-time, which is the first time
$X_t$ leaves or enters some set. 
\begin{thm}[First-Hitting time] Let $E \subset \R^n$ be open or closed and non-empty
 
then $$T := \inf\{ t \geq 0 | X_t \notin E\}$$ is a stopping time.
\end{thm}
 
The reason stopping times are very useful is that all the facts so far quoted
for It\^o calculus using integrals $\int_0^t \intd{W}$ remain true if the
fixed time $t$ is replaced by a stopping time $T$. Also it allows us to link
SDE's and PDEs using the generator, $A$, of the diffusion: 
\begin{thm}[Dynkin's formula] Given $T$ a stopping time, $\Exp[T] < \infty$.
Then:
$$
\Exp[u(X_T, T) | X_0 =x] =
u(x, 0) + \Exp\left[\int_0^T \di_t u(X_t, t) + A[u(X_t,t) | X_0=0] \intd{t}
\right].
 $$
\end{thm}
This provides a link between PDEs and stochastic processes and allows us to go
back and forth in that we can find probabilistic results by
solving a PDE or we can approximate a PDE by simulating a stochastic process
and averaging.

This thesis is primarily concerned with the analysis of first-hitting times.
Thus we will often refer to the probability of a first-hitting time, $T$, to
take the value $t$ as
$$g(t) := \frac{1}{\intd{t}} \Pr(T \in [t, t+\intd{t}) |X_0 = x_0).$$

There is a simple relation between the hitting-time density, $g$, and the
transition density of $X_t$. Let $T$ be the first-hitting time of $X$ to the
boundary $\di E$ of the closed domain $E\subset  \R^n$, i.e.\ $T$ is the first
time that $X_t$ leaves $E$.  
Then if $f$ solves the
Fokker-Planck equation for $X$, such that $f\equiv 0$ on $\R^n \setminus E$, and
if $\phi$ is its probability flux defined in \cref{eq:FokkerPlanck_prob_flux}, then
we will have that
\begin{equation}
g(t) = \int_{\di E} \phi \cdot n \intd{S},
\label{eq:hitting_time_g_in_terms_of_outflow}
\end{equation}
where $n$ is an outward normal to $E$, the integral is a surface integral. 

We give a quick derivation of \cref{eq:hitting_time_g_in_terms_of_outflow}. Let
$$G(t) = \Pr(T\leq t | X_0=x_0) = \int_0^t g(s) \intd{s}$$ be the cumulative
distribution corresponding to $g$. Then $G(t)$ is the probability
that at time $t$ the particle has left the domain and so 
$$G(t) = 1- \int_E f(x,t|x_0,0) \intd{x}$$

Since $g(t) = \di_t G(t)$, we then have that  
$$g(t) = - \di_t \int_E f(x,t|x_0,0) \intd{x} = \int_E \grad \cdot
\phi(x,t|x_0,0) \intd{x} = \int_{\di E} \phi \cdot n \intd{S}.$$
where the second equality comes from the probability form of the Fokker-Planck equation,
\cref{eq:Fokker_planck_as_prob_flux}, and the third  equality is the divergence
(Gauss') theorem.

In the simplest case, of working in 1-D, such that $E = (-\infty, \xth)$ and
$\xth>x_0$, we will have 
$$  g(t) = \phi(\xth, t) - \phi(-\infty,t) = \phi(\xth, t),$$
since, at the lower boundary the probability flux $\phi(-\infty,t)=0$. Using
the expression for $\phi$ from \cref{eq:FokkerPlanck_prob_flux}, we arrive at 
 $$  g(t) = B(\xth,t)f(\xth,t ) -\di_x[ D(\xth,t) \cdot f(\xth,t )]= -\di_x[
 D(\xth,t) \cdot f(\xth,t )].$$
 Since, $f(\xth,t)=0$. In neural applications this is the form of $g$, that we
 will encounter. 

% \begin{ex}[\cite{Evansb} pg 99 - Expected hitting time to a boundary]
% \label{ex:mean_hitting_time}
%  Let $\O
% \subset \R^n$ be a bounded open set with smooth boundary $\di \O$ then it is a
% basic fact from PDEs theory that
% \begin{equation}
% \begin{cases}
% -\frac{1}{2} \grad^2 u = 1  & \text{over } \O
% \\
% u =  0 &\text{on } \di \O
% \end{cases}
% \end{equation}
% has a unique $C^\infty(\O)$ solution.
% 
% Let $X = W_t + x$ for any $x \in \O$ and define
%  $$\t_x := \text{first time } X \text{ hits } \di \O$$
% then the generator of $X$ is $A[\psi] = -\grad^2(\psi)/2$ and we will
% have:
% 
% \begin{align*}
% \Exp[u(X_\t)] - \Exp[ u(x_0)] =&
%  \Exp \left[ -\frac{1}{2}\int_0^\t \grad_x^2 u(X_t) \intd{t} \right]
% \\
% =& \Exp \left[ - \int_0^\t 1  \intd{t} 
% \right]
% \\
% =-&\Exp [\t]
% \end{align*}
% Finally, invoke $u$'s BCs, $u|_{\di E} = 0$, and $X$'s ICs, $X_0=0 = x$ to
% conclude that $$ u(x) = \Exp [\t]$$
% The solution to the PDE evaluated at $x$ is the expected exit time from
% $E$ for an $X_t$  starting at $x$.
% \end{ex}


\subsection{Numerical Simulation of SDEs}
In general SDEs, such as \cref{eq:generic_Ito_SDE_Rn}, cannot be solved
analytically and if one wants to obtain approximate paths from their solution,
one needs numerical methods. There are many numerical methods for approximating
an SDE as in \cref{eq:generic_Ito_SDE_Rn}, see e.g. \cite{Higham2001} for a
popular introduction. Here we will only describe the most basic method, the
Euler-Maruyama scheme, which will suffice for our purposes. In the
Euler-Maruyama scheme, the approximate solution to \cref{eq:generic_Ito_SDE_Rn}
is computed at predetermined time-nodes, $\{t_k\}_{k=0}^N$ with time intervals
$\Delta t_k = t_{k+1} - t_{k }$, which are often assumed to be constant, i.e.
$\Delta t_k = \Delta t\,\,\, \forall k$. Then given an initial condition $X_0 =
X(t_0)$, the approximate path, $\{X_k\}_{k=0}^N = \{X(t_k)\}_{k=0}^N$ is
obtained iteratively via:

\begin{equation}
X_{k+1} = B(X_k,t_k)  \Delta t_k   + G(X_k,t_k) \xi_k \sqrt{\Delta t_k }
\label{eq:euler_maruyama_discretization_generic_Ito_SDE}
\end{equation}
where $\xi_k$ is an independent draw from the standard normal distribution. 


\section{Mathematical Models in Neuroscience}
\label{sec:math_models_in_neuroscience}
We now describe in detail the basic mathematical model of a neuron that will be
used throughout most of this thesis. It is just a basic linear SDE together with an associated
first-hitting time specification.

An introduction to mathematical models in neuroscience is given in Gerstner et
al., \cite{Gerstner2014}, also available online. The book Stochastic Methods in
Neuroscience \cite{Laing2009} provides a nice overview of several current research
applications of stochastic techniques to neuroscience.

A neuron's most important property is its membrane potential - the electric
potential difference between the cell's interior and its surroundings. Neurons
relay information by means of voltage spikes - sudden sharp increases in their
membrane potential, which are then transmitted to other neurons that are
chemically or electrically connected to the spiking neuron. The information
content is thought to be contained in the timing of the spike, or equivalently
in the length of the time-interval between subsequent spikes. In the simplest
case, this can be thought of as a {\sl firing rate}, i.e. the average number of
spikes over some time interval, but more complicated coding schemes have been hypothesized
to exist. A spike in a given neuron can be triggered as a result of spikes coming from neurons 
connected to it. Special neurons called sensory neurons interface with external physical stimuli
 such as mechanical vibrations in the case of auditory neurons and light photons in the case of visual neurons in the retina.  

Given a small positive stimulus, the voltage of a neuron increases linearly, but
then relaxes back down to its equilibrium, pre-stimulus, level. However, if the
stimulus is sufficiently strong or if there are many small stimuli in a
sufficiently short time interval, the voltage will then go through a stereotypical large
 non-linear excursion before resetting back down to its equilibrium
 - this is a spike, which can propagate along the axon to other neurons. 
 The physical underpinnings of this event involves ion
 channels with different time-scales, with an initial fast excitation due to sodium ions (known as a depolarization) 
 followed by a slightly slower counter-acting inhibition by potassium ions (repolarization). 
 The full spike unfolds on the order of 1-5 milliseconds. 
 
 The first successful mathematical model describing this
 phenomena is the famous Hodgkin-Huxley (HH) model, which forms a system of 4
 ODEs, one for the membrane voltage and three for the dynamics of the ion
 channels. From a practical point of view however, the HH model of
neuron dynamics is quite complicated if one is only interested in spike times,
 rather than the detailed contributions of ion channels to the voltage. Thus there have been several
 reductions to this model. The standard approach is to keep the voltage equation, 
 as well as one more equation as a 'recovery' variable which works on a slower
 scale than the voltage; together these two variables produce the very sharp up-and-down voltage
 excursions. Two popular examples of such reductions are the Fitzhugh-Nagumo
 model and the Morris-Lecar model. The Morris-Lecar model
 is a 2-dimensional ODE for, respectively, the voltage and recovery variables $v(t), w(t)$. It reads:
\begin{equation}
\left\{
\begin{array}{ccl}
\dot{v}(t)  &=& \frac{1}{C}\Big(-g_{Ca}m_\infty(v) (v-V_{C_a}) -
g_K w (v-V_K) \\ && 
-g_L(v-V_L)+I(t)  \Big) \\
\dot{w}(t)&=& \alpha(v)(1-w) - \beta(w)w
\end{array}
\right.
\label{eq:ML_original_deterministic}
\end{equation}
where the auxiliary channel gating functions, $m_\infty, \alpha, \beta$ are given by:
\begin{eqnarray*}
m_\infty(v)&=&\frac{1}{2}\left(1+\tanh\left(\frac{v-V_1}{V_2}\right)\right),\\
\alpha(v) &=& \frac{1}{2}\phi \cosh\left(\frac{v-V_3}{2V_4}\right)\left(1+\tanh\left(\frac{v-V_3}{V_4}\right)\right),\\
\beta(v) &=& \frac{1}{2}\phi \cosh\left(\frac{v-V_3}{2V_4}\right)\left(1-\tanh\left(\frac{v-V_3}{V_4}\right)\right).
\end{eqnarray*} 
The term $I(t)$ represents current applied on the neuron, either from natural
external stimulation, an experimental control, or synaptic or electrical inputs from other neurons. 
In \cref{sec:morris_lecar_control}, we describe how the Morris-Lecar
deterministic ODE of \cref{eq:ML_original_deterministic} is extended to an SDE, and we demonstrate how its spikes times can be optimally controlled in spite of the noise.
 
The 2-dimensional models like Morris-Lecar are much easier to work with
mathematically and experimentally, for example for parameter estimation, but
they still require a lot of effort for describing the details of a neural spike.
If one is only interested in the timing of the spike then yet another
simplification is to keep only the voltage dynamics, and then declare
that a spike has occurred whenever the voltage crosses some appropriate threshold,
$v_{thresh}$, which can be related to the sodium activation threshold in real neurons. 
For analytical purposes this turns out to be both convenient and
satisfactory. The basic integrate-and-fire model then is just
\begin{equation}
\begin{gathered}
\dot{v} = \left(I(t) - \frac{(v - \m)}{\tc} \right) \intd{t} 
\\
v(\ts) = v_{thresh} \implies  
\begin{cases}
v(\ts^+) = 0 &  
\end{cases}
\end{gathered}
\label{eq:deterministic_IF}
\end{equation}

In many areas of the brain, due to the large number of connections, 
the external stimulation to a neuron is highly erratic and
unpredictable. When the randomness is mostly in this input, $I$,
one can approximate its effect by adding to the input $I$ a Gaussian white 
noise. In the terminology of this thesis, this is just a term proportional to
a Wiener process increment, $\beta dW_t$. 
Thus the final model of the basic spike-generation mechanism that accounts for
its most important aspects yet retains analytical tractability is the noisy
leaky-integrate-and-fire model:
\begin{equation}
\begin{gathered}
dX_t = \left(\a(t) - \frac{(X_t - \m)}{\tc} \right) \intd{t} + \b \intd{W_t},
\\
X(0) = 0,
\\
X(\ts) = \xth \implies  
\begin{cases}
X(\ts^+) = 0 &  
\end{cases}
\end{gathered}
\label{eq:X_evolution_uo_math_ch}
\end{equation}
That is, $X_t$ follows an Ornstein-Uhlenbeck (OU) process, but upon reaching a pre-determined
threshold, $\xth$, a 'spike' is deemed to have occurred and the process is
'reset' to an initial value, here $0$.

Alternatives to the hard-threshold integrate-and-fire model are so-called
'soft-threshold' models which use a {\sl hazard} function, which is akin to the
intensity of a Poisson Process to determine the spike time. The hazard function
increases as the voltage increases, thus higher voltages imply higher likelihood of
spiking. The hard-threshold model can be seen as a special case of the
soft-threshold model, with the hazard function equal to 0 below the threshold
and infinity above the threshold. We will not address hazard-function-based
spiking models in the rest of the thesis. 

\section{Parameter Estimation for SDEs}
\label{sec:estimation}
Let us rewrite the SDE in \cref{eq:generic_Ito_SDE_Rn} so that the 
functions $B, G$ are explicitly parametrized by some parameter set, $\th$;
\begin{equation}
dX_t = B(X,t;\th) dt + G(X,t;\th) dW,
\label{eq:generic_Ito_SDE_Rn_parameterized}
\end{equation}
For example, in the case of the OU process, \cref{eq:OU_equation_generic},
$B(X,t;\th) = (\m - {X_t})/{\tc}$ and $G(X,t;\th) = \b$, and the
parameter set is $\th = \{\m, \tc, \b\}$. In the standard formulation of the SDE estimation problem,  
one has exact observations $\{x_n\}_{n=0}^N$ at times $\{t_n\}_{n=0}^N$ from a process $X_t$
satisfying \cref{eq:generic_Ito_SDE_Rn_parameterized}, and one seeks to find the
values of the parameter set $\th$.

\subsection{Maximum Likelihood Estimation}
A fundamental method, both
practically and theoretically, for estimating parameters in an SDE is the {\sl Maximum
Likelihoood} (ML) method, which proceeds by seeking those parameters which
maximize the likelihood of the observed data, $\{x_n\}_{n=0}^N$. In particular
let $$L(\{x_n\}; \th) = \Prob[ X_0 = x_0\ldots X_n = x_n\ldots X_N = x_N |
\th]$$ be the joint probability of observing the data $\{x_n\}$ given the
parameter set $\th$, also known as the likelihood. Then the ML method seeks to
maximize $L$.

In the case of independent observations, the
likelihood is just the product of the individual probabilities of each observation. In the case of SDEs, the problem is only slightly more complicated, due to the Markov nature of the stochastic process. In particular,
the likelihood becomes the product of the transition probabilities. Recall that
earlier we defined the transition probability $\f(x,t | y,s)$ in
\cref{eq:transition_prob_defn}. We shall also write this as $\f_\th(x,t|
y,s)$ if we need to be reminded of $\f$'s dependence on the
parameter set, $\th$. The likelihood of the observed $x_n$ then
becomes
\begin{equation}
L(\{x_n\}; \th) = \prod_{n=1}^{N} f_\th(x_n, t_n| x_{n-1}, t_{n-1})
\label{eq:SDE_discrete_likelihood}
\end{equation}
Here, we assume that $x_0$ is fixed, otherwise we would have to add a term
specifying the probability distribution of $X_0$.

In general the transition density for a generic SDE is impossible to find
analytically. There are several ways to approximate it numerically. The most
generic way relies on the numeric solution of the Fokker-Planck PDE in
\cref{eq:generic_FokkerPlanck}, but that is computationally quite expensive and suffers from the
'curse-of-dimensionality' for SDEs of higher dimension. 

In some simple cases, the transition density {\sl can} be calculated. The OU
example described above is one such case, where the transition density is
\begin{align*}
f(x_n, t_n| x_{n-1}, t_{n-1}) &=
 f(x_n, \Delta| x_{n-1}, 0)\\& =
 \frac{1}{ \b \sqrt{\tc 2\pi(1 -  e^{-2 \Delta/\tc}})}
 	\cdot \exp\left(\frac{\left( x - \mu)  - (x_{0} - \mu) \cdot
 	 e^{-\Delta/\tc} \right)^2  } {\t \s^2  (1-e^{-2 \Delta/\tc})}
 	\right) 
\end{align*}
assuming that $\Delta_n = t_n-t_{n-1} = \Delta$ is constant for all $n$.
With this expression, one can form the likelihood and solve analytically for the
maximizers $\{\hat\m , \hat\tc, \hat\b \}$:
\begin{eqnarray} 
\hat{ \mu} &=& 
\frac{  \sum_{n=1}^{N } 
     \left( X_n - e^{-\frac{\Delta} {\hat \tc}} X_{n-1} \right)} 
	 { N( 1-e^{-\frac {\Delta} {\hat \tc}}) }
\\
e^{-\frac {\Delta}{\hat{\tc}} } &=& 
\frac { \sum_{n=1}^{N} 
			( X_n -  \hat \mu)(X_{n-1} -  \hat \mu) }
    {   \sum_{n=1}^{N } \left( X_{n-1} - \hat \mu
    \right)^2 }
\\
\hat\beta^2 &=&  
\frac{ 2  \sum_{n=1}^{N}  \left( X_n - \hat \mu - (X_{n-1} -
\hat \mu) e^{-\frac {\Delta} {\hat \tc}} \right)^2 } 
	  { N (1-e^{-2\frac {\Delta} {\hat \tc}}) \hat \tc}
\end{eqnarray}
The solution for the ML estimates is almost explicit. It requires one numerical
single-dimensional root-finding, which is an easy numerical task. Note that
there is no guarantee that the estimate for $\hat{\beta}$ will be positive in
general.

\subsection{Fortet Equation} 
In the context of \cref{eq:X_evolution_uo_math_ch}, that is for a 1-dimensional
SDE with a boundary above the initial value of the process, $\xth>x_0$, there is
another relation between the hitting-time density, $g$ and the unconditional transition density,
$f$, that is $f$ for which no boundary conditions are applied at $\xth$.

Consider the space integral of $f$
$$ \F(x,t|x_0, t_0) = \int_{\xi<x} f( \xi, t |x_0, t_0)) \intd{\xi},$$
which is just the probability that $X_t \leq x$, conditional on the initial
state, $x_0$. The {\sl Fortet equation} \cite{Fortet1943} states that
\begin{equation}
1 - \F(\xth, t|0, 0) =
\int_0^t g(s) [1-\F (\xth,  t| \xth, s)] \intd{s}.
\label{eq:Fortet}
\end{equation}
The left hand side is simply the probability of exceeding $\xth$ at time
$t$ starting at $0$ at time $0$. This can also be written as the probability
of hitting $\xth$ for the first time at time $s < t$ and then exceeding
$\xth$ at time $t$ starting at $\xth$ at time $s$, integrated over all $s$.

The Fortet equation is particularly appealing to use in parameter estimation
contexts when we have an analytical expression for the unconditional $\F$ as is
the case for example for the Ornstein-Uhlenbeck process. We will see in
\cref{ch:estimate} that \cref{eq:Fortet} can be extended to the case when the
threshold boundary or the dynamics of $X_t$ are not time-homogeneous.

\subsection{Particle Filtering}
In {\sl Bayesian} approaches to parameter estimation, the unknown parameter,
$\th$ is also treated as a Random Variable, $\Th$, with some belief
distribution: 
\begin{equation}
\rho(\th) = \Prob(\Theta = \th)
\label{eq:bayesian_prior} 
\end{equation} which is also called a
prior. Via applications of the Bayes formula, observations from the system are
used to update the belief distribution.

Let us say that the Random Variable $T$ depends on the Random Variable, $\Th$,
which we denote by writing the probability density of $T$ as 
\begin{equation}g(t|\th) =
\Pr(T=t|\Th=\th).
\label{eq:bayesian_observation}
\end{equation}
Suppose we observe $T=t$. Then Bayes' formula
states that 
\begin{equation}
\rho(\th|  t  ) = 
\frac{  \rho(\th) \cdot g(t|\th ; \a) }
	 { \int_\Th  \rho(\th) \cdot  g(t|\th ; \a)  \intd{\th}}
\label{eq:bayesian_formula}
\end{equation}

In practice, exact calculation of $\rho(\th|t)$ would not be possible in our
context, so an approximation approach needs to be made. 

A standard technique is to approximate $\rho$ by a set of weighted particles. To
avoid repetition, we refer to a quick description of how this is done in
\cref{ch:optimal_design}, \cref{sec:intro_to_particle_filtering}, where it is
used.
 
\subsection{Optimal Design}
\label{sec:optimal_design}
There are estimation problems in which the experimenter has some control over
some of the parameters in the model and may choose to set them in order to
facilitate the estimation task. {\sl Optimal design} is the design approach for
statistical experiments that is guided by optimizing some formal measure of the
parameter estimates of the experiments, \cite{Pukelsheim2006}. For example,
perhaps one would like to perform linear regression with a polynomial model and
one has some latitude over the points at which to evaluate the polynomial.
Common criteria when selecting the design, e.g the polynomial points, are
minimizing the determinant or the trace of the covariance matrix of the
parameter estimators. However, the covariance matrix, directly related to the
so-called Fisher Information, often depends on the very parameters one seeks to
estimate.

An alternative to the Fisher Information as an objective for the formal design
of experiments is to use concepts from Information Theory, \cite{MacKay2003}.
A related topic in the Machine Learning literature is called 'Active Learning',
e.g. see \cite{Cohn1996,Settles2010,Seeger2008}. In the third part of the
thesis, \cref{ch:optimal_design}, we will use the {\sl Mutual Information}
criteria as a guideline for choosing the stimulation that best allows parameter
estimation. Thus we now define and discuss the Mutual Information between two
random variables $X, \Th$. For reference, we follow \cite{MacKay2003}.

\begin{defn}Mutual Information:
Given two random variables $T,\Th$ with joint probability density
$p(x,\th)$ and marginal densities $p(x), p(\th)$, the Mutual Information between
$T$ and $\Th$ is given by
\begin{equation}
I(T,\Theta) = \int_\Theta \int_T p(x,\th) \cdot \log \left(
\frac{p(x,\th)}{p(x)p(\th)}\right) \intd{x} \intd{\th}
\label{eq:mutual_info_defn}
\end{equation}
\end{defn}

It is obvious that if $T,\Th$ are independent, then $I(T,\Theta) = 0$. It can
be verified that $I\geq0$ and that it is maximized if $\Theta$ is a function of
$T$; that is, if the entropy of $\Th$ conditional on $T$ is zero. 
The mutual information, $I(T,\Th)$ represents the 'average
reduction in uncertainty about $\Th$ that results from learning the value of
$T$' \cite{MacKay2003}. This statement is formally correct if one takes
 'uncertainty' to mean the entropy of a random
 variable.
 
In order to make use of the Mutual Information in a parameter estimation
context, we again recall the prior and likelihood, $\rho(\th), g(t)$ as defined
in \cref{eq:bayesian_prior,eq:bayesian_observation}.

If we consider $\Th$ as a parameter and $T$ as an observation, it is then
natural to seek an experiment which maximizes the Mutual Information between
$\Th$ and the observation $T$.
 
 The joint distribution can be written as $p(t,\th) =
g(t|\th)\rho(\th)$; the marginal distribution of $\Theta$ is simply the prior, $p(\th) =
\rho(\th)$; and the marginal of $T$ is $p(t) =
\int_\Theta g(t|\th)\rho(\th) \intd{\th}$.
Plugging the three expressions into the definition in
\cref{eq:mutual_info_defn} and cancelling common terms yields
\begin{equation}
I(T,\Th) = \int_\Theta \int_0^{\infty} g(t|\th) 
\log \left( \frac{g(t|\th) }{\rho(\th)\int_\Theta g(t|\th)\rho(\th) \intd{\th}
 } \right)
\intd{t}\intd{\th}.
\label{eq:mutual_info_prior_trajectory}
\end{equation}
\Cref{eq:mutual_info_prior_trajectory}  is used in
\cref{ch:optimal_design}.


\section{Stochastic Optimal Control}
Optimal Control Theory has three main components - a state, $x$, a control,
$\a$, and an objective $J$ which is a functional of $\{x,\a\}$ and which we try
to either minimize or maximize. In {\sl Stochastic} Optimal Control, $x$ follows
a stochastic process, thus the objective, $J$ is most often expressed in terms
of some average or expectation over the random realizations of $x$. The general
theory of Optimal Control, as well as the subset dealing explicitly with random
systems, has relied on two main analytical techniques - {\sl the Maximum
Principle} and {\sl Dynamic Programming}, (a classic reference is
\cite{Fleming1975}). The Maximum Principle uses a variational approach to
characterize the optimal pair, $x_{opt}, \alpha_{opt}$ optimizing $J$, while Dynamic Programming recursively builds up
the optimal solution with a backwards induction from the terminal conditions.
Both techniques have their advantages and disadvantages and we use both in the
thesis. It turns out that there is a close relation between the two approaches,
which is well known in the deterministic finite-dimensional deterministic case,
\cite{Fleming1975,Evansb} and less so in the stochastic case,
\cite{Annunziato2014}.

To set the notation right, we consider the following functional
\begin{equation}
J[\alpha] = \Exp_X \left[ \int_0^\tf L(X_t, \a_t) \intd{t} + M(X_\tf) \right]
\label{eq:generic_objective_functional} 
\end{equation} 
over the realizations of $X_t$ governed by an It\^o SDE as in
\cref{eq:generic_Ito_SDE_Rn}, such that the drift $B$ and/or the diffusion
coefficient, $G$ are parametrized by the control $\a(t)$. Here $L$ is the
running cost function, which depends on the trajectory and the applied control,
while $M$ is the terminal cost function, which just depends on the value of the
trajectory at the terminal time, $t_f$. Here we assume that the terminal time is
a priori known, although this is not necessary in general. 

Given $J$ in \cref{eq:generic_objective_functional}, we then seek $\a$, which
maximizes it (for example): 

\begin{equation}
\a^* = \argmax_{\a \in \Udomain} [ J[\a] ]
\label{eq:generic_optimization_statement}
\end{equation} 

To be mathematically correct, we should specify that the optimization in
\cref{eq:generic_optimization_statement} is done over the space $\Udomain$ of
stochastic processes that are measurable with respect to the filtration
generated by the underlying Wiener process, $W_t$. Further constraints on
$\Udomain$ may be imposed by a specific problem.

The simple-looking \cref{eq:generic_objective_functional} can give rise to
different variations. For one, the final-time $\tf$ may be variable. Or we may
face path or terminal constraints for $X_t$. In the stochastic context such
constraints are only enforceable in a probabilistic sense and they can easily
make the problem much more difficult. However we will not deal with either of
these complications and in the sequel we will assume that $\tf$ is fixed and
that $X_t$ faces no constraints other than that it satisfies its SDE.

Given the objective and the optimization equations,
\cref{eq:generic_objective_functional,eq:generic_optimization_statement}, the
Maximum Principle relies on the forward Kolmogorov (Fokker-Planck)
equation-based definition of the SDE expectation, while Dynamic Programming
relies on the relation between an SDE expectation and the backward Kolmogorov
equation.


\subsection{The Maximum Principle for Transition Probability Densities}
\label{sec:maximum_principle_4_stochastic_control}
The Maximum Principle uses a variational principle to characterize the optimal
control, $\a$ and the optimal {\sl transition } density. Note that the
expectation in the objective, \cref{eq:generic_objective_functional} can be written in terms of the forward
probability density of the state $X_t$:
\begin{align}
J[\a] =&  \Exp_X \left[ \int_0^\tf L(X_s, \a_s) \intd{s} + M(X_\tf) \right]
\notag \\
&=  \int_0^\tf\int_X L(x, \a_s) \cdot f(x,s|x_0,0) \intd{s}\intd{x} 
+ \int_X  M(x)\cdot f(x,\tf|x_0,0)\intd{x}.
\label{eq:generic_objective_functional_in_terms_of_forward_density}
\end{align}
As such, the stochastic problem is reduced to a deterministic optimization
problem but for PDEs, given that $f$ follows the forward PDE in
\cref{eq:generic_FokkerPlanck}. 

Thus, for the purpose of this thesis, the Maximum Principle for Stochastic Optimal
Control is really a Maximum Principle for PDEs. Its variational argument is
similar in spirit to the Euler-Lagrange equations from the Calculus of
Variations; one can think of it as a generalization of the zero-tangent
rule (Fermat's Rule) for finding optima in single-variable calculus. In fact,
like both the Euler-Lagrange equations and Fermat's Rule, the Maximum Principle
provides necessary, but not sufficient conditions for an optimum.

Originally, the Maximum Principle was developed for finite-dimensional
deterministic systems, i.e.\ systems described by ODEs. In that context it is
known as the {\sl Pontryagin} Maximum Principle and for ODEs, the theoretical
results of existence and uniqueness of optimal controls are strongest. 
As is often the case, theoretical results in the infinite-dimensional context,
i.e.\ for PDEs are more difficult to obtain, but the general technique carries
over analogously. Recently, there has been a series of publications on PDE
control of the Fokker-Planck equation, see \cite{Annunziato2010,Annunziato2013,Annunziato2014}.
In particular, \cite{Annunziato2014} discusses the relation between the Dynamic
Programming approach to Stochastic Control and the approach based on PDE
optimization of the Fokker-Planck equation.

The basic idea of the Maximum Principle is that one 'adjoins' the dynamics' PDE
to the objective and introduces a Lagrange multiplier, which in this case is
called {\sl the adjoint state}.

Let us rewrite the governing SDE, \cref{eq:generic_Ito_SDE_Rn}, to explicitly
take into account the control variable $\a(t)$:
\begin{equation}
dX_t = B(X,t; \a) dt + G(X,t) dW.
\label{eq:generic_Ito_SDE_Rn_controlled}
\end{equation} 
Given the problems considered in this thesis, we only
illustrate  a one-dimensional SDE. Its corresponding forward density is
governed by the following Fokker-Planck equation: 
\begin{equation}
\di_t \f= -\di_{x } \left[ B (x,t;\a) \f(x,t) \right] +  
\di^2_{x} \left[ D(x,t) \f(x,t) \right],
\label{eq:generic_FokkerPlanck_controlled}
\end{equation} 
where $D(x,t) = G^2(x,t)/2$. We assume that for all $\a(t)$, both the SDE and the
PDE have unique solutions. 

For notational convenience, we will write
\cref{eq:generic_FokkerPlanck_controlled} as
$$ \dot{\f} = \L_{\a} [\f] $$ where $\L_{\a}$ is the differential operator corresponding to the
right-hand side of the Fokker-Planck equation parametrized by the control $\a$.
For now we will assume that there are no BCs, and the domain of $X$ is all of $\R$.

As we already alluded to, the key concept in Maximum Principle for PDEs is to
adjoin the dynamics, \cref{eq:generic_FokkerPlanck_controlled}, multiplied by
the adjoint state, $p$ to the objective,
\cref{eq:generic_objective_functional_in_terms_of_forward_density}
  
\begin{align*}
J[\a] =& \int_0^\tf\int_X L(x, \a_s) \cdot f(x,s) \intd{s}\intd{x} 
+ \int_X  M(x)\cdot f(x,\tf) \intd{x}
\\ &- \int_0^\tf\int_X p \cdot (\di_t f(x,s)  - \L_{\a} [f(x,s)] )
\intd{s}\intd{x}. 
\end{align*}
Since $f$ satisfies the PDE, we have added a term that equals zero. 
However, what this allows us to do is to 'transfer' the time and space
derivatives from $f$ to $p$. The reason why
that is productive is that we will then be able to form the 'variation'
of $J$ with respect to the control $\a$ and either set it to zero or use this as
a gradient. 

To illustrate the idea, let us perform this 'transfer' explicitly - it is basically an
application of integration-by-parts; in larger dimension this is also commonly
called {\sl Green's identities}.  We have that $$
 \int_0^\tf   p \cdot \di_t f   \intd{t} =
  p\cdot f|_0^\tf - \int_0^\tf   \di_t p \cdot   f  \intd{t}
$$ which 'transfers' the time-derivative to $p$. We also have that $$ \int_X  p \cdot 
\L_{\a} [f ]  \intd{x} = \int_X  ( B  \di_x p + D \di_x^2 p) \cdot  f  \intd{x} =
\int_X   \Lstar_{\a} [p] \cdot f  \intd{x} $$ which transfers' the
space-derivative to $p$. In doing so, we have  naturally introduced 
 the differential operator $\Lstar$, which is the adjoint, in a Banach-space
 sense, to $\L$. Actually, we have already met $\Lstar$ before - it is the
 generator of the SDE in \cref{eq:generic_Ito_SDE_Rn_controlled}.

In the above manipulations, we have assumed that the double integrals have the
same value independent of the order of integration of the space and time
variables  and that $f$ and all its partials go to zero uniformly for $|x|$
large enough. If we had boundary conditions on $f$, those will come up in the
spatial terms.

With the above integration-by-parts done, we can write the objective as
\begin{align}
J[\a] =& \int_0^\tf\int_X L(x_s, \a_s) \cdot f(x,s) \intd{s}\intd{x} 
+ \int_X  M(x)\cdot f(x,\tf) \intd{x} \notag
\\ &- 
\int_X  \left[ p\cdot f|_0^\tf  +
    \int_0^\tf  (\di_t p  + \Lstar_{\a} [p]) \cdot f  \intd{s} \right] \intd{x} 
\label{eq:generic_objective_functional_in_terms_of_forward_density_adjointed}
\end{align}
% (ALEX: check the sign in front of the last integral from 0 to t\_f inside the
% integral of X, as well as in Eq.2.28 below)  - it looks right to me(alex)

Now we assume that we apply a small variation around a given control $\a$: 
\begin{align*}
\a_\e = \a + \e \da
\\
f_\e = f + \e \df
\end{align*}

The variation of the objective, $J$, with respect to the control
variation at the current $\a$ can now be calculated as:
\begin{align}
\frac {dJ}{d\e} \Big|_{\e = 0} &= 
\int_0^\tf\int_X \grad_\a L(x_s, \a_s) \cdot \delta \a \cdot f(x,s) +
L(x_s, \a_s) \cdot \delta f(x,s)
\intd{s}\intd{x} 
\notag \\
&+ 
\int_X  M(x)\cdot \delta f(x,\tf) \intd{x} \notag
\\ &- 
\int_X    p\cdot \delta f(x, \tf)     \intd{x}  \notag \\+& 
    \int_X \int_0^\tf  (\di_t p  + \Lstar_{\a} [p]) \cdot \delta f 
    + \grad_\a B(x,t;\a) \cdot \delta \a \cdot  \di_xp \cdot f 
     \intd{s}
    \intd{x}
\label{eq:J_variation_PDE}
\end{align}
A few notes are required in order to better explain \cref{eq:J_variation_PDE}.
The initial conditions are considered fixed and as such $\delta f |_{t=0} \equiv
0 $, that is the variation in the control does not change $f(x, 0)$. This is
why only the term $p\cdot f|_{t=\tf}$ is retained from \cref{eq:generic_objective_functional_in_terms_of_forward_density_adjointed}.

Heuristically, we would like to infer from \cref{eq:J_variation_PDE} a gradient
with respect to the control, however we note that there are also variations with
respect to $f$ that make it impossible to do so. However, recall that $p$ is
our free variable - the Lagrange multiplier. Thus if we choose $p$
appropriately, we can eliminate $\delta f$ from the expression. Since we
have set up the problem with this in mind, this is now straight forward
to do - we let $p$ evolve (backwards) according to:
\begin{equation}
\begin{cases}
-\di_t p &= \Lstar[p] + L
\\
p(x, \tf) &=  M(x)  
\end{cases}
\label{eq:generic_adjoint_PDE}
\end{equation}

Thus, \cref{eq:J_variation_PDE} simplifies to
\begin{equation}
\frac {dJ}{d\e} \Big|_{\e = 0} =
\int_0^\tf\int_X \left[ \grad_\a L(x_s, \a_s) \cdot f(x,s) +
 \grad_\a B(x,t;\a) \cdot \di_xp \cdot f \right] \cdot \delta \a
     \intd{s}    \intd{x}
\label{eq:J_variation_PDE_simplified}
\end{equation}

% (ALEX: I think it should be \grad_\a in the first term below)
From \cref{eq:J_variation_PDE_simplified}, we can infer that
\begin{equation}
\grad_a J(x,t) = \grad_\a L(x, \a_t) \cdot f(x,t) +
 \grad_\a B(x,t;\a_t) \cdot \di_xp(x,t) \cdot f(x,t)  
\label{eq:J_gradient_wrt_control}
\end{equation}
can be considered as a pointwise gradient of the objective with respect to
changes in the control, given the current control. It is then natural to claim
that setting that equal to zero will give us a necessary condition for an
optimal $\a$.  If setting \cref{eq:J_gradient_wrt_control} to 0 and solving for
$\a$ is possible,  we would then  have an expression of $\a$ in terms of the
state, $f$ and the adjoint $p$, which we could then input in their respective
equations. Of course, this explicit representation of $\a$ would not always be
possible for any running cost function $L$ or drift fields $B$. Even if it were
possible, we  would still have to solve the pair of now-nonlinear
forward/backward equations for $f,p$. It should be clear by now why the Maximum
Principle theory for PDEs faces many practical challenges.

There are two practical approaches to obtain actual numerical results given
\cref{eq:J_gradient_wrt_control}, both of them iterative: 1) gradient
descent and 2) fixed point iteration. 

The fixed point approach is advocated in \cite{Lenhart2007}, for example, but
only for simple pedagogic examples and we will not discuss it here.
Alternatively, we can apply a gradient-based optimization method. Since
\cref{eq:J_gradient_wrt_control} gives us a gradient, we can take the current
control $\a$ and increment it in the direction of $\grad_a J$, if we are
maximizing $J$,  or  $-\grad_a J$, if we are minimizing $J$.

As with all gradient-based optimization, the standard disclaimer about local
minima and optimization initialization applies, since there is no guarantee -  and
 it is usually not the case - that the objective is convex with respect to the
control. 

Our derivations have not been very rigorous. More rigorous arguments can be
found in, e.g.\ \cite{Fattorini1999,Borzi2012}. 
  
As a closing note, we should mention that a Maximum Principle in Stochastic
Control can also be stated directly in terms of the SDE for $X$, in which
case the corresponding 'adjoint' variable satisfies a {\sl backwards} SDE. We
do not explore this approach, but the literature on this topic is also vast, for a
early introductory monograph see \cite{Haussmann1986}.  
 
\subsection{Dynamic Programming for Stochastic Optimal Control}
\label{sec:dynamic_programming}
Dynamic Programming uses backwards recursion to tabulate the optimal control
starting from the terminal time. The basic object in dynamic programming is the
value function, $v$. In order to introduce it, we first extend our definition
for the objective, $J$, to consider starting the state, $X_t$ at later
times, with different initial conditions.

The running cost-to-go corresponding to
\cref{eq:generic_objective_functional} is:
\begin{equation}
J[\a; x, t] = \Exp \left[ \int_t^\tf L(x_s, \a_s) \intd{s} + M(x_\tf) \right] ,
\quad X_t = x
\label{eq:generic_cost_to_go} 
\end{equation}
so that $J[\a; x_0, 0] = J[\a]$ is our original objective from
\cref{eq:generic_objective_functional}.
 
We now introduce the {\sl  value function}, $v$,
defined by: 
$$
v(x,t) = \inf_{\a \in \Udomain} J[\a; x,t];  
$$
it is also called the optimal cost-to-go. 

We immediately note the terminal conditions on $v$:
\begin{equation}
v(x,\tf ) = M(x)
\end{equation}

In other words, if $X_\tf = x$, there is no more time for a control to be applied and no
more running cost to be incurred and we just incur the terminal cost
corresponding to wherever $X$ is now, i.e. $M(x)$. It turns out that the value function, $v$, can be characterized as the solution
to the following non-linear PDE:
\begin{thm}[Hamilton-Jacobi-Bellman (HJB)] 
\label{thm:stochastic_hjb}
\begin{equation}
\begin{cases}
-\di_t v(x,t) &=  \max_{\a(t) \in \Udomain(t)} \big\{ L(x,\a)  +
B(x,t,\a) \cdot \di_x v
\big\} + \frac{G^2(x,t)}{2} \di_x^2v
\\
v(x,\tf) &= M(x)
\end{cases} 
\label{eq:generic_HJB}
\end{equation}
\begin{proof}[Heuristic Derivation adapted from \cite{Evansb}] Suppose we are
at time $t$ and $X_t = x$.Take a time increment $[t, t+h]$ and assume that during that time we apply a
constant control $\a$ and subsequently we apply the optimal control $\a^*$.
The running-cost will then break down as: 
$$
J[\a; x,t] = \Exp \left[ \int_t^{t+h} L(X_s, \a_s) \intd{s}  + v(X_{t+h},
t+h) \right] $$

Since $v(x,t) = \inf J(\a; x,t)$, we must have that:
$$
v(x,t) \leq  \Exp \left[ \int_t^{t+h} L(X_s, \a_s) \intd{s}  + v(X_{t+h},
t+h) \right] $$
or, rearranging, that
\begin{align*}
0 \leq^&  \Exp \left[ \int_t^{t+h} L(X_s, \a_s) \intd{s} \right]  
+ \underbrace{\Exp \left[ v(X_{t+h}, t+h) - v(x,t) \right]}_{\Exp\left[ dv
\right]}
\end{align*}
Now, $\Exp\left[ dv \right]$ can be expressed using Dynkin's Formula, as:
$$
\Exp[dv] = \int_t^{t+h} \di_t v +  B\cdot \di_x v + \frac{G^2 }{2}\cdot \di_x^2
v \intd{s} $$
Plugging that back, we get
$$
0 \leq \int_t^{t+h} L(X,\a) +  \di_tv +  B\cdot \di_x v + \frac{G^2 }{2}\cdot
\di_x^2 v \intd{s} $$
Taking $h \ra 0$ we get 
$$
0 \leq  L(x,\a) +  \di_tv(x,t) +  B\cdot \di_x v(x,t) + \frac{G^2 }{2}\cdot
\di_x^2 v(x,t) $$
and we conjecture that for the actual optimal control, the inequality becomes an
equality:
\begin{equation}
\begin{cases}
0 &=   L(x,\a^*)+ \di_t v(x,t) +
 B(x,t,\a^*)\cdot \di_x v(x,t) + \frac{G^2(x,t)}{2}\cdot \di_x^2 v(x,t)
\\
\a^*  &= \argmax_{\a \in \Udomain(t)}  
\big\{L(x,\a) + B(x,t,\a)\cdot \di_x v \big\}
\end{cases}
\end{equation}
\end{proof}
\end{thm}

It turns out that rigorous proofs of Theorem
\ref{thm:stochastic_hjb} , in particular specifying the exact
mathematical meaning for a solution, $v$, to the HJB PDE are quite complex and beyond the scope
of this text. Some references include \cite{Krylov2008,Fleming2006}. We will
assume that numerically solving \cref{eq:generic_HJB} is sufficient. 

\subsection{Numerical Solutions to PDEs - Finite Difference Methods}
Both the variational approach of the Maximum Principle and the backwards
induction of Dynamic Programming result in having to solve PDEs in order to
obtain the optimal control. For practical purposes, solving these PDEs requires
numerical discretization. In all cases the PDEs are {\sl
parabolic} (see \cite{Press1992}), which in general can be written as 
\begin{equation}
\di_t f(x,t) = \L[f(x,t)]
\label{eq:generic_parabolic_PDE}
\end{equation}
for some differential operator, $\L$. 
In one spatial dimension, a finite difference discretization of
\cref{eq:generic_parabolic_PDE} is to select a set of space- and time- nodes
$\{x_i\}, \{t_k\}$ and approximate $f$ by solving for $f(x_i, t_k)$, given the
initial conditions $f(x_i, t_0)$ and possibly boundary conditions.

A classic technique for parabolic PDEs is the
{\sl Crank-Nicholson} scheme, which time-discretizes
\cref{eq:generic_parabolic_PDE} as
$$
\frac{f(x_i, t_{k+1})- f(x_i, t_k)}{ t_{k+1}-t_k}
=
\frac 12 \left(\L[f(x_i, t_{k+1}  )] + \L[f(x_i, t_k)]
\right)
$$
and then solves for the resulting linear system before stepping
iteratively forward in time. See chapter 19 in \cite{Press1992} for a brief
discussion of the theoretical properties of the Crank-Nicholson method.   

This solution approach requires a finite spatial domain. If the spatial domain
is theoretically infinite, as it may be in some cases, we need to truncate it and
apply some reasonable boundary conditions at the artificial boundary which
approximate the solution of the theoretically infinite space. We will show
details of how this is done in the context of each specific problem
discussed in the thesis.  
 
\cleardoublepage
% \chapter{Mathematical Models in Neuroscience}
\label{ch:neuro_background}

We now describe in detail the basic mathematical model of a neuron that will be
used often in the sequel. 

An introduction to mathematical models in neuroscience is given in Gerstner and
Kistler, \cite{Gerstner2002}, also available online, while the book Stochastic
Methods in Neuroscience, \cite{Laing2009} gives a nice overview of several
current research applications of stochastic techniques to neuroscience.


Neurons relay information by means of voltage spikes - sudden sharp increases in
voltage. Although many details remain unclear, the information content is
thought to be contained in the length of the time-interval between these spikes.
In the simplest case, this can be thought of as a rate - the average number of
spikes per time interval, but more complicated coding schemes are hypothesized to
exist. 

Many experiments allow for manipulating an individual neural cell. A natural
goal then is to make a cell produce a given spike train. This may arise, for
example, in brain-machine interfaces or in artificial prosthetics. 

\section{Problem Formulation}
The most basic representative model for a neuron is the noisy
leaky-integrate-and-fire model:
\begin{equation}
\begin{gathered}
dX_s = \left(\a(t) - \frac{(X_s - \m}{\tc} \right) \intd{s} + \b \intd{W_s},
\\
X(0) = 0,
\\
X(\ts) = \xth \implies  
\begin{cases}
X(\ts^+) = 0 &  
\end{cases}
\end{gathered}
\label{eq:X_evolution_uo}
\end{equation}


 
% \cleardoublepage
\chapter{Estimation in the Sinusoidaly-driven Leaky-Integrate and Fire Model}
% \chapter[Estimation in the Sinusoidaly-driven LIF Model]{Estimation in the
% Sinusoidaly-driven Leaky-Integrate and Fire Model}
% \chaptermark{Estimation in the Sinusoidaly-driven LIF Model}
\label{ch:estimate}
\graphicspath{{../LIFEPaper/}}
<<<<<<< HEAD
=======

\usepackage{amsfonts}
\usepackage{mathrsfs}

>>>>>>> 6d1ee3c9eb52b6bed66343a6488d0f9a4ca3aef0
% DEFINITIONS:
\def \Prob 	  {{ \mathbbmtt{P}  }} %the expectation operator
\def \th 	  {{ \theta}}
\def \FI 		{{\Phi}}
\def \KL     {{ K\!L}}

\def \adot {{ \dot{\alpha} }}
\def \Udot {{ \dot{U}}}
\def \In	{{ i_n}}
\def \vt 	{{ v_{\textrm{th} } }}
\def \Ihat  {{ \hat{I}  }}
\def \p 	{{ \phi  }}

\def \G		{{ \bar{G}  }} %{{ \reflectbox{G} }}
\def \Gest  {{ \hat{\G} }}
\def \Gtilde		{{ \tilde{\G} }}
\def \D		{{ \reflectbox{D} }}
\def \dphi {{ \delta \phi }}

\def \L {{ \Lambda }}
\def \P {{ \Phi }}
\def \n {{\nu}}
\def \Fx {{ F_x}}
\def \Fxx {{ F_{xx} }}
\def \Ft {{ F_t }}
\def \xest {{ \hat{x}_t}}
\def \muncond {{ {m}_x}}
\def \mcond {{ {m}_x^c}}

\def \f {{\rho}}
\def \F {{\Phi}}
\def \Fn {{\mathcal{F}}}

\def \abg		{{\a,\b,\g }} 
\def \aest      {{ \hat{\a} }}
\def \best      {{ \hat{\b} }}
\def \gest      {{ \hat{\g} }}
\def \estabg	{{\aest, \best, \gest}}
\def \abgest	{{\estabg }}

\def \sAlg {{ \mathcal{A} }}
\def \N {{ \mathcal{N} }}
\def \Udomain {{ \mathcal{U} }}
\def \Umax {{ u_{\textrm{max}} }}
\def \umax	{{ u_{\textrm{max}} }}
\def \amin	{{ \a_{\textrm{min}} }}
\def \amax	{{ \a_{\textrm{max}} }}
\def \astar {{ \a^* }}
\def \xth	{{ x_{th} }} 
\def \yth	{{ y_{th} }}
\def \xmin	{{ x_{-} }}
\def \xmax	{{ x_{+} }}
\def \xmid  {{ x_{mid} }}

% \def \x {{ \boldsymbol{x} }}
% \def \u {{ \boldsymbol{u} }}
% \def \p {{ \boldsymbol{p} }}
% \def \q {{ \boldsymbol{Q} }}
% \def \f {{ \boldsymbol{f} }}
\def \tf {{ t_f }}
\def \tc {{ \tau_{c} }}
\def \lc {{ \lambda_{c} }}
\def \ts {{ t_{\textrm{sp} } }}
\def \tn {{ t_n }}
\def \tns {{ \{t_n \} }}
\def \T {{ T^*  }}

\def \free {{\textrm{free} }}


\def \Ttwo {{ \hat{T}_{(2)} }}
\def \Ttwol {{ \hat{T}^{\lambda}_{(2)} }}
\def \Tone {{ \hat{T}_{(1)} }}
\def \Ti {{ \hat{T}_{(i)} }}

\def \Normal {{ \mathcal{N} }}
\def \L {{ \mathcal{L} }}
\def \Lstar {{ \L^* }}
\def \H {{ \mathcal{H} }}
\def \dx {{ \delta\! x}}
\def \da {{ \delta\! \a}}
\def \df {{ \delta\! f}}

% COMP EXAM:
\def \Lonepm {{\mathbb{L}^1}}
\def \Ltwopm {{\mathbb{L}^2}}
\def \Fil 	{{\mathcal{F}}}

\def \x {{ \vec{x} }}
\def \X {{ \vec{X} }}
<<<<<<< HEAD
% parameterized densities / adjoints:
\def \ft {{ f_\th}}
\def \pt {{ p_\th}}
\def \dft {{ \delta f_\th}}
\def \wt {{ w_\th }}

\def \aopt {{\a_{opt} }}
=======

>>>>>>> 6d1ee3c9eb52b6bed66343a6488d0f9a4ca3aef0

   
\section{Thesis Context}
 In the first of three main chapters of the thesis, we focus on the problem of
 estimating model parameters in a sinusoidally-perturbed LIF model from the
 observation of spike timings only. We discuss two approaches - one based on
 solving an integral equation and one based on solving for the forward density.
 In both cases we need to make an approximation in order to account for the
 non-renewal property of the hitting-time process. 
Both estimation algorithms are iterative, therefore we propose a non-trivial
 method for initializing them with observations-driven initial guesses.  
 
\section{Problem Introduction} 
Information processing in the nervous system is carried out by spike timings in
neurons. To study the neural code in such a complicated system, a first step is
to understand signal processing and transmission in single neurons. Stochastic
leaky integrate-and-fire (LIF) neuronal models are a good compromise between
biophysical realism and mathematical tractability, and are commonly applied as
theoretical tools to study properties of real neuronal systems. A central issue
is then to perform statistical inference from experimental data and estimate
model parameters. Many electrophysiological experiments on neurons, namely
extra-cellular recordings, are only capable of detecting the time of the spike
and not the detailed voltage trajectory leading up to the spike. Estimating the
parameters of the LIF model from this type of data is equivalent to estimating
the parameters of a stochastic model from the statistics of the first-passage
times only. A common assumption is that the data are well described by a renewal
process, thus basing the statistical inference on the interspike intervals
(ISIs), assuming these are realizations of independent and identically
distributed random variables. Since only partial information about the process
is available, the statistical problem becomes more difficult, and no explicit
expression for the likelihood is available. 

Different methods have been
proposed. In the seminal paper \cite{Brillinger1988}, a point process approach
is proposed. The spike trains of a collection of neurons are 
represented as counting processes. Time is discretized and the point processes
approximated by 0-1 time series. Then the probability of firing in the
next time interval is modeled as a function of the
spike history. In this way maximum likelihood
estimation is feasible. External stimuli are not considered. 
In \cite{Inoue1995} a numerically involved moment method is
developed. It uses the first two moments of the first-passage times of
the Ornstein-Uhlenbeck process to a constant threshold, which are
given as series expressions, and equates them to their empirical
counterparts. In
\cite{DitlevsenLansky2005,DitlevsenLansky2006} certain explicit 
moment relations derived from the Laplace transform of the first-passage time
distribution are applied, but these are only valid under 
stimulation (supra-threshold regime). In
\cite{MullowneyIyengar2008} inference is based on numerical inversion of the Laplace transform. In \cite{Zhangetal2009}, a
functional of a 3-dimensional Bessel bridge is applied to obtain a maximum
likelihood estimator. None of these methods
are feasible to extend to the
non-timehomogenous case, which is of our interest. In \cite{Ditlevsen2008,Ditlevsen2007} an
integral equation is used to derive an estimator in the
time-homogenous setting. This approach is readily extended to time
varying input, which we will explore in this paper. Some of the above methods are compared in
\cite{Ditlevsen2008a}. Finally, a review of estimation methods is provided in
\cite{Lansky2008}. 
%We provide more details on previous methodologies in
%\cref{sec:estimation_algos} below.

Many sensory stimuli, like sound, contain an oscillatory component
\cite{Braunetal1994,Chacron2000}. Such inputs will cause oscillating membrane
potentials in the neuron, generating rhythmic spiking patterns. The oscillation
frequency determines the basic rhythm of spiking, and is considered to be
significant for neuronal information processing. The dynamics of periodically
forced neuron models have been extensively studied, see
\cite{Bulsaraetal1996,Burkitt2006b,Lansky1997,Longtingetal1994,SacerdoteGiraudo2013,Shimokawa2000}
and references therein. Even so, attempts to solve the estimation problem in
these non-stationary settings have been rare. One problem is that the ISIs are
no longer independent nor identically distributed. In
\cite{Paninski2004} a more complicated model with linear filters is considered, allowing
also for the spike history to influence the membrane potential
dynamics. The estimation problem is solved through numerical solutions to the
Fokker-Planck equation, and it is shown that the log-likelihood is
concave, thus ensuring a global maximum, see also
\cite{Dong2011,Sirovich2011a}. Because their model is more involved,
some approximations to the solution of the Fokker-Planck equation is
applied, to ensure acceptable computing times. We will apply the full 
Fokker-Planck equation to solve our estimation problem, since the
computing time is always lower than 2 seconds for a sample size of
1000 spikes.   
  

In this paper, we thus describe and discuss two methods to estimate parameters
of LIF models with the added complexity of a time-varying input current. We
assume that the time-varying current is a sinusoidal wave, but we believe that
the approaches generalize to an arbitrary periodic forcing with known frequency.
One approach relies on the Fortet integral equation, which is readily extended
to the time non-homogeneous case. An advantage of this approach is that if the
transition density of the diffusion in the LIF model is known, as is the case
for the Ornstein-Uhlenbeck and the Feller model, the computational burden is
limited. A second approach involves numerical solution of the Fokker-Planck
equation, where the time-dependence is explicitly accounted for. After a
numerical differentiation, the likelihood function can be calculated providing
the maximum likelihood estimator. Nevertheless, we chose an alternative loss
function which seem marginally more robust, directly comparing the survival
function provided by the solution of the Fokker-Planck equation with its
empirical counterpart. The two approaches give similar results and they are more
carefully compared in the supplementary online material.

Both methods need sensible starting values for the optimization
algorithms, and we provide an easy-to-implement initializer. The estimation
procedures are compared on simulated data and we find that both algorithms are
able to find estimates close to the true values for several different dynamical
regimes. We find that for small sample sizes the Fokker-Planck algorithm can be
considered marginally preferable, whereas for larger sample sizes the Fortet
algorithm becomes marginally superior. Moreover, at high frequencies of the
sinusoidal forcing, the Fortet is better at identifying the parameters, though in general
there is less information in the data to distinguish between a constant input
and the amplitude of the periodic forcing.

\section{Model}
The time evolution of the voltage of a spiking neuron is modelled by a
stochastic process, $V$, given as solution to the following stochastic
differential equation (SDE)
\begin{equation}
\begin{gathered}
\intd{V}(t) = \left(\m - \frac{V(t)}{\t} +  A \sin(\o t ) \right) \intd{t} + \s
\intd{W}(t),
\\
t_0 = 0; \quad V(t_0) = v_0,
\\
t_{n } = \inf \{ t > t_{n-1} : V(t) = \vt \} \quad \text{for } n \geq 1
\\
\begin{cases}
V(t_n^+) = v_0 &  
\\
J_n = t_n-t_{n-1} .
% \\
% \phi_{n+1} = t_n \mod 2 \pi
\end{cases}
\end{gathered}
\label{eq:v_evolution_uo}
\end{equation}
Here, $\mu$ is a bias current acting on the cell, $\t$ is the decay time, $A$
and $\o$ are the amplitude and (angular) frequency of the sinusoidal current
acting on the cell, $\s$ is the strength of the stochastic fluctuations, $W =
\{W_t\}_{t\geq0}$ is a standard Wiener process, and $t_n^+$ denotes the right
limit taken at $t_n$. A spike occurs when the membrane voltage $V(t)$ crosses a
voltage threshold, $\vt$, and then $V(t)$ is instantaneously reset to the
resting potential $v_0$. The difference between subsequent spike times, $J_n =
t_{n} - t_{n-1}$, is called the interspike interval (ISI).

We will assume that $\t$ is known (but see \cref{sec:optimal_design} for a discussion
of the alternative) and non-dimensionalize \cref{eq:v_evolution_uo} as follows
\begin{align*}
s &= \frac{t}{\t}	& 
X_s &= \frac{V(t) - v_0}{\vt - v_0}&
W_s &= \frac{W(t)}{\sqrt \t}	&
\xth&= 1
\\
\a &= \frac{\m \t}{\vt - v_0}	&
\b &= \frac{\s\sqrt{\t}}{\vt-v_0}	&
\g &= \frac{A \t}{ \vt - v_0 }	&
\th &= \o \t \notag
% \label{defn:nond_params}
\end{align*}
to obtain
\begin{equation}
\begin{gathered}
dX_s = (\a - X_s + \g \sin( \th s ) ) \intd{s} + \b \intd{W_s},
\\
s_0 = 0; \quad X_{s_0} = 0,
\\
s_{n } = \inf \{ s > s_{n-1} : X_s = \xth =1 \} \quad \text{for } n \geq 1 
\\
\begin{cases}
X_{s_n^+} = 0 &  
\\
I_n = s_n-s_{n-1} ,
% \\
% \phi_{n+1} = t_n \mod 2 \pi
\end{cases}
\end{gathered}
\label{eq:X_evolution_uo}
\end{equation}
where we have defined $I_n = J_n / \t$. We can also write the dynamics between two spike times $s_n$ and $s_{n+1}$ in terms of elapsed time since
the last spike, $s' = s- s_n$, $s' < I_{n+1}$,
\begin{equation}
\begin{gathered}
dX_{s'} = (\a - X_{s'} + \g \sin( \th (s' + \phi_n ) ) \intd{s'} + \b
\intd{W_{s'}},
\\
\begin{cases}
s' &= s - s_n
\\
 \phi_n &= s_n \mod \tfrac{2 \pi}{\th}
\end{cases}
\end{gathered}
\label{eq:X_evolution_uo_renewal}
\end{equation}
This form of the dynamics highlights that this is not a renewal process since
different trajectories between spikes have different phase shifts $\phi_n = s_n$
modulo ${2 \pi}/{\th}$. This will be important in the following discussion. The
shape of the ISI distribution depends on the model parameters, and it is natural
to divide the parameter space in different regimes characterized by their
qualitative behaviour. Four distinct parameter regimes will be considered;
supra-threshold, critical, sub-threshold and super-sinusoidal. 
To understand the reasoning behind the regime names, observe that in the absence
of noise, $\b=0$, the deterministic model will produce spikes if and only if $$
\a + \frac{\g}{\sqrt{1 + \th^2} } > 1, $$ see the discussion in
\cite{Burkitt2006b}, which can be directly inferred from the solution in
\cref{eq:LIF_deterministic_solution} below. In both the supra-threshold and
super-sinusoidal regimes, $ \a + \g/\sqrt{1 + \th^2}  > 1$. The
difference between the two is that in the supra-threshold regime the constant
bias current alone is sufficient for spikes to occur, also in absence of noise,
that is,  $\a > 1$. In the super-sinusoidal regime the sinusoidal current is
necessary for spikes to occur in absence of noise, that is, $\alpha +
\gamma /
\sqrt{1+\Omega^2} > 1$ and $\alpha \leq 1$. In the critical regime, the sum of
the two terms is just barely enough to guarantee deterministic spiking, that is
$\a + \g / \sqrt{1 + \th^2}  \approx 1$. Finally, in the sub-threshold regime,
there would be no spikes without the noise, $ \a + \g / \sqrt{1 + \th^2} < 1$.

\Cref{tab:regimes} tabulates examples of corresponding parameter values for each
regime, while \cref{fig:trajectory_examples} shows examples of individual
voltage trajectories and their associated spike trains.
\Cref{fig:4regimes_illustrated_SDF,fig:4regimes_illustrated_PDF} illustrate
 how each regime behaves for selected $\phi$'s by plotting the survivor
 distribution, $\G_{\phi}(t)$, and the probability density, $g_{\phi}(t)$, both
 defined in \cref{eq:ISI_distribution_functions} below.
 
With regards to
\cref{fig:4regimes_illustrated_SDF,fig:4regimes_illustrated_PDF}, it is worth
noting explicitly, that combinations of noise and sinusoidal forcing can cause
firing patterns in which spikes are phase locked, but skip a certain number of
cycles. This leads to multi-modal ISI densities. There are many different
dynamical mechanisms that can yield such patterns, and the particular
correlations between the ISIs will depend on the underlying voltage dynamics
(which, in our case, we assume to be given by \cref{eq:v_evolution_uo}); in
particular, it may be difficult to distinguish whether the dynamics are
sub-threshold or supra-threshold, since both can show similar ISI densities,
see \cite{Longtin1995}.


 
% \begin{table}[ht] \begin{center} \begin{tabular}{l|ccc}
% Regime Name & $\a$ & $\b$ & $\g$ \\
% \hline Supra threshold &
% 1.5 & 0.3 & 1.0 \\ 
% Critical &
% 0.5 & 0.3 & 1.12 \\ 
% Sub threshold &
% 0.4 & 0.3 & 0.4 \\ 
% Super-sinusoidal & 0.1 & 0.3 & 2.5 \end{tabular} \caption{Example $\abg$
% parameters for the different regimes, given $\th = 2.0$} \label{tab:regimes}
% \end{center} \end{table}
\begin{table}[ht]
\begin{center}
\begin{tabular}{l|ccc}
Regime Name & $\a$ & $\b$ & $\g$ \\ \hline
Supra-threshold&1.40&0.30&0.14 \\
Super-sinusoidal&0.10&0.30&1.98 \\
Critical&0.50&0.30&0.71 \\
Sub-threshold&0.40&0.30&0.57 \\
\end{tabular}
\caption[Parameter values for model regimes]{Example of $\abg$ parameter values
for the different regimes, given $\th = 1$.}
\label{tab:regimes}
\end{center} 
\end{table}
% \begin{figure}[ht] \begin{center} \subfloat {
% \includegraphics[width=0.49\textwidth] {Figs/FP/Illustrate_refinedsuperT.png}
% } \subfloat { \includegraphics[width=0.49\textwidth]
% {Figs/FP/Illustrate_refinedcrit.png} }
% \\
% \subfloat{ \includegraphics[width=0.49\textwidth]
% {Figs/FP/Illustrate_refinedsubT.png} } \subfloat{
% \includegraphics[width=0.49\textwidth]
% {Figs/FP/Illustrate_refinedsuperSin.png} } \caption{the Four Regimes -
% illustrated are the empirical and analytical SDF, $\G(t)$, for the 4 most
% populated bins. Recall that the analytical $\G(t)$ is derived using the
% representative $\phi_m$ for each bin, while the empirical $\G$ is obtained
% directly from the spike data, each spike having an associated $\phi_n \neq
% \phi_m$.} \label{fig:4regimes_illustrated} \end{center} \end{figure} \SDF:
\begin{figure}[ht]
\begin{center}
% \subfloat[Supra-threshold Regime]
% {
% \includegraphics[width =0.48\textwidth,height=.25\textheight]
% {Figs/Trajectories/path_T=24_TrajExample_superT.png}
% }
% \subfloat[Critical Regime]
% {
% \includegraphics[width =0.48\textwidth,height=.25\textheight]
% {Figs/Trajectories/path_T=24_TrajExample_crit.png}
% }
% \\
% \subfloat[Sub-threshold Regime]
% {
% \includegraphics[width =0.48\textwidth,height=.25\textheight]
% {Figs/Trajectories/path_T=24_TrajExample_subT.png}
% }
% \subfloat[Super-sinusoidal Regime]
% {
% \includegraphics[width=0.48\textwidth,height=.25\textheight]
% {Figs/Trajectories/path_T=24_TrajExample_superSin.png}
% }
\includegraphics[width=0.95\textwidth]{Figs/Trajectories/path_T=24_combined.pdf}
% \includegraphics[width=0.95\textwidth]{Figs/Trajectories/path_T=24_combined.png}1
\end{center}
\caption[Example model trajectories]{Example trajectories from
\cref{eq:X_evolution_uo} for the four different parameter regimes using the parameter values given in
\cref{tab:regimes}. A) supra-threshold, B) super-sinusoidal, C) critical, D)
sub-threshold. In the supra-threshold regime
spikes occur regularly and often; in the super-sinusoidal regime
spikes cluster near the peak of the sine wave; in the critical regime
they occur less often; and in the sub-threshold regime, spikes occur
rarely. For all regimes, $\th = 1$.} 
\label{fig:trajectory_examples}    
\end{figure}
% #PDF: 
\begin{figure}[ht]
\begin{center}
% \subfloat
% {
% \includegraphics[width=0.48\textwidth]
% {Figs/FP/Illustrate4superT_pdf.png}
% }
% \subfloat
% {
% \includegraphics[width=0.48\textwidth]
% {Figs/FP/Illustrate4crit_pdf.png}
% }
% \\
% \subfloat{
% \includegraphics[width=0.48\textwidth]
% {Figs/FP/Illustrate4subT_pdf.png}
% }
% \subfloat{
% \includegraphics[width=0.48\textwidth]
% {Figs/FP/Illustrate4superSin_pdf.png}
% }
\includegraphics[width=0.99\textwidth]{Figs/FP/regimes_pdf_combined.pdf}
\caption[Model Regimes hitting-time densities]{The four different parameter
regimes using the parameter values given in \cref{tab:regimes}. Illustrated are the
probability density functions, $g_{\phi_m}(t)$, for representative $\phi_m =
2\pi/\th \times \{0,0.25,0.5,0.75\}$. 
Varying $\phi_m$ has,
 for the most part, the effect of shifting the curves laterally, 
 while varying $\abg$ changes their characteristic form. For all regimes, $\th
 = 1$.
 A) supra-threshold, B) super-sinusoidal, C) critical, D) sub-threshold}
\label{fig:4regimes_illustrated_PDF}  
\end{center}      
\end{figure}             
\begin{figure}[ht]    
\begin{center} 
% \subfloat  
% {
% \includegraphics[width=0.48\textwidth]
% {Figs/FP/Illustrate4superT.png}
% } 
% \subfloat
% {
% \includegraphics[width=0.48\textwidth]
% {Figs/FP/Illustrate4crit.png}
% }
% \\
% \subfloat{
% \includegraphics[width=0.48\textwidth]
% {Figs/FP/Illustrate4subT.png}
% }
% \subfloat{
% \includegraphics[width=0.48\textwidth]
% {Figs/FP/Illustrate4superSin.png}
% }
\includegraphics[width=0.99\textwidth]{Figs/FP/regimes_sdf_combined.pdf}
\caption[Model Regimes hitting-time survivor distributions]{The four different
parameter regimes using the parameter values given in \cref{tab:regimes}. Illustrated are the survivor
distribution functions, $\G_{\phi_m}(t)$, for representative $\phi_m = 2\pi/\th
\times \{0,0.25,0.5,0.75\}$. Varying $\phi_m$ has, for the most part, the effect
of shifting the curves laterally, while varying $\abg$ changes their
characteristic form.
A) supra-threshold, B) super-sinusoidal, C) critical, D) sub-threshold.}
\label{fig:4regimes_illustrated_SDF}    
\end{center}
\end{figure}     

\clearpage

\subsection{Basic ISI probability density functions}
Here we introduce the notation for the  probability density, distribution and
survival functions of $I_n$, an ISI arising from a trajectory
produced by \cref{eq:X_evolution_uo_renewal},
\begin{equation} 
\begin{array}{rcll}
g_{\phi}(\t) \intd{\t} &:=& \Prob(I_{n+1} \in [\t, \t + \intd{\t})  | \phi_n =
\phi) &
 \textrm{(probability density)} 
\\ 
G_{\phi}(t) &:=& \Prob(I_{n+1} \leq t  |\phi_n = \phi ) = \int_0^t g_{\phi}(\t)
\intd{\t} &
 \textrm{(cumulative distribution)}
\\
\G_{\phi}(t) &:= & \Prob(I_{n+1}>t | \phi_n = \phi ) = 1 - G_{\phi}(t)
&
 \textrm{(survivor distribution)}
\end{array}
\label{eq:ISI_distribution_functions}
\end{equation}
The subscript $\phi$ is to stress that $g, G$ and $\G$ depend on the value of
$\phi_n$ in \cref{eq:X_evolution_uo_renewal}. This is the formal statement that
in a sinusoidally-driven neuron, the interspike intervals are not identically
distributed, and are only independent conditioned on the sinusoidal phase at an
interval's onset. Knowing these distributions would provide the likelihood function,
offering estimation by the preferred method of choice, the maximum likelihood
estimator. Unfortunately, explicit expressions for the ISI distribution are not
available except for the special case of $\g = 0$ and $\a=1$ , see
\cite{DitlevsenLansky2005}. Different representations of the likelihood function
are available though, see \cite{Alili2005}, one of which we will use below.

\subsection{Fokker-Planck Equation with Absorbing Boundaries}
\label{sec:fp_estimation}
The Fokker-Planck equation is a partial differential equation (PDE) describing
the evolution of the probability density, $f(x,t)$, of $X_t$. 
For the sinusoidally-forced Ornstein-Uhlenbeck process,
\cref{eq:X_evolution_uo_renewal}, with the threshold $x_{th} = 1$, the PDE is
\begin{equation}
\di_t f^{(\phi)}(x,t) = -\di_x[(\a - x + \g \sin(\th (t + \phi))\cdot
f^{(\phi)}] +\di^2_x[ \tfrac{\b^2}{2}f^{(\phi)}], \quad x \in (-\infty,
1).
\label{eq:FP_pde_OU_absorbBC}
\end{equation}
Due to the reset, we have that at time $t=0$, $X_t=0$  and so for the initial
conditions we can write
\begin{equation}
f^{(\phi)}(x,t=0) = \delta(x) ,
\label{eq:PDF_ICs}
\end{equation}
where $\delta(\cdot)$ is the Dirac delta function. The spike is represented as
a zero boundary condition for $f$ at $x = 1$ $$
f(1, t) =0.
$$

The natural way of using the Fokker-Planck equation in first-hitting-times
problems is as follows. Denote the integral of $f^{(\phi)}$ by $F^{(\phi)}(x,t)
= \int_{\xi \leq x} f^{(\phi)}(\xi, t) \intd{\xi}$. $F^{(\phi)}(x,t)$ can be
related to the ISI's survivor distribution function, $\G_{\phi}(t)$, by
\begin{equation}
\G_{\phi}(t) = F^{(\phi)}(1,t).
\label{eq:SDF_vs_F_at_thresh}
\end{equation}
\Cref{eq:SDF_vs_F_at_thresh} forms the
basis of one of the methods below for estimating the structural parameters from
the observed data.

Since \cref{eq:FP_pde_OU_absorbBC} has to be solved numerically, we will need to
truncate its domain from below. The most natural way to do this, given the
dynamics, is to impose reflecting boundary conditions at some $x=\xmin \ll
(\a-\g/\sqrt{1+\th^2})$ where the probability mass is very small. For the left
(lower) limit of the computational domain, we use the formula $$ \xmin =
\min(\underbrace{\a -\g/\sqrt{1+\th^2}}_{\textrm{mean}} - 2 \underbrace{\b/
\sqrt{2}}_{\textrm{std. dev}}, -0.25).$$ This choice requires some explanation.
In the $t\ra \infty$ limit, the distribution of $X_t$ in
\cref{eq:X_evolution_uo_renewal} {\sl without} thresholding is Gaussian with
mean given by \cref{eq:LIF_deterministic_solution} (below) and variance equal to
$\beta^2/2$. Thus to truncate the computational
domain for the thresholded process from below, we take the lowest value of
the asymptotic mean, $\alpha - \gamma / \sqrt{1 + \th^2}$, then from
this we subtract two standard deviations, $2\b/\sqrt{2}$ and set the result to be the lower
bound, $\xmin$. Finally, if this value for $\xmin$ happens to be larger than
$-.25$, we enforce that  $\xmin \leq -0.25$. 

% The reflecting BCs look like:
% \begin{equation}
% \big[ -(\a - x + \g \sin(\th (t + \phi) ) \cdot f + 
% 	\frac{\b^2}{2} \cdot \di_x f \Big] \Big|_{x=c} = 0.
% \label{eq:reflecting_BCs}
% \end{equation}

Numerical considerations lead us to solve for $F$, instead of $f$, since delta
functions are difficult to represent in floating point, while the initial
conditions for $F$, the Heaviside step function, $H(x)$, faces no such
difficulties \cite{Hurn2005}. The Heaviside step function is defined to be
equal to $0$ for $x<0$ and to be equal to $1$ for $x \geq 0$. At this point we
need to derive the PDE for the distribution $F$, starting from the PDE for the
density, $f$, \cref{eq:FP_pde_OU_absorbBC}.

First, at the lower boundary, it is intuitive that the distribution should be
zero, $ F(\xmin,t) = 0 $, while $f(1,t) = 0$ implies that at the upper boundary
$ \di_xF(1, t) = 0 $. Inside the domain, the PDE itself reformulates as
\begin{eqnarray*}
\di_t f(x,t) &=&  \di_x \left[\frac{1}{2}\di_x [\b^2 f] -  (\a - x + \g \sin(\th
(t - \phi)) f \right]
\end{eqnarray*}
so that
\begin{eqnarray*}
\di_x \di_t F(x,t) &=& \di_x \left[
\frac{\b^2 }{2}\cdot \di^2_x F -  
						(\a- x + \g \sin(\th (t + \phi))  \cdot \di_xF \right].
\end{eqnarray*}
Integrating with respect to $x$ then gives
$$
\di_t F(x,t) =
\frac{\b^2 }{2}\cdot \di^2_x F -  
						(\a- x + \g \sin(\th (t + \phi))  \cdot \di_xF + C(t)
$$
where $C(t)$ is a constant of integration depending on $t$. Now consider
the lower boundary condition, $x =
\xmin$. Here $F(\xmin,t) = 0$ implies that $\di_tF = 0$ and so 
\begin{equation}
C(t) = - \left[ \frac{\b^2 }{2}\cdot \di^2_x F -  
			(\a- x + \g \sin(\th (t + \phi))  \cdot \di_xF \right].	
\label{eq:const_of_integration}
\end{equation}
The right-hand side in eq.\ \eqref{eq:const_of_integration} is precisely the
reflecting boundary condition on $f$ once we recall that $\di_x F = f$. Therefore $C(t) \equiv 0$.

Thus, the fully specified PDE for $F$, which we will be solving frequently in what
follows, is
\begin{equation}
\begin{gathered}
	\di_t F^{(\phi)}(x,t) =
					\frac{\b^2 }{2}\cdot \di^2_x F^{(\phi)} -  
					\Big(\a- x + \g \sin(\th (t + \phi))\big)  \cdot \di_x F^{(\phi)},
	\\
	\\
	\left\{ \begin{array}{lcl}
	 F^{(\phi)}(x,0) &=& H(x)
	\\
	F^{(\phi)}(x,t) |_{x=\xmin} &\equiv& 0 
	\\
	\di_x F^{(\phi)}(x,t) |_{x=\vt} &\equiv& 0.
	\end{array} \right.
\label{eq:FP_pde_OU_absorbBC_CDF}
\end{gathered}
\end{equation}
Numerical solutions for \cref{eq:FP_pde_OU_absorbBC_CDF} are shown in
\cref{fig:FP_pde_OU_absorbBC_CDF}. We have used the standard Crank--Nicholson
finite-difference algorithm (central-differences in space with equally weighted
implicit-explicit terms in time, see \cite{Karniadakis2003}). 

\begin{figure}[h]
\begin{center}
% \subfloat[]{
% \includegraphics[width=0.45\textwidth]
% {Figs/FP/surf_phi_105_az=-50.png}
% } 
% \subfloat[] { 
% \includegraphics[width=0.45\textwidth]
% {Figs/FP/surf_phi_105_az=-10.png}
% }
% \\
% \subfloat[] {
% \includegraphics[width=0.5\textwidth]
% {Figs/FP/SDF_phi_105.png}
% }
\includegraphics[width=0.95\textwidth]{Figs/FP/SDF3D_combined.pdf} 
\caption[Transition distribution example solution]{Example solution to
\cref{eq:FP_pde_OU_absorbBC_CDF} for $(\a,\b,\g) = (0.5, 0.3, 0.5\sqrt{2})$; $\th= 1, \phi = \pi / 2$.
In A,B,C, we show the full solution in space-time $F(x,t)$. In (d) we show
the time solution at the upper boundary, $F(1,t)$.} 
\label{fig:FP_pde_OU_absorbBC_CDF} 
\end{center}
\end{figure}

\subsection{Fortet Equation}
\label{sec:fortet_estimation}
Consider a general form of \cref{eq:X_evolution_uo_renewal}, $$ dY_t =
b(t,Y_t)dt + \sigma(t,Y_t) dW_t. $$ Let $\F(y,t| y_0, t_0) :=   \Prob[Y_t \leq
y| Y_{t_0} = y_0]$ be the transition cumulative distribution of $Y$. Note that
this is the distribution of $Y_t$ in absence of a threshold, different from the
distribution given in \cref{eq:SDF_vs_F_at_thresh},
 which is the distribution 
of the process constrained to be below the threshold. 
Now consider an arbitrary time-dependent threshold $\vt(t)$. The Fortet
equation, see \cite{Fortet1943}, convolves the first-hitting time probabilities,
$g(t)$, with the transition density of the process. Integrating
over $(-\infty, \vt(t))$, we obtain
\begin{equation}
1 - \F(\vt(t), t|v_0, 0) =
\int_0^t g(\tau) [1-\F (\vt(t),  t| \vt(\t), \t)] \intd{\t}.
\label{eq:Fortet_moving_vth}
\end{equation}
The left hand side is simply the probability of exceeding $\vt$ at time
$t$ starting at $v_0$ at time $0$. This can also be written as the probability
of hitting $\vt$ for the first time at time $\t < t$ and then exceeding
$\vt$ at time $t$ starting at $\vt$ at time $\t$, integrated over all $\t$.
% Strictly speaking, \cref{eq:Fortet_moving_vth} is not the Fortet equation, but
% the integral of the Fortet equation, while the original Fortet equation uses
% $\phi(x,t) = \di_x \F(x,t)$; we use the integrated version for numerical
% convenience.

The Fortet equation is particularly appealing to use when we have an analytical
expression for $\F$. For the problem at hand, $\Phi$ is complicated
due to the time-dependent forcing. However, the following transformation yields
a time-homogeneous $Y$ for which $\Phi$ will be tractable, along with an associated
moving threshold, $\vt(t)$. This makes feasible
the use of the Fortet equation. To obtain this transformation, cf.\
\cite{Shimokawa1999}, consider the deterministic version of the SDE in
\cref{eq:X_evolution_uo_renewal} 
\begin{equation}
dv(t) = \left( \a - v + \g \sin(\th (t + \phi) ) \right) \intd{t},
\label{eq:LIF_deterministic}
\end{equation}
$$v(0) = 0
$$
with solution
\begin{equation}
v(t) = \a( 1- \exp(-t)) 
  + \frac{\g}{\sqrt{1+\th^2}} 
\left[  \sin(\th(t + \phi) - \psi )
 - \exp(-t)\sin(\phi\th - \psi ) \right]; \quad  
\psi = \arctan(\th).
\label{eq:LIF_deterministic_solution}
\end{equation}

Now take $X_t$, the solution to \cref{eq:X_evolution_uo_renewal} and
$v(t)$, \cref{eq:LIF_deterministic_solution}, and 
let $Y_t = X_t - v(t)$. Then
\begin{equation}
dY_t = -Y_t dt + \b \intd{W},
\label{eq:Y_evolution_uo_transformed}
\end{equation}
which has the time and parameter dependent threshold
\begin{equation}
\vt_{\{\a,\g;\phi\}}(t) = \vt - v(t).
\end{equation}
That is, $X_t$ hits the constant threshold $\vt$ if and only if $Y_t$ hits the
moving threshold $\vt_{\{\a,\g;\phi\}}(t)$, where the subindex
indicates the dependence on $\a,\g$ and $\phi$. Therefore the ISIs produced by $X$ and $Y$
are the same and so are their distributions. Thus, $g_\phi(\t)$ satisfies
\begin{equation}
1 - \F_{\{\b\}}(\vt_{\{\a,\g;\phi\}} (t), t|0, 0) =
\int_0^t g_{\phi} (\tau)
\left[1-\F_{\{\b\}} \left(\vt_{\{\a,\g;\phi\}} (t),  t|
						  \vt_{\{\a,\g;\phi\}} (\t), \t ) \right)
      \right] \intd{\t},
\label{eq:Fortet_moving_vth}
\end{equation}
where 
$$
\F_{\{\b\}}(y,t| y_0, t_0) = \frac{1}{ \sqrt{\pi \b^2(1-e^{-2(t-t_0)}) }}
\int_{-\infty}^{y} \exp \left( -\frac{(x - y_0e^{-(t-t_0)})^2}
							         {\b^2(1-e^{-2(t-t_0)})} \right) \intd{x}
$$
is the conditional cumulative distribution function of $Y_t$ defined in
\cref{eq:Y_evolution_uo_transformed}.

\section{Parameter Estimation Algorithms}
\label{sec:estimation_algos}
The unknown parameters in \cref{eq:X_evolution_uo_renewal} are $\a,\b$ and $\g$,
while we assume $\th$ known. The reason why the amplitude, $\g$, is often
unknown while the frequency, $\th$, is known is that one can usually observe the
sinusoidal input and thus its frequency. Further, the encoding of the input into
neuronal firing patterns often involves phase locking to the sinusoidal
component. However, the actual forcing amplitude at the level of the neuron is
usually modified by various synaptic and other filtering processes, unless the
cell receives direct sinusoidal current injection.

Our goal is to estimate the structural parameters $(\abg)$ from a sample of
spike time data, $\{i_1,\ldots,i_N\}$. There are several algorithms for
estimating the parameters for the simpler and more common case of $\g = 0$. One
such algorithm relies on the Fortet equation, see
\cite{Ditlevsen2008,Ditlevsen2007}, which we extend to the presence of a
time-varying current. A more basic approach is to directly solve the
Fokker-Planck equation for the probability density of $X_t$,
\cite{Sirovich2011a,Paninski2004,Dong2011}, from which one can derive the
survival distribution of $I_n$ and use this to compare against the empirical
survival distribution of $I_n$ obtained from data. An approximate maximum
likelihood approach is also possible by numerical differentiation. The relation
between Fokker-Planck equations and the first-passage time problem is discussed
in most introductory books on stochastic analysis, see, for example,
\cite{Jacobs}. A recent review of this approach for the simple $\g=0$ case in
neuronal modeling can be found in \cite{Sirovich2011a}, wherein the first
passage problem is discussed at great lengths in the context of spiking neurons.
We will use this in \cref{sec:fp_estimation}. A more elaborate approach using
the Fokker-Planck equation to approximate the hitting time distribution is given
in \cite{Lo2006}. The techniques in \cite{Lo2006} avoid the need to compute the
Fokker-Planck PDE numerically, instead approximating it with analytically known
solutions. This approach might offer significant computational savings, but
since this would at most amount to a computational speed-up of our algorithm, we
have left this unexplored for now.


The immediate problem in generalizing the aforementioned approaches to the case
of $\g \neq 0$ is that the $I_n$'s are no longer identically distributed since
the phase $\phi_{n-1}$ of the $n$th interval $I_n$ depends on $t_{n-1}$, the
time the previous spike occurred. The $I_n$'s are also dependent, but conditionally
independent given $\phi_{n-1}$. So the trajectories in each interval are
parametrized by the value of $\phi_{n-1}$ at the time of the last spike/reset.
We overcome this obstacle by splitting the $I_n$'s in groups, and
approximating the $I_n$'s within groups as coming from identically
distributed trajectories in a sense to be specified below. This approximation
which solves the challenge of dependent and non-identically distributed ISIs is
the primary contribution of this paper.

\subsection{$\phi$ - binning}
Before we can use \cref{eq:FP_pde_OU_absorbBC_CDF} or
\eqref{eq:Fortet_moving_vth}, we need to deal with the fact that $\phi$ is not
fixed, but instead each $I_n$ starts with a distinct $\phi_n$. Our approach is
to partition the interval $[0, 2 \pi/\th]$ into $M$ bins, where $M \ll N$, and
represent each bin by the midpoint of the bin, $\phi_m$. Then we approximate the
$N$ observed $\phi_n$'s by the closest $\phi_m$ and pretend that any observed
$I_n$ was not produced by a trajectory of the form in
\cref{eq:X_evolution_uo_renewal} with $\phi = \phi_n$, but with $\phi = \phi_m$.
Our hope is that for a judicious choice of $M$, we can balance the error of
$\phi_n \neq \phi_m$ with having enough data points in each bin in order to
obtain a useful estimate from \cref{eq:FP_pde_OU_absorbBC_CDF} or
\eqref{eq:Fortet_moving_vth}.

There is clearly much freedom in how one sets up these bins, but we will do the
simplest thing and make them all of equal width, $\dphi = 2 \pi / {(\th M)}$.
Each $\phi_n$ will belong to one and only one of the bins $ [\phi_m - \dphi/2,
\phi_m + \dphi/2)_{m=1}^M, $ with centre points $ \phi_m = \dphi / 2 + (m-1)
\dphi$, for $m = 1,\ldots, M$. Thus, given an empirically observed $I_n$ with
associated $\phi_n$, we will pretend that it was produced by the process $$ dX_s
= (\a - X_s) \intd{s}  + \g \sin(\th (s+\phi_m(n) ))
\intd{s}
+ \b \intd{W_s}, $$ where $$ \phi_m(n)  = \argmin_{\phi_m} {| \phi_n - \phi_m|}.
$$ This binning is illustrated in \cref{fig:binning_visualized}.
\begin{figure}[ht]
\begin{center}
%     \subfloat[Raw ISIs]{
% 	\includegraphics[width=0.45\textwidth]{Figs/Bins/Example_raw.png}
% 	}
%     \subfloat[Binned ISIs]{
% 	\includegraphics[width=0.45\textwidth]{Figs/Bins/Example_binned.png}
%   	}  
\includegraphics[width=0.9\textwidth]{Figs/Bins/Example_composite.pdf}
  \end{center}
\caption[Phase binning illustration]{The raw $(\In, \phi_n)$ pairs (left) are
binned into a set of $M$ bins with a representative $\phi_m$ (right) and the ISIs within each bin are treated
as a renewal process. In this illustration, $M=8$, $\th =
1$ while the parameters $\abg$ are taken from the
supra-threshold regime. }
\label{fig:binning_visualized} 
\end{figure}

While we have no rigorous approach to determine the value of $M$, our limited
experience suggests that given $N=1000$ ISIs, $M=10$ or $M=20$ gives satisfactory
results for very different parameter regimes. In general, choosing $M$ is a
balancing act. For $M$ too high, the resulting bins will have too few data
points to approximate $\G(I)$ accurately. Therefore $M$ is forced to be small
when sample size is not large. For $M$ too low, the approximation of the phase
shifts will be poor, leading to a biased estimate of $\G(I)$. We illustrate the
effect of increasing $M$ in \cref{fig:effect_of_M}. Generally, as long as there
are sufficient data points, as $M$ increases, the approximation of using the
survival distribution with $\phi_m$ instead of $\phi_n$ improves since
$\phi_m(n) \ra \phi_n$ as $M \ra \infty$. In the sequel, we will use $M=20$ for
sample sizes of $N=1000$ and $M=8$ for sample sizes of $N=100$.

\begin{figure}[h]
\begin{center}
% \subfloat[M=5]
% {
% \includegraphics[width=0.25\textwidth]
% {Figs/FP/SuperSin_M=5.png}
% }
% \subfloat[M=10] 
% {
% \includegraphics[width=0.25\textwidth]
% {Figs/FP/SuperSin_M=10.png}
% } 
% \subfloat[M=20]
% {
% \includegraphics[width=0.25\textwidth]
% {Figs/FP/SuperSin_M=20.png}
% }
% \subfloat[M=40]
% {
% \includegraphics[width=0.25\textwidth]
% {Figs/FP/SuperSin_M=40.png}
% }
% \includegraphics[width=0.99\textwidth]{Figs/FP/EffectOfM.pdf}
\includegraphics[width=\textwidth]{Figs/FP/EffectOfM_Referees.pdf}
\caption[Effect of Bin-size]{Effect of $M$, the number of bins, on the
approximate survival distribution. The full-drawn blue curve is the true survivor distribution
given in \cref{eq:FP_pde_OU_absorbBC_CDF}, the red points are the approximation
given in \cref{eq:SDF_estimate_per_bin}.  
In the figures, the least populous (above) and most populous (below) bin for
each $M$ is shown. The width of the bins is $\dphi = {2\pi}/{(\th M)}$.
We have used A,E) $M=5$, B,F) $M=10$, C,G) $M=20$,
D,I) $M=40$. As $M$ increases, the approximation of using the survival
distribution using $\phi_m$ instead of $\phi_n$ improves since $\phi_m(n) \ra
\phi_n$ as $M \ra \infty$. The data is generated using parameter
values from the super-sinusoidal regime and $N=1000$. For this
particular data set the largest generated ISI was 6.55 time units.}
\label{fig:effect_of_M}
\end{center}
\end{figure}

\subsection{Fokker-Planck Algorithm}
Within each bin it is clear how to apply \cref{eq:SDF_vs_F_at_thresh}. In the
$m$th bin, for a given $\phi_m$, we approximate $\G_{\phi}(t)$  by
\begin{equation}
\Gest_{\phi_m}(t) =
 \frac{\#[i_n > t \, \big| \, \phi_{n-1} \in [\phi_m - \dphi/2,
\phi_m + \dphi/2) ]}{N_m},
\label{eq:SDF_estimate_per_bin}
\end{equation}
where $N_m$ is the number of ISIs in bin $m$. Using \cref{eq:SDF_vs_F_at_thresh}
we define the loss function
\begin{equation}
L(\a,\b,\g) = 
\sum_{\phi_m} N_m \Big\{ 
% \int_0^T  \left( \G_{\phi_m}(t) - F^{\phi_m}_{\a,\b,\g}(\vt, t) \right)^2
% \intd{t} \Big\}.
\sup_{t>0} \left| \Gest_{\phi_m}(t) - F^{\phi_m}_{\a,\b,\g}(\xth,
t) \right| \Big \}.
\label{eq:loss_function_absorbingBC}
\end{equation}
The weight $N_m$ is included so that bins with larger sample sizes have a
larger influence on the estimates. 
 
To evaluate the supremum in \cref{eq:loss_function_absorbingBC}, we
spline interpolate the empirically discrete $\Gest$ for each $\phi_m$, sample at
the time nodes of the PDE discretization and 
finally take the maximum amongst the sampled values.
We then minimize $L$ using an optimization algorithm (see below,
\cref{sec:method_performance}) and take our estimates $\abgest$ to be
$$
\abgest = \argmin_{\abg} L(\abg).
$$

Note that the relation between the spike time survival density, $\G_{\phi}$ and
the transition distribution, $F_{\phi}$, in \cref{eq:SDF_vs_F_at_thresh} could
also allow for an approximate maximum likelihood estimator (MLE), based on
maximizing 
$$
L^{\textrm{MLE}}(\a,\b,\g) = \sum_n  \log ( g_{\phi_{n-1}}(i_n) )
= \sum_n \log \big[ -\di_t F^{\phi_{n-1}}_{\a,\b,\g}(\xth, t) \big] \Big|_{t =
i_n},
 $$ where the derivative has to be approximated by finite differences. We
can then again use binning to avoid having to compute the PDE separately for
each $(i_n, \phi_{n-1})$. Our experience with the MLE approach has been that the
quality of the estimates provided are similar to those obtained by minimizing
\cref{eq:loss_function_absorbingBC} and that the associated computing times are
on the same order. Due to this similarity and in order to keep the paper
concise, we include details of the MLE estimates only in the supplementary
online material.
\subsection{Fortet Algorithm}
An alternative approach is to form a loss function from
\cref{eq:Fortet_moving_vth}. This is similar to what is done in
\cite{Ditlevsen2008,Ditlevsen2007} for the simpler case of a constant threshold.
Noting that $\int_0^t g (\t) [1-\F ] \intd{\t} = \Exp[ (1 - \F ) \charf_{I \leq
t} ]$ where the expectation is taken with respect to the distribution of the
random variable $I$, we can use the fact that the ISIs are approximately
independent and invoke the law of large numbers to estimate the integral as
\begin{multline*}
\int_0^t g_{\phi_m}(\t) 
\big[1-\F^{(\phi)}_{\{\b\}}(\vt_{\{\a,\g;\phi\}}(t),  t \,|\,
\vt_{\{\a,\g;\phi\}}(\t), \t) \big] \intd{\t} \approx 
\\
1/N_m \sum_{\In < t} 
\big[1
- \F^{(\phi)}_{\{\b\}}(\vt_{\{\a,\g;\phi\}}(t), t \,|\,
\vt_{\{\a,\g;\phi\}}(\In),\In )\big].
\end{multline*}


We then define the loss function
\begin{align}
L(\a,\b,\g) = 
\sum_{\phi_m} N_m \Bigg\{ 
% \sum_\In \Bigg[
% \int_{\e}^{I_{\text{max}}}
% 		 \Big[& 1 - \F^{(\phi_m)}_{\abg}(\vt(s), s|v_0) ] -
% 		 \notag
% \\
% 		 &-  \tfrac{1}{N_m}\sum_{\In' < s} 
% 		 [1 - \F^{(\phi_m)}_{\abg}(\vt(s),s-\In'| \vt(\In') ) ]
% 		   \Big]^2	\intd{s}
\sup_{s > 0}
		 \Big|& 1 - \F^{(\phi_m)}_{\{\b\}}(\vt_{\{\a,\g;\phi\}}(s), s \,|\,0,0) ] 
		 \notag
\\
		 &-  \tfrac{1}{N_m}\sum_{\In < s}
		 \big[1 - \F^{(\phi_m)}_{\{\b\}}(\vt_{\{\a,\g;\phi\}}(s),s  \,|\,
		 \vt_{\{\a,\g;\phi\}}(\In), \In) \big] \Big| / \omega(\phi_m; \a,\b,\g)
		   \Bigg\}.
\label{eq:loss_function_fortet}
\end{align}
We divide each inner term by $\omega(\phi_m; \a,\b,\g) =
\sup_{s > 0} |1 - \F^{(\phi_m)}_{\abg}(\vt(s), s|v_0) |$,
following the suggestion in
\cite{Ditlevsen2007}. This scaling ensures that \cref{eq:Fortet_moving_vth}
divided by $\omega(\a,\b,\g)$ will vary between $0$ and $1$ for all parameter
values thus giving sense to the measure defined by the loss
function. Since we can solve in closed form for $\F$, we have all we
need given an observed spike train of 
$\In$'s. We evaluate the $\sup$ by sampling at 
$K=500$ uniformly spaced points in $(0, I_{\max} + \e]$ and taking the maximum
of the sampled values. 


\subsection{Initialization of the algorithms}
The parameter search can be initialized in a simple way using the fact that
the Fokker-Planck PDE is almost an 'advection-diffusion' equation whose solution is
almost a Gaussian. Then $\G(t)$ can be approximated by the
amount of probability mass of a Gaussian to the left of the threshold at time $t$. The
idea is as follows. Suppose we are solving the following PDE
\begin{equation}
\di_t \f = -U \di_x [\f] + \frac{\b^2}{2} \di^2_x [\f].
\label{eq:FP_pde}
\end{equation}
Its solution given an initial condition $\f(x,0) = \delta(x)$ will be a 
Gaussian bell moving to the right with speed $U$ and standard deviation $\s =
\b\sqrt{t}$.

The survivor function $\G(t)$ can be thought of as the amount of area that has
passed the threshold (from the left moving to the right). We can then invert the
information about $\G$ to estimate $U$ and $\b$. In particular, a Gaussian bell
has $\approx 0.158$ of its mass more than one standard deviation to the right of
its mean. Thus, at time $t_1$ such that $\G(t_1) = 0.842$, the right tail of
more than one standard deviation of the Gaussian bell has crossed the threshold.
The threshold is at $x = 1$ and we obtain the following equation
\begin{equation}
U t_1 + \b \sqrt{t_1} = 1.
\label{eq:right_1std}
\end{equation}
Similarly, at time $t_2$ such that $\G(t_2) = 1 - 0.842$, the Gaussian bell has
crossed the threshold except for the left tail and we have
\begin{equation}
U t_2 - \b \sqrt{t_2} = 1.
\label{eq:left_1std}
\end{equation}
If $U$ and $\b$ were constant, then \cref{eq:right_1std,eq:left_1std} provide
two equations in two unknowns.
% \begin{figure}[htp]
% \begin{center}
%   \includegraphics[width=.75\textwidth]{Figs/Diagrams/Normal_quantiles_from_wp.png}
%   \caption[labelInTOC]{Illustration of Gaussian bell -
%   from: http://en.wikipedia.org/wiki/File:Standard\_deviation\_diagram.svg }
%   \label{fig:gaussian_bell}
% \end{center}
% \end{figure}
However, $U = U(x,t) = (\a - x + \g \sin(\th (t + \phi)) )$ is not constant and
we approximate $U$ as
\begin{equation}
U(x,t ) \approx \a -0.5 + \g \frac{1}{t}\int_0^t \sin(\th (\t+\phi)) \intd{\t}, 
\end{equation}
i.e.\ we approximate the space-dependent term, $x$, with the mid-point between
the reset value, $v_0 = 0$, and the threshold, $\vt = 1$, and we approximate
the time-dependent term, $\sin(\th (\t+\phi))$, by its time-average value
between $0$ and $t$. If we use the \nth{0}, \nth{1} and \nth{2} standard 
deviation points, we can form 5 equations in 3 unknowns as follows
\begin{eqnarray*}
\a t_1 + \g s(t_1) + 2\b \sqrt{t_1}
&=& 1 + 0.5 t_1
\\
\a t_2 + \g s(t_2) + \b \sqrt{t_2}
&=& 1 + 0.5 t_2 
\\
\a t_3 + \g s(t_3) + 0\b
&=& 1 + 0.5 t_3
\\
\a t_4 + \g s(t_4)-1 \b \sqrt{t_4}
&=& 1 + 0.5 t_4 
\\
\a t_5 + \g s(t_5) - 2\b \sqrt{t_5}
&=& 1 + 0.5 t_5 
\end{eqnarray*}
with the time-average weighting function $s(t) = (\cos(\th \phi)
-\cos(\th(t+\phi)))/{\th} $. However, the  approximation is best for earlier
times, when the solution is closer to a Gaussian bell that is approaching the
threshold, but less correct for later times, since it neglects the loss of
probability mass and thus overestimates the backward probability current.
Indeed, we have found it to be best to use only $t_1$ and $t_2$. In the
following we use only these equations
\begin{eqnarray*}
\a t_1 + \g s(t_1) + 2\b \sqrt{t_1}
&=& 1 + 0.5 t_1
\\
\a t_2 + \g s(t_2) + \b \sqrt{t_2}
&=& 1 + 0.5 t_2 
\end{eqnarray*}
for the initializer.
We can form these equations separately for each $\phi_m$ bin, thus
resulting in $M \times 2$ equations for the unknowns $\a, \b$ and $\g$. Since
we have more equations than unknowns, we use least-squares estimates in a
regression to pick out unique $\a,\b$ and $\g$ estimates. 

The proposed initialization procedure has two advantages. First, it is
automatic, i.e.\ it requires only the data and no input or  guidance from the
user. Second, it is extremely fast. While the precise effect of the initializer
is shown in \cref{sec:method_performance}, it is intuitively clear that it will
work best in the supra-threshold parameter regime when the bell curve is truly
moving past the threshold as a whole and less so for sub-threshold regimes, when
only the diffusive force serves to propel the process to reach $\vt$. The
behaviour of the initializer in the different regimes is illustrated in
\cref{fig:sdf_real_vs_init_estimated}. What we show in
\cref{fig:sdf_real_vs_init_estimated} is the following: First we show the
survival distribution for a given regime and $\phi_m$ fixed. Then using data
generated from such a regime and with $\phi_n$ in the $m$th bin, the initializer
tries to find the best approximation by the motion of a Gaussian bell which will
fit this data, in the sense of solving for $\a,\b,\g$ as previously described.
Once this is done, we then show in red the amount of area under this Gaussian
bell to the left of the threshold. Of course the interpretation of the survival
distribution for an ISI as a fraction of the area under a moving bell with
conserved total area is wrong, but the assumption is useful in automatically
generating initial values for the more appropriate approximations to start their
work.

\begin{figure}[htp]
\begin{center}
\includegraphics[width=.99\textwidth]{Figs/FP/sdf_init_vs_exact.pdf}
\caption[Estimation initiation]{The blue curves are the numerically
obtained survivor distributions $\G_\phi$ for the exact parameters in the four regimes (as in
  \cref{tab:regimes}) and $\th=1$. The red curves are obtained in the following
  manner: Simulations using the true parameters were used to generate sample spikes.
  Using these samples, the initializer algorithm was used to generate estimates
  for $\a,\b,\g$. Using these estimates, the bell curve discussed in sec.\ 3.4
  was formed and evolved in time. 
  Thus, the red curve drawn in the figures measures the area under
  this bell that is to the left of the threshold at time $t$. 
 A) supra-threshold, B) super-sinusoidal, C) critical, D) sub-threshold.}
  \label{fig:sdf_real_vs_init_estimated}
\end{center}
\end{figure}


\section{Method Comparison on Simulated Data}
\label{sec:method_performance}
We will now use our algorithms on spike trains simulated from the four different
regimes; the supra-threshold, the critical, the super-sinusoidal and the
sub-threshold. We have used 100 sample spike trains per regime, with $N=100$ as
well as $N=1000$ spikes per train. In order to perform the numerical
minimization of \cref{eq:loss_function_absorbingBC,eq:loss_function_fortet}, we
have used an implementation of the Nelder-Mead algorithm from the SciPy library
\cite{scipy}. The Nelder-Mead algorithm is a non-linear minimization routine
which uses a bounding-polygon method to zero-in on the minimum and thus avoids
the need to provide the gradient of the loss function. It is the standard
non-gradient minimization algorithm. 

The estimation results are shown in
\cref{fig:comprehensive_test_SuperT_relerrors,fig:comprehensive_test_SubT_relerrors,fig:comprehensive_test_crit_relerrors,fig:comprehensive_test_SuperSin_relerrors},
where we plot box plots for the estimates, $\abgest$ in the four regimes. We
also tabulate the average and the empirical 95\% confidence intervals of the
estimates in \cref{tab:est_quantiles_100,tab:est_quantiles_1000}. Conclusions
that can be drawn from these results are as follows. The initializer method is
effective for the supra-threshold regime and gives reasonable ballpark estimates
for all regimes, though the error can be substantial for the super-sinusoidal
regime. In general, both the Fortet and Fokker-Planck algorithm estimate the
parameters well in the supra-threshold, critical and super-sinusoidal regimes.
The estimators variance is especially low in the supra-threshold regime, while
it is higher for the critical and super-sinusoidal regimes. In the
super-sinusoidal regime the two algorithms give accurate estimates even though
the initializer can be quite off. On the other hand, in the sub-threshold regime
the initializer has a performance comparable to that of the two more involved
methods. It seems that distinguishing between the constant bias and the
sinusoidal current is difficult if their sum is not sufficient to generate
spikes without noise.


The Fokker-Planck method has a larger bias but a smaller spread than the Fortet
method for $N=100$, \cref{tab:est_quantiles_100}. However for
$N=1000$, the two methods have comparable 
spreads, while the Fortet method retains a smaller bias, see
\cref{tab:est_quantiles_1000}. More precisely, for $N=1000$, the Fokker-Planck
method has a smaller spread in the sub-threshold regime, while the Fortet method
has a smaller spread in the super-sinusoidal regime. As such, at least in the
super-sinusoidal regime, the Fortet method seems superior.

The two algorithms are numerically intensive. For $N=100$ and $N=1000$ spikes,
we show the times taken for the estimation in \cref{tab:walltimes}. While we
have done most of our numerical work in Python/SciPy\cite{scipy}, we have
implemented the critical components of both algorithms in C. That is we solve
the inner part of \cref{eq:loss_function_fortet} and the Fokker-Planck PDE,
\cref{eq:FP_pde_OU_absorbBC_CDF}, in C using the GSL libraries\cite{gsl}. From
\cref{tab:walltimes}, we can verify that the computing time for the Fortet
algorithm scales proportionally with the number of spikes. This is to be
expected, since the Fortet equation has a term of the form $\sum_{i_n}$ which in
turn has $N$ terms and this forms the bulk of the computing time for the
Fortet equation. The Fokker-Planck algorithm, on the other hand, scales
less-than-linearly with $N$, since the dependency on $N$ is in forming the
approximation, $\Gest$ to the survivor function and that is not computationally
intensive (solving the PDE is).


\begin{table}
\begin{center}
%\subfloat[Supra-threshold]
{\begin{tabular}{|c|ccc|} 
Parameter
& Initializer
& Fokker-Planck
& Fortet
\\ \hline
\multicolumn{4}{|c|}{Supra-threshold regime} \\[1mm]
$\alpha=1.40$
& $1.43 : [1.29, 1.56]$
& $1.34 : [1.24, 1.43]$
& $1.41 : [1.33, 1.49]$
\\
$\beta=0.30$
& $0.17 : [0.10, 0.24]$
& $0.29 : [0.21, 0.39]$
& $0.29 : [0.22, 0.36]$
\\
$\gamma=0.14$
& $0.16 : [0.02, 0.33]$
& $0.12 : [0.02, 0.23]$
& $0.12 : [0.01, 0.24]$
\\
\hline \hline
\multicolumn{4}{|c|}{Super-sinusoidal regime} \\[1mm]
$\alpha=0.10$
& $0.92 : [0.83, 1.01]$
& $0.28 : [0.02, 0.59]$
& $0.24 : [-0.22, 0.42]$
\\
$\beta=0.30$
& $0.15 : [0.10, 0.25]$
& $0.31 : [0.14, 0.53]$
& $0.32 : [0.14, 0.46]$
\\
$\gamma=1.98$
& $1.35 : [1.13, 1.57]$
& $1.67 : [1.33, 2.05]$
& $1.77 : [1.44, 2.38]$
\\
\hline \hline
\multicolumn{4}{|c|}{Critical regime} \\[1mm]
$\alpha=0.50$
& $0.72 : [0.66, 0.80]$
& $0.57 : [0.32, 0.73]$
& $0.57 : [0.36, 0.73]$
\\
$\beta=0.30$
& $0.19 : [0.10, 0.26]$
& $0.27 : [0.17, 0.40]$
& $0.25 : [0.15, 0.40]$
\\
$\gamma=0.71$
& $0.57 : [0.44, 0.73]$
& $0.55 : [0.30, 0.83]$
& $0.62 : [0.38, 0.93]$
\\
\hline \hline
\multicolumn{4}{|c|}{Sub-threshold regime} \\[1mm]
$\alpha=0.40$
& $0.62 : [0.57, 0.67]$
& $0.63 : [0.33, 0.84]$
& $0.58 : [0.03, 1.00]$
\\
$\beta=0.30$
& $0.17 : [0.10, 0.29]$
& $0.20 : [0.10, 0.37]$
& $0.19 : [0.00, 0.41]$
\\
$\gamma=0.57$
& $0.32 : [0.00, 0.53]$
& $0.29 : [0.00, 0.62]$
& $0.46 : [0.00, 1.19]$
\\
\hline
 \end{tabular}}\\
\end{center}
\caption[Estimator Performance given 100 spikes]{Averages and empirical 95\%
confidence intervals of the estimates for $N=100$ spikes per train.}
\label{tab:est_quantiles_100}
\end{table}

%\end{document}

\begin{table}
\begin{center}
{\begin{tabular}{|c|ccc|} 
Parameter
& Initializer
& Fokker-Planck
& Fortet
\\ \hline
\multicolumn{4}{|c|}{Supra-threshold regime} \\[1mm]
$\alpha=1.40$
& $1.44 : [1.40, 1.50]$
& $1.36 : [1.33, 1.40]$
& $1.40 : [1.37, 1.42]$
\\
$\beta=0.30$
& $0.25 : [0.22, 0.28]$
& $0.29 : [0.26, 0.32]$
& $0.30 : [0.27, 0.32]$
\\
$\gamma=0.14$
& $0.14 : [0.10, 0.19]$
& $0.14 : [0.10, 0.17]$
& $0.14 : [0.10, 0.18]$
\\
\hline \hline
\multicolumn{4}{|c|}{Super-sinusoidal regime} \\[1mm]
$\alpha=0.10$
& $0.90 : [0.85, 0.92]$
& $0.11 : [0.03, 0.29]$
& $0.10 : [0.03, 0.16]$
\\
$\beta=0.30$
& $0.18 : [0.14, 0.23]$
& $0.30 : [0.21, 0.34]$
& $0.31 : [0.22, 0.34]$
\\
$\gamma=1.98$
& $1.26 : [1.16, 1.34]$
& $1.92 : [1.49, 2.05]$
& $1.96 : [1.86, 2.07]$
\\
\hline \hline
\multicolumn{4}{|c|}{Critical regime} \\[1mm]
$\alpha=0.50$
& $0.73 : [0.70, 0.75]$
& $0.51 : [0.43, 0.63]$
& $0.53 : [0.45, 0.64]$
\\
$\beta=0.30$
& $0.20 : [0.17, 0.24]$
& $0.29 : [0.24, 0.32]$
& $0.28 : [0.19, 0.33]$
\\
$\gamma=0.71$
& $0.54 : [0.44, 0.61]$
& $0.66 : [0.52, 0.76]$
& $0.67 : [0.54, 0.77]$
\\
\hline \hline
\multicolumn{4}{|c|}{Sub-threshold regime} \\[1mm]
$\alpha=0.40$
& $0.62 : [0.55, 0.65]$
& $0.57 : [0.45, 0.66]$
& $0.56 : [0.26, 0.71]$
\\
$\beta=0.30$
& $0.20 : [0.17, 0.26]$
& $0.22 : [0.18, 0.29]$
& $0.21 : [0.13, 0.35]$
\\
$\gamma=0.57$
& $0.36 : [0.18, 0.44]$
& $0.36 : [0.25, 0.50]$
& $0.43 : [0.28, 0.72]$
\\
\hline
 \end{tabular}}\\
\end{center}
\caption[Estimator Performance given 1000 spikes]{Averages and empirical 95\%
confidence intervals of the estimates for $N=1000$ spikes per train. }
\label{tab:est_quantiles_1000}
\end{table}


% \begin{table}
% \begin{center}
% \subfloat[N=100]{
% \begin{tabular}{c|ccc|ccc|ccc|}
% Regime & Initializer && & Fortet &&& FP && \\
% \hline
%  & $\aest$ &$\best$&$\gest$& &&&&& \\
% \hline
%  superT
% & $1.61 \pm [1.34, 1.76]$
% & $0.16 \pm [0.10, 0.32]$
% & $1.15 \pm [0.88, 1.39]$
% & $1.44 \pm [1.31, 1.56]$
% & $0.30 \pm [0.13, 0.42]$
% & $0.95 \pm [0.75, 1.09]$
% & $1.53 \pm [1.41, 1.73]$
% & $0.33 \pm [0.13, 0.45]$
% & $0.91 \pm [0.61, 1.08]$
% \\
% subT
% & $0.60 \pm [0.53, 0.67]$
% & $0.15 \pm [0.10, 0.23]$
% & $0.16 \pm [0.00, 0.42]$
% & $0.68 \pm [0.43, 0.83]$
% & $0.17 \pm [0.10, 0.29]$
% & $0.15 \pm [-0.00, 0.50]$
% & $0.74 \pm [0.07, 1.00]$
% & $0.11 \pm [0.00, 0.40]$
% & $0.28 \pm [-0.05, 1.24]$
% \\
% crit
% & $0.87 \pm [0.78, 0.95]$
% & $0.17 \pm [0.10, 0.27]$
% & $0.91 \pm [0.73, 1.07]$
% & $0.53 \pm [0.34, 0.76]$
% & $0.32 \pm [0.13, 0.44]$
% & $0.98 \pm [0.73, 1.27]$
% & $0.54 \pm [0.40, 0.68]$
% & $0.30 \pm [0.12, 0.41]$
% & $1.03 \pm [0.83, 1.25]$
% \\
% superSin
% & $1.06 \pm [0.92, 1.19]$
% & $0.17 \pm [0.10, 0.32]$
% & $1.75 \pm [1.51, 1.97]$
% & $0.19 \pm [-0.16, 0.95]$
% & $0.30 \pm [0.14, 0.50]$
% & $2.33 \pm [1.64, 2.69]$
% & $0.29 \pm [-0.16, 0.59]$
% & $0.32 \pm [0.13, 0.43]$
% & $2.29 \pm [1.90, 2.78]$
% \\
% \end{tabular}
% }
% \end{center}
% \label{tab:est_quantiles}
% \end{table}

% #ABSOLUTE ERRORS
% \begin{figure}[h]
% \begin{center}
% \subfloat[SUPER THRESHOLD]
% {
% \label{fig:comp_test_superT}
% \includegraphics[width=0.48\textwidth]
% {Figs/Estimates/FP_vs_Fortet_100x1000superT_est_errors.png}
% }
% \subfloat[CRITICAL]
% {
% \label{fig:comp_test_critical}
% \includegraphics[width=0.48\textwidth]
% {Figs/Estimates/FP_vs_Fortet_100x1000crit_est_errors.png} 
% }
% \\
% \subfloat[SUB THRESHOLD]
% {
% \label{fig:comp_test_subT}
% \includegraphics[width=0.48\textwidth]
% {Figs/Estimates/FP_vs_Fortet_100x1000subT_est_errors.png}
% }
% \subfloat[SUPER SINUSOID]
% {
% \label{fig:comp_test_superSin}
% \includegraphics[width=0.48\textwidth]
% {Figs/Estimates/FP_vs_Fortet_100x1000superSin_est_errors.png}
% }
% \caption{Absolute Errors of the parameter estimation routines for the 4
% different spike regimes. For each regime, we plot the difference, e.g.\
% $\aest - \a$, between the simulation parameters and each estimated
% parameter for each of the three estimation routines (upper plots) and in the
% lower plot, we show the sum of the absolute values of the errors,
% $|\aest - \a| + |\best - \b| +|\gest - \g|$. Note that figures
% for different regimes have different scales.}
% \label{fig:comprehensive_tests_abserrors}
% \end{center}
% \end{figure}
% % #RELATIVE ERRORS:
% \begin{figure}[h]
% \begin{center}
% \subfloat[SUPER THRESHOLD]
% {
% \label{fig:comp_test_superT}
% \includegraphics[width=0.48\textwidth]
% {Figs/Estimates/FP_vs_Fortet_100x1000superT_est_rel_errors.png}
% }
% \subfloat[CRITICAL]
% {
% \label{fig:comp_test_critical}
% \includegraphics[width=0.48\textwidth]
% {Figs/Estimates/FP_vs_Fortet_100x1000crit_est_rel_errors.png} 
% }
% \\
% \subfloat[SUB THRESHOLD]
% {
% \label{fig:comp_test_subT}
% \includegraphics[width=0.48\textwidth]
% {Figs/Estimates/FP_vs_Fortet_100x1000subT_est_rel_errors.png}
% }
% \subfloat[SUPER SINUSOID]
% { 
% \label{fig:comp_test_superSin}
% \includegraphics[width=0.48\textwidth]
% {Figs/Estimates/FP_vs_Fortet_100x1000superSin_est_rel_errors.png}
% }
% \caption{Relative Errors of the parameter estimation routines for the 4
% different spike regimes. For each regime, we plot the difference, e.g.\
% $\tfrac{\aest - \a}{\a}$, between the simulation parameters and each
% estimated parameter for each of the three estimation routines (upper plots) and in the
% lower plot, we show the sum of the absolute values of the errors,
% $|\tfrac{\aest - \a}{\a}|+
%  |\tfrac{\best - \b}{\b}|+
%  |\tfrac{\gest  - \g}{\g}|$. Note that figures for different regimes
%  have different scales!}
% \label{fig:comprehensive_tests_relerrors}
% \end{center}
% \end{figure}

\begin{figure}[p]
\begin{center}
% \subfloat[N=100]
% {
% \label{fig:comp_test_superT_100}
% \includegraphics[width=0.48\textwidth]
% {Figs/Estimates/FP_vs_Fortet_100x100superT_est_rel_errors.png}
% }
% \subfloat[N=1000]
% {
% \label{fig:comp_test_superT_1000}
% \includegraphics[width=0.48\textwidth]
% {Figs/Estimates/FP_vs_Fortet_100x1000superT_est_rel_errors.png}
% }
\includegraphics[width=0.99\textwidth]
{Figs/Estimates/FP_vs_Fortet_100x100_x1000superT_est_rel_errors_joint.pdf}
\caption[Estimates box-plots for supra-threshold regime]{Boxplots of parameter
estimates for the supra-threshold regime. The upper plots (A,B,C) show estimates using $N=100$ sample spikes per
estimation, while the lower plots (D,E,F) use $N=1000$. The dashed line
indicates the true parameter value, while the red line inside the boxes
indicates the median of the estimates.
\\
The boxes contain the central 50\% of the estimates. The bars indicate
the range of the estimates, except for outliers given by the points
outside the bars, and defined to be more than 1.5 times the
interquantile range (the height of the box) from the box.}
\label{fig:comprehensive_test_SuperT_relerrors}
\end{center}
\end{figure}
\begin{figure}[p]
\begin{center}
% \subfloat[N=100]
% {
% \label{fig:comp_test_superT_100}
% \includegraphics[width=0.48\textwidth]
% {Figs/Estimates/FP_vs_Fortet_100x100superSin_est_rel_errors.png}
% }
% \subfloat[N=1000]
% {
% \label{fig:comp_test_superT_1000}
% \includegraphics[width=0.48\textwidth]
% {Figs/Estimates/FP_vs_Fortet_100x1000superSin_est_rel_errors.png}
% }
\includegraphics[width=0.99\textwidth]{Figs/Estimates/FP_vs_Fortet_100x100_x1000superSin_est_rel_errors_joint.pdf}
\caption[Estimates box-plots for super-sinusoidal regime]{Boxplots of parameter
estimates for the super-sinusoidal regime. The upper plots (A,B,C) show estimates using $N=100$
sample spikes per estimation, while the lower plots (D,E,F) use $N=1000$. The dashed line
indicates the true parameter value, while the red line inside the boxes
indicates the median of the estimates.\\
The boxes contain the central 50\% of the estimates. The bars indicate
the range of the estimates, except for outliers given by the points
outside the bars, and defined to be more than 1.5 times the
interquantile range (the height of the box) from the box.}
\label{fig:comprehensive_test_SuperSin_relerrors}
\end{center}
\end{figure}
\begin{figure}[p]
\begin{center}
% \subfloat[N=100]
% {
% \label{fig:comp_test_superT_100}
% \includegraphics[width=0.48\textwidth]
% {Figs/Estimates/FP_vs_Fortet_100x100crit_est_rel_errors.png}
% }
% \subfloat[N=1000]
% {
% \label{fig:comp_test_superT_1000}
% \includegraphics[width=0.48\textwidth]
% {Figs/Estimates/FP_vs_Fortet_100x1000crit_est_rel_errors.png}
% }
\includegraphics[width=0.99\textwidth]{Figs/Estimates/FP_vs_Fortet_100x100_x1000crit_est_rel_errors_joint.pdf}
\caption[Estimates box-plots for critical regime]{Boxplots of parameter
estimates for the critical regime.
The upper plots (A,B,C) show estimates using $N=100$ sample spikes per
estimation, while the lower plots (D,E,F) use $N=1000$. The dashed line
indicates the true parameter value, while the red line inside the boxes
indicates the median of the estimates.
\\
The boxes contain the central 50\% of the estimates. The bars indicate
the range of the estimates, except for outliers given by the points
outside the bars, and defined to be more than 1.5 times the
interquantile range (the height of the box) from the box.}  
\label{fig:comprehensive_test_crit_relerrors}
\end{center}    
\end{figure} 
\begin{figure}[p]  
\begin{center}
% \subfloat[N=100]
% {
% \label{fig:comp_test_superT_100}
% \includegraphics[width=0.48\textwidth]
% {Figs/Estimates/FP_vs_Fortet_100x100subT_est_rel_errors.png} 
% }
% \subfloat[N=1000]
% {
% \label{fig:comp_test_superT_1000}
% \includegraphics[width=0.48\textwidth]
% {Figs/Estimates/FP_vs_Fortet_100x1000subT_est_rel_errors.png}
% }
\includegraphics[width=\textwidth]{Figs/Estimates/FP_vs_Fortet_100x100_x1000subT_est_rel_errors_joint.pdf}
\caption[Estimates box-plots for sub-threshold regime]{Boxplots of parameter
estimates for the sub-threshold regime.
The upper plots (A,B,C) show estimates using $N=100$ sample spikes per
estimation, while the lower plots (D,E,F) use $N=1000$. The dashed line
indicates the true parameter value, while the red line inside the boxes
indicates the median of the estimates.
\\
The boxes contain the central 50\% of the estimates. The bars indicate
the range of the estimates, except for outliers given by the points
outside the bars, and defined to be more than 1.5 times the
interquantile range (the height of the box) from the box.}
\label{fig:comprehensive_test_SubT_relerrors}
\end{center}
\end{figure}
% #CROSS ERROS:
\begin{figure}[htp]
\begin{center}
% \subfloat[SUPRA THRESHOLD]
% {
% \label{fig:comp_test_superT}
% \includegraphics[width=0.90\textwidth]
% {Figs/Estimates/FP_vs_Fortet_100x100superT_cross_compare.png} 
% }
% \\ 
% \subfloat[CRITICAL] 
% {
% \label{fig:comp_test_critical}
% \includegraphics[width=0.90\textwidth] 
% {Figs/Estimates/FP_vs_Fortet_100x100crit_cross_compare.png}  
% }
% \\
% \subfloat[SUB THRESHOLD] 
% {
% \label{fig:comp_test_subT}
% \includegraphics[width=0.90\textwidth]
% {Figs/Estimates/FP_vs_Fortet_100x100subT_cross_compare.png}
% } 
% \\
% \subfloat[SUPER SINUSOID] 
% {
% \label{fig:comp_test_superSin} 
% \includegraphics[width=0.90\textwidth]
% {Figs/Estimates/FP_vs_Fortet_100x100superSin_cross_compare.png}
% }     
\includegraphics[width=0.99\textwidth]
{Figs/Estimates/FP_vs_Fortet_100x100_cross_compare_joint.pdf}
\caption[Fortet-based vs. Fokker-Planck-based algorithm performance
with 100 spikes]{Estimates based on samples of $N = 100$ spikes obtained from
the Fokker-Planck algorithm against the Fortet algorithm for the four different parameter regimes, with parameter values given in table
\cref{tab:regimes}, fixing $\th=1$. Each row corresponds to one regime
and one set of simulations. Each column corresponds to a parameter,
with the specific value indicated above each plot.  
A,B,C) Supra-threshold; D,E,F) Super-sinusoidal; G,H,I) 
Critical; J,K,L) Sub-threshold.}
\label{fig:comprehensive_tests_cross_comparison}
\end{center}
\end{figure}
\begin{figure}[htp]
\begin{center}
% \subfloat[SUPRA THRESHOLD]
% {
% \label{fig:comp_test_superT}
% \includegraphics[width=0.90\textwidth]
% {Figs/Estimates/FP_vs_Fortet_100x1000superT_cross_compare.png} 
% }
% \\ 
% \subfloat[CRITICAL] 
% {
% \label{fig:comp_test_critical}
% \includegraphics[width=0.90\textwidth] 
% {Figs/Estimates/FP_vs_Fortet_100x1000crit_cross_compare.png}  
% }
% \\
% \subfloat[SUB THRESHOLD] 
% {
% \label{fig:comp_test_subT}
% \includegraphics[width=0.90\textwidth]
% {Figs/Estimates/FP_vs_Fortet_100x1000subT_cross_compare.png}
% } 
% \\
% \subfloat[SUPER SINUSOID] 
% {
% \label{fig:comp_test_superSin}
% \includegraphics[width=0.90\textwidth]
% {Figs/Estimates/FP_vs_Fortet_100x1000superSin_cross_compare.png}
% }  
\includegraphics[width=0.99\textwidth]    
{Figs/Estimates/FP_vs_Fortet_100x1000_cross_compare_joint.pdf}
\caption[Fortet-based vs. Fokker-Planck-based algorithm performance
with 1000 spikes]{Estimates based on samples of $N = 1000$ spikes obtained from
the Fokker-Planck algorithm against the Fortet algorithm for the four different
parameter regimes, with parameter values given in table
\cref{tab:regimes}, fixing $\th=1$. Each row corresponds to one regime
and one set of simulations. Each column corresponds to a parameter,
with the specific value indicated above each plot.  
A,B,C) Supra-threshold; D,E,F) Super-sinusoidal; G,H,I) 
Critical; J,K,L) Sub-threshold.}
\label{fig:comprehensive_tests_cross_comparison}
\end{center}
\end{figure}
\begin{table}
\begin{center}
\subfloat[N=100]{
\begin{tabular}{c|cc|}
Regime & Fortet & Fokker-Planck \\
\hline
Sub-threshold
& 1.29 $\pm$ 0.72
& 0.52 $\pm$ 0.21
\\
Supra-threshold
& 0.83 $\pm$ 0.28
& 0.18 $\pm$ 0.20
\\
Critical
& 0.94 $\pm$ 0.42
& 0.36 $\pm$ 0.16
\\
Super-sinusoidal
& 1.36 $\pm$ 0.46
& 0.43 $\pm$ 0.17
\\
\end{tabular}
}
\subfloat[N=1000]{
\begin{tabular}{|c|cc}
Regime & Fortet & Fokker-Planck \\
\hline
Sub-threshold
& 9.68 $\pm$ 4.98
& 1.69 $\pm$ 0.91
\\
Supra-threshold
& 3.90 $\pm$ 1.05
& 0.21 $\pm$ 0.06
\\
Critical
& 10.03 $\pm$ 2.88
& 1.28 $\pm$ 0.41
\\
Super-sinusoidal
& 10.13 $\pm$ 2.24
& 1.06 $\pm$ 0.33
\\
\end{tabular}
}
\end{center}
\caption{Estimator Algorithm Computational Time}
\label{tab:walltimes}
\end{table} 

\section{The effect of $\th$}
So far we have held $\th$ constant and equal to $1$. We now investigate the
effect of varying $\th$ on the quality of estimates. To narrow the scope, we
focus on increasing $\th$ while keeping the parameters in the critical regime
such that $\a + \g/\sqrt{1+\th^2} = 1$ and $\a=0.5$. This amounts to increasing
$\g$ with $\th$. We do the estimations for four values of $\th=[1,5,10,20]$.
Similarly to the previous section, we use 100 sample spike trains per 
parameter set, with each spike train consisting of $N=1000$ ISIs.

We show box plots of the estimates for each $\th$ in
\cref{fig:comprehensive_test_thetas_relerrors}. We then directly compare the two
algorithms, Fortet vs.\ Fokker-Planck, in
\cref{fig:comprehensive_test_thetas_cross_compare}. The immediate observation is
that the Fokker-Planck algorithm fails to keep up at the higher frequencies and
consistently underestimates $\g$. The Fortet algorithm does better, but still
underestimates $\g$. In general, this underestimation of $\g$ is accompanied by
an over-estimation of $\a$. This is exacerbated at higher $\th$. We illustrate
the relation between estimates for $\a$ vs. $\g$ in
\cref{fig:comprehensive_test_thetas_alpha_vs_gamma}, where it is quite clear
that an underestimation of $\g$ is proportional to the overestimation of $\a$.
\begin{figure}[htp]
\begin{center}
% \subfloat[$\th=1$]
% {
% \includegraphics[width=0.48\textwidth]
% {Figs/Estimates/thetas_100x1000theta1_est_rel_errors.png}
% }
% \subfloat[$\th=5$]
% {
% \includegraphics[width=0.48\textwidth]
% {Figs/Estimates/thetas_100x1000theta5_est_rel_errors.png}
% }
% \\
% \subfloat[$\th=10$]
% {
% \includegraphics[width=0.48\textwidth]
% {Figs/Estimates/thetas_100x1000theta10_est_rel_errors.png}
% }
% \subfloat[$\th=20$]
% {
% \includegraphics[width=0.48\textwidth]
% {Figs/Estimates/thetas_100x1000theta20_est_rel_errors.png}
% }
\includegraphics[width=0.99\textwidth]  
{Figs/Estimates/thetas_100x1000thetas_est_rel_errors.pdf}
\caption[Estimates' box-plots for varying sinusoidal frequency]{Boxplots of
parameter estimates for varying $\th$ across $[1, 5, 10, 20]$ while holding $\g / \sqrt{1+\th^2}$ constant as to keep the parameters in the critical
regime.
A-C) $\th=1$,  D-F) $\th=5$,        
G-I) $\th=10$, J-L) $\th=20$.
\\
The boxes contain the central 50\% of the estimates. The bars indicate
the range of the estimates, except for outliers given by the points
outside the bars, and defined to be more than 1.5 times the
interquantile range (the height of the box) from the box.}  
\label{fig:comprehensive_test_thetas_relerrors}    
\end{center}
\end{figure}   
\begin{figure}[htp]
\begin{center}
% \subfloat[$\th=1$]
% {
% \includegraphics[width=0.48\textwidth]
% {Figs/Estimates/FP_vs_Fortet_thetastheta1_cross_compare.png}
% }
% \subfloat[$\th=5$]
% {
% \includegraphics[width=0.48\textwidth]
% {Figs/Estimates/FP_vs_Fortet_thetastheta5_cross_compare.png}
% }
% \\
% \subfloat[$\th=10$]
% {
% \includegraphics[width=0.48\textwidth]
% {Figs/Estimates/FP_vs_Fortet_thetastheta10_cross_compare.png}
% }
% \subfloat[$\th=20$]
% {
% \includegraphics[width=0.48\textwidth]
% {Figs/Estimates/FP_vs_Fortet_thetastheta20_cross_compare.png}
% } 
\includegraphics[width=0.99\textwidth]
{Figs/Estimates/FP_vs_Fortet_thetas_cross_compare_joint.pdf} 
\caption[Fortet-based vs Fokker-Planck-based algorithms for varying
sinusoidal frequency]{Estimates based on samples of $N = 1000$ spikes obtained
from the Fokker-Planck algorithm against the Fortet algorithm for a parameter set in the critical regime, while varying $\th$ across $[1, 5, 10, 20]$ and holding 
$\g / \sqrt{1+\th^2}$ and $\a$ constant.
A,B,C) $\th=1$; D,E,F) $\th=5$; G,H,I)
$\th=10$; J,K,L) $\th=20$. } 
\label{fig:comprehensive_test_thetas_cross_compare}
\end{center}
\end{figure}
\begin{figure}[htp]
\begin{center}
% \subfloat[$\th=1$]
% {
% \includegraphics[width=0.48\textwidth]
% {Figs/Estimates/thetavariation_100x1000theta1_alphagamma_compare.png}
% }
% \subfloat[$\th=5$]
% {
% \includegraphics[width=0.48\textwidth]
% {Figs/Estimates/thetavariation_100x1000theta5_alphagamma_compare.png}
% }
% \\
% \subfloat[$\th=10$]
% {
% \includegraphics[width=0.48\textwidth]
% {Figs/Estimates/thetavariation_100x1000theta10_alphagamma_compare.png}
% }
% \subfloat[$\th=20$] 
% {
% \includegraphics[width=0.48\textwidth]
% {Figs/Estimates/thetavariation_100x1000theta20_alphagamma_compare.png}
% }
\includegraphics[width=\textwidth]  
{Figs/Estimates/thetavariation_100x1000_alphagamma_compare_joint.pdf}
\caption[Fortet-based vs. Fokker-Planck-based algorithm performance
for varying sinusoidal frequency]{Comparison of $\aest$ vs.
$\gest$ parameter estimates while varying $\th$ across $[1, 5, 10, 20]$,  holding $\g / \sqrt{1+\th^2}$ constant as to
keep the parameters in the critical regime.
A,B,C) $\th=1$; D,E,F) $\th=5$; G,H,I)
$\th=10$; J,K,L) $\th=20$.   
}
\label{fig:comprehensive_test_thetas_alpha_vs_gamma}
\end{center}
\end{figure}


For completeness we also include the estimates' average and empirical 95\%
confidence intervals in \cref{tab:thetas_est_quantiles_1000}.
\begin{table}[htp]
\begin{center}
{\begin{tabular}{|c|ccc|} 
Parameter
& Initializer
& Fokker-Planck
& Fortet
\\ 
\hline \hline
\multicolumn{4}{|c|}{$\Omega=1$} \\[1mm]
$\alpha=0.50$
& $0.73 : [0.69, 0.75]$
& $0.52 : [0.45, 0.61]$
& $0.52 : [0.44, 0.62]$
\\
$\beta=0.30$
& $0.20 : [0.17, 0.25]$
& $0.29 : [0.24, 0.33]$
& $0.29 : [0.22, 0.34]$
\\
$\gamma=0.71$
& $0.54 : [0.44, 0.62]$
& $0.64 : [0.53, 0.75]$
& $0.68 : [0.55, 0.81]$
\\
\hline \hline
\multicolumn{4}{|c|}{$\Omega=5$} \\[1mm]
$\alpha=0.50$
& $0.88 : [0.76, 0.99]$
& $0.78 : [0.61, 0.89]$
& $0.64 : [0.39, 0.99]$
\\
$\beta=0.30$
& $0.24 : [0.17, 0.31]$
& $0.26 : [0.20, 0.34]$
& $0.27 : [0.12, 0.34]$
\\
$\gamma=2.55$
& $0.85 : [0.00, 1.65]$
& $0.92 : [0.00, 1.68]$
& $1.86 : [0.00, 3.10]$
\\
\hline \hline
\multicolumn{4}{|c|}{$\Omega=10$} \\[1mm]
$\alpha=0.50$
& $0.90 : [0.78, 0.99]$
& $0.71 : [0.52, 0.88]$
& $0.58 : [0.37, 0.86]$
\\
$\beta=0.30$
& $0.25 : [0.18, 0.33]$
& $0.26 : [0.20, 0.35]$
& $0.28 : [0.23, 0.32]$
\\
$\gamma=5.02$
& $2.82 : [0.92, 4.38]$
& $2.72 : [0.95, 3.88]$
& $4.32 : [1.20, 6.49]$
\\
\hline \hline
\multicolumn{4}{|c|}{$\Omega=20$} \\[1mm]
$\alpha=0.50$
& $0.93 : [0.76, 1.02]$
& $0.75 : [0.50, 0.92]$
& $0.62 : [0.31, 0.97]$
\\
$\beta=0.30$
& $0.27 : [0.20, 0.33]$
& $0.29 : [0.20, 0.43]$
& $0.29 : [0.25, 0.33]$
\\
$\gamma=10.01$
& $5.35 : [0.00, 12.29]$
& $3.98 : [0.00, 6.83]$
& $7.48 : [0.00, 13.96]$
\\
\hline
\end{tabular}}\\
\end{center}
\caption[Impact of sinusoidal frequency on estimators]{Averages and empirical
95\% confidence intervals of estimates for $N=1000$ spikes per train in the critical regime for varying $\th$ across
[1,5,10,20]. Note that the upper subtable corresponds to the third
subtable in \cref{tab:est_quantiles_1000}. Numbers differ slightly due to statistical
fluctuations in the simulations. }
\label{tab:thetas_est_quantiles_1000}
\end{table}


\section{Maximum Likelihood Estimation Procedure}
We detail the maximum likelihood (ML) approach to estimating the parameters
$\abg$ in the sinusoidal, noisy LIF model as discussed in the main text
\cite{Iolov2013}. We assume the reader is familiar with the notions and notation
in \cite{Iolov2013}, most pertinently the material in sec.\ 3.2.

We start by recalling the relationship between the ISI survival density, $g$,
the survival distribution $\G$ and the Fokker-Planck transition distribution,
$F$. $$\G_{\phi}(t) = F^{(\phi)}(1,t)$$ and $$g_{\phi}( t)  = -\di_t
\G_{\phi}(t). $$

Given a set of $N$ observed spikes and phase angles, $(i_n,
\phi_{n-1})_{n=1}^N$, the ML estimates are obtained by considering the
log-likelihood function, $L$,
\begin{equation}
 L(\a,\b,\g) = \sum_n  \log (
g_{\phi_{n-1}}(i_n) ) = \sum_n \log \big[ -\di_t F^{\phi_{n-1}}_{\a,\b,\g}(\xth, t) \big] \Big|_{t =
i_n}. 
\label{eq:loss_function_MLE}
\tag{A1}
\end{equation}
and maximizing it,$\abgest = \argmax L.$
Theoretically, one could stop there and start the number crunching. However, for
each evaluation of $L$ we would have to solve $N$ PDEs for $F$. That might
become computationally burdensome. Instead, in keeping with the spirit of the
paper, we choose a set of $M$ representative $\phi$'s $\{\phi_m\}$, chosen as
the same phase bin midpoints as in the main text, and approximate $\G_\phi$ as
a weighted average of $\{F^{\phi_m} \}$.



In particular given $\phi \in [\phi_m, \phi_{m+1}]$, we take a simple
linear average:
$$\G_\phi(t) \approx
\frac{\phi_{m+1} - \phi}{\phi_{m+1} - \phi_{m}} \cdot F^{\phi_m}(t)
+ 
\frac{\phi - \phi_{m}}{\phi_{m+1} - \phi_{m}} \cdot F^{\phi_{m+1}}(t).
$$
The basic idea is that if $\phi \approx \phi_m$ we use $F^{\phi_m}$ and
conversely if $\phi \approx \phi_{m+1}$ we use $F^{\phi_{m+1}}$.
Note that this assures that $0 \leq \G \leq 1$.

Finally, recall that $F$ is solved numerically, using a simple finite difference
to estimate the derivative in \cref{eq:loss_function_MLE}. If a spike time $i_n$
falls between two time slices $t_k, t_{k+1}$, $i_n \in [t_k, t_{k+1})$ then we approximate $-\di_t F(\xth, i_n)$ as $$ -\di_t F(\xth, i_n)
\approx -\frac{F(\xth, t_{k+1}) - F(\xth, t_{k})}{t_{k+1} - t_{k}}. $$


Putting it all together, the estimates are obtained as the maximizers of 
\begin{align}
\tag{A2}
\label{eq:loss_function_ML}
L(\a,\b,\g) = \sum_n  \log \Bigg(
&-\frac{\phi_{m+1} - \phi}{\phi_{m+1} - \phi_{m}} \cdot 
\frac{F_{\a,\b,\g}^{\phi_m}(\xth, t_{k+1}) - F_{\a,\b,\g}^{\phi_m}(\xth,
t_{k})}{t_{k+1} - t_{k}}
\\&-
\frac{\phi - \phi_{m}}{\phi_{m+1} - \phi_{m}} \cdot 
\frac{F_{\a,\b,\g}^{\phi_{m+1}}(\xth, t_{k+1}) - F_{\a,\b,\g}^{\phi_{m+1}}(\xth,
t_{k})}{t_{k+1} - t_{k}} 
 \Bigg)
 \Bigg|_{\substack{i_n \in [t_k, t_{k+1}] \\
 			 \phi_{n-1} \in [\phi_m, \phi_{m+1}]}} .
 			 \notag
\end{align} 
after numerically solving the PDE for $F$, eq. 9 in \cite{Iolov2013}. 

To maximize $L$ we again use the Nelder-Mead optimizer from the SciPy library,
\cite{scipy}. We also tried the gradient-based optimizers in SciPy (BFGS,
Truncated Newton, Sequential Least Squares), but they turned out slower and
provided less accurate estimates.
  
\section{ML Estimates}
We label estimates based on maximizing \cref{eq:loss_function_ML} as ML
estimates, while estimates obtained by minimizing eq.\ 17 in the main text, \cite{Iolov2013},
are called FP estimates. Of course, both of them rely on solving the
Focker-Planck equation, but we do this in order to be consistent with the main text
where estimates based on eq. 17 are labeled with 'FP'. We summarize the two
estimators in \cref{tab:estimators}. \begin{table} \begin{tabular}{ccp{4cm}}
Name & Loss function $L$ & Short description
\\
\hline FP &$\sum_{m} N_m \Big\{ \sup_{t} \left| \Gest_{\phi_m}(t) -
F^{\phi_m}_{\a,\b,\g}(\xth,
t) \right| \Big \}$ & Minimize the $sup$ of the  differences between a
numerically calculated survivor distribution and the empirically observed
survivor function
\\
ML & $-\sum_n  \log \Big( g_\phi(i_n)
 \Big)$
& Maximize the sum of the log-likelihoods of the observed ISIs
 		\\
\hline  \end{tabular}
\caption[Comparison of Maximum Likelihood vs.\ Fokker-Planck distributional
algorithm]{The two estimators we compare: the ML estimator and  the FP
estimator, which has already been discussed at length in the main text,
\cite{Iolov2013}} \label{tab:estimators} \end{table}

To compare the FP vs.\ ML estimators, we draw cross-plots between
the two in the same style as figs. 12,13 in the main text,
\cite{Iolov2013}. As always, we use 100 spike trains to obtain 100
estimates for $\abg$ using each of the two methods. In all simulations, we set
$\th = 1$.

We use the same regimes, super-threshold, super-sinusoidal, critical,
sub-threshold, with the same parameters as in the main text,

We start by using $N=100$ spikes per spike train - see
\cref{fig:MLCompare_N100}. We use 16 bins for the ML estimate and 8 for the FP.
The differences between the two methods appear minor. Neither method seems to be
consistently better than the other. In some cases, the ML estimators are
noticeably worse - especially for the super-sinusoidal regime, where the ML
method shows a bimodal distribution of the $\a$ parameter. In other cases, the
ML estimator has a smaller variance - see $\g$ in the critical regime and maybe
$\a$ in the supra-threshold regime, but that is already estimated quite
accurately.

Next, we try using the larger sample size, i.e.\ $N=1000$ spikes per spike train
- see \cref{fig:MLCompare_N1000}. Again the differences are minor and similar to
what was seen in the smaller sample size, $N=100$. Most notably the ML is now
better at estimating $\a$ in the super-sinusoidal regime, except for one rare
case.

At this point, we would conclude that the performance of the ML and FP
methods is comparable. However the ML method has the potential advantage that
one can use much smaller bins (more $\phi_m$s) no matter how big or small the
data size. This is because the FP method relies on there being a sufficient
amount of spikes in each bin in order to approximate the survivor distribution,
while the ML method does not have this requirement.

So as a final comparison we redo the estimation for $N=100$, but now
we use $M=32$ bins for the ML method while still using only $M=8$ bins for
FP. The results are in \cref{fig:MLCompare_N100_32bins}. It is evident
that the differences between using \cref{fig:MLCompare_N100} using 16 bins for
the ML method and \cref{fig:MLCompare_N100_32bins} using 32 bins are not
significant.

As a final note, we mention that the computing times for both methods are
roughly the same, modulo that one uses the same number of bins.

As such we conclude that the FP estimates and the ML estimates behave comparably.
\begin{figure}[htp]
\begin{center}
  \includegraphics[width=1\textwidth]{Figs/Estimates/MLE_N100_cross_compare_joint.pdf}
  \caption[Maximum-Likelihood vs. 'Fokker-Planck' algorithm performance with 100 spikes]
  {Comparison between the FP vs.\ ML
  estimators in the four regimes, using $N=100$ spikes per sample. We used 8
  bins for the FP and 16 bins for the ML. In all simulations $\th = 1$.}
  \label{fig:MLCompare_N100}
\end{center}
\end{figure}

\begin{figure}[htp]
\begin{center}
  \includegraphics[width=1\textwidth]{Figs/Estimates/MLE_N1000_cross_compare_joint.pdf}
  \caption[Maximum-Likelihood vs. 'Fokker-Planck' algorithm performance with
  1000 spikes] {Comparison between the FP vs.\ ML
  estimators in the four regimes, using $N=1000$ spikes per sample. We used 8
  bins for the FP and 16 bins for the ML. In all simulations $\th = 1$.}
  \label{fig:MLCompare_N1000}
\end{center}
\end{figure} 

\begin{figure}[htp]
\begin{center}
  \includegraphics[width=1\textwidth]
  {Figs/Estimates/MLE_N100_32bins_cross_compare_joint.pdf}
  \caption[Effect of bin-refinement on Maximum-Likelihood estimator]
  {Same as \cref{fig:MLCompare_N100}, but using 32
  bins for ML instead of 16. $N=100$. In all simulations $\th = 1$.}
  \label{fig:MLCompare_N100_32bins}
\end{center}
\end{figure}



\section{Discussion and Outlook}
\label{sec:sin_estimate_discusion}
We have shown two methods to estimate parameters in
\cref{eq:X_evolution_uo} from ISI data. Our methods are based
on binning the spikes into bins with representative phase shifts. We have
devised a constructive procedure to automatically initialize the methods from
the data.

Our computational results suggest that for low frequencies the Fortet algorithm
is superior for large sample sizes, especially in the super-sinusoidal regime, while the
Fokker-Planck algorithm has a comparable accuracy and a lower variance for small
sample sizes. Both algorithms find sensible estimates most of the time, although
they seem less effective in the sub-threshold regime. Their performance can
be partially attributed to the ability of the initializer algorithm to supply
good guesses for starting the optimization iterations.

The Fokker-Planck equation allows for approximate maximum likelihood
estimation. We chose an alternative loss function, though, because it
marginally appeared more robust, possibly because a numerical derivation
step is avoided. This is further investigated by simulations in the
supplementary online material. The simulations suggest that the
distribution of the maximum likelihood estimates in the super-sinusoidal
regime appears bimodal, which is not occurring for the alternative loss
function, \cref{eq:loss_function_absorbingBC}. %eq. \eqref{}. (((Eq. 17))) 

We have also made a preliminary exploration of the effect of $\th$ on the
quality of the estimates. Our results show that an increase in $\th$ makes the
parameters $\a$ and $\g$ more difficult to estimate accurately and at high
$\th$, $\g$ is underestimated, while $\a$ is over-estimated. We find that in
this scenario, the Fortet algorithm does a markedly more accurate job then the
Fokker-Planck algorithm.
 
We have assumed  the time-constant $\t$ of the leak term known. In most
experiments that is not realistic, and it would be preferable to estimate $\t$
alongside the other parameters. However, it is difficult to estimate
\cite{DitlevsenLansky212}. When we tried to estimate it together with the other
parameters, we usually obtained results which were not accurate. The obtained
estimates resulted in ISIs that very well matched the data, no worse than the
ISIs obtained from the true parameters. This leads us to believe that the
simultaneous estimation of $\t$ along with $\abg$ using only ISI data suffers
from identifiability problems. In \cite{MullowneyIyengar2008}, they were able to
estimate $\t$ in the simpler non-sinusoidally-driven model, but concluded that
adding $\t$ as an unknown dramatically reduced the accuracy in the estimation of
the other unknown parameters. The reason is that if $\t$ is also estimated from
a single dataset alongside the other parameters, then a reasonable fit can be
found to the data for various combinations of $\abg$ and $\t$, but the so-obtained
parameter values can be far from the true values.

Our model is relatively simple and ignores neurophysiological realism, such as
the fact that the spiking threshold is likely non-constant, with a
time-dependent functional form that would involve further unknown parameters. A
recent paper attempting the parameter estimation in such a model, but without
sinusoidal forcing, is \cite{Dong2011}. Furthermore, intra-cellular recordings
suggest that a hard threshold is a rough approximation and an exponential
voltage-dependent spiking intensity is more realistic \cite{Jahn2011}.

While our work has used a very specific form of the periodic forcing term,
namely $\g \sin(\th t)$,  it is clear how to apply the approach to an arbitrary
periodic function. This can be done as long as one knows where in the period of
oscillation a spike has occurred. If that is the case then the binning procedure
can be applied and the estimation methods proposed can be attempted. 


\section*{Implementation Details}
All the code used in this chapter including code to generate the figures
can be found on
\href{https://github.com/aviolov/SinSpikePython}{github @
https://github.com/aviolov/SinSpikePython}. Please see the README.md file for
for guide to the codes.

\cleardoublepage
\chapter{Optimal Control of Single Spikes}
\label{ch:spike_control}
\graphicspath{{../OptSpike/}}
<<<<<<< HEAD
=======

\usepackage{amsfonts}
\usepackage{mathrsfs}

>>>>>>> 6d1ee3c9eb52b6bed66343a6488d0f9a4ca3aef0
% DEFINITIONS:
\def \Prob 	  {{ \mathbbmtt{P}  }} %the expectation operator
\def \th 	  {{ \theta}}
\def \FI 		{{\Phi}}
\def \KL     {{ K\!L}}

\def \adot {{ \dot{\alpha} }}
\def \Udot {{ \dot{U}}}
\def \In	{{ i_n}}
\def \vt 	{{ v_{\textrm{th} } }}
\def \Ihat  {{ \hat{I}  }}
\def \p 	{{ \phi  }}

\def \G		{{ \bar{G}  }} %{{ \reflectbox{G} }}
\def \Gest  {{ \hat{\G} }}
\def \Gtilde		{{ \tilde{\G} }}
\def \D		{{ \reflectbox{D} }}
\def \dphi {{ \delta \phi }}

\def \L {{ \Lambda }}
\def \P {{ \Phi }}
\def \n {{\nu}}
\def \Fx {{ F_x}}
\def \Fxx {{ F_{xx} }}
\def \Ft {{ F_t }}
\def \xest {{ \hat{x}_t}}
\def \muncond {{ {m}_x}}
\def \mcond {{ {m}_x^c}}

\def \f {{\rho}}
\def \F {{\Phi}}
\def \Fn {{\mathcal{F}}}

\def \abg		{{\a,\b,\g }} 
\def \aest      {{ \hat{\a} }}
\def \best      {{ \hat{\b} }}
\def \gest      {{ \hat{\g} }}
\def \estabg	{{\aest, \best, \gest}}
\def \abgest	{{\estabg }}

\def \sAlg {{ \mathcal{A} }}
\def \N {{ \mathcal{N} }}
\def \Udomain {{ \mathcal{U} }}
\def \Umax {{ u_{\textrm{max}} }}
\def \umax	{{ u_{\textrm{max}} }}
\def \amin	{{ \a_{\textrm{min}} }}
\def \amax	{{ \a_{\textrm{max}} }}
\def \astar {{ \a^* }}
\def \xth	{{ x_{th} }} 
\def \yth	{{ y_{th} }}
\def \xmin	{{ x_{-} }}
\def \xmax	{{ x_{+} }}
\def \xmid  {{ x_{mid} }}

% \def \x {{ \boldsymbol{x} }}
% \def \u {{ \boldsymbol{u} }}
% \def \p {{ \boldsymbol{p} }}
% \def \q {{ \boldsymbol{Q} }}
% \def \f {{ \boldsymbol{f} }}
\def \tf {{ t_f }}
\def \tc {{ \tau_{c} }}
\def \lc {{ \lambda_{c} }}
\def \ts {{ t_{\textrm{sp} } }}
\def \tn {{ t_n }}
\def \tns {{ \{t_n \} }}
\def \T {{ T^*  }}

\def \free {{\textrm{free} }}


\def \Ttwo {{ \hat{T}_{(2)} }}
\def \Ttwol {{ \hat{T}^{\lambda}_{(2)} }}
\def \Tone {{ \hat{T}_{(1)} }}
\def \Ti {{ \hat{T}_{(i)} }}

\def \Normal {{ \mathcal{N} }}
\def \L {{ \mathcal{L} }}
\def \Lstar {{ \L^* }}
\def \H {{ \mathcal{H} }}
\def \dx {{ \delta\! x}}
\def \da {{ \delta\! \a}}
\def \df {{ \delta\! f}}

% COMP EXAM:
\def \Lonepm {{\mathbb{L}^1}}
\def \Ltwopm {{\mathbb{L}^2}}
\def \Fil 	{{\mathcal{F}}}

\def \x {{ \vec{x} }}
\def \X {{ \vec{X} }}
<<<<<<< HEAD
% parameterized densities / adjoints:
\def \ft {{ f_\th}}
\def \pt {{ p_\th}}
\def \dft {{ \delta f_\th}}
\def \wt {{ w_\th }}

\def \aopt {{\a_{opt} }}
=======

>>>>>>> 6d1ee3c9eb52b6bed66343a6488d0f9a4ca3aef0




\section{Introduction} 
Manipulation of individual neurones through electrical stimulation provides a
mean of controlling their spiking activity. In applications such as
brain-machine interfaces and neuroprosthetics, a common goal is to record from a
neurone and interpret the firing activity. Conversely, one may wish to stimulate
a cell in order that it produce a desired firing rate, either fixed or varying
in time. It is known that many cells fire sequences of spikes where the spike
times - rather than just the rate - matter to post-synaptic neurons. In this
paper we explore the possibility of controlling a neuron in a way that it
generates a sequence of spike times close to a desired sequence. 
We consider this problem in the framework of Stochastic Optimal Control and
give both a feedback solution (closed-loop), when the cell voltage is
explicitly observable, as well as an open-loop solution when only the
occurrence of spikes is observable. Importantly, we allow for an
arbitrary noise intensity.

Both theoretically and in practice related problems have been
addressed in the literature. One objective has been to
obtain either minimum or maximum interspike intervals lengths,
when the input is constrained to be between some prespecified upper
and lower bounds, see e.g. \cite{Lee1994,Lefebvre1987} for a
mathematical treatment, or \cite{Danzl2009,Nabi2012,Wu2009} in a neuronal
context. Another objective is to break a pathological synchronous
firing pattern in clusters of neurons, highly relevant for neurological
disorders such as epilepsy and Parkinson's disease,
\cite{Nabi2013a,Nabi2011}, see also \cite{Feng2007b}.

Our objective, namely targeting exact spike times in single neurons, has been
considered mainly in the open-loop context, and in either absence of or for
small noise. In \cite{Ahmadian2011} they use the Spike Response Model,
\cite{Gerstner2002}, to control output target spike trains and implement their
scheme on pyramidal cells in mouse cortical slices. Their method is numerically
efficient and allows for the simultaneous control of many neurons. However, it strongly relies on the assumption
that the noise (the value of $\beta$ in \cref{eq:X_evolution_uo} below) is small.
They only work with open-loop control.
The difference between the objective in \cite{Ahmadian2011} and what we consider
below is that they maximize the probability of spiking at some given time $t^*$, whereas we minimize the mean squared
difference between the realized spike time, $T_{sp}$ and the desired, $t^*$.
Moehlis et al. \cite{Moehlis2006} work with a Phase Response Model to obtain
spikes at exact times, while keeping the root mean square of the input to a
minimum. A similar approach is taken in \cite{Dasanayake2011}. They do not
consider noise, though, and only work with open-loop control. In
\cite{Nabi2013}, their methods are implemented on brain slices of pyramidal
neurons of rat hippocampus. While the
Phase Response Curve Model is a parsimonious and effective way to describe a
neuron's response to a stimulus, it is only valid in the supra-threshold regime,
where the unstimulated neuron is periodically spiking. Reference
\cite{Feng2003} investigates the control of the firing times of a
leaky integrate-and-fire neuron by varying the intensity of a noise process that
drives the voltage (there is no other intrinsic noise in the model). 
To obtain a given spike time, they choose the objective of minimizing the
variance of the membrane potential at the desired spike time, while forcing the
mean of the membrane potential at this time point to equal the threshold. This
provides exact solutions since it does not involve first-passage times, but has
the drawback that there is a non-negligible probability that the obtained spike
time will be far from the desired spike time.

Our objective of imposing a certain timing sequence for the spike train using an
externally applied control is obtained in both the closed- and
open-loop settings, and we specifically include the noise in the
calculations of the controls. We restrict the controlled input to stay
within pre-specified bounds, and also include a cost function to
minimize intervention.
Our main contributions are that we specifically allow for a
non-negligible noise component in the neural activity when calculating
the control, and consider both open- and closed-loop controls. The
noise component is given by a Wiener process with a noise intensity
$\b \gg 0$. 
In particular we do not restrict our attention to the autonomously-spiking,
supra-threshold regime.

The paper is structured as follows: First we describe the neuron model and
formalize the control objective. Then we describe a feedback-based solution,
which assumes that the controller has detailed access to the voltage trajectory.
Then we relax the observation assumption so that the controller only has access
to the spike times. Finally, we compare the two methods through
simulations against a simple-minded control technique which ignores
the stochastic input to the neuron.

\section{Problem Formulation}
A basic but useful model for the neural membrane potential evolution is the
noisy leaky-integrate-and-fire (LIF) model:
\begin{equation}
\begin{gathered}
dX(t) = \left(\Iext(t) - \frac{X(t)}{\tc} \right) \intd{t} + \b \intd{W},
\\
X(0) = 0,
\\
X(\ts) = \xth \implies
\begin{cases}
X(\ts^+) = 0. & 
\end{cases}
\end{gathered}
\label{eq:X_evolution_uo}
\end{equation}
Here $X(t)$ represents the membrane electric potential at time $t$, which in
absence of input decays to $0$ with a time constant of $\tc$, $dW$ is a Brownian
motion increment scaled by $\b$ and $\Iext(t)$ is the deterministic external input to the cell. Having last
spiked at time $0$, the potential hits $\xth$ at some random time $\ts$, the
potential resets to $0$ and the process starts all over again. 
We write $\ts^+$ for the limit from the right at $\ts$.
Throughout the paper the threshold is set to one, $\xth = 1$.

Suppose that we have some control over the external current such that it can be
decomposed as
\begin{equation}
\Iext(t) = \m + \a(t),
\label{eq:current_mu_alpha}
\end{equation}
where $\m$ is an uncontrollable, but constant part of the external current and
$\a(t)$ is controllable, i.e., it can be chosen to achieve some goal.
A natural goal is to attempt to control the spike time, $\ts$.
That is, how do we choose $\a(t)$ such that $\ts \approx \T$, where $\T$ is the
desired spike-time. A natural optimal control objective to achieve this is the
least squares solution
\begin{equation}
\a(\cdot) = \argmin_{\a(\cdot)} \{\, \Exp[(\ts- \T)^2] \,\},
\label{eq:OC_LS_variance}
\end{equation}
where expectation is taken with respect to the distribution of the trajectories
of $X$.

Often the control has certain constraints. The most common are simple box
constraints:
\begin{equation}
\a(t) \in [\amin, \amax] \quad \forall t.
\label{eq:bound_constraints_alpha}
\end{equation}

% \subsection{Optimal Control Formulation}
In addition to \cref{eq:OC_LS_variance}, we will also add to our objective a
running energy cost based on the control. This regularizes the problem
eliminating the subtleties of singular-control situations and it serves to avoid excessive
control as well as to avoid excessive charge building up on the cell, see
\cite{Ahmadian2011}. 

So we seek an optimal control, $\a^*$, that solves
\begin{align}
J[\a(\cdot)] =&
\Exp\left[
\e \int_0^\ts  \a^2(s) \intd{s}
+
(\ts - \T \big)^2 \right]
\label{eq:OC_LS_variance_energy}
\\ \notag
\a^*(\cdot) =& \argmin_{\a(\cdot)} J[\a(\cdot)],
\end{align}
where $\e$ measures how much weight we put on minimizing the energy cost.
If $\e = 0$, then we do not care at all about the expended energy cost.

It will often be the case that $\a$ is either a function or a value of that
function at a particular time. This function could be random, i.e., a stochastic
process, which is naturally the case when it is a function of the random
realizations of $X$. We will try to make clear below when we are refering to
$\a$ as a function and when we are merely referring to its particular value at
some particular time. For example, in \cref{eq:OC_LS_variance_energy},
$\a(\cdot)$ refers to the function, while $\a(s)$ refers to that function's
value, possibly random, at time $s$.

We will consider two control contexts -- {\sl closed-loop} and  {\sl open-loop}
control. In closed-loop control, the value of $X$ at time $t$ is observable and
can be used in determining the control. In open-loop control only the spike
times are observable. In the closed-loop context, we write the control as $$\a =
\a(x,t)$$ to indicate its dependence on $X(t)$ and to express that it will be
updated based on the time-course of $X$. In the open-loop context, we write $$\a
= \a(t)$$ to indicate that $\a(\cdot)$ is decided for all times at time 0. The
techniques used to obtain the optimal controls in the two scenarios will be
different. For the closed-loop scenario we use Dynamic Programing, see
\cite{Fleming1975}, while for the open-loop scenario we use a form of the
Maximum Principle applied to the transition density of the controlled process,
see \cite{Borzi2012}. The transition density is the probability density function
of the process being at a state $y$ at time $t$, given it was at some state $x$
at some earlier time $s$.

Crucially, we assume that the model parameters, $\m,\tc,\b$ in
\cref{eq:X_evolution_uo,eq:current_mu_alpha} are known. This is a strong assumption and will be
discussed later. 

\subsection{Parameter Regimes}
Different  parameter regimes can be envisioned given \cref{eq:X_evolution_uo},
depending on whether the noise intensity, $\b$, is
relatively high or low, and whether the external, uncontrollable bias current,
$\m$, induces spikes in the absence of noise or not. 
Spikes will occur in the absence of noise, if and only
if $ \m > 1 / \tc,$ which is called the supra-threshold regime. When $\m
\leq 1/\tc$, the regime is called the sub-threshold regime.
In addition we will investigate two values of $\b$, which we
will call high-noise and low-noise, respectively.
Example values for each parameter regime are given in \cref{tab:regimes}, and we
visualize a single path for each regime in \cref{fig:regime_path_examples}.
\begin{table}
\begin{tabular}{|l||{c}|{c}|}
\hline
\backslashbox{$\m$}{$\b$}
& $1.5$ & $0.3$ \\
\hline
$1.5 / \tc $ &Supra-threshold-High-noise & Supra-threshold-Low-noise \\
\hline
$0.1 / \tc$   &Sub-threshold-High-noise & Sub-threshold-Low-noise \\
\hline
\end{tabular}
\caption{Regime labels and example values. Note that for the numerical
experiments below, we use $\tc = 0.5$}
\label{tab:regimes}
\end{table}
%\usepackage{graphics} is needed for \includegraphics
\begin{figure}[htp]
\begin{center}
  \includegraphics[width=0.99\textwidth]{Figs/PathSimulator/path_T=14_combined.pdf}
  \caption[labelInTOC]{Example Trajectories from \cref{eq:X_evolution_uo}
  using the parameter values from \cref{tab:regimes}. 
  A) Supra-Threshold-Low-Noise 
  B) Supra-Threshold-High-Noise 
  C) Sub-Threshold-Low-Noise
  D) Sub-Threshold-High-Noise. 
  Note the multiple crossings very close together in the high-noise regimes
  in B and D. }
  \label{fig:regime_path_examples}
\end{center}
\end{figure}

\section{Closed-Loop Solution - Dynamic Programing}
We now detail the Dynamic Programing approach to obtaining an optimal feedback
control. In closed-loop, the controller can be continuously updated depending on
the realization of the stochastic process, $X$.


Given a time $t$ and a value $X(t) = x$ of the voltage, let $(\ts - t)$ be
the unknown and remaining time to spike. Note that $\ts$ is a random variable.
Given arbitrary $t, x$, our remaining-cost objective, $J[\a(\cdot);
x,t]$, will be
\begin{equation}
J[\a(\cdot); x,t]  =
\Exp\left[
\big((\ts-t) - (\T-t) \big)^2  
+
\e \int_t^\ts  \a^2(s) \intd{s} 
\,\Big|\, X(t) = x
\right].
\label{eq:OC_LS_cost_to_go}
\end{equation}
That is, if time $t$ has elapsed without a spike, we now want to minimize
the difference between $(\ts - t)$ and $(\T-t)$, given the current state $x$.
The mean in \cref{eq:OC_LS_cost_to_go} is taken over the distribution of
hitting times, $\ts$, conditional on $X(t)$ or equivalently over the
distribution of forward trajectories of $X$ starting at $x$ and ending at the
threshold. 
Recall that in the closed-loop scenario, we assume that
the value of $X(t)$ is known to the controller. A similar problem is discussed
analytically at length in the book on optimal control by Whittle,
\cite{Whittle1996}, although there is no discussion there of the numerics
required to solve it. 

\subsection{Hamilton-Jacobi-Bellman equation}
The Hamilton-Jacobi-Bellman (HJB) equation, see \cite{Fleming1975,Whittle1996}
or the articles \cite{Danzl2009,Nabi2013a} in a neuroscience context, associated
with the optimal control for \cref{eq:OC_LS_cost_to_go} is obtained as follows.
We introduce the value function, $\v(x,t)$, as the minimum of the
remaining-cost objective, i.e., of the cost function between the current time
$t$ and the desired spike time $\T$:
\begin{align}
\v(x,t) =&
 \min_{ \substack{\a(\cdot)_{s \geq t}\\
 	     \amin \leq \a(\cdot)\leq \amax}}
 \{J[\a(\cdot); x,t] \}
\notag
\\
=&
\min_{\substack{\a(\cdot)_{s \geq t}\\
 \amin \leq \a(\cdot)\leq \amax}}
\Exp\left[
\big((\ts-t) - (\T-t) \big)^2
+
\e \int_t^\ts  \a^2(s) \intd{s}
\,\Big|\,X(t) = x
\right].
\label{eq:OC_LS_mean_value}
\end{align}
Then $\v$ satisfies the following HJB partial differential equation (PDE):
\begin{equation}
\begin{gathered}
\di_t \v(x,t) + \tfrac{\b^2}{2} \di_x^2 \v(x,t) + \\
\min_{\a(x,t) \in [\amin,
\amax]}\Big\{ \e \a^2 (x,t)+ \left(\m + \a(x,t)-\tfrac{x}{\tc}\right) \di_x
\v(x,t) \Big\} = 0.
\label{eq:OC_LS_mean_HJB}
\end{gathered}
\end{equation}
The special feature of \cref{eq:OC_LS_mean_HJB} in contrast to a generic
parabolic PDE is that it contains an embedded optimization that depends on the
solution, $\v$. For each $x,t$ in the computational domain,
$\a$ is chosen such as to minimize $\{\e \a^2 + \left(\m + \a -
\tfrac{x}{\tc}\right) \di_x \v \}$. Here we can solve for the optimal control, $\a^*(x,t)$ analytically as:
\begin{align}
\a^*(x,t)  =& \argmin_{\a \in [\amin, \amax]}\big\{\e \a^2 + (\m + \a-\tfrac{x}{\tc}) \di_x\v\big\}
\notag
\\  
=&
\min \left(\amax, \max\left(\amin, -\frac{\di_x\v(x,t)}{2\e}\right)\right).
\label{eq:OC_LS_variance_energy_quadratic_control}
\end{align}
% With this the HJB PDE becomes
% \begin{equation}
% \di_t \v(x,t) + \tfrac{\b^2}{2} \di_x^2\v+
% \e (\astar(x,t))^2 + (\m + \astar(x,t)-\tfrac{x}{\tc}) \di_x\v
% = 0.
% \label{eq:OC_LS_mean_HJB_bounded_control}
% \end{equation}

We need to consider boundary conditions (BCs) for $\v$. If $X(t) = \xth$ then we
have a spike now and $\ts = t$. Thus $$ \v(\xth,t) = (t-\T)^2.$$ At the
threshold, the value function equals the squared difference between the desired
spike time and the realized one. 

For large, negative values of $x$, we assume that $w$ is not 
significantly affected by the change in $x$, i.e., that $$ \partial_x
w(x_{-}, t) = 0 $$ for some lower boundary $x_{-}$. Such a boundary
condition will be justified if we choose $\xmin$ such that the probability for the
process to take values smaller than $x_{-}$ is small. For example, we can take $\xmin$ to be two
standard deviations below the mean of the stationary distribution of the
maximally inhibited process, i.e., setting $\alpha=\amin$ in
\cref{eq:current_mu_alpha}. That is, we set $\xmin = \tc(\mu + \amin) - 2 \beta
/ \sqrt{\tc/2}$. We further enforce that $\xmin \leq -0.5$. 

Note that Dynamic Programing and the HJB equation work backwards. Thus the
evolution of the value function proceeds from the future to the past and we need
some Terminal Condition (TCs) at some point in the future, possibly infinity,
from which to start incrementing $\v$ using the dynamics and the BCs. To
determine TCs for $\v$, our idea is simple: if we reach $\T$ without having
spiked we apply maximum control in the positive direction, i.e., 
$$t>\T \implies \a(t) = \amax.$$ Thus: \begin{equation}\v(x,\T) = \Exp
\Big[\trem^2 \,\Big|\, X(\T) = x, \a(t) = \amax \Big].
\label{eq:valuefun_TCs}
\end{equation}
Note that we are making an approximation here
- we are ignoring the energy term, $\eps u^2$, in the objective for
$t>\T$. Naturally, this approximation is ever more accurate for $\eps \ll 1$.
We discuss in more detail the validity of the approximation in
\cref{sec:effect_of_eps}.
Alternatively, we could impose this approximate terminal condition at some $t^+
> \T$.

We will see the quantity on the right hand side of \cref{eq:valuefun_TCs}
repeatedly so we will give it a special name.
\begin{equation}
\Ttwo(x) := \Exp \Big[\trem^2 \,\Big|\, X(\T) = x, \a(t) = \amax \Big].
\label{eq:Trem_squared}
\end{equation}
$\Ttwo$ is the second moment of the remaining time to reach the
threshold starting at $X(\T) = x$ and applying $\amax$ throughout. This quantity
can be found easily for all $x$ in the domain by solving a stationary backward Kolmogorov
equation. This is an extension to the calculation of the first moment of an
exit-time as seen in textbooks such as \cite{Jacobs} and we give the
details in \ref{sec:valuefun_TCs}.

Thus, we restate the HJB equation in its fully specified form:
\begin{equation}
\begin{gathered}
\di_t \v(x,t) + \tfrac{\b^2}{2} \di_x^2\v(x,t)+
\e \a^2(x,t) + \big(\m + \a(x,t) -\tfrac{x}{\tc}\big)\cdot \di_x\v(x,t)
= 0,
\\
\a (x,t) = \min \left(\amax, \max\left(\amin, -\frac{\di_x \v(x,t)}{2\e}\right)
\right).
\\
\begin{cases}
\v(\xth,t) = (t-\T)^2  \quad & \textrm{upper BC}
\\
\di_x \v(\xmin, t)  = 0  \quad &\textrm{lower BC}
\\
\v(x,\T)  =
% \Exp \Big[\trem^2 \,\Big|\, X_\T = x, \a(t) = \amax \Big]
% \Exp[\t^2|x, \amax]  \quad
\Ttwo(x)
\,& \textrm{TC}
\end{cases}
\end{gathered}
\label{eq:OC_LS_HJB_full}
\end{equation}
We are solving $\v(x,t)$ over the domain $[\xmin, \xth] \times [0,\T]$.

\subsection{The numerical method for the HJB equation}
\label{sec:hjb_numerix}
We now have a PDE for $\v$ and an algorithm for computing all the BCs and TCs of
this PDE. It is time to discuss the numerical method for solving
\cref{eq:OC_LS_HJB_full}. Since it is one-dimensional in space, it is straight
forward to apply the standard centred finite difference using the
Crank-Nicholson scheme to step in time, see ch.\ 19.2 in \cite{Press1992}. To
resolve the non-linearity, $(\di_x\v)^2$, in the PDE, we treat it as a mixed
implicit-explicit term $$(\di_x \v(x,t_k))^2 = \di_x \v(x,t_k) \cdot \di_x
\v(x,t_k) \approx \underbrace{\di_x \v(x,t_k)}_{\textrm{implicit}} \cdot
\underbrace{\di_x\v(x,t_{k+1})}_{\textrm{explicit}}.$$ Note that the implicit
term is in the previous time $t_k$ instead of, as is conventional, the next
$t_{k+1}$, because we are solving for $\v$ {\sl backwards} in time from
$t_{k+1}$ to $t_k$ down to  $t_0 = 0$.

The numerical scheme is implemented in Python, using the Scipy/Numpy library,
\cite{Scipy}. For the discretization in time and space, we choose $\Delta x$ and
$\Delta t$ in relation to the parameter regimes. In particular, we take $\Delta
x$ to be less than $\b$ divided by the largest possible absolute value of $(\m +
\a - x/\tc)$ for $\a \in [\amin, \amax], x \in [\xmin, \xth]$; in our examples
this always equals $\mu + \amax - \xmin/\tc$. This is an attempt to ensure
that numerically we are in the diffusion-dominated regime. We set $\Delta t$ to
be twice the value of $\Delta x$ divided by the largest possible absolute
value of $(\m + \a - x/\tc)$. For example in the Supra-Threshold Low-Noise regime, we
have $\Delta x = 0.0043$ and $\Delta t = 0.0013$. In all regimes, our
discretization results in point-wise convergence of up to at least three
significant digits for the value function. We consider this to be a sufficiently
fine discretization. 

We next demonstrate solutions for the value function and the associated
control law for a parameter set from each regime in \cref{tab:regimes}.

\subsection{Solutions of the HJB equation} 
We set $\T=1.5$ and $\eps = 0.001$, so that the primary focus is the accurate
spiking and energy minimization considerations are secondary.

In \cref{fig:HJB_4regimes_value_surf} we show the computed value
function, $\v(x,t)$, the solution to \cref{eq:OC_LS_HJB_full}, for each of the
parameter regimes; while in \cref{fig:HJB__4regimes_control_surf}, we show the
corresponding surface plots of $\a(x,t)$. 

% THE CUTS:
% \begin{figure}[htp]
% \begin{center}
%   \includegraphics[width=\textwidth]{Figs/HJB/Regimes_vc_cuts.pdf}
%   \caption[labelInTOC]{Snapshots of the value function, $\v$, on the left and on
%   the optimal control, $\a$, on the right for $t$ fixed, at the start (blue),
%   mid-point (green) and end-point (red), for each of the four regimes.
%   The desired spike time is set to $\T=1.5$ assuming the previous spike was at
%   $t=0$, the energy penalty, $\eps = 0.001$. The control bounds are $\a \in
%   [-2,2]$. 
%   Note that all red value function curves go to zero at the threshold to satisfy
%   the upper BC in \cref{eq:OC_LS_HJB_full}.
%   Note that the green and blue
%   curves in D) are lying on top of each other for all $x$.
%   \\
%   A,B) Supra-Threshold-Low-Noise
%   C,D) Supra-Threshold-High-Noise
%   E,F) Sub-Threshold-Low-Noise
%   G,H) Sub-Threshold-High-Noise
%   }
% \label{fig:HJB_4regimes_value_control_cuts}
% \end{center}
% \end{figure}
% THE SURFS:
\begin{figure}[h!]
\begin{center}
\includegraphics[width=\textwidth]{Figs/HJB/Regimes_valuesurf.pdf}
\caption{The numerical solution for the value function $\v(x,t)$ HJB PDE,
\cref{eq:OC_LS_HJB_full}, for the four different parameter regimes.
The desired spike time is set to $\T=1.5$.
The control bounds are $\a \in [-2,2]$. 
The coloured contours for fixed $t$ plotted in thick correspond to the
beginning, middle and end times of the interval (blue: $t=0$, green:
$t=\T/2$, red: $t=\T$).
The energy penalty is $\eps = .001$. 
Note that all red curves for the value function go to zero at the threshold to
satisfy the upper BC in \cref{eq:OC_LS_HJB_full}.
\\
A) Supra-Threshold-Low-Noise
B) Supra-Threshold-High-Noise
C) Sub-Threshold-Low-Noise 
D) Sub-Threshold-High-Noise  }
\label{fig:HJB_4regimes_value_surf}
\end{center}
\end{figure}
\begin{figure}[htp]
\begin{center}
  \includegraphics[width=\textwidth]{Figs/HJB/Regimes_controlsurf.pdf}
  \caption{Surface plots for $\alpha(x,t)$ analogous to the surface
  plots for $v(x,t)$ in \cref{fig:HJB_4regimes_value_surf}. 
Note that the aspect of this figure is rotated
relative to the surface plots of $w(x,t)$ in \cref{fig:HJB_4regimes_value_surf}
because the different surfaces are best seen from different angles.}
  \label{fig:HJB__4regimes_control_surf}
\end{center}
\end{figure}    

Let us discuss the most salient features in
\cref{fig:HJB_4regimes_value_surf,fig:HJB__4regimes_control_surf}. Recall
that lower values for the value function, $\v(x,t)$, are preferred, since we are
minimizing. We see the same general shape in all four space-time surface plots
of the value function in \cref{fig:HJB_4regimes_value_surf}. At the end of the
interval, $\v(x,\T)$ is monotonically decreasing in $x$, which is to be expected
given its terminal condition in \cref{eq:valuefun_TCs}. That is, the lower the
value of $X(\T)$ the more on average will we have to wait for the spike to
occur.
% This same basic shape represents the idea that we would like to be close to
% the threshold, $x = 1$, near the end of the interval when $t \approx \T$, but
% we would like to be far from threshold, for $t \ll \T$. Indeed, at the very
% end of the interval, when $t = \T$, the value function, $v(x,\T)$, is
% monotonically decreasing in $x$, because the lower the value of $X_\T$ the
% more time on average will it take us to spike and the more delay will we have
% from our desired spike time.
As we go back in time $\v$ inverts, with a clear peak near the upper, threshold
boundary. That is, for intermediate values of $t$ we have to consider the risk
of spiking too early, represented by the high value of $\v$ near the upper
boundary {\sl and} the risk of spiking too late, represented by the high values
of $\v$ at the lower boundary. As we progress even further back in time to the
beginning of the interval, the peak near the threshold rises further, since we are spiking
{\sl earlier}, while the peak at the lower end flattens, since now there is
enough time to reach threshold despite starting far from it. The full surface
plots for $\v(x,t)$ in \cref{fig:HJB_4regimes_value_surf} can be thought of as
the interpolation between these three basic phases.

% Note, that exact shape is generated by Since we have put a very small weight
% on the energy penalty, $\e = .001$, and we have a significantly high bound on
% the maximum / minimum of the applied control, the value function becomes quite
% flat for $x \ll 1$ and $t \ll \T$, since it is likely that we can apply enough
% control to avoid an early spike and still drive the system close to threshold
% at the desired time, $t = \T$
Now focus on the controls in \cref{fig:HJB__4regimes_control_surf}. Naturally,
at the end of the interval, the control takes its maximum value, $\a(x,\T) =
\amax$ for all $x$, i.e., it gives the maximum available push for the neuron to
spike. This is consistent with \cref{eq:valuefun_TCs}. As we go back in time,
however, the control $\a$ decreases for $x \ll \xth$, and it becomes negative,
i.e., inhibitory for $x \lesssim \xth$. That is intuitively consistent with the
problem objective. For $t<\T$, we want to bring $X(t)$ closer to the threshold,
but not too close, to avoid early spiking.

Now consider the differences between the individual regimes, i.e., the
effect of changing the bias and the noise intensity, $\m,\b$. All things being
equal, the effect of increasing the noise intensity, $\b$, is to lift the value
function, i.e., to make our objective worse, at the beginning of the interval
and to decrease it at the end. This is to be expected since we are attempting to
minimize the variance in the spike time and without noise there would be no
variance at all. However, at the end of the interval, noise only helps, since
now we are only interested in spiking as soon as possible, and a higher noise
intensity will tend to decrease the spike time on average. Increasing $\b$ also
has the effect of increasing the size of the boundary layer near $\xth$, where
the value function rises steeply. That is, for small $\b$ that layer is
small since the risk of spiking early is significant only close to the
threshold. For larger $\b$ this layer increases, see panels A,C, with a thin layer, vs.\ B,D, with a
thicker layer, in \cref{fig:HJB_4regimes_value_surf}. Similarly,
increasing the bias, $\mu$, tends to decrease the value function, especially at
the end since that has the unequivocal effect of preventing late spikes.

 
\section{Open-loop Stochastic Control}
When the value of $X(t)$ is unobservable, the transition density can be used to
perform the optimization. Since the transition density follows a deterministic
PDE, we apply a Maximum Principle for PDEs as a method of obtaining the optimal
control, see \cite{Borzi2012} for details on optimization with PDEs, or the
article, \cite{Annunziato2010}, for optimization with a
Fokker-Planck-type of system.

\subsection{Fokker-Planck equation for the state density evolution}
We write the transition density of $X$, conditional on no
spikes having occurred since $t=0$ as:
\begin{align*}
f(x,t) \intd{x} &= \Prob[X(t)\in \intd{x} \,\big|\, X(0) = 0,\, X(s) < \xth
\quad \forall s < t].
\end{align*}
Then $f$ satisfies a Fokker-Planck equation, see Ch.\ 7 in \cite{Jacobs},
\begin{equation}  
\begin{gathered}
\di_t f(x,t) =
				\frac{\b^2 }{2}\cdot \di^2_x f -
				\di_x \Big[ \left( \m + \a(t)- \frac{x}{\tc}\right)  \cdot f \Big].
\\
\\
\begin{array}{ll}
	& 
	\left\{ \begin{array}{lcl}
	 f(x,0) &=& \delta(x) \quad \textrm{delta function}
	\\
	(\m + \a(t)- \tfrac x \tc)f - \di_x \tfrac {\b^2}2 f] \big|_{x=\xmin} &\equiv&
	0 \quad \textrm{lower BCs at some } \xmin
	\\
	f(x,t) |_{x=\xth} &\equiv& 0 \quad \textrm{upper BCs at } \xth
\end{array} \right.
\end{array}
\label{eq:FP_pde_OU_PDF}
\end{gathered}
\end{equation}
In theory, $\xmin = -\infty$, but in the numerics below we will need to
truncate it to some finite value, exactly as in the HJB equation,
\cref{eq:OC_LS_HJB_full}.
Note that we write $\a(t)$ here instead of $\a(x,t)$ since we cannot use the
value of $X(t)$.

The $f$ dynamics can also be written as
$$
\di_t f(x,t) = - \di_x \phi(x,t)
$$
for the probability flux
$$
\phi(x,t) = (\m + \a(t) - \tfrac x \tc)f - \di_x [\tfrac {\b^2}2 f].
$$
Then the lower BC is
$$
\phi(x,t) |_{x=\xmin} \equiv 0.
$$

We will also need a short hand notation for the differential operator on the
right side of \cref{eq:FP_pde_OU_PDF}. Let
$$ \L_\a[\cdot] := \di^2_x \Big[\frac{\b^2 }{2} \cdot\Big] -
 \di_x \Big[ (\m + \a(t)- \frac{x}{\tc}) \cdot \Big],$$
 where
 $[\cdot]$ indicates the argument of the operator $\L_\a$. We will usually
 omit the subscript $\a$, but have written it now to emphasize that the
 differential operator is parametrized by the control, $\a$.

\subsection{Restating the objective in terms of the transition density}
We now derive the optimality conditions for the optimal control $\a^*$ for the
open-loop context. Recall our objective, \cref{eq:OC_LS_variance}.  
% $$
% J[\a(\cdot)] = \Exp\left[
% (\ts - \T \big)^2
% +
% \e \int_0^\ts  \a^2(s) \intd{s}
% \right].
% $$
We can write this in terms of the transition density, $f$, as:
\begin{align}
J[\a(\cdot)] =&
\int_\xmin^\xth \Ttwo(x) \cdot f(x,\T) \intd{x}
\notag
\\
&+ \int_0^\T \phi(\xth, t) (t-\T)^2 \intd{t}
\label{eq:OC_LS_variance_density}
\\
&+  \e \int_0^\ts  \a^2(t)  \cdot \left(  \int_\xmin^\xth f(x,t) \intd{x}
\right)
\intd{t}.
\notag
\end{align}
Let us explain in more detail each term on the right-hand side in
\cref{eq:OC_LS_variance_density}.

The first term, $$ \int_\xmin^\xth \Ttwo(x) \cdot f(x,\T)\intd{x}, $$ counts the
cost of trajectories which spike too late. This cost is the expected squared time-to-hit starting at $x$, with  $\a = \amax$, weighted by the
probability of $X(\T) = x$, which is just $f(x,\T)$. Recall that $\Ttwo$ is
defined in \cref{eq:Trem_squared}.

The second term, $$ \int_0^\T \phi(\xth,t) (t-\T)^2 \intd{t}, $$ counts the cost
of trajectories which spike too early, that is the squared difference between
some realized spike time, $\ts = t$, and desired spike time $\T$, weighted by
the probability of a spike at $t$ which is just the outward probability flux at
the threshold, $\phi(\xth,t)$. Recall that we assume that $\xth = 1$ throughout.
Note further that due to the homogeneous Dirichlet BC at $\xth$, $f(\xth) = 0$, the outward flux is simply $$ \phi(\xth, t) = -\frac{\b^2 }{2} \di_x f(\xth,
t).$$

Finally, the third term,
$$
\e \int_0^\ts  \a^2(t)  \cdot \left(  \int_\xmin^\xth f(x,t) \intd{x} \right)
\intd{t}, $$
is the energy cost; the inner integral, $\int_\xmin^\xth f(x,t) \intd{x}$,
takes into account that we incur an energy cost only for those trajectories
which have not yet spiked.

With that our optimal control $\a^*(\cdot)$ will naturally be found via:
$$
\a^*(\cdot) = \argmin_{\a(\cdot)} J[\a(\cdot)].
$$

\subsection{Optimizing using a Maximum Principle}
\label{sec:PDE_max_principle_for_pdf}
By now our stochastic optimal control problem modelled by SDEs
has become a deterministic optimal control problem modelled by PDEs. Our control
$\a$ influences the evolution density $f$ and via $f$, the integrals in the
objective, $J$.

The Maximum Principle for PDEs, which is an extension to the famous Pontryagin
Maximum Principle from finite dimensional systems, introduces an adjoint
variable, $p$, which solves a PDE related to the PDE satisfied by the density
$f$ and then calculates the optimal control, $\a(\cdot)$, as a functional of $f$
and $p$.

In short, the equation for the adjoint function, $p$, is
\begin{equation}
\begin{gathered}
\begin{aligned}
\di_t p =& - \Lstar[p]
\\
		=&
			- \Big[ \frac{\b^2 }{2}\cdot \di^2_x p +
			(\m + \a(t)- \frac{x}{\tc})  \cdot \di_x p \Big].
\end{aligned}
\\
\begin{cases}
	p \big|_{x=\xth} &= (t-\T)^2
	\\
	\di_x p  \big|_{x=\xmin} &= 0
	\\
	p(x,\T) &= \Ttwo(x)
\end{cases}
\label{eq:adjoint_pde_OU}
\end{gathered}
\end{equation}

and then $\a$ can be found via
\begin{equation}
\Big\{
 \e 2 \a(t)
+ p f \Big|_\xmin
- \int _\xmin^\xth p \cdot \di_x f \intd{x}
\Big\} = 0
\quad \forall t \in [0,T].
\label{eq:J_necessary_condition}
\end{equation}
Roughly speaking, \cref{eq:J_necessary_condition} corresponds to setting the
derivative of the objective $J$  with
respect to $\a$ to zero, in \cref{eq:OC_LS_variance_density} and the
adjoint state $p$ acts as a Lagrange multiplier corresponding to the constraints of
the transition density dynamics.

More practically, the quantity

\begin{equation}
\delta_\a J (t) =  \Big\{
- \int _\xmin^\xth p(x,t) \cdot \di_x f(x,t) \intd{x}
+ p(x,t) f(x,t) \Big|_\xmin
+ \e  2 \a(t)
\Big\}
\label{eq:hamiltonian_gradient}
\end{equation}
gives the direction of increase of $J$ at $\a(t)$
and we can use it as a gradient in a descent algorithm, given some initial
guess, $\a_0(t)$, for the control.

Finally, we are ready to calculate the open-loop stochastic optimal control for
the four parameter regimes. We use the same numerical method for solving the
PDEs as described in \cref{sec:hjb_numerix}, since the PDEs for $f,p$ are very
similar in structure to the PDE for $\v$ except they do not contain a nonlinear
term. The details of the simple gradient descent algorithm for obtaining the
optimal open-loop control are given in algorithm
\ref{alg:gradient_descent_4_OC}. See \cite{Annunziato2013} for a more
sophisticated descent method based on the conjugate-gradient method. For our
applications, the algorithm in \ref{alg:gradient_descent_4_OC} typically
converges in less than 10 iterations, although sometimes it can take longer. 

The optimal controls obtained using algorithm \ref{alg:gradient_descent_4_OC} for each regime are in
\cref{fig:FBK_Regimes_cs}. Recall that the optimal control, $\a^*(t)$, is open-loop and thus it is only a function of time.
% \usepackage{graphics} is needed for \includegraphics
\begin{figure}[htp]
\begin{center}
%   \includegraphics[width=\textwidth]{Figs/FP_Adjoint/Regimes_cs.pdf}
  \includegraphics[width=\textwidth]{Figs/FP_Adjoint/Regimes_cs_singleplot.pdf}
  \caption[labelInTOC]{The deterministic optimal controls for each parameter
  regime as functions of time, $t \in [0, \T]$.
  The desired spike time is set to $\T=1.5$, the energy penalty, $\eps
  = 0.001$ and the bounds are $\a \in [-2,2]$.
  The initial value of the process is always the reset, i.e. $X_t=0$ at
time $t=0$, see \cref{eq:X_evolution_uo}.
  From left-to-right, the control curves correspond to the
  Sub-Threshold-Low-Noise, Sub-Threshold-High-Noise, Supra-Threshold-High-Noise,
  Supra-Threshold-Low-Noise regimes
%   A) Supra-Threshold-Low-Noise 
%   B) Supra-Threshold-High-Noise
%   C) Sub-Threshold-Low-Noise
%   D) Sub-Threshold-High-Noise  
  }
  \label{fig:FBK_Regimes_cs}
\end{center}  
\end{figure}   

Let us discuss the most salient features of the controls calculated in
\cref{fig:FBK_Regimes_cs}. In general, the strategy is to inhibit the neuron at
the beginning of the interval, $\a(t) = \amin$ for $t \ll \T$ and to excite it
near the end, $\a(t) = \amax$ for $t \approx \T$. The smooth portion that
connects these two segments in the middle of the interval is due to the energy
penalty, $\e \int \a^2(s) \intd{s}$. The behaviour of the control in different
regimes is relatively simple to explain and we see what we would expect. In a
sense the bias current $\mu$ can be absorbed by the control and so increasing
$\mu$ amounts to decreasing $\a$ modulo its bounds. That is exactly the
difference between the Supra-Threshold vs.\ Sub-Threshold plots in
\cref{fig:FBK_Regimes_cs} when holding the noise intensity fixed, either
$\b=0.3$ or $\b=1.5$. Both for low and high noise intensity, a reduction of $\mu$
results in an increase in $\a(t)$. 
The effect of varying the noise intensity, $\b$, is more subtle and it is not
the same in the Supra vs.\ the Sub-Threshold regimes. In the
Supra-Threshold regime, reducing $\b$ reduces the need for
excitatory control towards the end, since the bias alone should put the neuron
over the threshold. Thus the main part of the control with low noise in the
Supra-Threshold regime is to stop the neuron from spiking too early. In the
Sub-Threshold regime, the situation is somewhat reversed - reducing
$\b$ obviates the need to apply an inhibitory control in the early part of the
interval and also prompts the controller to apply excitatory input earlier,
since it is the controller which is now the main drive for a spike.

% An example for . The results are in
% \cref{fig:FP_adjoint_objective_control_convergence}.
% \begin{figure}[h]
% \begin{center}
% \subfloat[$\a_k(t)$ ]
% {
% \includegraphics[width=0.48\textwidth]
% {Figs/FP_Adjoint/ExampleControlConvergence_control.png}
% }
% \subfloat[$J_k$ ]
% {
% \includegraphics[width=0.48\textwidth]
% {Figs/FP_Adjoint/ExampleControlConvergence_objective.png}
% }
% \caption[ ]{The result of an entire optimization iteration, on the left the
% iterates of the control, $\a_k(t)$, on the right the progress of the
% objective with each iteration. The final control, in purple, is inhibitory at
% the beginning of the time interval, $\a < 0$, unlike the the deterministic
% control solution.
% We see a significant
% reduction in the objective value, $J$, and thus an improvement, on the
% order of 50\%.}
% \label{fig:FP_adjoint_objective_control_convergence}
% \end{center}
% \end{figure}
%
% \begin{table}[h]
% \centering
% \begin{tabular}{lc}
% Control Law & Squared Error \\
% \hline
% Deterministic &  0.647 \\
% Stochastic &  0.336\\
% Theory (Value function) & 0.285
% \end{tabular}
% \caption{Realized performance of the different control laws and the theoretical
% expected performance of the stochastic law (last row)}
% \label{tab:realized_avg_errors_det_vs_stoch}
% \end{table}
%
% \begin{figure}[h]
% \begin{center}
% \subfloat[A]
% {
% \label{fig:controlled_traj_ex1}
% \includegraphics[width=0.33\textwidth]
% {Figs/ControlSimulator/example_controlled_trajectories_id1.png}
% }
% \subfloat[B]
% {
% \label{fig:controlled_traj_ex2}
% \includegraphics[width=0.33\textwidth]
% {Figs/ControlSimulator/example_controlled_trajectories_id4.png}
% }
% \subfloat[C]
% {
% \label{fig:controlled_traj_ex3}
% \includegraphics[width=0.33\textwidth]
% {Figs/ControlSimulator/example_controlled_trajectories_id3.png}
% }
% \caption[]{Examples for the controlled trajectories using both the deterministic
% and the stochastic control approaches. The red vertical line in the plots
% indicates the desired spike-time, $\T$. In A, B the performance of both
% control laws is essentially the same, but in C we see the advantage of the
% stochastic approach. $\tc, \b = [.75, 1.25]$.}
% \label{fig:control_trajectories_examples}
% \end{center}
% \end{figure}
% \begin{figure}[htp]
% \begin{center}
%   \includegraphics[width=.9\textwidth]{Figs/ControlSimulator/example_controlled_trajectories_hists.png}
%   \caption[labelInTOC]{Histogram of the spike timing error for the
%   deterministic (left) vs. the stochastic (right) control laws.}
%   \label{fig:error_histograms_det_vs_stoch}
% \end{center}
% \end{figure}

\section{Simulations} 
\subsection{Single-Spike Control}
\label{sec:probabilistic_numerical_test}
Having obtained the optimal controllers, both closed- and open-loop, we evaluate
their performance with simulated realizations of the voltage process,
\cref{eq:X_evolution_uo}. In particular, while they both minimize the expected
squared deviation of the spike-time from some desired spike time, we analyze
the distribution of the squared deviation and the behaviour of the
controls for different parameter regimes.

In addition to the optimal controllers, we also show the behaviour of
another control law - perhaps the most naive one - which is obtained by ignoring
the noise and the energy penalty. That is, we find a {\sl constant} value of
$\a$ which satisfies the desired boundary conditions, $x(0) = 0, x(\T) = \xth$. This naive
controller will be called 'Deterministic', since it assumes deterministic
dynamics in $X$, i.e., $\b = 0$.

For the comparison, we sample $N$ realizations of the controlled system
and apply in turn each of the three controls. Naturally, we reuse the same
realization of the underlying stochastic process, $W_t$, for each of the
three different controls.

We set $\e = 0.001$ and $\amax = 2.0$. As such the energy cost is of secondary
importance and the paramount effect on the objective is the squared difference,
$(\ts - \T)^2$, which we are trying to minimize.

The performance of the three controllers for each parameter regime is given in
\cref{tab:realized_avg_errors_det_vs_openloop_vs_stoch} and
\cref{fig:error_histograms_det_vs_openloop_vs_stoch}. Naturally, the closed-loop
achieves a lower error than the open-loop, and the naive controller fares worst.
The difference in performance mostly depends on the strength of the noise. Thus,
for low noise, the performance of the stochastic controllers, be they open- or
closed-loop is not much superior to the naive deterministic controller. On the
contrary - in the high-noise regime, using a stochastic controller gives a much
lower error on average between the desired $\T$ and the realized $\ts$.

To give a better view of the performance of the controllers as a function of the
noise intensity we plot the 'Percentage of Correct Spikes' and the 'Average
Squared Spike-Time Deviation' in \cref{fig:noise_intensity_study}.
In \cref{fig:noise_intensity_study} we also illustrate the effect of two
values of the control bounds, $\amax=2$ or $4$. Naturally with a higher $\amax$
the error is reduced and the percentage of correct spikes is increased.
As expected, the performance gets worse with increasing noise intensity. When
the noise is small, the Supra-Threshold regime behaves best, but when the noise
increases the Sub-Threshold regime provides more flexibility to correct for
large perturbations caused by the noise. 

For illustration sake, we also visualize some trajectories from the
Sub-Threshold-High-Noise regime in \cref{fig:control_trajectories_examples}, see
panels A),C),E). The most notable feature of the $\a(\cdot)$ plots in
\cref{fig:control_trajectories_examples} is that especially for $\eps=0.001$,
there is a high level of fluctuations in the closed-loop $\a$. That is to be
expected from the optimality condition in
\cref{eq:OC_LS_variance_energy_quadratic_control}. Since the control is
proportional to the derivative of the value function and we are minimizing,
roughly speaking, the control attempts to push the process, $X$, to the bottom
of the valley that is formed by the value function, $\v$. However as the
stochastic fluctuations of $X(t)$ push it randomly to either side of that
'valley' and in turn force the control to change signs to counter-act.
These fluctuations are then amplified by the division by the small parameter
$\eps$ so that the control mostly bangs up or down to its extremes
$\amin, \amax$.
 
\begin{table}[h]
\begin{center}  
\subfloat[Supra-Threshold Low-Noise]{
   \input{../OptSpike/Figs/ControlSimulator/MeanSquaredErrors__SUPT_ln.txt}}\\ 
\subfloat[Supra-Threshold High-Noise]{   
\input{../OptSpike/Figs/ControlSimulator/MeanSquaredErrors__SUPT_HN.txt}}\\
 \subfloat[Sub-Threshold Low-Noise]{
 \input{../OptSpike/Figs/ControlSimulator/MeanSquaredErrors__subt_ln.txt}}\\ 
 \subfloat[Sub-Threshold High-Noise]{
 \input{../OptSpike/Figs/ControlSimulator/MeanSquaredErrors__subt_HN.txt}}\\
\caption{
Realized and theoretical performance of the different control laws. The
empirical performance is obtained using $N=10000$ sample paths. The theoretical
performance is found using the optimal value for $J$ for the open-loop
stochastic control and the value function $\v(x=0, t =0)$ for the closed-loop
stochastic control.}
\label{tab:realized_avg_errors_det_vs_openloop_vs_stoch}   
\end{center} 
\end{table} 

\begin{figure}[h]
\begin{center}
\subfloat[Supra-Threshold Low-Noise]{
  \includegraphics[width=\textwidth]
  {Figs/ControlSimulator/Regimes_SUPT_ln_errors_hist.pdf}
}
\\
\subfloat[Supra-Threshold High-Noise]{
  \includegraphics[width=\textwidth]
  {Figs/ControlSimulator/Regimes_SUPT_HN_errors_hist.pdf}
  }
  \\
\subfloat[Sub-Threshold Low-Noise]{
  \includegraphics[width=\textwidth]
  {Figs/ControlSimulator/Regimes_subt_ln_errors_hist.pdf}
}
\\
\subfloat[Sub-Threshold High-Noise]{
  \includegraphics[width=\textwidth]
  {Figs/ControlSimulator/Regimes_subt_HN_errors_hist.pdf}
}\\
  \caption[labelInTOC]{Histogram of the spike timing error for the
  deterministic (left) vs. open-loop stochastic (centre) vs. closed-loop
  stochastic (right) control laws.
  Recall that $\ts$ is the random realized spike time and $\T$ is the target
  spike time.  
  This is for the same problem as in
  \cref{fig:control_trajectories_examples}.
  We have used $N=10000$ sample paths
  to form the statistics.}
  \label{fig:error_histograms_det_vs_openloop_vs_stoch}
\end{center}
\end{figure}

%\usepackage{graphics} is needed for \includegraphics
\begin{figure}[htp]
\begin{center}
  \includegraphics[width=\textwidth]{Figs/TrainController/SingleISI_ControlError_Stats.pdf}
  \caption[labelInTOC]{The performance of the controllers (open-loop and
  closed-loop) as a function of the noise intensity $\b$. The target spike-time
  is again $\T = 1.5$. 
  On the left, panels A,C), we show the percent of correct spikes,
  which is defined as the number of controlled spikes which occur within 10\% of
  $\T$, the target spike time. On the right, panels B,D), we show the average
  squared difference between the controlled spike and the target spike. 
  Above, in  A,B), we show the Supra-Threshold regime. 
  Below, in  C,D), we show the Sub-Threshold regime} 
  \label{fig:noise_intensity_study}
\end{center}
\end{figure}      

\begin{figure}[h]
\begin{center}
\includegraphics[width=0.99\textwidth]
{Figs/ControlSimulator/Composite_Traj.pdf} 
\caption[]{Three examples for the controlled trajectories using the
deterministic, open-loop stochastic and closed-loop stochastic control approaches.
The plots on the left, A,C,E, are obtained using a value of $\eps=0.001$,
while the plots on the right, B,D,F, use $\eps=0.1$ (for a discussion on the
effect of $\eps$ see \cref{sec:effect_of_eps}). The black vertical line in the plots indicates the
desired spike-time, $\T$. The parameter values are $\m, \tc, \b = [0.2, 0.5,  1.5]$ (Sub-Threshold
High-Noise regime), with $\T = 1.5$. 
The upper plots show the voltage evolution, $X(t)$, while the lower
plots show the applied control, $\a(t)$. Note that the optimal control, $\a(t)$, obtained from
the deterministic and from the open-loop controls are the same across all three
samples.}
\label{fig:control_trajectories_examples}   
\end{center}  
\end{figure}

% \begin{figure}[h]
% \begin{center}
% \subfloat[]
% {
% \label{fig:controlled_traj_ex1}
% \includegraphics[width=0.33\textwidth]
% {Figs/ControlSimulator/SubTHighNoise_Traj0.pdf}
% }
% \subfloat[ ]
% {
% \label{fig:controlled_traj_ex2}
% \includegraphics[width=0.33\textwidth]
% {Figs/ControlSimulator/SubTHighNoise_Traj4.pdf}
% }
% \subfloat[ ]
% {
% \label{fig:controlled_traj_ex3}
% \includegraphics[width=0.33\textwidth]
% {Figs/ControlSimulator/SubTHighNoise_Traj7.pdf}
% }
% \caption[]{Examples for the controlled trajectories using the
% deterministic, open-loop stochastic and closed-loop stochastic control
% approaches. The black vertical line in the plots indicates the desired spike-time, $\T$.
% For the parameter values are $\m, \tc, \b = [.2, .5,  1.5]$ (Sub-Threshold
% High-Noise regime). $\T = 1.5, \eps = 0.001$. The three panels from left to
% right are three different realizations of the model dynamics. On the upper
% plots, we show the voltage evolution, $X_t$, on the lower plots we show the
% applied control, $\a(t)$. Note that $\a(t)$ obtained from the
% deterministic and open-loop controls are the same for all three samples.}
% \label{fig:control_trajectories_examples}
% \end{center}
% \end{figure}

\clearpage

\subsection{Multi-Spike Control}
We now turn to the ultimate goal of our analysis, the control of spike {\sl
trains}. The major challenge when working with spike trains, that is not
present when considering intervals in isolation, is that the target spike train
might contain several closely clustered spike times, such that if
the first controlled spike time occurs with delay, there will be too little time
to hit the next target spike times and they will all be
delayed. However, we adapt the controller to the actual controlled spike
occurrences and compensate so that it is still targeting the originally posited train. Thus there should be no accumulation of errors unless
the controller is unable to catch up to an unusually high firing rate.

We proceed by generating $M=50$ realizations of $N=16$ spikes
each in an attempt to meet a prescribed spike train. We focus on two
regimes, the Supra-Threshold regime with either low or high noise. The results
for parameters from the low-noise regime are shown in
\cref{fig:targettrain_cl_lownoise,fig:targettrain_ol_lownoise} and
results for the high-noise regime are shown in
\cref{fig:targettrain_cl_highnoise,fig:targettrain_ol_highnoise}. In all cases,
the target train is obtained by a simulation of the Supra-Threshold High Noise
regime with no additional control, i.e., with $\a(\cdot) = 0$.

In each figure,
\cref{fig:targettrain_cl_lownoise,fig:targettrain_ol_lownoise,fig:targettrain_cl_highnoise,fig:targettrain_ol_highnoise}
on the top panel we show a smoothed version of the target train vs.\ an
empirical firing rate from the simulations. The red curve is a smoothed Gaussian
kernel applied at the target times where the standard deviation of the kernel is
$\sigma = 0.1$. The blue curve is the empirical firing rate equal to the number
of spikes in a small window divided by the number of realizations and divided by
the width of the window $w_b$. We have used $w_b = 0.1$. In the bottom panels of
\cref{fig:targettrain_cl_lownoise,fig:targettrain_ol_lownoise,fig:targettrain_cl_highnoise,fig:targettrain_ol_highnoise},
we show in detail $10$ out of the $M=50$ realizations.


It is immediately clear that while the evoked trains closely follow the target
train in the low-noise case,
\cref{fig:targettrain_cl_lownoise,fig:targettrain_ol_lownoise}, this is a lot
more difficult in the high-noise case,
\cref{fig:targettrain_cl_highnoise,fig:targettrain_ol_highnoise} and the
restricted control is not able to produce reliable results. A stronger
control is needed to obtain the same level of accuracy in the target. In
\cref{fig:targettrain_cl_highnoise_aplus}, we relax the bound constraints on the
control $[\amin, \amax]$ from $[-2; 2]$ to $[-4; 4]$. Then a significantly
better tracking of the target train is achieved. Thus, to control a system under
higher noise comes at the price of allowing a more powerful control signal.

The main question in this section is whether the separate optimization of each
interval independently of the others is a suitable approach when targeting a
whole spike train. In particular, we do not account for the possibility that two
target spikes occur closely together. Then it might make sense to attempt
generating the first one a little earlier, on average, to give more time to
generate the second. For example, in
\cref{fig:targettrain_cl_lownoise,fig:targettrain_ol_lownoise}, the 13th target
spike is successfully produced, but the 14th target spike occurs so soon after
that in all trials it is delayed. However, if the target interspike intervals
are within reasonable reach of the neuron, this is not a problem. 
Here, 'reasonable reach' is to be understood as any period longer than
the time it takes to reach the threshold in the case of no noise and
under maximum excitation, i.e. $\alpha (t) = \alpha_{max}$.

% TODO: (Review) Note that the main question in this section is whether the
% separate
% optimization
% of each interval independently of the others 'works' when we are working with a
% whole train. In particular, we do not account for the following possible
% occurence, if two target spikes occur closely together, then it might make sense
% to attempt generating the first one a little earlier, on average, to give more
% time to generate the second. At least visually however, we see that this is not
% necessarily a problem in
% \cref{fig:targettrain_cl_lownoise,fig:targettrain_ol_lownoise,fig:targettrain_cl_highnoise,fig:targettrain_ol_highnoise}.
% 
 
 
% %%%%%%%%%%%%%%%%%%%%%%%%%%%%%%%%%%%%%%%%%%%%%%%%%%%%%%%%%%%%%%%%%%%% %%%%%%%%%
% SUPER-THRESHOLD REGIME LOW-NOISe: %%%%%%%%%%%%%%%%%%%%%%%%%%%%%%
% %%%%%%%%%%%%%%%%%%%%%%%%%%%%%%%%%%%%%%
\begin{figure}[htp]
\begin{center}
\subfloat[Closed-Loop Controller]{
  \includegraphics[width=\textwidth]
  {Figs/TrainController/SUPT_ln_cl_Amax2_trains_sim_50.pdf}
  \label{fig:targettrain_cl_lownoise}
}\\
\subfloat[Open-Loop Controller]{
  \includegraphics[width=\textwidth]
  {Figs/TrainController/SUPT_ln_ol_Amax2_trains_sim_50.pdf}
  \label{fig:targettrain_ol_lownoise}
} 
  \caption[]{Controller performance in the Supra-Threshold Low-Noise regime,
  given $\a \in [-2,2]$.
  In (a) is depicted the closed-loop controller, in (b) the open-llop
  controller.
  In A) panels, the blue curve is the empirical firing
  rate of $M=50$ controlled trains, while the red curve is a
  smoothed-version of the target train. 
  (See text for details of how they are calculated). 
  In the B) panels, the dashed, red lines indicate the target times,
  $\t_n$, while the blue dots are the realizations of the controlled trains.  
  The upper spike train above the solid line is the target spike train.
  The parameters from the Supra-Threshold, Low-Noise regime are used, see
  \cref{tab:regimes}. 
  The target spike train was generated
  using the model itself, with parameters from the Supra-Threshold, High-Noise regime, without an applied
  control, $\a=0$.
  } 
  \label{fig:targettrain_lownoise} 
\end{center} 
\end{figure}

%%%%%%%%%%%%%%%%%%%%%%%%%%%%%%%%%%%%%%%%%%%%%%%%%%%%%%%%%%%%%%%%%%%%%
%%%%%%%%%% SUPER-THRESHOLD REGIME HIGH-NOISE:
%%%%%%%%%%%%%%%%%%%%%%%%%%%%%%% %%%%%%%%%%%%%%%%%%%%%%%%%%%%%%%%%%%%%%
\begin{figure}[htp]
\begin{center}
\subfloat[Closed-Loop Controller]{
\label{fig:targettrain_cl_highnoise}
  \includegraphics[width=\textwidth]{Figs/TrainController/SUPT_HN_cl_Amax2_trains_sim_50.pdf}
  }
  \\
\subfloat[Open-Loop controller]{
   \label{fig:targettrain_ol_highnoise} 
  \includegraphics[width=\textwidth]{Figs/TrainController/SUPT_HN_ol_Amax2_trains_sim_50.pdf}
  }
\caption[]{Controller performance in the Supra-Threshold High-Noise regime
given $\a \in [-2,2]$. 
  In (a) is depicted the closed-loop controller, in (b) the open-llop
  controller.
  In A) panels, the blue curve is the empirical firing
  rate of $M=50$ controlled trains, while the red curve is a
  smoothed-version of the target train. 
  (See text for details of how they are calculated). 
  In the B) panels below A) the dashed, red lines indicate the target times,
  $\t_n$, while the blue dots are the realizations of the controlled trains.  
  The upper spike train above the solid line is the target spike train.
  The parameters from the Supra-Threshold, Low-Noise regime are used, see \cref{tab:regimes}. 
  The target spike train was generated
  using the model itself, with parameters from the Supra-Threshold, High-Noise regime, without an applied
  control, $\a=0$.
}
  \label{fig:targettrain_highnoise}     
\end{center}
\end{figure}

\begin{figure}[htp]
\begin{center}
\subfloat[Closed-Loop Controller]{
  \label{fig:targettrain_cl_highnoise_aplus}    
  \includegraphics[width=\textwidth]{Figs/TrainController/SUPT_HN_cl_Amax4_trains_sim_50.pdf}
  }\\
\subfloat[Open-Loop Controller]{
\includegraphics[width=\textwidth]
  {Figs/TrainController/SUPT_HN_ol_Amax4_trains_sim_50.pdf}
}
 \caption[ ]{Same as \cref{fig:targettrain_highnoise}, but with the control
  constraints, $[\amin, \amax]$ increased to $[-4,4]$ instead of $[-2,2]$.}
  \label{fig:targettrain_cl_highnoise_aplus}     
\end{center}
\end{figure} 
% \clearpage

\subsection{Controlling a Biophysical Model}
To verify our method in a more biophysically realistic model, we use
the established Morris-Lecar model \cite{MorrisLecar1981}, a
Hodgkin-Huxley type of conductance-based model. To mimick an
experimental situation we will not assume that we know the model nor
the parameters, and simply use the LIF model
\eqref{eq:X_evolution_uo}--\eqref{eq:current_mu_alpha} where
parameters 
are estimated from data sampled before the control is initiated. First we estimate parameters for the LIF
model from observations of the Morris-Lecar model. Then we control the Morris-Lecar
model using the scheme derived for the estimated LIF model. Thus, the
control strategies will be conducted in a realistic experimental
setting with no specific knowledge of the underlying mechanisms,
considering data as if coming from a black-box. 


\subsubsection{The Stochastic Morris-Lecar model}
\def \Vt {{ V_s }} \def \Wt {{ W_s }} \def \Vz {{ V_z}} \def \Wz {{ W_z}}

The stochastic Morris-Lecar model including both current and channel noise is
defined as the solution to
\begin{equation}
\left\{
\begin{array}{ccl}
d\Vt &=& \frac{1}{C}\Big(-g_{Ca}m_\infty(\Vt) (\Vt-V_{C_a}) - g_K\Wt(\Vt-V_K)
\\ && 
-g_L(\Vt-V_L)+I + A(s) \Big)dt +\gamma d\tilde{B}_s,\\
d\Wt&=&\left(\alpha(\Vt)(1-\Wt) - \beta(\Vt)\Wt\right) dt  \, + \sigma(\Vt,\Wt)dB_s,
\end{array}
\right.
\label{eq:ML}
\end{equation}
where 
\begin{eqnarray*}
m_\infty(v)&=&\frac{1}{2}\left(1+\tanh\left(\frac{v-V_1}{V_2}\right)\right),\\
\alpha(v) &=& \frac{1}{2}\phi \cosh\left(\frac{v-V_3}{2V_4}\right)\left(1+\tanh\left(\frac{v-V_3}{V_4}\right)\right),\\
\beta(v) &=& \frac{1}{2}\phi \cosh\left(\frac{v-V_3}{2V_4}\right)\left(1-\tanh\left(\frac{v-V_3}{V_4}\right)\right).
\end{eqnarray*}
The processes $\tilde{B}_s$ and $B_s$ are
independent Brownian motions. The variable $\Vt$ represents the membrane
potential of the neuron at time $s$, and $\Wt$ represents the normalized
conductance of the K$^+$ current. It varies between 0 and 1, and can be
interpreted as the probability that a K$^+$ ion channel is open at
time $s$. The 
equation for the dynamics of $V_s$ contains four terms,
corresponding to Ca$^{2+}$ current, K$^+$ current, a general leak
current, and the
input current $I$. In addition it contains the externally applied control
current $A(s)$.
   
 The functions $\alpha (\cdot )$ and $\beta (\cdot )$ model the rates of opening
 and closing of the K$^+$ ion channels. The function $m_{\infty}(\cdot)$
 represents the  equilibrium value of the normalized Ca$^{2+}$ conductance for a
 given value of the membrane potential. The parameters $V_1, V_2, V_3$ and $V_4$
 are scaling  parameters; $g_{Ca}, g_K$ and $g_L$ are conductances associated
 with Ca$^{2+}$, K$^{+}$ and leak currents; $V_{Ca}, V_K$ and $V_L$ are reversal
 potentials for Ca$^{2+}$, K$^+$ and leak currents; $C$ is the membrane
 capacitance; and $\phi$ is a rate scaling parameter for the opening and closing of
 the K$^+$ ion channels.
 
The parameter $\gamma$ scales the additive current noise. Conductance
fluctuations caused by random opening and closing of ion channels leads to
multiplicative noise on the conductance equation. Function $\sigma(\Vt,\Wt)$
models this channel or conductance noise. We consider the following function
that ensures that $\Wt$ stays bounded in the unit interval  if $\sigma \leq 1$
\cite{DitlevsenGreenwood2013}: $$\sigma(\Vt, \Wt ) = \sigma \sqrt{2
\frac{\alpha(\Vt) \beta(\Vt)}{\alpha(\Vt)+ \beta(\Vt)} \Wt (1-\Wt)}. $$

Parameter values used in the simulations are
the same as those of 
\cite{RinzelErmentrout1989,TatenoPakdaman2004} for a {\sl class I}
membrane: $V_K = -84$ mV, $V_L = -60$ mV, $V_{Ca} = 120$ mV, $C = 20 \mu
F/cm^2, 
g_L = 2 \mu S/cm^2, g_{Ca} = 4, g_K = 8, 
V_1 = -1.2 mV, V_2 = 18 mV,
V_3 = 12 mV, V_4 = 17.4,\phi = 0.07, I = 40$. The noise intensities
are taken from \cite{DitlevsenSamson2014}:  
$\gamma = 1, \sigma = 0.2$. 
Trajectories are simulated with an Euler scheme with a time step of
$0.01$ ms, which is then sub-sampled every ten points to compensate
for approximation errors of the simulation scheme. Finally, we obtain a data
set with observations every $\Delta = 0.1$ ms.  An example
of a simulated trajectory is given in the top and bottom panels ($V,W$
respectively) of Figure \ref{fig:ML_estimates_data}. The peaks correspond
to spikes of the neuron. 

\subsubsection{Relating recorded voltage to a LIF model}

\def \ss {S_{sp}} \def \ts {T_{sp}} \def \yreset {y_r}
\def \yth {y_{th}} 
\def \s {\sigma} \def \b {\beta} \def \a {\alpha}
\def \tc {\tau_c} \def \meanI {\bar T}
\def \vreset {v_r}
\def \vth {v_{th}} 

Assume discrete observations of the membrane potential, here generated
by the Morris-Lecar model as described above. These have to be related to the LIF model
\eqref{eq:X_evolution_uo}--\eqref{eq:current_mu_alpha}, which 
assumes that spikes are point events. Thus, the first problem is to
partition the data into sub-threshold fluctuations and spikes. To make
the method more robust we operate with two thresholds. A lower
threshold, where fluctuations below can be well approximated by the
LIF model, is employed to identify the data used for estimation of LIF
parameters. We set this to $\vth = -22$ mV, see
Fig. \ref{fig:ML_estimates_data}, upper and middle panel. A higher threshold is set to
determine the start of the spike, a point of no return, from which the potential can only
relax by going through a spike. We set this to $-14$ mV, see
Fig. \ref{fig:ML_estimates_data}, upper panel. When implementing the control
assuming a LIF model, we attenuate the non-linear effects of being
close to spiking in the biophysical model (or real data) by treating the interval between $-22$ and $-14$ mV
as a grey zone, artificially setting the data to a constant value just
below the threshold. Finally, we need to fix the resetting, which we
set to $\vreset = -44.53$ mV.

In the Morris-Lecar model the spike is not a point event, and we have
to cut out the dynamics during spikes, where we will make no control
and just wait for the reset. The effect of a spike for this setting of
parameters is around $t_{ref} = 40$ ms, see lower panel of
Fig. \ref{fig:ML_estimates_data}, at
which point we restart. The time $t_{ref}$ between the start of the spike and
the reset is collapsed to
a point as far as the LIF model is concerned. The value $40$ is obtained from studying
the voltage traces and is specific to the spike-dynamics of the Morris-Lecar
model. In a real experimental setting it will probably be shorter,
being the sum of the spike duration and the refractory period, 
maybe 10 ms.

The LIF model \eqref{eq:X_evolution_uo}--\eqref{eq:current_mu_alpha}  is
non-dimensional and normalized, resetting at 0 and spiking at a threshold of 1.
Therefore the data have to be transformed. First we non-dimensionalize time by
the transformation $t=s/\meanI$, where $s$ is the time-scale of the
measurements, and $\meanI$ is some mean interspike-interval, here we use the
average of the target spike trains, $\meanI=177.48$ ms. Then we transform an
observation $V_s$ to $X_t =(V_s - \vreset)/(\vth - \vreset)$. On the transformed
data we estimate the parameters $\mu, \tau_c$ and $\beta$. Then we are ready to
start the control. The control $\alpha (t)$ is calculated on the transformed
data, transformed back to a control on the original scale, $A(s)=(\vth -
\vreset)\tc \alpha (t)$, and fed into the Morris-Lecar model (or the cell in
an experiment).

\subsubsection{Estimating LIF parameters from Morris-Lecar data }
\label{sec:estimaing_LIF_from_ML}


\begin{figure}
\begin{center}
  \includegraphics[width=\textwidth]{Figs/MLBox/ML_estimates_annotated.pdf} 
  \caption[labelInTOC]{Annotating (spike-segmenting) the raw Morris-Lecar data. 
  The top panel shows a trajectory of $V_s$ from the Morris-Lecar model. 
  The middle plot shows the data segments that are used to estimate the LIF
  parameters from Morris-Lecar data. 
  The bottom plot shows the stereotypical spike shape and justifies why 40 ms
  is a reasonable recovery time}
  \label{fig:ML_estimates_data} 
\end{center}
\end{figure} 
Assume $\alpha(\cdot)=0$ and discrete observations of model \eqref{eq:X_evolution_uo}--\eqref{eq:current_mu_alpha}, $X_n^{(k)}$, for $k=1,\ldots ,
K$, where $K$ is the number of interspike intervals, and $n=0,1, \ldots
, N_k$, where $N_k+1$ is the number of observations in the $k$th
interspike interval. 
The maximum likelihood estimators for the parameters $\mu, \tc$ and
$\beta^2$ in absence of a threshold are given by 
\begin{eqnarray*} 
\hat{ \mu} &=& 
\frac{ \sum_{k=1}^K \sum_{n=1}^{N_k} \left( X^{(k)}_n - e^{-\frac{\Delta} {\hat \tc}} X^{(k)}_{n-1} \right)} 
	 { N( 1-e^{-\frac {\Delta} {\hat \tc}})}
\\
e^{-\frac {\Delta}{\hat{\tc}} } &=& 
\frac { \sum_{k=1}^K \sum_{n=1}^{N_k} ( X^{(k)}_n - \hat \mu)(X^{(k)}_{n-1} - \hat \mu) }
						  { \sum_{k=1}^K \sum_{n=1}^{N_k} \left( X^{(k)}_{n-1} - \hat \mu \right)^2 } 
\\
\hat\beta^2 &=&  
\frac{ 2\sum_{k=1}^K \sum_{n=1}^{N_k}  \left( X^{(k)}_n - \hat \mu - (X^{(k)}_{n-1} -
\hat \mu) e^{-\frac {\Delta} {\hat \tc}} \right)^2 } 
	  { N (1-e^{-2\frac {\Delta} {\hat \tc}}) \hat \tc}
\end{eqnarray*}
where $N= \sum_{k=1}^K N_k$.

Given Morris-Lecar data points over the time interval from 0 to 2 sec, we get
the following estimates: $\hat \mu = 3.57, \hat \tc = 0.20$ and $\hat \beta = 0.59$. For the bounds on our 
control we set $A(s) \in [\pm 10]$, which corresponds to $\alpha(t)
\in [\pm 2.27]$.

Now we consider the more difficult case when only spike-times are
available. Estimating $\tc$ from spikes is not accurate, it is
(almost) un-identifiable. Therefore, we assume a
fixed value
of $\tc=0.11$. This is quite different from the value of $\tc$
estimated from intra-cellular recordings, so that we investigate robustness
to misspecification of this value. Thus we only estimate $\mu$ and
$\beta$, using the 
Fortet equation, for details see \cite{Ditlevsen2007}. 
The estimates are $\hat \mu = 6.17$ and $\hat \beta = 0.78$. This
corresponds to $\alpha(t)
\in [\pm 3.94]$ if $A(s) \in [\pm 10]$.

\subsubsection{Controlling the Morris-Lecar model}
\label{sec:controlling_ML}
We now have all the pieces to control the Morris-Lecar model. In the closed-loop control strategy we use the 
voltage-based estimates for the LIF model parameters, while for the
open-loop 
strategy we use the intervals-based estimates. This is a disadvantage
for the open-loop strategy, since on average it will have poorer estimates of the
real system, but provides a realistic picture of an experimental
situation. 
The closed-loop results are shown in Fig. \ref{fig:ML_controled_simulation_cl}, and the open-loop results are
shown in 
\cref{fig:ML_controled_simulation_ol}. The results are comparable to
the results of \cref{fig:targettrain_lownoise}, and show that for
the purpose of stochastic control, it is not crucial to use a
biophysical realistic model nor knowing exactly the parameters. This
is partly due to a lower value of $\tc$ in the Morris-Lecar model,
compared to what was used in the earlier simulations, and to slightly higher
values of the bounds of the control. The variance is
proportional to $\tc \beta^2$, which in the simulations are 1.125
(high noise) or 0.045 (low noise), and in the estimated Morris-Lecar
is 0.07.  There might also be some error cancelling, in the sense that
a wrong model and 
wrongly estimated parameters might be the best choice for
the model at hand, thus providing a robust control.

%%%%%%%%%%%% CL %%%%%%%%%%%%%%%%%%%%%%%%%%%%%%%
\begin{figure}[h]
\begin{center}
\subfloat[Example controlled trajectory]
{
\label{fig:ML_example_controled_traj}
\includegraphics[width=\textwidth]
{Figs/MLTrainController/MLSim_example_cl_Amax10.pdf} 
}\\
\subfloat[Raster plot for controlled simulations]
{
\label{fig:ML_raster_plot}
\includegraphics[width=\textwidth] 
{Figs/MLTrainController/MLSim_raster_cl_Amax10.pdf}
}
\caption[labelInTOC]{Closed-loop control of the Morris-Lecar
  dynamics. In (a) are plots of the trajectory and the control, in (b)
  is a raster plot of realized spikes. The upper spike train above the
solid line is the target spike train}
\label{fig:ML_controled_simulation_cl}
\end{center}
\end{figure} 
%%%%%%%%%%%% OL %%%%%%%%%%%%%%%%%%%%%%%%%%%%%%%
\begin{figure}[h] 
\begin{center}
\subfloat[Example controlled trajectory] 
{ 
\label{fig:ML_example_controled_traj}
\includegraphics[width=\textwidth]
{Figs/MLTrainController/MLSim_example_ol_Amax10_spikesonly_C20.pdf} 
}\\
\subfloat[Raster plot for controlled simulations]
{
\label{fig:ML_raster_plot}
\includegraphics[width=\textwidth]
{Figs/MLTrainController/MLSim_raster_ol_Amax10_spikesonly_C20.pdf} 
}
\caption[labelInTOC]{Open-loop control of the Morris-Lecar
  dynamics. In (a) are plots of the trajectory and the control, in (b)
  is a raster plot of realized spikes. The upper spike train above the
solid line is the target spike train}
\label{fig:ML_controled_simulation_ol}
\end{center}
\end{figure} 

\section{Effect of energy penalty}
\label{sec:effect_of_eps}
So far we have assumed a low value for the energy penalty parameter, $\eps$, in
the objective defined in \cref{eq:OC_LS_variance_energy}. Our approach has been
to be primarily concerned with the accurate spiking and that penalizing energy
is rather intended to regularize the control and avoid excessive chattering of
the control between its extreme values, than as a goal in minimizing energy in
its own right. Now, we take some time to explore the effect of a higher $\eps$
on the optimal controls. Intuitively, a higher value of $\eps$ will tend to
bring the optimal control, $\a^*$, closer to zero. We show the effect of setting
$\eps = 0.1$ in each of the four regimes for both the closed-loop control and
the open-loop control in
\cref{fig:HJB_4regimes_control_different_eps,fig:FBK_Regimes_cs_different_es}
respectively. These should be compared to
\cref{fig:HJB__4regimes_control_surf,fig:FBK_Regimes_cs}, respectively.
Also, in \cref{fig:control_trajectories_examples} on the right we
show example trajectories for the higher energy parameter value, $\eps=0.1$, in
the Sub-Threshold High-Noise regime.

Indeed, for both the closed-loop and open-loop optimal controls, increasing
$\eps$ tends to reduce the absolute value of $\a^*(t)$. In particular, for the
closed-loop control it tends to broaden the area of transition for decreasing
$x$ when the control swings from its minimal, i.e., inhibitory value, to its
maximal, i.e., excitatory value. Similarly for the open-loop control, instead of
banging from its minimal bound at the beginning of the interval to its maximal
bound at the end, the optimal control for $\e = 0.1$ tends to have a more mild
transition from a slightly inhibitory to slightly excitatory values.
  
Note however that increasing $\eps$ has an important effect on the validity of
the approximation used to form the terminal conditions for the value function,
$\v$, and the adjoint variable, $p$. We assumed that applying $\amax$ is optimal
for $t>\T$. This is indeed true in the limit $\eps \ra 0$. However, even for a
finite value of $\eps$, this may still be the optimal thing to do. Indeed, in
the original calculations, where $\eps=0.001$, see the panels on the left in
\cref{fig:HJB_4regimes_control_different_eps,fig:FBK_Regimes_cs_different_es},
the so-obtained value function {\sl implied} that $\a(x,\T) = \amax$. In other
words, our guess that $\a(x, t) = \amax$ for $t > \T$, is self-consistent with
the so-obtained value function. This is akin to confirming an ansatz. For a
higher value of $\eps$, like $\eps = 0.1$, this is no longer the case. The red
curves on the right-side of \cref{fig:HJB_4regimes_control_different_eps} are no
longer pushed up at $\a = \amax$, and as such it is no longer valid, strictly
speaking, to assume that the value function can be obtained by assuming
$\a(t)|_{t>\T} = \amax$ and then calculating the expected remaining
time-to-spike. The exact same conclusion can be drawn from the open-loop
control. There the optimal control always (almost) reaches its maximum, $\amax$
while choosing $\eps = 0.001$, this is no longer so for the higher $\eps = 0.1$.
In fact, as we mentioned after \cref{eq:valuefun_TCs}, the correct thing to do
for higher values of $\eps$ is to push the calculation interval to some $t^+ >
\T$, apply the same terminal condition at $t^+$ instead of $\T$ and then solve
backwards (either for the closed- or open-loop control). One will be guaranteed
to find some $t^+ > \T$ that works since eventually the quadratic term
$(t-\T)^2$ will dominate the energy term in the objective, which is linear in
$t$. We do not explore this further here.
% CLOSED_LOOP EPS:
\begin{figure}[htp]
\begin{center}
  \includegraphics[width=\textwidth]{Figs/HJB/RegimesHighEps_controlsurf.pdf}
  \caption[labelInTOC]{The effect of the energy parameter $\e$ on the
  closed-loop control. This is the same as \cref{fig:HJB__4regimes_control_surf}, but for
  the higher value of $\eps = 0.1$.
  The desired spike time is set to
  $\T=1.5$ and the control bounds are $\a \in [-2,2]$. 
  A,B)
   Supra-Threshold-Low-Noise 
  C,D) 
   Supra-Threshold-High-Noise 
  E,F)
   Sub-Threshold-Low-Noise 
  G,H)
   Sub-Threshold-High-Noise}
\label{fig:HJB_4regimes_control_different_eps} 
\end{center}
\end{figure}
% OPEN_LOOP EPS:
\begin{figure}[htp]
\begin{center}
  \includegraphics[width=\textwidth]{Figs/FP_Adjoint/RegimesHighEps_cs_singleplot.pdf}
  \caption[labelInTOC]{The effect of the energy parameter $\e$ on the open-loop
  control. This is the same as \cref{fig:FBK_Regimes_cs}, but with the higher
  value of $\eps =0.1$. 
  The desired spike time is set to $\T=1.5$ and the bounds are $\a \in [-2,2]$.}
    \label{fig:FBK_Regimes_cs_different_es} 
\end{center}
\end{figure}

% 
% \begin{figure}[h]
% \begin{center}
% \subfloat[]
% {
% \label{fig:controlled_traj_ex1}
% \includegraphics[width=0.33\textwidth]
% {Figs/ControlSimulator/HighEps_Traj5.pdf}
% }
% \subfloat[ ]
% {
% \label{fig:controlled_traj_ex2}
% \includegraphics[width=0.33\textwidth]
% {Figs/ControlSimulator/HighEps_Traj6.pdf}
% }
% \subfloat[ ]
% {
% \label{fig:controlled_traj_ex3}
% \includegraphics[width=0.33\textwidth]
% {Figs/ControlSimulator/HighEps_Traj0.pdf}
% }
% \caption[]{The effect of higher energy penalty, $\eps = 0.1$.
% Examples for the controlled trajectories using the deterministic,
% open-loop stochastic and closed-loop stochastic control approaches. The black vertical line in the plots indicates the desired
% spike-time, $\T$.
% The parameter values are $\m, \tc, \b = [0.2, 0.5,  1.5]$ (Sub-Threshold
% High-Noise regime). $\T = 1.5, \eps = 0.1$. The three panels from left to
% right are three different realizations of the model dynamics. On the upper
% plots, we show the voltage evolution, $X_t$, on the lower plots we show the
% applied control, $\a(t)$. Note that the optimal control, $\a(t)$, obtained from
% the deterministic and from the open-loop controls are the same across all three
% samples.}
% \label{fig:control_trajectories_examples_high_epsilon}
% \end{center}
% \end{figure}


\section{Discussion}
We have analyzed and computed the optimal control of spikes in two scenarios -
one where the underlying voltage of the neuron is observable to the controller
(closed-loop control) and one where only the spikes are observable (open-loop
control).

Naturally, the ability to control the system is most clearly affected by the
level of the noise. In a sense, the noise acts like an adversary - in our
context it has no beneficial role, except to obstruct precise spiking.
When the level of the noise is high, a stronger control is needed, and
the trade-off between accuracy (hitting the target spikes), possible
damage of the system (the limits of the control) and energy expenditure
has to be considered.

The bias current has a more helpful role at least in our examples, where a
high value of the bias tends to help precise spiking. Of course, where it helps
or hinders depends on the value of the desired spike time - a combination of a
high positive bias and a 'distant' spike time, will tend to be difficult to
control as the system will naturally tend to spike earlier than desired.
In summary, we have found that systems in the Supra-Threshold, Low-Noise regime
tend to be most accurately controlled, while the Supra-Threshold, High-Noise
regime are the least accurate. 

It should be noted that we do not require that the controlled
cell is already in the Supra-Threshold regime - the control scheme
applies in the same manner in the Sub-Threshold regime as well. It is
only required that the maximum value of the control puts the 
model in Supra-Threshold regime, implying that the control can 
always achieve Supra-Threshold behaviour. The Supra- and Sub-Threshold
distinction only applies to the intrinsic dynamics in
absence of control, and ignoring the noise.

In both contexts, open-loop and closed-loop, finding the optimal control is
computationally intensive as we numerically solve partial differential
equations. If these algorithms were to be practical in an online setting, some
more work would have to be done in order to ensure they can be computed very
quickly (in the millisecond range) or that they can be pre-computed.

Our work has stayed close to the paradigm of \cite{Ahmadian2011} in
trying to make the neuron spike at a particular time. 
Similiar to them, we have also incorporated into our objective
the minimization of total energy used to achieve our goal, which as discussed in
\cite{Ahmadian2011} is sensible given the potential damage to the cell of
accumulating charge. Furthermore, we have assumed that our control is
constrained in magnitude, which is natural given physical limitations on
equipment and safety considerations. Another possible constraint on the control
is that it is charge-balanced, meaning that its time integral is zero, $\int u
\intd{t} = 0$. Such constraints are most easily posed in the context of
deterministic spiking models like the Phase-Response Curve, see e.g. Danzl et
al. \cite{Danzl2010}. It is possible to add them to the open-loop control, since
the control is still deterministic, but it is non-trivial. It is even less
trivial when one uses Dynamic Programing. In
particular, insisting on charge-balance in our schemes will make it much more
difficult to apply the Terminal Conditions used in deriving both the closed- and
open-loop controls.

% Alternative objectives to spike-timing control in the literature have been
% formulated, most notably the idea of desynchronizing a population of neurons,
% \cite{Nabi2011}, where it is not so important when the neurons spike as long as
% they do not spike at the same time. It is  Maybe you would like to
% write that as a future application, in which case there would be many neurons in parallel to be
% controlled. Alternately, one can imagine that each neuron is firing in the
% suprathreshold regime with a bit of noise, and you wish to apply independent
% controls that makes each one fire the most Poisson-like. If this were achieved
% for each neuron in a population, it would surely have the result of
% desynchronizing them.

As points of outlook, we mention that our scheme can potentially be applied to
produce a periodic train in the face of noise with applications to various
biological pacemakers, or it can also be used to produce the opposite effect - a
random behaviour at the single cell level, for example by specifying a target
train with Poisson-like characteristics.

The novelty in our work is primarily in obtaining a stochastic control scheme,
which is not limited by the intensity of the noise in the system. In fact, by
comparison with the naive 'deterministic' controller, we have shown how wrong
the control strategy can be if we assume that the noise is negligible. Given
this scope, the numerical demands on our scheme are non-trivial, but our
experience and experiments show that these demands can be addressed.
It however remains to be seen whether they can be successfully applied in the lab
and beyond.


% \appendix
\section{Calculating the Terminal Conditions for the HJB equation}
\label{sec:valuefun_TCs}
Here we explain how to calculate the first two moments of the remaining
time-to-spike, $\Exp \big[\trem^n \,\Big|\, X(\T) = x, \a(t) = \amax
\big]$, $n=1,2$, which is needed for the terminal conditions for $\v(x,\T)$
and $p(x,\T)$ in, respectively, \cref{eq:OC_LS_HJB_full} and
\cref{eq:adjoint_pde_OU}.

Let
\begin{equation}
\begin{array}{lcl}
\Tone(x) &=& \Exp \Big[\trem \,\,\Big|\, X(\T) = x, \a(t) = \amax
\Big],
\\
\Ttwo(x) &=&
\Exp \Big[\trem^2 \,\Big|\, X(\T) = x, \a(t) = \amax \Big],
\end{array}
\end{equation}
where as before, $\ts-\T$ is the remaining time-to-spike in
\cref{eq:X_evolution_uo}, with the voltage being $X(\T) = x$ at the desired
spike-time $t=\T$ and applying the control $\a(t) = \amax$ afterwards.
Naturally, the $\Ti$'s depend on $x$ and are affected by the value of the
various parameters, $\a,\b,\tc$. We only  need $\Ttwo$ for the TCs in
\cref{eq:OC_LS_HJB_full} and \cref{eq:adjoint_pde_OU}, but $\Tone$ is needed as
an auxiliary. 
  
Analytical expressions exist for $\Ti$, e.g.\ see \cite{Inoue1995}.
However in our experience, they are awkward to work with numerically, especially
as they involve infinite alternating sums which can easily overflow in numeric
calculations. Moreover, they are not particularly convenient if you need $\Ti
(x)$ for a wide range of $x$'s as we do here, since they calculate the value
independently for each $x$. Instead we will follow and slightly extend Jacobs\footnote{Jacobs only discusses how to calculate the first moment
$\Tone$, however it is quite clear from his discussion how to proceed
recursively to obtain any higher moment $\Ti$, by using previously calculated
lower moments. This is what we do here. Moreover the relation between mean exit
times and boundary value problems is discussed in most
introductory books on stochastic processes, see \cite{Oksendal2007} for
example.}, \cite{Jacobs}, in deriving and solving a differential equation for
$\Ti(x)$.

Basically we are calculating the mean squared time to exit an interval $[\xmin,
\xth]$. Theoretically, $\xmin = -\infty$, but for our purposes we will set it to
some finite value and impose reflecting boundaries there just as we did for the
value function $\w$. This is a very good approximation for $\xmin \ll 0$, as
long as $\a = \amax >0$.

Let $B(y,t|x) = \Prob[X_0 = y| X(t) = x]$ for a fixed $x$. Then $B$ solves the
following (backward Kolmogorov) PDE
\begin{equation}
-\di_t B = (\m + \a - \frac y\tc)\di_yB - \frac {\b^2} 2 \di_y^2 B.
\label{eq:backward_Xdensity}
\end{equation}
Let $\G(t,y)$ be the survival function for $\ts$ given $X(\T) = y$,
$$\G(t,y) = \Prob[\T>t | X(\T) = y] = \Prob[X(s)|_{s \leq t} \in [\xmin,
\xth] \,\big|\, X(\T) = y].$$
Because the dynamics do not depend explicitly on time, we have
$$
\G(t,y) = \int_\xmin^\xth B(y,t|x) \intd{x}
$$
and so $\G$, being a definite integral of $B$ satisfies the same PDE:
\begin{equation}
-\di_t \G = (\m + \a - \frac y\tc) \di_y\G - \frac {\b^2} 2 \di_y^2 \G,
\label{eq:backward_SDF}
\end{equation}
with the BCs, ICs:
\begin{equation}
\begin{cases}
\G(t,\xth) = 0 & \quad (\ts = t \implies \Prob[\ts>t] = 0)
\\
\di_x \G(t,\xmin) = 0  & \quad (\textrm{lower boundary condition})
\\
\G(\T, y) = 1. & \quad (\ts \textrm{ is definitely} > \t \textrm{ for } X(\T)
\in (\xmin, \xth) )
\end{cases}
\end{equation}

If we solve the PDE for $\G$ over $t \in [0,\infty)$, then we can calculate
$\Ttwo(x)$ as:
\begin{eqnarray}
\Ttwo(x) &=& \int_0^\infty t^2 \cdot (-\di_t\G(t,x)) \intd{t}
\\
		   &=& \int_0^\infty 2t \cdot \G(t,x) \intd{t},
\end{eqnarray}
assuming it exists.

Since $\Ttwo$ is an integral over time, we can reduce the PDE in
\cref{eq:backward_SDF} for the distribution to an ordinary differential equation
(ODE) for the moments by integrating over time. Multiply both sides of
\cref{eq:backward_SDF} by $2t$, and integrate with respect to time:
\begin{equation}
\int_0^\infty 2t (-\di_t \G) \intd{t}
=
\int_0^\infty  2t \left[ (\m + \amax - \frac y\tc) \di_y\G - \frac {\b^2} 2
\di_y^2 \G\right]
\intd{t}.
\end{equation}
The left-hand side turns out to be $2\Tone$, so
\begin{equation}  
2\Tone
=(\m + \amax - \frac y\tc)   \di_y \Ttwo
- \frac {\b^2} 2
\di_y^2 \Ttwo.
\label{eq:BVP_Ttwo}
\end{equation}
Similarly we can derive an equation for $\Tone$, namely:
\begin{equation}
1
=(\m + \amax - \frac y\tc)   \di_y \Tone
- \frac {\b^2} 2
\di_y^2 \Tone.
\label{eq:BVP_Tone}
\end{equation}
In both cases, the boundary condtions will be:
\begin{equation}
\begin{cases}
\Ti(\xth) = 0 & \quad (\textrm{we are already spking})
\end{cases}
\label{eq:BVP_Ti_BCs}
\end{equation}
Equations \ref{eq:BVP_Tone} and \ref{eq:BVP_Ttwo} can be reduced to first order
differential equations and solved analytically, at least up to some definite integral,
but the analytical solution is not particularly useful or necessary for
numerical calculations. Instead, we opt to calculate the solutions to
\ref{eq:BVP_Tone} and \ref{eq:BVP_Ttwo} directly via an ODE solver, starting
from $T_i(\xth) = 0$ and integrating down for $x  \in [\xmin, \xth]$.
That is how we have obtained the terminal conditions for both $\v(x,\T)$ and
$p(x,\T)$ to apply in, respectively, \cref{eq:OC_LS_HJB_full} and
\cref{eq:adjoint_pde_OU}.

\section{Gradient Descent Algorithm for the Maximum Principle}
For completeness we give the full pseudo-code for the gradient-descent algorithm
used to compute the optimal control in the case of minimal-observation (the
non-feedback case), see Algorithm \ref{alg:gradient_descent_4_OC}. 
\begin{algorithm}
\begin{algorithmic}
\State Fix $\T, \e, \ldots$ the problem parameters
\State Fix $\{t_n\}_0^{N_t}$ a time-discretization of $[0,\T]$
\State Fix $g_{tol}$ a convergence tolerance for the gradient
\State Fix $K_{\max}, J_{\max}$ number of maximum iterations in outer, resp.
inner descent loops
\State Fix $s$ the initial step-size. 
\\ {\itshape $\#$ we use $g_{tol}=1e-5,K_{\max}=100,J_{\max}=10,s=10$}
\State $\a_1(t) \gets (\amax-\amin) \cdot t / \T + \amin$ 
\\{\itshape  $\#$ $\a_1(t) \sim$ initial guess for the control, linear
interpolate between $\amin, \amax$}
\For { $k= 1\dots K_{\max}$} 
\\ {\itshape $\#$ This is the outer loop where we descend down different
gradients}
 \State Calculate $f_k,J_{k},p_k, \delta_\a J_k$ corresponding to
	$\a_k$ from
	\cref{eq:hamiltonian_gradient,eq:adjoint_pde_OU,eq:FP_pde_OU_PDF,eq:OC_LS_variance_density}
	\State $N_{active}\gets$   Number of time nodes $t_n$, where either
	$\a_k(t_n) \neq \{\amin, \amax\}$ or $\delta_\a J_k(t_n)$ points inwards
	\If{ $|| \delta_\a J_k||_{\R^{N_{active}}} \leq g_{tol}\cdot N_{active}$}
		  \\ {\itshape  $\#$ 'Active' gradient is small enough,
		 consider converged:}
		 \State BREAK
	\EndIf
	\\ {\itshape $\#$ Find the step size, $s$, of how far to move $\a$ in the
	direction $\delta_\a J_k$:}
	\For { $j= 1\dots K_{\max}$}
	\\ {\itshape $\#$ This is the inner loop where we find how much to descend down
	the current gradient}
	\State $\a_{k,j} \gets \a_{k} - s_j \cdot \delta_\a J_k  $
	\\ {\itshape $\#$ $\a_{k,j}$ is the new control strategy to try}
	\State Calculate $f_{k,j}, J_{k,j}$ corresponding to
		$\a_{k,j}$ from	\cref{eq:FP_pde_OU_PDF,eq:OC_LS_variance_density}
	\\ {\itshape $\#$ Recall $f_{k,j}$ is a probability density resulting from the
	control $a_{k,j}$ and $J_{k,j}$ is the objective value resulting from
	$f_{k,j}$}
	\If {$J_{k,j} < J_k$}
		\\ {\itshape $\#$ We found a better (smaller) objective value}
		\State $s \gets 2 s_j$ {\itshape $\#$ try a more aggressive step in the
		next iteration}
		\State BREAK
		\EndIf
	\If {$j == J_{\max}$}
		\\ {\itshape $\#$ We exhausted the step search without finding a 
		smaller $J$, return current values}
		\State Return $J_k, \a_k$
	\EndIf
 	\\ {\itshape  $ \#$ Continue the inner looop, now try with a smaller step:}
	 	\State $s_{j+1}$ \gets $s_j / 2.$
    \EndFor  {\itshape $\quad \#$ single step loop}
	\If {$k == K_{\max}$}
		\State ERROR 'Could not converge'
	\EndIf
    \\{\itshape $\#$ Assign the new candidate for the optimal control and
    re-loop}
		\State $\a_{k+1} \gets \a_{k,j}$
\EndFor {\itshape $\quad \#$ gradient descent loop}
\State \Return $J_k, \a_k$
\end{algorithmic}
\caption{ Gradient descent algorithm for obtaining the optimal open-loop 
control}
\label{alg:gradient_descent_4_OC}
\end{algorithm}



\cleardoublepage
\chapter{Optimal Design for Estimation in SDEs}
\label{ch:optimal_design}
\graphicspath{{../OptEstimate/}}
<<<<<<< HEAD
=======

\usepackage{amsfonts}
\usepackage{mathrsfs}

>>>>>>> 6d1ee3c9eb52b6bed66343a6488d0f9a4ca3aef0
% DEFINITIONS:
\def \Prob 	  {{ \mathbbmtt{P}  }} %the expectation operator
\def \th 	  {{ \theta}}
\def \FI 		{{\Phi}}
\def \KL     {{ K\!L}}

\def \adot {{ \dot{\alpha} }}
\def \Udot {{ \dot{U}}}
\def \In	{{ i_n}}
\def \vt 	{{ v_{\textrm{th} } }}
\def \Ihat  {{ \hat{I}  }}
\def \p 	{{ \phi  }}

\def \G		{{ \bar{G}  }} %{{ \reflectbox{G} }}
\def \Gest  {{ \hat{\G} }}
\def \Gtilde		{{ \tilde{\G} }}
\def \D		{{ \reflectbox{D} }}
\def \dphi {{ \delta \phi }}

\def \L {{ \Lambda }}
\def \P {{ \Phi }}
\def \n {{\nu}}
\def \Fx {{ F_x}}
\def \Fxx {{ F_{xx} }}
\def \Ft {{ F_t }}
\def \xest {{ \hat{x}_t}}
\def \muncond {{ {m}_x}}
\def \mcond {{ {m}_x^c}}

\def \f {{\rho}}
\def \F {{\Phi}}
\def \Fn {{\mathcal{F}}}

\def \abg		{{\a,\b,\g }} 
\def \aest      {{ \hat{\a} }}
\def \best      {{ \hat{\b} }}
\def \gest      {{ \hat{\g} }}
\def \estabg	{{\aest, \best, \gest}}
\def \abgest	{{\estabg }}

\def \sAlg {{ \mathcal{A} }}
\def \N {{ \mathcal{N} }}
\def \Udomain {{ \mathcal{U} }}
\def \Umax {{ u_{\textrm{max}} }}
\def \umax	{{ u_{\textrm{max}} }}
\def \amin	{{ \a_{\textrm{min}} }}
\def \amax	{{ \a_{\textrm{max}} }}
\def \astar {{ \a^* }}
\def \xth	{{ x_{th} }} 
\def \yth	{{ y_{th} }}
\def \xmin	{{ x_{-} }}
\def \xmax	{{ x_{+} }}
\def \xmid  {{ x_{mid} }}

% \def \x {{ \boldsymbol{x} }}
% \def \u {{ \boldsymbol{u} }}
% \def \p {{ \boldsymbol{p} }}
% \def \q {{ \boldsymbol{Q} }}
% \def \f {{ \boldsymbol{f} }}
\def \tf {{ t_f }}
\def \tc {{ \tau_{c} }}
\def \lc {{ \lambda_{c} }}
\def \ts {{ t_{\textrm{sp} } }}
\def \tn {{ t_n }}
\def \tns {{ \{t_n \} }}
\def \T {{ T^*  }}

\def \free {{\textrm{free} }}


\def \Ttwo {{ \hat{T}_{(2)} }}
\def \Ttwol {{ \hat{T}^{\lambda}_{(2)} }}
\def \Tone {{ \hat{T}_{(1)} }}
\def \Ti {{ \hat{T}_{(i)} }}

\def \Normal {{ \mathcal{N} }}
\def \L {{ \mathcal{L} }}
\def \Lstar {{ \L^* }}
\def \H {{ \mathcal{H} }}
\def \dx {{ \delta\! x}}
\def \da {{ \delta\! \a}}
\def \df {{ \delta\! f}}

% COMP EXAM:
\def \Lonepm {{\mathbb{L}^1}}
\def \Ltwopm {{\mathbb{L}^2}}
\def \Fil 	{{\mathcal{F}}}

\def \x {{ \vec{x} }}
\def \X {{ \vec{X} }}
<<<<<<< HEAD
% parameterized densities / adjoints:
\def \ft {{ f_\th}}
\def \pt {{ p_\th}}
\def \dft {{ \delta f_\th}}
\def \wt {{ w_\th }}

\def \aopt {{\a_{opt} }}
=======

>>>>>>> 6d1ee3c9eb52b6bed66343a6488d0f9a4ca3aef0


% \def\input@path{{../OptEstimate}} 
%TODO: change to consistent notation:
% TODO: Fix the .txt includes
 
\section{Thesis Context}
 In the last of the three main chapters of the thesis, we discuss the problem of
 stimulating a neuron model system in order to obtain higher fidelity model
 parameter estimates. We use the Mutual Information as the formal criterion for optimization. Since the
 parameters of the system are not known, we demonstrate why it is impossible to derive a HJB equation for the
 optimization problem in the full-observation case. However, one can still
 derive a variational, Maximum-Principle-based iterative schema to obtain the
 optimal stimulus. We then use this to find the optimal stimulation and perform
 several simulated estimations to demonstrate the superiority of the
 MI-optimal control over common-sense alternatives.
 
 This chapter can be seen to combine many of the features of the previous two
 chapters - we are again trying to control spike-times as in the second chapter,
 but with the ultimate goal of estimating parameters from the spike observations
 as in the first. Mathemtaically the greatest difference between the
 optimization problem in this chapter and that in the second is that we no
 longer assume a definite single set of model parameters, but instead must work
 with {\sl distributions} over the parameters. This renders the problem more
 difficult and in particular makes the adjoint derivations more complex. Instead
 of a single-pair of forward-backward PDEs, we now must deal with a ensemble of
 loosely coupled forward-backward PDEs.
  
    
\section{Introduction}
We consider the problem of optimal design for estimation of parameters in a
stochastic differential equation (SDE) given only observations of hitting times
from the SDE. We assume that the 'controller' has influence over some part of
the parameters of the SDE and the objective is to obtain observations that are
most informative in both formal and informal senses. That is, we choose the
control as the solution to a well-posed optimization problem, but our ultimate
goal is to obtain more accurate and more precise estimates of the unknown
parameter(s) given the observed hitting times.

We espouse a Bayesian perspective and proceed by assuming a prior
distribution of the unknown parameters. Then we seek to maximize the Mutual
Information (MI) between the posterior of the unknown parameters given the
observations and the distribution of the hitting times. Note that even for a
fixed and known parameter set the resulting observations, i.e.\ the hitting
times, are still random, due to the inherent stochasticity of the SDE. 

Using Mutual Information for the selection of 'maximally informative'
experiments has been advocated by several recent lines of research, for example,
in experimental psychology, \cite{Cavagnaro2010,Myung2013}, computational
neuroscience \cite{Paninski2006a,Paninski2005,Lewi2009} and quantum physics,
\cite{Granade2012}.
 
An alternative approach is to maximize the Expected Fisher Information of the
experiment, see \cite{Hooker2015}, which is especially effective if the Fisher
Information of the parameter estimators is available as an analytical formula.
That is the case when one observes full trajectories from the diffusion process,
as in \cite{Hooker2015}, but not in the more limited hitting-times observations
context.
 
The Mutual Information measures the expected reduction in uncertainty in the
parameter value given an observation, under a fixed experimental stimulation.
Thus the stimulation that maximizes the Mutual Information should extract the
maximum information about the model's parameters, on average. Using Mutual
Information for estimating statistical quantities is formally justified by
\cite{Paninski2005}, where proofs are provided showing that given fairly weak
modeling conditions, using a stimulation that iteratively maximizes the mutual
information between future observations and the parameters of interest leads to
consistent and efficient parameter estimates. However, while theoretically
sound, optimizing or even merely computing the Mutual Information may be
computationally prohibitive. For our problem of interest though, that is not a
primary issue.

We should explicitly single out a very recent paper on estimation
parameters in a quantum system,  
\cite{Granade2012}, where the authors describe a strategy to estimate
the parameters of the Hamiltonian of a quantum experiment. They use a simple
filtering technique originally suggested by \cite{Liu2001} to maintain and
update the prior distribution of the unknown parameters. In principle, their
approach is quite similar to ours, the main difference is the nature of the
observations, which in our case are the hitting times.

As posed, our optimization problem reduces to a slightly non-standard Partial
Differential Equation (PDE) optimization problem, in particular a Fokker-Planck
optimization problem. The field of PDE optimization is vast and we can just
mention one of many good references, \cite{Borzi2012}. There have recently
appeared a series of articles on Fokker-Planck based PDE-optimization, e.g.
\cite{Annunziato2010,Annunziato2014}. None of these however deal with the
boundary term that arises out of the hitting-time-based objective. We
should also mention our own work, \cite{Iolov2014a}, where we again solve a
hitting-time inspired optimization problem. The fundamental difference here is
that the uncertainty in the model parameters renders the problem more
complicated as it necessitates us to solve a system of (loosely-) coupled PDEs
rather than just one as in \cite{Iolov2014a}.

Our paper is also part of the general literature on estimating parameters of
diffusion processes from hitting-times observations only. Partly due to the
applications of this problem in theoretical neuroscience a lot has been written
on the subject for example in
\cite{Ditlevsen2007,MullowneyIyengar2008,Alili2005} and our own \cite{Iolov2013}
where we deal with the more exotic case of a sinusoidal driving force. The
literature on estimation from hitting-times of the class of diffusion models we
consider is clear that one parameter in particular, the characteristic time, is
most difficult to estimate having the widest confidence bounds by far. Thus we
focus on devising stimulations that will render most accurate estimations for
this parameter assuming that the other parameters are known or at least
nominally known in a sense that we explain later. In fact, we show that even if
the other parameters are also to-be estimated, for this problem, a control
designed with the goal of estimating the characteristic time parameter only,
ends up helping estimate all parameters.  

Thus our paper can be summarized as solving a PDE-optimzation arising from
maximizing the Mutual Information between a prior of the characteristic time
parameter and the random (future) hitting-times. The paper outline is as
follows: in \cref{sec:problem_formulation} we describe the mathematical formulation of the optimal design problem; in
\cref{sec:maximizing_MI}, we describe the optimization procedure, in particular
the derivation of the objective gradient and the numerical methods used; in
\cref{sec:batch_estimation}, we show the performance of the optimally designed
stimulation when many observations are first collected and only a single-pass,
{\sl batch} estimation is performed afterwards; in \cref{sec:online_estimation},
we turn to the {\sl online} problem, where the estimation is performed
after each observation and the stimulation is adjusted to reflect the newly
updated prior of the unknown parameters; in \cref{sec:discussion} we make
concluding and forward-looking remarks.

 
\section{Problem Formulation}
\label{sec:problem_formulation}
We would like to estimate the parameters in a noisy, leaky integrate-and-fire
(LIF) neuronal model. This is a diffusion process that has an absorbing barrier
at at an upper boundary and is unbounded below. In particular it is governed by,
\begin{equation}
\begin{gathered}
dX_t = \left(\underbrace{\a(t)}_{\textrm{control}} + \frac 1\tc(\m %\g \sin(\ot)
 - {X_t} ) \right) \intd{s} + \s\intd{W_t},
\\
X(0) = 0,
\\
X(\ts) = \xth \implies  
\begin{cases}
X(\ts^+) &= 0   
\\
t_k &=  \ts
\\
k  &= k+1
\end{cases}
\end{gathered} 
\label{eq:X_evolution_uo_control}
\end{equation}
The 'hitting time', $\ts$ is the first-time the process touches the upper
boundary, $\xth$, which is here set to $\xth=1$. Once a hitting-time is
recorded, the process is reset to $0$ and then continues from there. The control
$\a(t)$ is arbitrary and can be altered at any time; however, the parameter set,
$\th = \{\m, \tc, \s\}$, is assumed to be constant.

In the context of parameter estimation, we will assume that (a subset of) the
parameter set $\th = \{\m, \tc, \s\}$ is unknown and only the hitting times,
$\{t_k\}$, are observable.

In fact, we will concentrate on estimating the characteristic time ($\tc$). As
we mentioned in the introduction, previous papers on estimating parameters of
LIF from hitting times, e.g.\ \cite{Ditlevsen2007,MullowneyIyengar2008}, have
at length discussed that the estimation of $\tau$ is clearly the
hardest. In fact the literature is still not quite in agreement whether $\tau$ is hard to estimate or
whether it is simply unidentifiable. Since these studies were done in the
context of no control $\a=0$ or at least $\a= \textrm{const}$, we posit that
a judicious choice for the shape of $\a$ can significantly improve the
estimation of $\tau$.

Our goal is to choose the control, $\a(t)$, such as to 'best' estimate $\tc$
given only that the spike times $\{t_k\}$ are observed. Equivalently, setting
$t_0 = 0$, we can talk about the inter-spike intervals, $\{S_k\}$, which are
just the differences of subsequent spike times, $S_k = t_k - t_{k-1}$.

The probability density of the length of the $k$th interval,
conditional on some applied control $\a(\cdot)$, will be denoted as
\begin{equation} 
\begin{array}{rcll} 
g_k(t) \intd{t} &:=& \Prob[I_{k} \in [t, t + \intd{t})  \,|\,
 \a(\cdot)] &
 \textrm{(probability density)} 
% \\ 
% G_{n}(t) &:=& \Prob \left[T_{n} \leq t  \,|\,
%  \a(\cdot) \right] = \int_0^t g_{\phi}(\t) \intd{\t} &
%  \textrm{(cumulative distribution)}
% \\
% \G_{n}(t) &:= & \Prob(T_{n}>t \,|\, \a(\cdot) ) = 1 - G_{n}(t)
% &
%  \textrm{(survivor distribution)}
\end{array}
\label{eq:ISI_distribution_functions}
\end{equation}
We will drop the hitting-time indexing subscript, $k$, when there is no
confusion. 

The transition density for the state variable, $X_t$, $t \in [0, I_{n})$, will
be denoted as
\begin{equation}
f(x,t) := \Prob \left[X_{t} \in [x, x+ \intd{x})  \,|\,
 X_0 = 0, X_{s < t} < 1  \right]  \quad
 \textrm{(transition density)}
 \label{eq:transition_distribution}
\end{equation} 
The transition density satisfies a Fokker-Planck PDE over the domain, $x \in
(-\inf, \xth]$
\begin{equation}
\begin{gathered}
\begin{array}{lcl}
	\di_t f (x,t) &=&
					\underbrace{\frac{\s^2 }{2}}_{D}\cdot \di^2_x f 
					+ \di_x \Bigg(  
					\underbrace{\Big( \frac 1\tc (x-\m) - \a(t) \Big)}_{U(x,t)}  \cdot  f
					\Bigg)
					\\
					&=&
					D \cdot \di^2_x f +
					\di_x  \Big( U(x,t) \cdot f \Big)
					\\
					&=&
					- \di_x \phi(x,t)
					\\
					&=&
					\L[f] 
					\end{array}
	\\
	\left\{ \begin{array}{lcl}
	 f (x,0) &=& \delta(x)
	\\
	D \di_xf + U f |_{x=\xmin} &=& 0 
	\\
	f |_{x=\xth} &=& 0.
	\end{array} \right.
\label{eq:FP_pde_OU_absorbBC}
\end{gathered}
\end{equation}

The probability flux-out at the threshold boundary, $\phi(\xth, t)$, is very
important as it is related to the spike-time density,
\cref{eq:ISI_distribution_functions} via 
$$g(t)  = \phi(\xth, t) = -D\cdot \di_x
f|_{x=\xth}.$$
Recall that $U \cdot f = 0$ at the absorbing boundary, $\xth$.


%  In a typical (maximum likelihood) estimation experiment, we will see a lot of
% spikes and form the likelihood as $$ L(\th| t_n ) = \prod_n g_n(t_n) $$ We
% will then take logs and proceed as usual: $$ l(\th| t_n ) = \sum_n \log
% (g_n(t_n)) =  \sum_n \log ( -\di_t F(1,t_n)) $$ and then maximize $l$ over the
% parameters $\th$.  The associated {\sl score} function is $$ S(\th | \ts ) =
% \grad_\th l(\th | \ts) $$ The score function is a vector\footnote{We write
% $\grad$ for the vector differential and $\di$ for its scalar components, i.e.\
% $\grad_\th = [\di_{\th_1},\ldots\di_{\th_i}],\ldots$}.  The typical Maximum
% Likelihood process is to maximize the likelihood, $l$ which, if one uses a
% gradient-based approach amounts to finding the roots of the score, $S$.

In our Bayesian approach, we assume some {\sl  prior} distribution over
the possible values of $\th$. We denote the prior distribution as
$$
\begin{array}{rcll} 
\rho(\th) \intd{\th} &:=& \Prob(\Theta \in [\th, \th + \intd{\th})  ] &
 \textrm{(prior density)} 
 \end{array}
$$.

Given a single observation, $\ts$, the parameter posterior 
distribution is 
\begin{equation}
p(\th| \ts; \a) =
\frac{g(\ts |\th; \a)\cdot \rho(\th)}{\int_\Theta g(\ts|\th; \a)\cdot \rho(\th)
\intd{\th}}
\label{eq:parameter_posterior_defn}
\end{equation} 
Where $ g( \ts |\th; \a)$ is the likelihood of the hitting time given in
\cref{eq:ISI_distribution_functions}.

Our goal is to choose the control, $\a(\cdot)$, that maximizes the mutual
information  between the two random variables, $\Th, \ts$. Conditional on
$\a(\cdot)$, the Mutual Information, $I$, is given by
\begin{equation}
I[\a]= 
\int_\Theta \int_0^\infty g(t|\th)\rho(\th) \cdot 
\log \left( \frac{g(t|\th)}
{\int_\Theta g(t|\th)\rho(\th) \intd{\th}   } \right)
\intd{t}\intd{\th}.
\label{eq:J_mutual_info_objective}
\end{equation}

See \cref{sec:mutual_info_defn} for how this is derived from the usual textbook
definition of the Mutual Information. For different controls, $\a(\cdot)$,
the mutual information, $I$, will be different since $g$, the hitting time
density depends on the shape of $\a$ (the prior, $\rho$, does not). 
  
In short, we want to find the control input $\a(\cdot)$, which maximizes $I$ in
\cref{eq:J_mutual_info_objective} and we then verify that observations obtained
under such 'optimal' stimulation lead to parameter estimates that are more
accurate and more precise than parameters obtained for observations when the
system is perturbed by 'sub-optimal' stimulations.

\section{Maximizing the Mutual Information,
\cref{eq:J_mutual_info_objective}, using the Adjoint Method}
\label{sec:maximizing_MI}
Let us discuss the optimization problem
$$
\a(\cdot) = \argmax_{\a \sim \textrm{admissible}} I[\a]
$$ 

in our case, admissible controls are simply those that are continuous and with
the upper and lower bounds, $\a(t) \in [\amin, \amax] \,  \forall t$.

Our goal will be to obtain the gradient of $I$ with respect to $\a$, and then
use it in a standard gradient-based optimization procedure. There is the added
complexity that the gradient here is an infinite dimensional object and there
are some subtle functional analysis questions from a purely mathematical
point-of-view, we refer to the references, \cite{Lenhart2007,Borzi2012} for
rigorous justification of the manipulations. 

To proceed with obtaining the gradient with respect to $\a$, let us rewrite \cref{eq:J_mutual_info_objective}) in terms of the transition density, $f$. 
\begin{equation}
I[\a] = 
- \int_\Theta \int_0^\infty
	   \di_xf_\th(\xth, t|\th)  \rho(\th) \cdot 
		\log \left( \frac{\di_xf_\th(\xth, t|\th)}
						{\int_\Theta \di_xf_\th(\xth, t|\th)\rho(\th)
\intd{\th} } \right)
\intd{t}\intd{\th}
\label{eq:I_mutual_info_objective_in_terms_of_dixf} 
\end{equation}
We have dropped the constant, $D$ from
\cref{eq:I_mutual_info_objective_in_terms_of_dixf} as it is irrelevant for
the maximizing value of $\a$.

 In a standard approach for what
is known as the {\sl adjoint}  method of deriving a gradient of a
functional of a solution to a PDE with respect to one of the PDE's inputs, we
augment the objective, $I$, with the dynamics of $f$, by introducing the {\sl
adjoint} variable, $p$: 
\begin{align}
I=&  -
\int_\Theta \int_0^\infty 
	  \di_xf_\th(\xth, t|\th)  \rho(\th) \cdot 
		\log \left( \frac{\di_xf_\th(\xth, t|\th)}
						{\int_\Theta \di_xf_\th(\xth, t|\th)\rho(\th)\intd{\th} } \right)
\intd{t}\intd{\th} 
\\
	  &+ \int_\Theta  \rho(\th) \cdot \int_0^\infty \int_{x_-}^{\xth}
	  		p_\th \cdot (\di_t f_\th -	  \L[f_\th]) 
  				\intd{x}	  \intd{t} \intd{\th}
	  \label{eq:adjoint_term_in_objective} 
\end{align}
where, recall, we write $\L$ for the spatial differential operator in 
\cref{eq:FP_pde_OU_absorbBC}.

Note that the term \cref{eq:adjoint_term_in_objective} that is added to
\cref{eq:I_mutual_info_objective_in_terms_of_dixf}, is always zero, due to the
fact that $f$ satisfies the PDE in \cref{eq:FP_pde_OU_absorbBC}. However, the
reason for adding this null term is that via integration by parts, we can
remove the dependence of $I$ on first-order perturbations of $f$. This allows us to
compute the 'differential' of $I$ with respect to $\alpha$, without needing to
additionally compute the 'differential' of $f$ with respect of $\alpha$ as th
chain rule would require us to do. This is quite useful, since the
'differential' of $f$ with respect of $\alpha$ is a complicated mathematical
object because any value of $f(x,t)$ depends on the entire history of
$\a(\cdot)$, i.e.\ on all values of $\alpha(s)$ for $s<t$.

The elimination of $I$'s dependence on $f$ is done by 'transferring' the
derivatives of $f$ to $p$ using repeated integration-by-parts. For completeness,
we demonstrate in detail how this is done for $\th$ fixed:
\begin{align*}
\int_0^{\tf} \int_{x_-}^{\xth}
p \cdot \left[\di_t f  + \di_x \phi\right]
	\intd{x}\intd{t} 
=& 
\int_{x_-}^{\xth} p \cdot f \intd{x} \Big|_0^{\tf} -
\int_0^{\tf}\int_{x_-}^{\xth} \di_tp \cdot  f \intd{x}\intd{t}  
\\
&+ \int_{0}^{\tf} p\cdot \phi   \Big|_{x_-}^{\xth} \intd{t}
-  \int_0^{\tf}\int_{x_-}^{\xth} \di_xp \cdot  (\underbrace{-D\di_xf - U
f}_{\phi})
\intd{x}\intd{t}
\\
=&
\int_{x_-}^{\xth} p \cdot f \intd{x} \Big|_0^{\tf} -
\int_0^{\tf}\int_{x_-}^{\xth} \di_tp \cdot  f \intd{x}\intd{t}  
\\ 
&+ \int_{0}^{\tf} p\cdot \phi \Big|_{x_-}^{\xth} \intd{t} 
 + \int_0^{\tf}\int_{x_-}^{\xth} \di_xp \cdot  U f\intd{x}\intd{t}
\\
&+  
\int_{0}^{\tf} D\di_xp\cdot f \intd{t}  \Big|_{x_-}^{\xth} -
\int_0^{\tf}\int_{x_-}^{\xth} D\di^2_xp \cdot  f\intd{x}\intd{t}
\end{align*}

We can make a few simplifications based on the terminal and boundary
conditions for $f$
\begin{itemize}
  \item The initial conditions for $f$ are fixed and so that perturbations of
  $f$ at $t=0$ do not exist, equivalently, perturbations of the  control, $\a$,
  do not perturb the initial conditions of $f$, thus we can also disregard any
  terms that involve $f(x,0)$.
  \item $\phi(x_-, t)=0$
  \item $f(\xth, t) = 0$ 
\end{itemize}
with that we can simplify the adjoint term to 
\def \tf {{t_f}}
\begin{align*}
\int_0^{\tf} \int_{x_-}^{\xth}
p \cdot \left[\di_t f  + \di_x \phi\right]
	\intd{x}\intd{t} 
=&
-\int_0^{\tf}\int_{x_-}^{\xth} \di_tp \cdot  f \intd{x}\intd{t}
+
\int_{x_-}^{\xth} p(x, \tf) \cdot f(x, \tf) \intd{x}  
\\ 
&- \int_{0}^{\tf} p(\xth,t) \cdot D\di_xf(\xth,t) \intd{t}
 + \int_0^{\tf}\int_{x_-}^{\xth} \di_xp \cdot  U f\intd{x}\intd{t}
\\
&- \int_{0}^{\tf} D\di_xp(x_-,t) \cdot f(x_-,t) \intd{t} 
 - \int_0^{\tf}\int_{x_-}^{\xth} D\di^2_xp \cdot  f\intd{x}\intd{t}
\\
=&
\int_0^{\tf}\int_{x_-}^{\xth} 
	\left[-\di_tp -  D\di^2_xp + U \di_xp \right]\cdot  f
\intd{x}\intd{t}
\\ 
&- \int_{0}^{\tf} p(\xth,t) \cdot D\di_xf(\xth,t) \intd{t}
\\
& + \int_{0}^{\tf} D\di_xp(x_-,t) \cdot f(x_-,t) \intd{t}
\\
& +
\int_{x_-}^{\xth} p(x, \tf) \cdot f(x, \tf) \intd{x} 
\end{align*}
We can see that the adjoint term breaks down into several sub-terms, the spatial
sub-term, which gives the (backwards) evolution for the adjoint variable, $p$,
and two boundary sub-terms, which will provide the boundary conditions for $p$. 

We now replace the adjoint term in the original equation for $I$,
\cref{eq:adjoint_term_in_objective}. Recall that ultimate goal is to find the
differential of $I$ with respect to $\a$ while eliminating the
dependence on perturbations in $f_\th$. 

\begin{align*}
I=  
\int_\Theta  \rho(\th)\cdot \Bigg[ 
&-\int_0^{\tf} 
	  \di_xf_\th(\xth, t)  \cdot 
		\log \left( \frac{\di_xf_\th(\xth, t)}
						{\int_\Theta \di_xf_\th(\xth, t)\rho(\th)\intd{\th} } \right)
\intd{t} 
\\ &+
\int_0^{\tf}\int_{x_-}^{\xth} 
	\left[-\di_tp_\th -  D\di^2_xp_\th + U \di_xp_\th \right]\cdot  f_\th
\intd{x}\intd{t}
\\ 
&- \int_{0}^{\tf} D p_\th(\xth,t) \cdot \di_xf_\th(\xth,t) \intd{t}
\\
& + \int_{0}^{\tf} D\di_xp_\th(x_-,t) \cdot f_\th(x_-,t) \intd{t}
\\	& +
\int_{x_-}^{\xth} p(x, \tf) \cdot f(x, \tf) \intd{x}  
\Bigg]				\intd{\th}   
\end{align*}

We are now in position to consider the effect of small
perturbations, $\delta \a$, of the control, $\a$, on our objective, $I$,
$$
\delta I_\epsilon =\frac{I(\a + \epsilon \delta \a) - I(\a)}{\eps}
$$

The natural assumption is that given $\a +  \epsilon \delta \a$, there is a
corresponding solution to the evolution PDE, $f+ \epsilon \delta f$

Taking the limit of $\epsilon\ra0$, we get
\begin{equation}
\begin{aligned}
\delta I =  
\int_\Theta  \rho(\th)\cdot\Bigg[ \int_0^{\tf} \Bigg[ 
&-   \di_x \delta f_\th(\xth, t)  \cdot 
		\log \left( \frac{\di_xf_\th(\xth, t)}
						{\int_\Theta \di_xf_\th(\xth, t)\rho(\th)\intd{\th} } \right)
\\ 
	&-  \frac{ \di_x f_\th(\xth, t)}{ \di_x f_\th(\xth, t)}  \cdot \di_x
		\delta f_\th(\xth, t)
\\ 
	&+    \frac{ \di_x f_\th(\xth, t)}{	{\int_\Theta \di_xf_\th(\xth,
	t)\rho(\th)\intd{\th} }} \cdot
		 \boxed{\int_\Theta \rho(\th) \di_x \delta f_\th(\xth, t) 	\intd{\th}}				 
\\
 &+ \int_{x_-}^{\xth} 
	\left[-\di_tp_\th +  D\di^2_xp_\th  - U \di_xp_\th \right]\cdot  \delta f_\th
\intd{x}
\\ & - \int_{x_-}^{\xth}
\di_xp_\th \cdot f_\th\cdot \delta \a \intd{x}
\\
 &-   D p_\th(\xth,t) \cdot \di_x\delta f_\th(\xth,t) 
\\
 &+  D\di_xp_\th(x_-,t) \cdot\delta f_\th(x_-,t)
  \Bigg]			 \intd{t}	
\\
&+\int_{x_-}^{\xth} p(x, \tf) \cdot \delta f_\th(x, \tf) \intd{x}  
   \Bigg]\intd{\th}	   
\end{aligned} 
\label{eq:differential_I_before_refactoring}
\end{equation}
We see that $\delta I$ depends on both $\delta \a$ and $\delta f_\th$. 
In fact, it only depends on $\delta \a$ though the third last term, and on
$\delta f_\th$ through all the others.
However, we can select $p_\th$ in a judicious matter, such that the
dependence on $\delta f$ vanishes, which was our ultimate intention.

In order to eliminate $\delta f$ on the lower boundary,
$x_-$, we need that $$ \di_xp(x_-,t)=0.$$

To eliminate the dependence on the terminal value of $\delta f(x, \tf)$, we set
$p_\th(x, \tf)=0$.

In order to eliminate the spatial dependence on $\delta f$, we need to
enforce that $p$ evolves (backwards) as $$\di_t p + D\di_x^2p - U \di_x p = 0$$

Finally, in order to nullify the effect of $\di_x \delta f$ on the upper,
threshold, boundary, we note that we can rewrite the terms in
\cref{eq:differential_I_before_refactoring} that depend on $\di_x \delta
f(\xth, t)$ as
\begin{align*}
\int_\Theta  \rho(\th)\cdot \int_0^{\tf} \Bigg[ 
&-   \di_x \delta f_\th(\xth, t)  \cdot 
		\log \left( \frac{\di_xf_\th(\xth, t)}
						{\int_\Theta \di_xf_\th(\xth, t)\rho(\th)\intd{\th} } \right)
\\ 
	&-  \frac{ \di_x f_\th(\xth, t)}{ \di_x f_\th(\xth, t)}  \cdot \di_x
		\delta f_\th(\xth, t)
\\ 
	&+    \frac{ \di_x f_\th(\xth, t)}{	{\int_\Theta \di_xf_\th(\xth,
	t)\rho(\th)\intd{\th} }} \cdot
		 \boxed{\int_\Theta \rho(\th) \di_x \delta f_\th(\xth, t) 	\intd{\th}}				 
\\ 
 &-   D p_\th(\xth,t) \cdot \di_x\delta f_\th(\xth,t) \Bigg] \intd{t}	\intd{\th}	   
 \end{align*}
 \begin{align*}
 =
 \int_\Theta  \rho(\th)\cdot \int_0^{\tf} \Bigg[ 
& \bigg[ - \log \left( \frac{\di_xf_\th(\xth, t)}
						{\int_\Theta \di_xf_\th(\xth, t)\rho(\th)\intd{\th} } \right)
\\ 
	&- 1
\\ 
	&+    \frac{ \int_\Theta \rho(\th)\di_x f_\th(\xth, t)\intd{\th}}
	     		{	{\int_\Theta \di_xf_\th(\xth,	t)\rho(\th)\intd{\th} }} 
		 \bigg]  \cdot \di_x \delta f_\th(\xth, t)				 
\\ 
 &-   D p_\th(\xth,t) \cdot \di_x\delta f_\th(\xth,t) \Bigg] \intd{t}	\intd{\th}
\end{align*}
This is more clear if we replace the integrals with respect to $\th$ by discrete
sums. 

We then note that the latter two of the three terms before $\di_x \delta
f_\th$, sum to zero, and thus in order to eliminate the dependence of  $\delta
I$ on $\di_x \delta f_\th$, we need merely apply the simple boundary
condition: 
$$ p_\th(\xth,t) = - \frac 1D \log\left( \frac{\di_xf_\th(\xth, t)}
						{\int_\Theta \di_xf_\th(\xth, t)\rho(\th)\intd{\th} } \right) $$

\subsection{Terminal Time}
In principle, there is no {\sl a priori} obvious way how to select the terminal
time, $\tf$ in general. Intuitively, we would like the terminal time to be after
the bulk of the hitting-time density (say the 99th percentile). But, of course,
that depends on the chosen control, $\a$. For example, if we happen to select a
control that is always maximally-negative, $\a(t) = \amin$, we might not see a
hitting time for a long time.

\def \topt {{ t_{opt}}}

For practical purposes we proceed as follows. Given a set of plausible
parameters we set $\tf$ large enough and we further select a sub-interval,
$[0, \topt] \subset [0, \tf]$, such that we only seek to optimize over $t\in
[0, \topt]$ and for $t > \topt$ we let $\a= \amax$. I.e. after some interval we
force the system to spike so that to ensure that $$\int_{t>\tf} g(t) \intd{t}
\ll 1$$

\subsection{Adjoint PDE}
In summary, the adjoint variables, $p_\th$, must satisfy the following, adjoint,
PDE
\begin{equation}
\begin{gathered}
\begin{aligned}
-\di_t p_\th =&  \Lstar[p_\th]
\\ 		=&  \Big[ D\cdot \di^2_x p_\th - U(x,t;\th)   \cdot \di_x p_\th \Big].
\end{aligned}
\\
\begin{cases}
	p_\th \big|_{x=\xth} &=  -\log\left( \frac{\di_xf_\th(\xth, t)}
						{\int_\Theta \di_xf_\th(\xth, t)\rho(\th)\intd{\th} } \right) 
							- 1 
							+ \frac{\di_xf_\th(\xth, t)}
				   					{\int_\Theta \di_xf_\th(\xth, t)\rho(\th)\intd{\th} } $$
	\\
	\di_x p_\th  \big|_{x=\xmin} &= 0
	\\
	p_\th(x,\tf) &= 0
\end{cases}
\label{eq:adjoint_pde_OU}
\end{gathered}
\end{equation}

\subsection{Objective Gradient}
Given $p_\th, f_\th$, we can calculate the differential of $I$ in
\cref{eq:I_mutual_info_objective_in_terms_of_dixf}, with respect to the
control $\a(t)$, $\delta I / \delta \a$ using the terms in
\cref{eq:differential_I_before_refactoring} that multiply $\delta \a$ as 
\begin{align}
\delta I =&  \Bigg[ 
-\int_\Theta  \rho(\th) \cdot \bigg(  
 \int_\xmin^{1} \di_x p_\th \cdot f_\th \intd{x} 
    \bigg) \intd{\th} \Bigg]\cdot \delta \a
    \label{eq:objective_gradient_continuous}
\end{align}

\subsection{Approximation of the prior distribution}
For tractability purposes, we need to approximate the $\th-$integral via a
sample sum. I.e for a given $\alpha(t)$, we solve for $p,f$ for a few sampled
values $\th_i$ from the updated prior distribution $\rho(\th)$ and their
corresponding probabilities, $\rho(\th_i)$.

This is equivalent to assuming a discrete, i.e.\ a 'particle', prior, a sum
of Dirac delta masses. In the simplest case, these are $N$ values with equal
probability.
\begin{equation}
\rho(\tc) = \sum_i  
	\tfrac 1N \delta(\tc- \tc_i) 
\label{eq:basic_prior_over_tau}
\end{equation} 


% \subsection{Effect of the Prior}
% Here we show that the optimal control is sensitive to the {\sl spread} of the
% prior, for example if we have a tightly clustered vs. loosely spread prior, both
% centred at roughly the same mean (the log-prior has the same mean). 
% 
% Very interestingly, we see that while for a wide prior, the optimal control has
% its characteristic double hump shape, that we have seen already, for a tight
% prior, that is no longer the case
% 
% Thus we see that the shape of the prior {\sl has!}
% an effect on the optimal control.
% 
% \begin{figure}[h]
% \begin{center}
% \subfloat[Wide Prior]
% {
% \label{fig:prior_spread_wide}
% \includegraphics[width=0.48\textwidth]
% {Figs/FP_Adjoint/PriorBox_wide_prior.pdf}
% }
% \subfloat[Tight Prior]
% {
% \label{fig:prior_spread_tight}
% \includegraphics[width=0.48\textwidth]
% {Figs/FP_Adjoint/PriorBox_concentrated_prior.pdf}
% }
% \caption[labelInTOC]{The effect of the spread (variance) of the prior on the
% resulting optimal control}
% \label{fig:prior_spread}
% \end{center}
% \end{figure}
% 
% Let's look at it another way, we will consider our basic prior as a function of
% $w$
% \begin{equation}
% \rho(\tc) = 
% \begin{cases}
% 	\tfrac 12 & \textrm{if } \tc= \in \{1- w, 1/(1-w) \}\\
% 	0   &\textrm{o/w }
% \end{cases} 
% \end{equation} 
% and sweep for $w = .1:.1:.9$ (in matlab notation).
% 
% The results are in \cref{fig:effect_of_prior_width}. Looking at
% \cref{fig:effect_of_prior_width}, we might be optimistic to hypothesize that we
% should be doing this online and as the uncertainty (roughly speaking $w$) of the
% parameter decreases, we should be changing the applied control\ldots This
% brings us to {\sl adaptive } versions of our scheme which is NOT something we
% have yet implemented. 
%  
% \begin{figure}[h]
% \begin{center} 
% \includegraphics[width=\textwidth]
% {Figs/FP_Adjoint/Effect_of_prior_spread.pdf} 
% \caption[labelInTOC]{The effect of the width ($w$, a measure
% of uncertainty) of the prior on the resulting optimal control}
% \label{fig:effect_of_prior_width}
% \end{center}
% \end{figure}

\section{Implementation and Illustration of the Gradient Ascent Procedure}
\label{sec:gradient_ascent}
The optimization of the Mutual Information, $I$, in
\cref{eq:I_mutual_info_objective_in_terms_of_dixf} is approached as a gradient-based iterative method, where we approximate the
infinte-dimensional gradient, $\delta I$ in
\cref{eq:objective_gradient_continuous} by some finite-dimensional
approximation. 

The process involves three basic stages
\begin{enumerate}
  \item Given the current control iterate, $\a_n(t))$, numerically solve for the
  corresponding $f,p$ from their respective PDEs,
  \cref{eq:FP_pde_OU_absorbBC,eq:adjoint_pde_OU}.
  \item From the adjoint differential,
  $\delta I / \delta \a$, using \cref{eq:objective_gradient_continuous}
  \item Increment $\a_{n+1}$ in the direction of increasing $\delta I / \delta
  \a$
  \begin{equation}
\a_{n+1}(t) \ra \a_n(t) + \frac {\delta I }{\delta \a} (t) \cdot \Delta \a
\label{eq:simple_gradient_ascent_increment}
\end{equation}
for some step-size $\Delta \a$.
\end{enumerate}

In practice, we need to align the various discretization grids, and the simplest
thing to do is to discretize $\a,f,p$ at the same time points, $t_k$.

That is, the gradient evaluated at $t_k$ is just $$ 
\frac {\delta I }{\delta \a}(t_k) =   
	-\int_\Theta  \rho(\th) \cdot \bigg(  
	\int_\xmin^{1} \di_x p_\th(t_k,x) f_\th(t_k,x) \intd{x}    
	    \bigg) \intd{\th}
$$

\Cref{eq:simple_gradient_ascent_increment} is the simplest gradient ascent
scheme one can devise. The literature on gradient-based optimization,
\cite{Nocedal1999}, and the subset on PDE-based gradient optimization,
\cite{Borzi2012} offers more sophisticated schemes, but for simplicity
sake we start with this basic one. An example of a more sophisticated scheme
is the nonlinear conjugate-gradient ascent, as recommended in the literature on
Fokker-Planck control, \cite{Annunziato2013}.
 
We provide a full description of the optimization algorithm in the appendix, see
Algorithm \ref{alg:gradient_ascent_4_OC}.

An illustration of a single control increment is shown in
\cref{fig:example_control_increment}; while an example of a full optimization ascent is given in \cref{fig:example_gradient_ascent}.

%\usepackage{graphics} is needed for \includegraphics
\begin{figure}[htp] 
\begin{center}
  \includegraphics[width=\textwidth]{Figs/FP_Adjoint/control_increment_example.pdf}
  \caption[Single control Increment Illustration]{Illustration of how the
  control is incremented given the gradient. In the top panel, we show the
  initial control in dashed and the incremented control in solid with arrows
  indicating the incremental change. In the bottom panel, we visualize the
  corresponding value of $\delta I/ \delta \a (t)$. }
  \label{fig:example_control_increment}    
\end{center}
\end{figure}   
           
\begin{figure}[htp] 
\begin{center}
\includegraphics[width=1\textwidth]{Figs/AdjointOptimizer/GradientAscent_Nt2.pdf}
  \caption[Gradient Ascent for the Optimal Stimulation]{Example gradient
  ascent for the optimization of $I$ in
  \cref{eq:I_mutual_info_objective_in_terms_of_dixf} illustrating that the
  effect of the optimal control $\a_{opt}$ is to effectively 'separate' the
  hitting time distributions corresponding to the potential values of $\tc$.
  Top
  panel, the initial and the optimal optimal controls, $\a_{0}(t), \a_{opt}(t)$. Second panel shows the hitting time distributions $g(t|\tc)$ conditional on
  using the initial control, $\a_0$.
  The third panel again shows $g(t|\tc)$, but this time conditional on the
  optimal, $\a_{opt}$.
 The bottom panel shows the progress of the Objective Mutual Information ($I$) 
  (\cref{eq:J_mutual_info_objective}) as a function of the gradient ascent
  iterations (here the algorithm converged in 9 iterations). It is clear that
  th optimal control significantly improves the objective in comparison to the
  initial guess.}
  \label{fig:example_gradient_ascent}   
\end{center}   
\end{figure}    
 
In \cref{fig:example_gradient_ascent}, we get an intuitive 
feeling for what the 'optimal control' is attempting to do - it tries to
'pull apart' the hitting time distributions, such that an observed hitting time
can most cleanly be attributed to one of the potential parameter values, $\t$.
For the combination of known parameters, $\m, \s$ and prior $\rho$, used in
\cref{fig:example_gradient_ascent}, the optimal control first suppresses firing
for an initial time-segment and then maximally stimulates firing in the
remaining time. 
It is visually obvious that the result on $g(t|\tc)$ is to go
from a case where the two possible, hitting time distributions lie on top of each other, to one where they are clearly
delineated. Thus given an observation from the optimally stimulated system, one
would be able to much more confidently estimate what was the underlying value of
$\tau$ that produced the observation. 

\subsection{Switch point sweep}
Given the optimal solution obtained in \cref{sec:gradient_ascent}, we suspect
that the optimal solution is bang-bang, in the sense of 'hold  everything back
and then excite maximally'. It is interesting to check how sensitive the
objective is to the exact value of the switching time, $t_{switch}$. If it is
not very sensitive, it would be practically useful as the exact
optimization of the PDE-based optimization will not be
necessary, and any 'bang-bang' solution will achieve satisfactory improvement in
estimation. 

Therfore, we sweep through the switch point of exactly when the
bang-bang switch occurs and see how it impacts the resulting objective, $I$.
The results are in \cref{fig:sweep_switchtime}. It seems that, at least for this
parameter set, there is an improvement in the objective by putting the switching
point past $t=6$, but going much beyond that there is no real difference. 

\begin{figure}[htp]
\begin{center}
  \includegraphics[width=1\textwidth]{Figs/AdjointOptimizer/SweepSwitchpoint_wide.pdf}
  \caption[Effect of Switching time on Mutual Info Objective]{Effect of
  Switching time on Mutual Info Objective. Plotted is the objective, $I$, as a
  function of the switching time, $t_{switch}$, of a bang-bang switching control
  as in the top panel of \cref{fig:example_gradient_ascent}. The same
  simulation parameters are used as in \cref{fig:example_gradient_ascent}. The
  red line is the value of $I$ as function of a switching time, $t_{switch}$
  between maximally-inhibitive to maximally-excitatory control (a bang-bang
  control). The blue line, plotted for refernce, is the value of
  $I$ for no control ($\a=0, \forall t$))}
  \label{fig:sweep_switchtime}
\end{center}
\end{figure}


\subsection{Optimization Method Sensitivity to Problem Parameters}
We explore the effect of the number of points
$\t_i$ in the prior and the dispersion in the prior; the initial
guess for $\a(\cdot)$ and the values of the known parameters $\m, \s$. 

\subsubsection{Number of Points, higher moments of the prior}
We wonder whether the exact shape of the prior matter or whether the first two
moments (mean, variance) are sufficient to determine the optimal control. We
make a simple experiment where we compute the optimal control as in
\cref{sec:gradient_ascent} for two very similar priors, the first using 2 points
in the prior and in the second using 3 points, but both priors have the same
log-mean and log-variance. See \cref{fig:prior_shape_impact}. We observe that
the optimal control is essentially the same. This suggests that for practical
purposes it suffices to make the optimization with a 2-point prior that matches
the (log-) variance of the more detailed prior distribution.
 
%\usepackage{graphics} is needed for \includegraphics
\begin{figure}[htp]
\begin{center}
  \includegraphics[width=1\textwidth]{Figs/AdjointOptimizer/NumberOfTausEffect.pdf}
  \caption[Detailed-Shape-of-Prior Impact]{Effect of detailed shape of the prior
  on the Optimal Control.} 
  \label{fig:prior_shape_impact} 
\end{center}
\end{figure}

\subsubsection{The dispersion of the prior}
We wonder whether the dispersion of the prior, quantified by the ($log$-)
variance changes the optimal control. Again, we take two 2-point priors, one with
particles at $\tc = [.25, 4]$ and the other at $\tc = [.75,1.3]$, the results
are shown in \cref{fig:prior_dispersion_impact}. It is clear that while the
general shape of the optimal control is the same, regardless of the 'width' of
hte prior, there is some difference. We further investigate the
practical difference between the two priors in the section on estimation,
\cref{sec:batch_estimation}.

%\usepackage{graphics} is needed for \includegraphics
\begin{figure}[htp]
\begin{center}
  \includegraphics[width=\textwidth]{Figs/AdjointOptimizer/PriorSpread.pdf}
  \caption[Variance-of-Prior Effect]{Effect of variance-of-prior on
  optimal stimulation. For the 'wide prior', we set the simple, uniform 2-point prior at
   $\tc =  [0.25, 4]$, while for the 'narrow prior', they are positioned much
   closer to each other at $\tc =   [0.75, 1/.75 ]$  }
  \label{fig:prior_dispersion_impact} 
\end{center}
\end{figure}

\subsubsection{The initial guess for the control $\a_0$}
Like all gradient-based optimization procedures, ours is sensitive to the choice
of initial guess for the independent variable, i.e.\ the choice of $\a_0$. 

It turns out that indeed the choice of $\a_0$ has a strong influence on what the
scheme finds as optimal. We illustrate this in \cref{fig:ICs_for_control}. We
proceed as follows, for three different priors, a wide, medium and concentrated
one, we run the optimization scheme from four different initial guesses for
$a_0$ - 
\begin{enumerate}
\item zero for the entire (optimization) time-interval
\item linearly increasing from the lower to the upper control bound over the
optimization interval
\item a sinusoidal wave from max to min and again to max
\item max for the entire optimization time-interval 
\end{enumerate}
In \cref{fig:ICs_for_control}, we see that for different initial guesses the
optimization routine finds different optimal controls. It seems that it is
easier to find a good optimal control for the wide prior and more difficult for
the concentrated prior in which case, the optimizer cannot improve on the
objective except if started with the 'sinusoidal' initial guess.

\begin{figure}[h]
\begin{center}  
\subfloat[Wide Prior]
{
\label{fig:wide_prior_opt_ics}
\includegraphics[width=0.5\textwidth]
{Figs/AdjointOptimizer/OptimizerICswide_prior.pdf}
}
\subfloat[Medium Prior] 
{
\label{fig:medium_prior_opt_ics}
\includegraphics[width=0.5\textwidth]
{Figs/AdjointOptimizer/OptimizerICsmedium_prior.pdf}
}
\\
\subfloat[Concentrated Prior]  
{
\label{fig:medium_prior_opt_ics}
\includegraphics[width=0.5\textwidth]
{Figs/AdjointOptimizer/OptimizerICsconcentrated_prior.pdf}
}
\caption[Dependence on initial guess for control]{Illustrating the dependence
of the optimization scheme on the initial guess for the optimal control. 
On the top panel of each sub-float we have the initial guess, in the middle
panel we show the corresponding solution obtained by the optimization routine;
in the bottom panel we show the corresponding evolution of the objective. Note
that the general shape of the optimal control depends more on the IC then the
spread of the prior distribution of $\tc$.}
\label{fig:ICs_for_control}
\end{center}
\end{figure}

\subsubsection{The value of the known parameters $\mu, \s$}
So far we have restricted ourselves to a base-case scenario of the assumed known
parameters, $\mu, \s$. We now redo the basic analysis by varying these.

We look at 6 qualitatively different parameter values, in
\cref{tab:mu_sigma_perturbation}.
\begin{centering} 
\begin{table}
\begin{tabular}{cccc}
Regime meaning & $\m$ & $ \s$ & Meaningful Objective Improvement?\\ \hline
 Negative $\m$ & -0.5& 1 & YES\\ 
 Small-perturbation $\m$ &0.1&1   & YES\\ 
 Very-Positive $\m$ & 1&1  & YES\\
    Smaller $\s$ &                 0&0.3 & YES\\ 
 Small-perturbation $\s$  &        0&0.9 & YES\\ 
                Larger $\s$ &      0&1.5 & YES\\ 
\end{tabular}
\caption{TODO: This table is just for internal consumption - what is the best
way to present this table?

Perturbation parameter values and a YES/NO as to whether the
optimization routine can find significant improvements in the
objective, recall that small-perturbations, mean small relative to
the base case $\m, \s=0,1$.
All examples use the same, 2-pt, prior, $\tc = 
[0.5, 2]$}
\label{tab:mu_sigma_perturbation}
\end{table}
\end{centering}

Unfortunately, while for small perturbation, lines 2,5 in
\cref{tab:mu_sigma_perturbation}, the results are the same as for the base-case,
$\m,\s = 0,1$, for the more significant perturbations, the iterative
 optimization method is not able to progress to significantly better objective
 values and the ascent terminates after only one or two iterations. Clearly some 
improvement can be made here.

\clearpage

\section{Batch Estimation}
\label{sec:batch_estimation}
We now proceed to generate a large sample of hitting times with several type
of controlled stimulation and then estimate the unknown
parameter after observing the whole sample. We call this 'batch' estimation as opposed to
'online' estimation where estimates are updated after every single observation
and the control is also updated according to the updated prior distribution.
Online estimation is performed in the next
section, \cref{sec:online_estimation}).

We assume that the true parameters governing \cref{eq:X_evolution_uo_control} are
$$ \m = 0; \tc = 1; \s = 1;
$$ We will assume we know $\s, \m$ and do not know the time constant, $\tc$ so
we are trying to maximize the Mutual Information between the random hitting
time, $\ts$, and $\tc$. In the parlance of the neuroscience literature this
parameter set corresponds to what is known as the 'Sub-Threshold, High-Noise'
regime, \cite{Iolov2013}, as long as the control, $\a$ equals zero. Of course,
changing the control sufficiently will move it into the qualitatively different,
'Super-Threshold' regime, which is the case when the process can hit the
threshold even in the absence of noise (in the 'Sub-Threshold' regime, a
non-zero noise parameter is required if any hitting-times are to occur.)

For obtaining the MI-optimal stimulation, we will take a uniform prior on
$\tc$, using 10 $\tc_i$ uniformly spaced in the interval $[0.25, 4.0]$, i.e.
each of the $\tc_i$ in the prior has 1/10. Note that neither the mean nor the
log-mean of the prior correspond to the unknown true, $\tc$ or $\log(\tc)$.
 
The control obtained thus will be tagged as 'opt. In addition we consider the
two optimal control obtained using the 'wide' and 'narrow' 2-point priors as
shown in \cref{fig:prior_dispersion_impact}. We call these 'opt-wide' for the
prior with $\tc =  [0.25, 4]$, and 'opt-narrow' for the prior with  $\tc =  
[0.75, 1/0.75]$.  

In addition, we also use the control where $\a$ is set to the 'critical' value,
$\a = \xth/\tc = 1$, and one where the control is set to its upper
bound, $\a = \amax$.

\subsection{The Estimation Algorithm for the Batch Problem}
We have posed a fairly-simple estimation objective, to estimate $\tau$, which
amounts to single-variable optimization. The negative log-likelihood of an
observed hitting-time set $\{t_n\}$ is
\begin{equation}
l(\tc) = - \sum_n \log ( g(t_n | \tc) ) =  - \sum_n \log \left( -D \di_x f(\xth,
t_n |\tc) \right)
\label{eq:MLE_likelihood}
\end{equation}
We minimize \cref{eq:MLE_likelihood} using the standard single-variable
optimization routine in NumPy based on Brent's method.

% The distributions are exemplified in
% \cref{fig:log_likelihood_beta_examples_100000},
% for three different values of $N_s =  1e5$. We see that for the constant
% stimulations, $\a = \a_{crit}, \a_{max}$ it is very hard to
% distinguish between different values of $\tc$. 
% 
% In \cref{fig:log_likelihood_beta_examples_100000} as well as in 
% \cref{fig:hitting_time_density_g_aopt_bprior}, we get an indication for why
% the 'optimal control' is better than the constants. For the constant control
% the different hitting time densities look like local perturbations of each
% other, either a little more or a little less, but for the optimal control they
% are shifted, which means that we see the first indications that the Opt Control,
% might have some superiority over the 'Crit' Control (for example) as it seems to estimate a $\tc$ closer to 1 (the 'true' value). However, on average, the different shapes of $\a(t)$ seems to have a very limited impact on the estimates for $\tc$ (even though it has a very obvious impact on the shape of the hitting time distribution $g(t)$).

% \begin{figure}[h] 
% \begin{center}
% \subfloat[opt]
% {
% \includegraphics[width=.75\textwidth]
% {Figs/HitTime_MI_TauChar_Adjoint_Estimate/Adjoint_TauChar_Estimator_estimatorWorkbench_b=0x100000_a0.pdf}
% }
% \\
% \subfloat[crit] 
% {
% \includegraphics[width=.75\textwidth]
% {Figs/HitTime_MI_TauChar_Adjoint_Estimate/Adjoint_TauChar_Estimator_estimatorWorkbench_b=0x100000_a1.pdf}
% }
% \\
% \subfloat[max]
% {
% \includegraphics[width=.75\textwidth]
% {Figs/HitTime_MI_TauChar_Adjoint_Estimate/Adjoint_TauChar_Estimator_estimatorWorkbench_b=0x100000_a2.pdf}
% }
% \caption[labelInTOC]{Example of Empirical vs.\ Analytical Hitting time
% distributions, $g(t|\t;\a)$, and the associated log-likelihoods. $N_s = 1e5$
% hits}
% \label{fig:log_likelihood_beta_examples_100000}
% \end{center}
% \end{figure}  
% 
% \clearpage



\subsection{Estimator Comparison}
Recall that we will stimulate the system using 5 stimulation
waveforms.  
\begin{enumerate}
  \item 'opt' - the optimal gradient-ascent-based  control $\a_{opt}$, based on
  a 10-point uniform prior between $[0.25, 4]$
  \item  'opt-wide' - the optimal control using a 2-point prior on  $[0.25, 4]$
  \item 'opt-narrow' - the optimal control using a 2-point prior on $[0.75,
  1/0.75]$
\item   'crit' - the constant control
$\a_{crit}$, ($\a_{crit}(t) =  \xth/\tc$
\item  'max' - the max constant control, $\amax$ ($=2$)
\end{enumerate} 

We now simulate $N_b $ blocks of $N_s$ hitting times each for the
5 alphas and then estimate $\tc$ over each set using MaxLikelihood over our
computed expression for the density, $g(t|\tc; \a(t) )$. 
Naturally, for each control, we use the same Gaussian random draws per block of
$N_s$ hitting of times.

%\usepackage{graphics} is needed for \includegraphics
% \begin{figure}[htp]
% \begin{center}
%   \includegraphics[width=\textwidth]{Figs/HitTime_MI_TauChar_Adjoint_Estimate/three_pt_prior_thits_distn.pdf}
%   \caption[labelInTOC]{Empirical Hitting-TIme distributions for the different
%   choices of $\a$}
%   \label{fig:empirical_hitting_times_3alphas}
% \end{center}
% \end{figure}

The estimation results are tabulated in in
\cref{tab:beta_estimates_from_hitting_times_different_alphas}.

\begin{table}
% \subfloat[$N_b=1000, N_s = 1e2$]{
% \begin{tabular}{ccc}
% \input{../OptEstiamte/Figs/HitTime_MI_TauChar_Adjoint_Estimate/tauchar_hit_time_100.txt}
% \end{tabular}
% }
% \subfloat[$N_b=100, N_s = 1e3$]{
% \begin{tabular}{ccc}
% \input{Figs/HitTime_MI_TauChar_Adjoint_Estimate/tauchar_hit_time_1000.txt}
% \end{tabular}
% }\\
% \subfloat[$N_b=10, N_s = 1e4$]{
% \begin{tabular}{ccc}
% \input{Figs/HitTime_MI_TauChar_Adjoint_Estimate/tauchar_hit_time_10000.txt}
% \end{tabular}
% } 
% \subfloat[$N_b=1, N_s = 1e5$]{
% \begin{tabular}{ccc}
% \input{Figs/HitTime_MI_TauChar_Adjoint_Estimate/tauchar_hit_time_100000.txt}
% \end{tabular}
% }
\caption[Batch $\tau$ MLE estimates]
{Results for the estimates arising from simulations using various values of $\a$
(opt, crit, max). In each sub-table there are $N_b$ parameter estimates for each distinct $\a$, with $N_s$ hitting times used to
form a $\tc-$estimate.  The 'true' value of $\tc$ is $\tc=1$. 
Also see \cref{fig:beta_estimates_from_hitting_times_different_alphas}.}
\label{tab:beta_estimates_from_hitting_times_different_alphas} 
\end{table}   

\begin{figure}[h]
\begin{center}
\subfloat[$N_b=1e3, N_s = 1e2$]
{
\includegraphics[width=0.48\textwidth]
{Figs/HitTime_MI_TauChar_Adjoint_Estimate/miestimates_scatterplot_Nb1000_Ns100.pdf}
}
\subfloat[$N_b=1e2, N_s = 1e3$]
{
\includegraphics[width=0.48\textwidth]
{Figs/HitTime_MI_TauChar_Adjoint_Estimate/miestimates_scatterplot_Nb100_Ns1000.pdf}
}
\\
\subfloat[$N_b=1e2, N_s = 1e4$]  
{
\includegraphics[width=0.48\textwidth]
{Figs/HitTime_MI_TauChar_Adjoint_Estimate/miestimates_scatterplot_Nb10_Ns10000.pdf}
}  
\subfloat[$N_b=1e1, N_s = 1e5$]
{ 
\includegraphics[width=0.48\textwidth]
{Figs/HitTime_MI_TauChar_Adjoint_Estimate/miestimates_scatterplot_Nb1_Ns100000.pdf}
}
\caption[Illustration of the individual MLE Estimates]{Fine Visualization of the
MLE estimates for the different controls, also see
\cref{tab:beta_estimates_from_hitting_times_different_alphas}}
\label{fig:beta_estimates_from_hitting_times_different_alphas}
\end{center}
\end{figure}
 
Given
\cref{tab:beta_estimates_from_hitting_times_different_alphas,fig:beta_estimates_from_hitting_times_different_alphas},
we see a clear advantage to using the Optimal Control, $\a_{opt}$ over the
simpler, constant controls. In particular, the bias of the estimates seems to
be significantly reduced. The variance is greater for the optimal controls, but
a look at the detailed distribution of estimates, top panels in
\cref{fig:beta_estimates_from_hitting_times_different_alphas} indicates that
this is due to a few outliers. Meanwhile, if we consider the various optimal
controls, 'opt' vs. 'opt-wide', 'opt-narrow', we see that there is no big
difference in the estimator quality of each - they all perform approximately
equivalently. 


\subsection{Closer look when the estimation goes wrong}
TODO: Move to appendix?

In general, the optimally-stimulated samples give (much) more accurate estimates
than the ones that are naively-stimulated. However, we see in
\cref{fig:beta_estimates_from_hitting_times_different_alphas}, panel b) that
occasionally (once), the optimally-stimulated sample can give very wrong
estimates.

We now plot what goes wrong in that sample vs. the base-case where the estimate
is almost exact, see \cref{fig:batch_estimtion_in_detail}. It seems that in the
bad case, there are enough extreme (very long) hitting times that the estimation
procedure infers that the characteristic time must be very small (i.e the
attractive force towards $\mu=0$ is very strong). In the larger sample, there
are no other examples of these extremely delayed hitting times, they are thus
seen to be a rarity and the correct parameter value is inferred.

\begin{figure}[h]
\begin{center} 
\subfloat[1e3 Hits]  
{ 
\includegraphics[width=0.98\textwidth]
{Figs/HitTime_MI_TauChar_Adjoint_Estimate/Adjoint_TauChar_Estimator_good_vs_bad_estimates_Nh1000.pdf}
}\\
\subfloat[1e4 Hits]
{
\includegraphics[width=0.98\textwidth]
{Figs/HitTime_MI_TauChar_Adjoint_Estimate/Adjoint_TauChar_Estimator_good_vs_bad_estimates_Nh10000.pdf}
}
\caption[Estimation Detail View for the Optimally-Stimulated
Experiment]{Comparing the case when the estimation does best vs. when it does
worst for the optimally-stimulated experiment. In the case with less samples, a
few extreme observations can skew the estimate, however with more data points,
those are seen to be rare cases and a more accurate estimate is found.}
\label{fig:batch_estimtion_in_detail}
\end{center}
\end{figure}


\subsection{Proof of concept of estimating the entire parameter set}
So far we have assumed that $\mu, \s \in \th$ are known. We now relax that
assumption and attempt to estimate those two parameters in addition to
characterisitc time $\tc$. We still use the optimal control obtained assuming
$\m,\s$ known. Why is this expected to work? First of all we found that the
optimal control seems essentially irrelevant to the exact values of $\m,\s$ as
long as they are not 'too far' from the nominal values ($\m=0, \s=1$.). Moreover
as discussed in the introduction, the $\tc$ parameter is well-known to be the
hardest to estimate. Thus a scheme designed to estimate it well is conjectured
to help with estimation of the entire parameter set, since removing variation
from one parameter estimate necessarily removes variation from the others.

We use the same simulated hitting-time data-set as in
\cref{tab:beta_estimates_from_hitting_times_different_alphas,fig:beta_estimates_from_hitting_times_different_alphas},
using $N_s=1e4$ hits and $N_b=10$ separate sample-sets, i.e.\ we make 10
estimates.

The results are given in \cref{fig:all_theta_estimates_batch}, it is clear that
the 'optimally' stimulated samples result in much more accurate and precise
estimates. In particular all stimulation schemes come-up with high-fidelity
estimates for $\s$, it is in resolving the interplay between $\m,\tc$ that the
'optimal' stimulation excels. In particular the 'naive' stimulations does
not allow to distinguish a high $\m$ and a small $\tc$ from a low $\m$
and a high $\tc$, while the dynamic, bang-bang situation makes it possible.

\begin{figure}[htp]
\begin{center}
  \includegraphics[width=\textwidth]{/home/alex/Workspaces/Latex/OptEstimate/Figs/MuTauSigma_Batch_estimates/all_theta_estimates_scatterplot_Nb10_Ns10000.pdf}
  \caption[Batch estimates for all 3 parameters]{Optimal Stimulation greatly
  improves estimates when all parameters are being estimated. 
  From the top to bottom, we show $N_b=10$ estimates for $\m,\tc,\s$,
  formed after observing $N_s=1e4$ hits per-estimate. There are three 'optimal'
  stimulations as discussed in the text, shown to the left and two naive ones,
  shown to the right. The scatter dots are the individual estimates per
  stimulation-parameter pair. The horizontal line indicates the true value of
  the paramters, i.e.\ the value used to simulate the hitting-time observations
  in the data samples.}
  \label{fig:all_theta_estimates_batch}
\end{center}
\end{figure}

\clearpage

\section{Online Estimation}
\label{sec:online_estimation} 
In the {\sl online} optimization of the MI / estimation, we proceed in a
slightly more complicated way. In particular we 

\begin{enumerate}
  \item Find $\aopt$ using the gradient ascent, for the prior $\rho$
  \item Apply $\aopt$ and measure several $1,2\ldots,N_{s}$ hitting times
  $t_k$
  \item Update the $\rho$ into a posterior conditional on the observed $\{t_k\}$
  \item Recalibrate $\aopt$ using the new $\rho$, i.e. go back to 1. 
\end{enumerate}

In practice we incrementally, increase by starting with $N_s=1$ and then
doubling it, i.e.\ we re-compute $\a$ after the 1st, 3rd, 7th etc hitting time.

Computational efficiency considerations aside, we have already discussed and
illustrated all the techniques needed to perform pts. 1,2,4, it is
only the prior update that needs to be discussed.

\subsection{Quick Intro to Particle Filtering}
Recall Bayes' formula
$$
\rho(\th| \{t_k\} ) = 
\frac{  \rho(\th) \cdot \prod_k g(t_k|\th ; \a) }
	 { \int_\Th  \rho(\th) \cdot \prod_k g(t_k|\th ; \a)  \intd{\th}}
$$

In practice, exact calculation of $\rho(\th|\t_k)$ would not be possible in our
context, so an approximation approach needs to be made.

The standard approach is to describe the prior distribution by an ensemble of
points (particles). We now describe the basic aspect of how the particle
ensemble is constructed, how it is updated and how it is resampled. We use the
reference \cite{Granade2012}, in particular section 4 therein.

We have a prior
$ \rho(\th)$, which we represent or approximate  via an ensemble of points $\th_i$. as 
$$ \rho(\th) \sim \sum_i w_i \delta(\th-\th_i)$$
This is what we have been doing up to now for the MI optimization routines.

Again recall that a bayesian update, given the $k$th hitting time is
$$ \rho(\th|t_k) \propto g(t_k|\th)\rho(\th)$$

thus the weights are iteratively updated as 

$$w_i \ra w_i g(t_k|\th_i)$$

Given the particle ensemble, $\{\th_i, w_i\}$, we can  then approximate the
mean/variance of $\Th$ as
$$ \Exp[\Th] \approx \sum_i w_i \th_i$$
and 
$$ \Var[\Th] \approx \sum_i w_i \th_i \th_i - (\Exp[\Th])^2$$
In fact, in general 
$$\Exp[f(\Th)]\approx\sum_i w_i f(\th_i)$$ approximates the expectation of any
function, $f$, of the random variable $\Th$.

Here's the crux of the update/resample algorithm. Given a new observation, i.e.
the latest hitting time $t_k$, the weights are updated according to
$$ w_{i,k} = w_{i,k-1}\cdot g(t_k| \th_i, \a_k),$$where $g(|\th_i, \a_k)$ is
the probability density given the parameter value $\th_i$ and the chosen
stimulation that was applied during the $k$th sample, $\a_k(\cdot)$.
The weights are then re-normalized so that at all time $$\sum_i w_i = 1$$ 

The literature suggests that this procedure will tend to concentrate all the
'mass' on one location and most of the weights will decrease to 0. This
concentration is 'bad' since eventually, all the weights go to zero and so
effectively the distribution has converged  artificially to a point, which might
be the most likely point from the initial ensemble, but still be far from the
'true' value of the parameter. This adverse effect can be ameliorated by
resampling the {\sl locations} of the particle ensemble, $\th_i$. This
resampling can be done in many ways, but a standard way is described in
algorithm. \ref{alg:particle_resampling} (which is copied from Algo4 in Granade
et. al. \cite{Granade2012}, who in turn closely follow \cite{Liu2001}, but
are very nice and pedagogical).
While, of course, updating happens after every iteration, re-sampling happens
only when  
$$ \frac{1}{\sum_i^{N_p} w_i^2} < \frac {N_p}{2} \implies \textrm{ resample!}$$
For reference, the complete filtering algorithm is provided in 
Algorithm \ref{alg:particle_resampling} in the Appendix.



% In \cref{fig:poissonian_rate_filtering} we give an example for the filtering
% procedure to estimate the rate of a Poissonian process, $1/\t$, for which  the
% likelihood is just $\tfrac 1\t exp(- \tfrac t\t)$. From
% \cref{fig:poissonian_rate_filtering} we can conclude that the basic mechanics of
% the filtering update/resample procedure is working. (We've tried it with other
% values of $\tau$ and it works for those as well.)
% 
% %\usepackage{graphics} is needed for \includegraphics
% \begin{figure}[htp]
% \begin{center}
%   \includegraphics[width=\textwidth]{Figs/TauParticleEnsemble/poisson_rate_filtering.pdf}
%   \caption{Filtering Example for a test-case. We are trying to estimate the
%   inverse of the Poissonian rate, $1/\tau$, given inter-event intervals $t_k$.
%   The true value is $\tau  = 1$. The Maximum Likelihood estimate for $\tau$ in
%   this problem  is just the mean of the observed times $\bar{t_k}$, which
%   happens to be 0.9583 in this example, while after the last observation, the
%   ensemble mean $\pm$ std-dev is $0.9811\pm 0.0199$}
%   \label{fig:poissonian_rate_filtering}
% \end{center}
% \end{figure}

% \clearpage

\subsection{Simulation Results}
We now illustrate how the Optimal Design procedure works while using the
particle filtering methodology to represent and update our parameter prior.

% \subsubsection{Single Hitting-Time Illustration}
% In \cref{fig:example_online_miopt_single_iteration}, we
% illustrate one iteration of the update, that is one hitting time given a
% stimulation from either the MI Optimal Controller or a random controller, which
% just gives a random constant stimulation per each hitting time.
% 
% Let's discuss what happens in \cref{fig:example_online_miopt_single_iteration}.
% Recall that lower values of $\tc$ imply higher restoring force and therefore
% longer hitting times (waiting times). Since in this sample, the observed hitting
% times were fairly long, especially for the MI-optimal stimulation, weights for
% smaller $\t$s grow larger, while weights for bigger $\t$s become smaller.
% However it is immediately clear that the MI-optimal stimulation is more
% discerning as it has almost entirely discarded (correctly) the
% possibility that $\tau>2$, while the other two stimulations have resulted in
% only mild perturbation in the prior distribution. 
% 
% WARNING: The new results are much better primarily through a judicious choice of
% the initial condition of the MI-optimal optimization (the initial guess for the
% optimal control.)
% 
% \begin{figure}[h]
% \begin{center}
% \subfloat[Example Hitting Times]
% {
% \label{fig:example_hitting_times_for_online_miopt}
% \includegraphics[width=0.48\textwidth]
% {Figs/HTOnlineEstimator/single_trial_example_hittimes.pdf}
% } 
% \subfloat[Resulting Particle Ensemble Updates]
% {
% \label{fig:example_particle_ensemble_updates_for_online_miopt}
% \includegraphics[width=0.48\textwidth]
% {Figs/HTOnlineEstimator/single_trial_example_weights.pdf}
% }
% \caption[Effect of the First Observation on the Belief
% Distribution]{Examples of a single iteration of the Online Stimulation-Estimation scheme.}
% \label{fig:example_online_miopt_single_iteration}
% \end{center}
% \end{figure}
% 
% \subsubsection{Full Multiple Hitting-Times Experiment}
We stimulate a sequence
of hitting times and online update our parameter prior distribution, after
every observation and then online-update our MI-optimal stimulation as the
prior evolves. 

% The main result is shown in \cref{fig:example_miopt_vs_rand_ensemble_evolution},
% where we visualize the mean and confidence intervals for the belief
% distributions for the three protocols (MI-Optimal vs. Random Constant vs Zero),
% using $N_\t = 32$ particles and $N_k=251$ hitting times. In
% \cref{fig:example_miopt_controls_evolution}, we show the different stimulations
% that were chosen by the Mutual-Info Maximization Algorithm.
 
% \begin{figure}[htp]
% \begin{center}
%   \includegraphics[width=\textwidth]{Figs/HTOnlineEstimator/single_experiment_example_ensemble_distn_evolution.pdf}
%   \caption[labelInTOC]{Evolution of the belief distributions given Optimal
%   (red) or Random (green) stimulation. We used $N=32$ particles to represent
%   the ensemble for both stimulation protocols. There are 250 hitting times used}
%   \label{fig:example_miopt_vs_rand_ensemble_evolution}
% \end{center}
% \end{figure}
% \begin{figure}[htp]
% \begin{center}
%   \includegraphics[width=\textwidth]{Figs/HTOnlineEstimator/single_experiment_example_controls_evolution.pdf}
%   \caption[labelInTOC]{Different MI-OPtimal Stimulations computed 
%   as the Experiment evolves and the parameter belief distribution changes}
%   \label{fig:example_miopt_controls_evolution}
% \end{center}
% \end{figure}
 
% \clearpage


We now simulate $50$ independent experiments of 500 hitting times each and
then average the ensembles.

% We visualize them in \cref{fig:online_optimization_more_examples}. Here, we
% see that the MI-optimal stimulation is indeed producing more accurate
% estimates, faster (earlier in the experiment). This is true in 7 out of the 10
% experiments, in 2 it is hard to determine whether there is a 'better' protocol
% and only once is the MI-based stimulation resulting in worse estimates than
% one of the alternatives ((f) in \cref{fig:online_optimization_more_examples}).
% \begin{figure}[h] \begin{center} \subfloat[] {
% \includegraphics[width=0.48\textwidth]
% {Figs/HTOnlineEstimator/single_experiment_exampleNts=32_Ntrls=495_ensemble_distn_evolution.pdf}
% } \subfloat[] { \includegraphics[width=0.48\textwidth]
% {Figs/HTOnlineEstimator/single_experiment_exampleNts=32_Ntrls=496_ensemble_distn_evolution.pdf}
% }\\
% \subfloat[] { \includegraphics[width=0.48\textwidth]
% {Figs/HTOnlineEstimator/single_experiment_exampleNts=32_Ntrls=497_ensemble_distn_evolution.pdf}
% } \subfloat[] { \includegraphics[width=0.48\textwidth]
% {Figs/HTOnlineEstimator/single_experiment_exampleNts=32_Ntrls=498_ensemble_distn_evolution.pdf}
% }\\
% \subfloat[] { \includegraphics[width=0.48\textwidth]
% {Figs/HTOnlineEstimator/single_experiment_exampleNts=32_Ntrls=499_ensemble_distn_evolution.pdf}
% } \subfloat[] { \includegraphics[width=0.48\textwidth]
% {Figs/HTOnlineEstimator/single_experiment_exampleNts=32_Ntrls=500_ensemble_distn_evolution.pdf}
% }\\
% \subfloat[] { \includegraphics[width=0.48\textwidth]
% {Figs/HTOnlineEstimator/single_experiment_exampleNts=32_Ntrls=501_ensemble_distn_evolution.pdf}
% } \subfloat[] { \includegraphics[width=0.48\textwidth]
% {Figs/HTOnlineEstimator/single_experiment_exampleNts=32_Ntrls=502_ensemble_distn_evolution.pdf}
% }\\
% \subfloat[] { \includegraphics[width=0.48\textwidth]
% {Figs/HTOnlineEstimator/single_experiment_exampleNts=32_Ntrls=503_ensemble_distn_evolution.pdf}
% } \subfloat[] { \includegraphics[width=0.48\textwidth]
% {Figs/HTOnlineEstimator/single_experiment_exampleNts=32_Ntrls=504_ensemble_distn_evolution.pdf}
% } \caption[labelInTOC]{Various examples of the perturbation-estimation
% protocol with the belief distribution plotted against the hitting time $k$,
% with the MI-optimal stimulation (red), the constant random stimulation (green)
% and the zero, $\a\equiv 0\,\forall t$, stimulation (blue)}
% \label{fig:online_optimization_more_examples} \end{center} \end{figure}
The aggregated evolutions for the updated prior distributions are
visualized in \cref{fig:online_optimization_aggregated_belief_evolution}. It is now clear that
the MI-Optimal procedure produces more accurate and more precise estiamtes and
that the increase in precision is especially clear in the earlier parts in the
experiment. That is, using the MI-optimal protocol reduces the uncertainty in
the parameters much faster.

In \cref{fig:online_optimization_quantiles_belief_evolution}, we plot a similar
illustration, here we show the median of the mean of the prior (median over the
$N=16$ experiments) as well as the min and the max. We see a similar effect in
that the 'opt'-estimates are much more accurate and precise. 

TODO: Should we include only one of these figs - they are kinda showing the
same thing?
 
% \usepackage{graphics} is needed for \includegraphics
\begin{figure}[htp]
\begin{center}
  \includegraphics[width=\textwidth]{Figs/HTOnlineEstimator/online_updated_prior_mean_aggregated_ensemble.pdf}
  \caption[MI-Optimal Stimulation produces more precise parameter estimates]
  {MI-Optimal Stimulation produces more precise and more accurate parameter
  estimates. Visualized are the mean and confidence intervals for the parameter
  distribution of $\log (\tau)$, averaged over $N=50$ independent
  experiments. The black line indicates the 'true' value of the unknown parameter $\tau$.
  The solid (red,green and blue) lines indicate the mean of $\rho_{avg}$, while
  the dashed lines indicates 2 standard deviations in either direction of the
  mean. }  
  \label{fig:online_optimization_aggregated_belief_evolution}
\end{center}
\end{figure}

%\usepackage{graphics} is needed for \includegraphics
\begin{figure}[htp]
\begin{center}
  \includegraphics[width=\textwidth]{Figs/HTOnlineEstimator/online_updated_prior_quantiles_mean_per_experiment.pdf}
  \caption[Quantiles of the Mean Estimates]{Visualized are the median and
  2.5/97.5th percentiles of the mean of the updated prior
  distribution over the $N=50$ experiments. I.e.\ for each experiment and each
  hitting time, $k$, we compute the mean of the updated prior $\rho(\tc)$ and then we plot the median
  )crosses) of these means as well as the min and the max (dashes). Again we see
  that the median and the min/max of the optimally stimulated estimates are much closer to the
  true value of the parameter, $\tau$ then the estimates obtained from the
  naive stimulations.}
  \label{fig:online_optimization_quantiles_belief_evolution}
\end{center}
\end{figure}

\section{Discussion}
\label{sec:discussion}
We introduce a method for optimal design given hitting-time observations from a
Fokker-Planck system. Our method is based on maximizing the Mutual Information
between the observed hitting times and the parameter posterior distribution
of the parameters. The optimal control tends to separate
the hitting-time distributions associated with alternative values of the unknown
parameter, as such facilitating the identification of the parameter once an
observation is made. 

Our optimal stimulation selection uses PDE-based adjoint
optimization methods, which can be expensive and sensitive to the various
parameters of the optimization. 

The simulations show that the resulting estimates from the
optimally-stimulated system have higher precision and accuracy than sensible
alternatives, such as random stimulation or 'critical' stimulation or no
 stimulation at all.
 
TODO: WITHOUT BEING VERBOSE, GO OVER WHAT HAPPENED IN THE PAPER:)

Also compare with Existing Lit (e.g. Hooker paper)

Stress Novelty (of hitting time problem)
 

\clearpage
\section{Mutual Information definition}
\label{sec:mutual_info_defn} 

There are two standard references on Information Theory, 
\cite{Cover2006,MacKay2003}. The Mutual Information, $I$ between two random
variables, $X,Y$ is defined, for example, in Chapter 8 of \cite{MacKay2003},
where it says that $I(X,Y)$ represents the average reduction in uncertainty
about $x$ that results from learning the value of $y$ or vice versa! 

Here we show why our \cref{eq:J_mutual_info_objective} for the Mutual
Information agrees with the usual definition of the $I$,
which for the random variables, $X,\Theta$ is
\begin{equation} 
I(X,\Theta) = \int_\Theta \int_X p(x,\th) \cdot \log \left(
\frac{p(x,\th)}{p(x)p(\th)}\right) \intd{x} \intd{\th}
\label{eq:mutual_info_defn}
\end{equation}
 
First of all, the marginal distribution, $p(\th)$,  is just the prior of
$\Theta$, $$p(\th) = \rho(\th)$$ The joint distribution is $$p(x,y) =
L(x|\th)\rho(\th)$$ while the $X$ marginal is $$p(x) = \int_\Theta L(x|\th)\rho(\th) \intd{\th}$$
Plugging the three expressions into the definition in
\cref{eq:mutual_info_defn} gives:
\begin{equation}
I = \int_\Theta \int_X L(x|\th)\rho(\th) \cdot 
\log \left( \frac{L(x|\th)\rho(\th) }{\int_\Theta L(x|\th)\rho(\th) \intd{\th}
\cdot \rho(\th) } \right)
\intd{x}\intd{\th}.
\label{eq:mutual_info_prior_trajectory}
\end{equation}
And after canceling $\rho(\th)$ inside the $\log$, we get
\cref{eq:J_mutual_info_objective} .

\clearpage

\section{Pseudo-Code for Algorithms}
\begin{algorithm}
\begin{algorithmic}
\State Fix $\T, \ldots$ the problem parameters
\State Fix $\{t_n\}_0^{N_t}$ a time-discretization of $[0,\T]$
\State Fix $g_{tol}$ a convergence tolerance for the gradient
\State Fix $K_{\max}, I_{\max}$ number of maximum iterations in outer, inner
loops
\State Fix $s$ the initial step-size. 
\\ {\itshape $\#$ we use $g_{tol}=1e-5,K_{\max}=100,I_{\max}=10,s=10$}
\State $\a_1(t) \gets (\amax-\amin) \cdot t / \T + \amin$ 
\\{\itshape  $\#$ $\a_1(t) \sim$ initial guess for the control, linear
interpolate between $\amin, \amax$}
\For { $k= 1\dots K_{\max}$} 
\\ {\itshape $\#$ This is the outer loop where we descend down different
gradients}
 \State Calculate $f_k,I_{k},p_k, \delta_\a I_k$ corresponding to
	$\a_k$ from
	\cref{eq:hamiltonian_gradient,eq:adjoint_pde_OU,eq:FP_pde_OU_PDF,eq:OC_LS_variance_density}
	\State $N_{active}\gets$   Number of time nodes $t_n$, where either
	$\a_k(t_n) \neq \{\amin, \amax\}$ or $\delta_\a I_k(t_n)$ points inwards
	\If{ $|| \delta_\a I_k||_{\R^{N_{active}}} \leq g_{tol}\cdot N_{active}$}
		  \\ {\itshape  $\#$ 'Active' gradient is small enough,
		 consider converged:}
		 \State BREAK
	\EndIf
	\\ {\itshape $\#$ Find the step size, $s$, of how far to move $\a$ in the
	direction $\delta_\a I_k$:}
	\For { $j= 1\dots K_{\max}$}
	\\ {\itshape $\#$ This is the inner loop where we find how much to descend down
	the current gradient}
	\State $\a_{k,j} \gets \a_{k} + s_j \cdot \delta_\a I_k  $
	\\ {\itshape $\#$ $\a_{k,j}$ is the new control strategy to try}
	\State Calculate $f_{\th,k,j}, I_{k,j}$ corresponding to
		$\a_{k,j}$ from
		\cref{eq:FP_pde_OU_absorbBC,eq:I_mutual_info_objective_in_terms_of_dixf} \\
		{\itshape $\#$ Recall $f_{\th,k,j}$ is a probability density resulting from
		the control $a_{k,j}$ and $I_{k,j}$ is the objective value resulting from $f_{k,j}$}  
	\If {$I_{k,j} < I_k$}
		\\ {\itshape $\#$ We found a better (smaller) objective value}
		\State $s \gets 2 s_j$ {\itshape $\#$ start from a more aggressive step in the
		next iteration}
		\State BREAK
		\EndIf
	\If {$j == J_{\max}$}
		\\ {\itshape $\#$ We exhausted the step search without finding a 
		smaller $I$, return current values}
		\State Return $I_k, \a_k$
	\EndIf
 	\\ {\itshape  $ \#$ Continue inner loop, try a smaller step:}
	 	\State $s_{j+1}  \gets  s_j / 2$
    \EndFor  {\itshape $\quad \#$ single step loop}
	\If {$k == K_{\max}$}
		\State ERROR 'Could not converge'
	\EndIf
    \\{\itshape $\#$ Assign the new candidate for the optimal control and
    re-loop}
		\State $\a_{k+1} \gets \a_{k,j}$
\EndFor {\itshape $\quad \#$ gradient ascent loop}
\State \Return $J_k, \a_k$
\end{algorithmic}
\caption{Gradient ascent algorithm for obtaining the optimal open-loop 
control}
\label{alg:gradient_ascent_4_OC}
\end{algorithm}

\clearpage 


\begin{algorithm}
\begin{algorithmic}
\State Given  $\{w_i, \th_i\}_1^N_p$ the current particle ensemble (weight,
locations)  
\\ {\itshape $\#$ Observe a single hitting-time, $t_k$ from the real (or
simulated) system conditional on the applied control $\a_k(\cdot)$}
\\ {\itshape $\#$ Update weights with likelihood given $t_k$ and normalize:}
\State $w_{i } \gets w_{i }\cdot g(t_k| \th_i, \a_k)$
\State $w_{i } \gets w_{i }/ \sum w_i$
\If{$$\frac{1}{\sum_i^{N_p} w_i^2} < \frac {N_p}{2}$$} 
\\ {\itshape $\#$ Re-sample the ensemble:}
\State $\m $ \gets
$\Exp[\Th] = \sum w_i \th_i$ 
\State $a = 0.98$, see
\cite{Granade2012,Liu2001} 
\State $h = \sqrt{1-a^2}$ i.e. $h \approx 0.1990$
\State $\Xi$\gets $h^2\cdot \Var[\Th]$
	\\ {\itshape $\#$ Re-sample each particle individually:}
\For {$i= 1\dots N_p$}
	\State draw $j$ with probability $w_j$ 
% 	\\ {\itshape $\#$ the bigger $w_i$ the more likely to choose
% 	$\th_i$}
	\State $\m_i$ \gets  $a \th_j + (1-a) \m$
	\State Resample $\th_i$ from $N(\m_i, \Xi)$
	\State $w_i$ \gets $1/N_p$ 
    \EndFor{\itshape $\quad \#   1 \ldots N_p$ particle resampling}
    \EndIf{\itshape Conditional Resampling}
\State \Return $w_i, \th_i$ the updated and possibly-resampled particle
ensemble
\end{algorithmic}
\caption{Particle Filtering for Parameter Estimation}
\label{alg:particle_resampling}
\end{algorithm}

\section{Code Repository}
All the code used in the paper including code to generate the figures can be
found on
\href{https://github.com/aviolov/OptEstimatePython}{github @
https://github.com/aviolov/OptEstimatePython}. Please contact the first author
for furhter information.



\section{Extra Stuff}
TODO: Move all discussion of Double WEll here\ldots


 
\clearpage
\cleardoublepage

\chapter{Conclusion and Outlook}
\label{ch:conclusion}
Ah, la conclusion:-) 

In the first task, of estimating, one can see that there remains one interesting
problem, and that is why the estimation quality degrades with higher sinusoidal
frequency in particular for one of the two methods (the
Fokker-Planck / Likelihood based method).

In the second task, the possible avenues for progress are much broader\ldots 

The third task is currently barely scratched. Optimal Design for SDEs is still
an opening area, but hopefully, much more work will be done especially in the
areas of neuroscience and physiology, where there are many systems which are
amenable to direct external perturbation.


\cleardoublepage
% \bibliographystyle{plain}
%TODO: Use JNE biblio-style???
\bibliographystyle{bmc_article}
%TODO: Check for ?? (include local.bib from the various sub-folders) 
\bibliography{library,local}

\end{document}
