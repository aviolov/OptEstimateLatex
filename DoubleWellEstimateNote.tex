%%This is a very basic article template.
%%There is just one section and two subsections.
\documentclass{article}

\usepackage{amsmath} 
\usepackage{amscd}
\usepackage{amssymb}
\usepackage{amsfonts}
\usepackage{amsthm}
\usepackage{amsfonts}
\usepackage{amsthm}

\usepackage{circuitikz}
\usepackage{pgf}
\usepackage{tikz}
\usetikzlibrary{arrows,snakes,backgrounds}
% \usetikz
\usepackage{subfig}

\usepackage[super]{nth}
% \usepackage{appendix}
% \usepackage{listings}
% \usepackage{color}
\usepackage{ulem}
\usepackage{hyperref}
%\usepackage{url}
\usepackage{cancel}
\usepackage{cleveref}
 
\usepackage{aviolov_style}
\usepackage{local_style} 


\begin{document}


\title{Estimating the Double WEll} \author{Alexandre Iolov, Susanne
Ditlevsen, Andr\'e Longtin  \\ $<$\href{mailto:aiolo040@uottawa.ca}
		{aiolo040 at uottawa dot ca}$>$, alongtin at uottawa dot ca}

\date{\today}

\maketitle

\abstract{Supposed we have a double --well potential SDE, how do we estimate
its parameters}

\tableofcontents

\section{The Double Well}
Consider the generic, parametrized,  Double-Well System 
\begin{equation}
dX = U(X; \th) \intd{t} + \sqrt{2D(X; \th)} dW
\label{eq:SDE_evolution} 
\end{equation}
where $U$ is a potential field gradient, $U = -\grad V$ for example something
like
\begin{equation}
dX = \underbrace{-\left( aX^3 - bX -A \frac Xc e^{-(X/c)^2/2} \right) }_{U(X;
\th)} \intd{t} + \underbrace{\s}_{\sqrt{2D}} dW
\label{eq:OU_evolution} 
\end{equation}
Whose parameter set is:
$$
\th = \{a, b, A, c, \s\}
$$
or a subset thereof.

We want to estimate $\th$  from observations $\{ X_t \}$. Now assume that we
have very high frequency (cts.!) observations, $\{X_{k\Delta t}\}_0^K$ such that
$\Delta t \ra 0, K\Delta t \ra T$, i.e.\ we know the full realization
$\{X_t\}_0^T$ over some interval $[0,T]$.

We want to estimate the values of $\th$. 

\subsection{Estimating $\s$ for cts. observations}

We now discuss why the estimation of $\s$ can be fairly accurate in the
high-frequency (cts.) context. 

Suppose, for illustration sake that $\s$ is constant. 

Then recall that the quadratic variation formula for an Ito Diffusion (such as
$X_t$) states that:
\begin{equation}
\lim_{\Delta \ra 0} \sum_{k\geq 1} (X_k - X_{k-1})^2 = \int_0^T \sigma(X_t)
\intd{t}   
\label{eq:sigma_est_quad_variation}
\end{equation}
i.e for $\Delta$ very small and $\s$ constant, we will have:

\begin{equation}
\sigma \approx \frac 1T \sum_{k\geq 1} (X_k - X_{k-1})^2
\label{eq:sigma_est_quad_variation_sigma_constant}
\end{equation}
and the probability of estimating $\s$ incorrectly due to $X$ fluctuations goes
to zero.

Note that this formula does not depend on the other possibly unknown parameters
in the drift.
% Furthermore, this does not depend on the exact nature of $U$, only on the
% observations $X_t$. Although, in the finite $\Delta$ case there will be some
% effect of the drift parameters in $U$ on the approximation of $\s$. 
% 
% The converse, that $\s$ does not impact estimates for the drift parameters is
% also possible, for example, in the Ornsteing-Uhlenbeck process, the estimates
% for the time-constant and the equilibrium are independent of the  estimate for
% $\s$, although $\s$ indirectly influences them though its effect on $X$. 
 

\Cref{eq:sigma_est_quad_variation} can readily be used for
inference when $\s$ is NOT constant.

I.e if we think that $\sigma$ is a nonlinear function of $X$ such as for example
$\s(X) = \s_0 + \tanh(X_t/d)$, then one needs to employ a nonlinear root-finder 
for the equation:
$$
\lim_{\Delta \ra 0} \sum_{k\geq 1} (X_k - X_{k-1})^2 = \int_0^T \s_0 +
\tanh(X_t/d) \intd{t} 
$$

Finally, one can try a non-parametric approach to estimate $\s(X)$ over
different segments of the realized $X_t$ (e.g. in the high state vs. low state
for the double-well potential).

Finally, one can devise all kind of SDE-based estimations to estimate all the
parameters, $\th$ simultaneously \ldots



 
\bibliographystyle{plain} 
\bibliography{library,local}

\end{document}