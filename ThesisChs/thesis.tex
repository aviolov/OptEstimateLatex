%%This is a very basic article template.
%%There is just one section and two subsections.
\documentclass{report}

\usepackage{amsmath}
\usepackage{amscd}
\usepackage{amssymb}
\usepackage{amsfonts}
\usepackage{amsthm}
\usepackage{amsfonts}
\usepackage{amsthm}

\usepackage{circuitikz}
\usepackage{pgf}
\usepackage{tikz}
\usetikzlibrary{arrows,snakes,backgrounds}
% \usetikz
\usepackage{subfig}

\usepackage[super]{nth}
\usepackage{appendix}
\usepackage{listings}
% \usepackage{color}

\usepackage{algpseudocode}
\usepackage{algorithm}

\usepackage{hyperref}
%\usepackage{url}

\usepackage{cancel}
\usepackage{slashbox}
\usepackage{cleveref}

\usepackage{aviolov_style} 
\usepackage{local_style}

\newtheorem{thm}{Theorem}[section]
\newtheorem{lemma}{Theorem}[thm]
% \theoremstyle{definition}
\newtheorem{ex}{Example}[thm]
\newtheorem{defn}{Definition}[thm]

\includeonly{chapters/ch2_Math}
% \includeonly{chapters/ch2_Math,chapters/ch6_OptDesign}
% \includeonly{chapters/ch6_OptDesign}
%%
\begin{document}

\pagenumbering{roman}
\pagestyle{empty}
\begin{titlepage}
\centering
\vspace*{1in}
\begin{Large}\bfseries
Parameter Estimation, Optimal Control and Optimal Design in Stochastic Neural
Models
\par
\end{Large}
\vspace{1.5in}
\begin{large}\bfseries
Alexandre V. Iolov\par
\end{large}
\vfill
A Thesis submitted for the degree of Doctor of Philosophy
\par
\vspace{0.5in}
Department of Mathematics
\par
University of Ottawa
\par
\vspace{0.5in}
%TODO: date: September 2015
\today
\par
\vspace{0.5in}
\includegraphics[width=0.15\textwidth]{./UOlogoBW.jpg}
\par
\end{titlepage}

\pagestyle{plain}
\abstract{This thesis poses and solves estimation and control problems
in computational neuroscience, in particular problems dealing with the
stochastic nature of neural systems. The main tool used is the description of
the system by a Fokker-Planck partial differential equation for evolution of
probability densities. 

The thesis deals with three problems in escalating degree of mathematical
sophistication and computational difficulty

The thesis focuses on developing computational schemes in order to solve the
problems. The schemes are tested for a wide range of parameters to demonstrate
their robustness}

% TODO: Acknowledgements
% \chapter*{Acknowledgements}
% I would like to thank my supervisor Andre Longtin, Sussane Ditlevsen at KU,
% colleagues at Ottawa and Copenhagen, parents, \ldots and Kirsten, and Daniel
% and Bombi
 
\tableofcontents 
\cleardoublepage

%TODO (caption[] for each table for list of tables): 
\listoftables
\clearpage 
%TODO (caption[] for each figure for list of figures): 
\listoffigures

\pagenumbering{arabic}
\pagestyle{headings} 


\chapter{Introduction}
This thesis deals with problems in parameter estimation and
stochastic optimal control arising in computational neuroscience. 

Neuroscience is the study of how information is processed by living beings. Its
main building block is a single neuron cell. A neural cell maintains a certain
membrane potential, $v(t)$, an electrochemical gradient between its interior and
exterior, which is the key mechanism in which information is processed and
transmitted. For our purposed we will assume that $v$ is essentially uniform
throughout the neuron - that is we will idealize the neuron as a single point in
space. In all neural systems, information is processed and transmitted by the
sharp, transient changes in the membrane potential called spikes (TODO: see a
figure). Currently, the dominant theory is that the exact shape of the potential
excursion is irrelevant, but that all information is contained entirely in the
time occurrence of the spike or equivalently the information is in
the duration of the interval between two subsequent spikes.

There is a rich history on modelling the dynamics of the membrane potential
starting from the Nobel-prize winning work of Hodgkin and Huxley in the 50's.
Their model, still widely considered as a benchmark for the dynamics of the
membrane potential, is a 4-dimensional ordinary differential equation (ODE),
where one of the states is the voltage itself and the other three are phenomenological
equations describing the behaviour of ion channels responcible for the
generation of the non-linear excursion - the spike itself. There has been many
reductions of the Hodgkin and Huxley model to a smaller dimension, including the
Fitzhugh-Nagumo model and the Morris-Lecar model, which still retain the basic
excursion non-linearity of the spiking mechanism. A even-more drastic
simplification is to linearize the dynamics and then impose the non-linear
voltage excursion of the neural spike artificially. This leads to the leaky
integrate-and-fire (LIF) model, which
forms the foundation of all the models used in our work.

Neural cell response, especially {\sl in vivo}, is stochastic. Given the same
stimulus the resulting inter-spike intervals will not be the same. There are
several reasons and explanations for why that is. The simplest one is that in
addition to the stimulus the neuron is bombarded by extraneous, random
stimulation, for example from other neurons. To account for this random
behaviour the dynamics of the membrane potential, $v(t)$, are modelled as
following a stochastic differential equation (SDE). SDEs are a generalization
of ODEs, which allow for a stochastic force-term. 

This thesis deals with two main topics - estimation of single-cell neural models
and control of neural dynamics. Mathematically, this amounts to estimation of
parameters in SDEs and control of SDEs. A complication arising from the
practical aspects of neuroscience is that often experimentally the exact value
of the voltage is difficult to observe and it is only the spikes that are
distinctly observable. Mathematically, this implies that our estimation
and control algorithms will often be based on observation of {\sl first
passage} times of the system rather than detailed trajectories. This makes the problem
significantly more challenging and requires some non-standard
algorithms. At the same time, the study of First-Passage Times (FPTs) is a
classic part of Stochastic Analysis and there are many known
techniques and results that we can build on.

The three problems that are discussed in this thesis can be summarized as
such: in the first problem, we estimate parameters from spike observations; 
in the second, we discuss how to control a neuron to spike at some pre-selected
time; and in the third problem, we combine both approaches whereas we
control the system in order to best estimate its parameters.

The thesis is structured as follows: First, in \cref{ch:math_background} we
introduce the mathematical background from stochastic processes, parameter
estimation and optimal control used in the rest of the thesis; in
\cref{sec:math_models_in_neuroscience} we discuss the standard stochastic models
of a single neuron. The next three chapters,
\cref{ch:estimate,ch:spike_control,ch:optimal_design} form the novel portion of
this thesis where we discuss each of the three problems introduced above. These
three chapters have been adapted from journal papers either accepted for
publication or currently in review. \Cref{ch:estimate} has been published in
TODO: (cite it); \cref{ch:spike_control} has been submitted in TODO: (cite it);
and \cref{ch:optimal_design} is being prepared for submission at \ldots.
Although the journal articles form the foundation of the chapters, each chapter
has been rewritten to achieve a uniform notation and to fit into the overall
flow of the thesis.

In the Conclusion, \cref{ch:conclusion}, we summarize the main findings and
provide a brief outlook.

\cleardoublepage
\chapter{Mathematical Background}
\label{ch:math_background}
\input{../OptEstimate/local_style.sty}
Here we collect a list of mathematical tools that are used in the thesis

We cite amongst others Oksendal for SDEs \cite{Oksendal2007} and Fleming and
Rishel for Optimal Control \cite{Fleming1975}.

A very readable introduction to the field of both SDEs and Optimal Control are
the online notes of Professor L. Evans \cite{Evansa,Evansb}. We have also used
Jacobs as our main tutorial on first-passage times for SDEs \cite{Jacobs}. 
 
\section{Stochastic Differential Equations}
\label{sec:SDEs}
In view of our ultimate goals, we will restrict ourselves to  stochastic
processes whose sample paths have continuous paths, i.e.\ to SDEs driven by
Brownian motion.  

Since the reader is more likely well familiar with the material in this section,
we will not provide proofs, but only state the results with a view towards
establishing the notation for the sequel.

We will assume that the reader is familiar with the following concepts:
\begin{itemize} 
  \item a probability space, $\{\O, \sAlg, P\}$ consisting of a
  probability space, a sigma-algebra and a probability measure
  \item a continuous-time stochastic process, $X_t$
  \item a filtration $\Fil(t)$ and in particular the filtration generated by a
  stochastic process.
\end{itemize}

\subsection{The Wiener Process and the Ito Integral}
The Wiener Process is the fundamental building block of the stochastic calculus,
it is often called Brownian Motion and we will denote it $\{W_t\}_{t\geq 0}$. It
satisfies:
\begin{defn}Wiener Process, $W_t$:
\begin{enumerate}
  \item $W_0 = 0$
  \item $W_t - W_s = N(0, |t-s|)$ , i.e. normally distributed increments with
  mean 0 and variance $|t-s|$)
  \item $\forall \{t_i\}_1^N, \quad \{W_{t_i} - W_{t_{i-1}} \}_2^N \sim$
  independent, i.e $W_t$ has independent increments
\end{enumerate}
\end{defn}
The fact that the finite incremental distributions suffice to specify a unique
continuous-time stochastic process is known as Kolmogorov's extension theorem.
Now, we collect a few more relevant properties of $W_t$:
\begin{enumerate}
  \item The sample paths $W_{[0,\infty)}(\o)$
are almost surely (a.s.) continuous and are in fact Holder continuous for any
exponent $\g < 1/2$
\item The sample paths $W_{[0,\infty)}(\o)$ are nowhere differentiable
\item $W_t$ is a Markov process: $\Prob[W_t \in B | \s(W_{s' \leq s })] =
\Prob[W_t \in B \,| \,W_s]$, for any Borel set, $B \subset \R$, where $\s(W_{s'
\leq s })$ is the filtration generated by the $W_t$
\end{enumerate}



We now turn to defining stochastic integrals based on the Wiener Process.
 
\begin{defn} Progressively Measurable Functions:

Let $\Fil_t$ be the filtration generated by the Wiener Process.

Let $X_t$ be a stochastic process which is $\Fil_t$-measurable $\forall t$ and
which is jointly measurable in $(t,\o)$. We call such an $X_t$
\emph{progressively measurable}
\end{defn}

\begin{defn} $\Ltwopm, \Lonepm$

We define $\Ltwopm[0,T]$ as the space of all progressively measurable
$X$ such that
\begin{equation*}
\Exp \left[ \int_{[0,T]} X_t^2 \intd{t} \right] < \infty
\end{equation*}

Similarly, we define $\Lonepm[0,T]$ as the space of all progressively measurable
$X$ such that
\begin{equation*}
\Exp \left[ \int_{[0,T]} X_t \intd{t} \right] < \infty
\end{equation*} 
\end{defn}

$\Ltwopm$ will be the class of functions for which the Ito integral is
well-defined. 
\begin{defn} Ito Integral:
\label{defn:ito_integral}

Let $P^n := {a = t^n_1 \ldots t^n_{m_n} = b}$ be a partition of the interval
$[a,b] \subset [0, \infty)$. Let $|P_n| = \sup_i|t_i - t_{i-1}|$. 
Take $|P_n| \rightarrow_n 0 $ and consider an $ X_t \in \Ltwopm[a,b]$,

then
\begin{equation}
\int_{[a,b]} X_t \intd{W} := \lim_{n \rightarrow \infty}  
\sum_{i=1}^{m_n} X_{t_i}\left( W(t_{i+1}) - W(t_{i})\right)
\end{equation}

\end{defn}

To be precise, our definition is actually a theorem, and the real
definition is one that uses step functions and passes to the limit.  Also we
will write the limits of integration $\int_0^T \cdot  \intd{W}$ or $\int_{[0,T]}
\cdot  \intd{W}$ interchangeably.
% To justify def'n \ref{defn:ito_integral}, we recall:
% \begin{lemma}[Quadratic Variation]
% \begin{equation}
% \lim_{n \rightarrow \infty} \sum_{i=1}^{m_n} \left( W(t_i) - W(t_{i-1})\right)^2
% = b-a \quad \text{in }  L^2(\O)
% \end{equation}
% \end{lemma}
% The above can be read, heuristically as $\lim \sum (\Delta W)^2 \rightarrow \sum
% \Delta t$, which is the meaning behind the colloquial $\intd{W} =
% \sqrt{dt}$.

Again, we state without proof a few interesting properties of $\int X \intd{W}$:
\begin{thm} Ito Integral Properties

\begin{enumerate}
  \item $\Exp[\int_0^T X \intd{W} ] = 0$ 
  \item $\Exp[\left(\int_0^T X \intd{W}\right)^2 ] = \Exp[\int_0^T X^2
  \intd{t}]$
  \item $I(t) = \int_0^t X \intd{W} $ is a martingale 
  \item $I(t) = \int_0^t X \intd{W} $ has continuous sample paths
\end{enumerate}
\end{thm}

\subsection{Ito SDEs and Ito's Lemma}
\begin{defn}[Ito SDE]

We write
\begin{equation}
dX_t = Fdt + G dW
\end{equation}
on $0 \leq t \leq T$, if $X$ is a real-valued stochastic process satisfying:
\begin{equation*}
X(r) = X(s) + \int_{[s,r]} F \intd{t} + \int_{[s,r]} G \intd{W}
\quad 0\leq s \leq r \leq T
\end{equation*}
for some $F \in \Lonepm[0,T]$ and $G \in \Ltwopm[0,T]$
\end{defn}

We are now ready to present the celebrated Ito Lemma which is the chain-rule of
stochastic calculus:

\begin{thm}[Ito Lemma]

Suppose $X$ satisfies the stochastic differential $dX_t =
Fdt + G dW$ as above and take $v(x,t) \in C^{2,1}[ \R \times [0,T]]$.

Set $Y(t) = v(X,t)$
\\
then
$$
dY =  \left( \di_t v + \di_x v \cdot F + \di^2_x v \cdot \frac{G^2}2 \right)
\intd{t} + \left(   \di_x v\cdot G  \right)\intd{W}
$$
\end{thm}

% We now present two examples, one fundamental and one more involved:
% \begin{ex}[pg 67 in \cite{Evansa}] The 'Stochastic' Exponential:
% 
% Let $Y = \exp(\l W - \l^2 t / 2)$. In the notation of the above Thm, we have:
% $dX = dW, F=0, G=1$ and $u(x,t) = \exp(\l x - \l^2 t / 2)$.
% 
% then
% \begin{align*}
% dY =& 
% \left( \di_t u + \di_x u \cdot F + \di^2_x u \cdot \frac{G^2}2 \right)\intd{t} +
% \left(   \di_x u \cdot G  \right)\intd{W}
% \\
% =&
% \left( -\frac{\l^2 t}{2}\exp(\l x - \l^2 t / 2) + \l^2\exp(\l x - \l^2 t / 2)  \right)\intd{t}
% \\ 
% &+ \left(   \l \exp(\l x - \l^2 t / 2) \right)\intd{W}  
% \\
% =&  \l \left( \exp(\l x - \l^2 t / 2) \right)\intd{W}  
% \\
% =& \l Y dW
% \end{align*}
% \end{ex}
% 
% \begin{ex}[pg. 67 in \cite{Evansa}] Hermite Polynomials:
% 
% Let $h_n(x,t) = \frac{(-t)^n}{n!} e^{x^2/2t} \di^n_x \left( e^{x^2/2t} 
% \right)$ be the $n$th Hermite polynomial
% \\
% then
% $$
% \int_0^t h_n(W, t) \intd{W} = h_{n+1}(W(t), t)) 
% $$
% i.e. in Ito calculus, the Hermite polynomials are like the power monomials
% $\frac{t^n}{n!}$ in the ordinary calculus.
% \begin{proof} Since
% $$
% d_\l^n [\exp(- \frac{(x-\l t)^2}{2t} ] |_{\l=0} =
%  (-t)^n d_x^n [ \exp(-\frac{x^2}{2t})] $$
% We will have:
% \begin{align*}
% d_\l^n [\exp(\l x - \l^2 t / 2) ] |_{\l=0} &= (-t)^n  e^{x^2/2t} d_x^n [
% \exp(-\frac{x^2}{2t})]
% \\
% &= n! h_n(x,t)
% \end{align*}
% Thus we can expand $\exp(\l x - \l^2 t / 2) $ in a Taylor series at $\l=0$ as:
% $$
% \exp(\l x - \l^2 t / 2) = \sum_n \l^n h_n(x,t)
% $$
% Now recall $Y$ from the previous example:
% \begin{equation*}
% Y(t) = \exp(\l W(t) - \frac{\l^2 t }2) = \sum_n \l^n h_n(W(t),t)
% \end{equation*}
% which satisfies:
% \begin{equation*}
% Y(t) = 1 + \l \int Y \intd{W}
% \end{equation*}
% Naturally we will plug the series expansion in the SDE to obtain:
% \begin{align*}
% \sum_{n=0}^{\infty} \l^n h_n(x,t) =&
% 1 + \l \int_0^t  \sum_{n=0}^{\infty} \l^n
% h_n(x,t)\intd{W}
% \\
% =& 1 + \sum_{n=1}^{\infty} \l^n\int_0^t 
% h_{n-1}(x,t)\intd{W} \quad {\textrm{bump $n$ up by one}}
% \end{align*}
% Equating the coefficients for $\l^n$ on both sides leads to the result
% \end{proof}
% \end{ex} 

Finally, we are in a position to define an Ito SDE:

\begin{defn}[Ito SDE] Let
$F(x,t):\R\times [0,T]\ra \R,  G(x,t):\R\times [0,T]\ra \R,$ be given functions.
We say that a stochastic process $X$ satisfies:
\begin{equation}
dX =F dt + G dW
\end{equation} 
over $[0,T]$ if
\begin{enumerate}
  \item $X$ is progressively measurable wrt. $\Fil_t$
  \item $F(X,t) \in \Lonepm$
  \item $G(X,t) \in \Ltwopm$ 
  \item $X_t = X_0 + \int_0^t F(X_s, s) \intd{s} + \int_0^t G(X_s, s) \intd{W_s}$
\end{enumerate}
\end{defn}

% \begin{ex} The solution to 
% $$
% \begin{cases}
% dX = F(t)X dt + G(t) X dW
% \\
% X_0 = 1
% \end{cases}
% $$
% is 
% $$
% X_t = \exp\left( \int_0^t F(s) - \frac{G^2(s)}{2} \intd{s} + \int_0^t G
% \intd{W} \right) $$
% \begin{proof}
% Let $Y = \int_0^t f(s) - \frac{g^2(s)}{2} \intd{s} + \int_0^t g
% \intd{W} $
% and guess that $X = e^Y$
% then, using Ito's lemma:
% \begin{align*}
% dX =& (\di_y [e^y] dY + \frac{1}{2}\di^2_y[e^y]  g^2 dt
% \\ =&
% (e^y  f dt - e^y g^2 / 2) dt + e^y g dW + \frac{1}{2} e^y \cdot  g^2 dt
% \\ =& e^y \left( f dt +g dW \right)
% \\ =& X \left( f dt +g dW \right)
% \end{align*}
% \end{proof}
% \end{ex}

% \begin{ex}[Brownian Bridge, pg 80 in \cite{Evansa}] The solution of the SDE
% $$
% \begin{cases}
% dB = -\frac{B}{1-t} dt + dW
% \\
% B_0= 0
% \end{cases}
% $$
% is
% $$
% B(t) = (1-t) \int_0^t \frac{1}{1-s} dW
% $$
% \begin{proof}
% It is not clear how to apply Ito's lemma to this problem, so instead we just
% directly calculate the differential:
% \begin{align*}
% B(t^+) - B(t) =& (1-t^+) \int_0^{t^+} \frac{1}{1-s} dW - (1-t) \int_0^t
% \frac{1}{1-s} dW
% \\ =&
% (1-t^+) \int_t^{t^+}\frac{1}{1-s} dW  - (t^+-t) \int_0^t
% \frac{1}{1-s} dW
% \\ \approx&
% \int_t^{t^+} dW - \Delta t \frac{B(t)}{1-t}  
% \end{align*}
% \end{proof}
% \end{ex}
% Actually we can say a bit more about the Brownian bridge, in particular we
% will calculate $\Exp[B^2]$
%  \begin{ex}[Evans .37]
% \begin{align*}
% \Exp[B^2] =& \Exp\left[\left( (1-t) \int_0^t \frac{1}{1-s} dW \right)^2 \right]
% \\
% =&(1-t)^2 \Exp\left[ \int_0^t\left(  \frac{1}{1-s}\right)^2 \intd{s} \right]
% \\
% =&(1-t)^2 [ \frac{1}{1-t} - 1 ]
% \\
% =&(1-t) - (1-t)^2
% \end{align*}
% 
% Further note that  $\lim_{t\ra 1^-}\Exp[B^2] = 0$ and that $B$ is a martingale
% and so its first moment is always equal to its initial value, $0$. So indeed the 2nd
% moment is the variance and with the variance converging to zero, we can use
% Chebyshev's inequality to conclude that $\lim_{t\ra 1^-}B(t) = 0 $(a.s). This
% justifies the name of the Brownian bridge - it is a normally distributed RV.
% which is clamped to equal 0 at both $t= \{0,1\}$.
% \end{ex}

As an example we will discuss the Ornstein-Uhlenbeck process, which is the
basis for the models we will face later on.
%TODO: Decide on whether it is (m-x)/t or m - x/t, once and for all!
\begin{ex}[O-U Process] Let $X$ follow:
\begin{equation}
dX = \left( \frac {\m -X_t}{\tc} \right) dt + \b dW
\label{eq:OU_equation_generic}
\end{equation}
with an initial condition, $X_0$, which may be an arbitrary distribution
independent of the Wiener Process, $W$.
We can solve this as follows:
%TODO: Solve OU with this parameter set:
\begin{align*}
dX =& \left( \frac {\m -X_t}{\tc} \right)  dt + \b dW
\\
dX + \frac {X_t}{\tc} dt=&  \frac\m\tc dt + \b dW
\\
e^{t/\tc} dX + e^{t/\tc}\frac {X_t}{\tc} dt
=& e^{t/\tc}\frac\m\tc dt + \b e^{t/\tc} dW
\\
Xe^{t/\tc} - X_0 
=& \int  e^{t/\tc}\frac\m\tc dt +  \int \b e^{t/\tc} dW
\\
X_t =& e^{-t/\tc} X_0 + \m(1-e^{-t/\tc}) +  \frac{\sqrt{\tc}\b
e^{-t/\tc}}{\sqrt{2}} W(e^{2t/\tc}-1)
\end{align*}
which means that $X$ forgets its initial conditions exponentially fast and
converges to a normal random variable with mean $\m\tc$ and variance
$\tfrac{\tc \b^2}{2}$
\end{ex}

At the end of this sub-section, we state Ito's Lemma with multiple Wiener
processes and then for a multidimensional state.

\begin{thm}[Ito Lemma for $dW \in  \R^{m}$] Suppose $X$ satisfies the stochastic
differential $dX_t = F dt + G dW$, where $ F \in \R, G \in \R^{1\times m}$ and
$dW = \left(dW^{(i)}\right) \in  \R^{m}$ is a vector of independent Wiener
Processes. Take $u(x,t) \in C^{2,1}[ \R \times [0,T]]$. Let $D :=\left[{\sum_k
G_{k} G_{k} } \right] \in \R$

Set $Y(t) = u(X,t)$
\\
then
$$
dY =  \left( \di_t u +  \di_{x} u \cdot F + 
  \di^2_{x} u \cdot D \right)
\intd{t} +
 \left(  \di_{x} u  \cdot  G \right) dW  
$$
\end{thm}

\begin{thm}[Ito Lemma for $X \in R^n$]
Suppose $X$ satisfies the stochastic differential $dX_t = F dt + G dW$, where $
F \in \R^n, G \in \R^{n\times m}$ and $dW = \left(dW^{(i)}\right) \in  \R^{ 
m}$ is a vector of independent Wiener Processes. Take $u(x,t) \in C^{2,1}[ \R^n
\times [0,T]]$. Let $D_{ij} := \tfrac{1}{2} \left[{\sum_k G_{ik} G_{jk} }
\right]_{ij} \in \R^{n\times n} $

Set $Y(t) = u(X,t)$
\\
then
$$
dY =  \left( \di_t u + \sum_i \di_{x_i} u \cdot f_i + 
\sum_{i,j} \di^2_{x_i x_j} u \cdot D_{ij} \right)
\intd{t} +
 \left(  \sum_{ij} \di_{x_i} u  \cdot G_{ij} dW^{(j)} 
\right)$$
\end{thm} 

\subsection{Fokker-Planck and Kolmogorov's Backward Equations}
The Fokker-Planck equation and Kolmogorov's Backward equation describe the
forward (resp. backward)  evolution of the probability density of $X_t$

For the rest of this section we will work in $ \R^n$, i.e. $X$ will be an $n-$
dimensional vector that satisfies the stochastic differential 
\begin{equation}
dX_t = F(X,t) dt + G(X,t) dW,
\label{eq:generic_Ito_SDE_Rn}
\end{equation}
$F \in \R^n, G \in \R^{n \times m}$,
$dW = (dW^{(i)}) \in  \R^{  m}$ is an $m$-dimensional Brownian motion and we
write $D_{ij} :=
\tfrac{1}{2} 
\left[{\sum_k G_{ik} G_{jk} } \right]_{ij} \in \R^{n\times n} $.

Let 
\begin{equation}
 \f(x,t| y,s) \intd{x} =  \Prob[X_t \in \intd{x} | X_s = y]
 \label{eq:transition_prob_defn} 
 \end{equation}
be the transition probability density. Then $\f$ satisfies:
\begin{equation}
\di_t \f= -\sum_i \di_{x_i} \left[ F_i(x,t) \f(x,t) \right] 
+ 
\sum_{i,j}  \di^2_{x_i x_j} \left[ D_{ij}(x,t) \f(x,t) \right]
\label{eq:FP_pde}
\end{equation}
This is called the \emph{Fokker-Planck} or \emph{Forward Kolmogorov} equation.
It can also be seen as a continuity euquation or a cosevation of probability
equation. To this end define the probability current, $\p \in \R^n$ as:
$$
\phi_i =   F_i(x,t) \f(x,t) 
+ 
\sum_{j}  \di_{x_j} \left[ D_{ij}(x,t) \f(x,t)\right]
$$
then the Fokker-Planck equation, \cref{eq:FP_pde}, can be written as:
$$
\di_t \f= -\grad \cdot \phi
$$
which just says that the change in probability is the
difference between the flow in and the flow out. 

Initial conditions for $\f$ are given by the distribution of $X_s$, if $s=0$
then this is the initial distribution of $X$. Boundary conditions (BCs) depend
on the domain of $X$ and what happens to $X$ once it hits its boundary. If the domain
is all of $\R^n$ then we only insist that $\lim_{|x| \ra \infty} \f = 0$. If the
domain has boundaries, then there are two common scenarios which we will also
encounter in the sequel:
\begin{enumerate}  
  \item absorbing BCs
  \item reflecting BCs
\end{enumerate}
At an absorbing boundary, the particle $X$ is removed and $\f=0$ there.
In this situation, we will not have conservation of probability and the integral
of $\f$ over $X$'s domain will monotonically decrease.

At a reflecting boundary, the particle $X$ bounces back into its domain and we
will have that the probability current $\p\cdot n = 0$ where $n$ is the outward
normal at the reflecting boundary.

We now consider $\f$ as functions of $y,s$, holding $x,t$ fixed, then
\begin{equation}
-\di_s \f= \sum_i  \left[F_i(y,s) \cdot \di_{y_i}\f(|y,s) \right] 
+ 
\sum_{i,j}   \left[ D_{ij}(x,t) \di^2_{y_i y_j}\f(|y,s) \right]
\label{eq:FP_backward_pde}
\end{equation} 
This is the \emph{Backward Kolmogorov} equation for $\f$. Note the minus sign
in front $\di_s \f$. A mnemonic for the signs of the Backward vs. Forward
equation is that as $t$ increases $\f$ diffuses and so $\di_t$ and $\di^2_{x}$
have the same sign, but as $s$ increases, that is as $s$ approaches $t$, $\f$
anti-diffuses and they have opposite signs. 
  
The differential operator on the right-hand side of
\cref{eq:FP_backward_pde} occurs often in the study of SDEs and has its own
name.
\begin{defn}[Generator of an SDE] the Generator $A$ of $X$ is defined by:
$$
A[\psi(x)] = \lim_{t \searrow 0^+} \frac{\Exp[\psi(X_t)]  - \psi(x)}{t} ;  \quad X_0 =
x \in \R^n
$$
\end{defn} 
\begin{lemma} For an Ito SDE as in \cref{eq:generic_Ito_SDE_Rn} 
$$
A[\psi(x)] = \sum_i F_i(x,t) \cdot \di_{x_i} \psi + \sum_{i,j} D_{ij}(x,t)
\di^2_{x_i x_j}\psi $$
\end{lemma}


\subsection{Stopping Times}
\begin{defn}
Let $\Fil(t)$ be some filtration, a random variable $\t$ is called a
\emph{stopping time} if 
$$
\{\o : \t(\o) \leq t\} \in \Fil(t) \, \forall t  
$$
\end{defn}
The colloquial way of describing stopping times is at any time we know
whether $\t$ has occurred or not. For a counterexample, the time that a Wiener
Process achieves its maximum over some interval is not a stopping time, since at any
given time, we do not know if the maximum has occurred or not. The most common
example of a stopping time is the first hitting-time, which is the first time
$X_t$ leaves or enters some set. 
\begin{thm}[First-Hitting time] Let $E \subset \R^n, E \neq
\phi$ be open or closed
 
then $$\t := \inf\{ t \geq 0 | X_t \in E\}$$ is a stopping time.
\end{thm}
 
The reason stopping times are very useful is that all the facts so far quoted
for Ito calculus using integrals $\int_0^T \intd{W}$ remain true if $T$ is
replaced by a stopping time $\t$.
 
Also it allows us to link SDE's and PDEs using the generator, $A$, of the
diffusion: 
\begin{thm}[Dynkin's formula] Given $\t$ a stopping time, $\Exp[\t] < \infty$.
Then:
$$
\Exp[u(X_\t, \t)] =
u(x, 0) + \Exp\left[\int_0^\t \di_t u + A[u] \intd{s}
\right]
 $$
\end{thm}
This provides a link between PDEs and stochastic processes and allows us to go
back and forth in that we can find probabilistic results by
solving a PDE or we can approximate a PDE by simulating a stochastic process
and averaging.

\begin{ex}[\cite{Evansb} pg 99 - Expected hitting time to a boundary]
\label{ex:mean_hitting_time}
 Let $\O
\subset \R^n$ be a bounded open set with smooth boundary $\di \O$ then it is a
basic fact from PDEs theory that
\begin{equation}
\begin{cases}
-\frac{1}{2} \grad^2 u = 1  & \text{over } \O
\\
u =  0 &\text{on } \di \O
\end{cases}
\end{equation}
has a unique $C^\infty(\O)$ solution.

Let $X = W_t + x$ for any $x \in \O$ and define
 $$\t_x := \text{first time } X \text{ hits } \di \O$$
then the generator of $X$ is $A[\psi] = -\grad^2(\psi)/2$ and we will
have:

\begin{align*}
\Exp[u(X_\t)] - \Exp[ u(x_0)] =& \Exp \left[ -\frac{1}{2}\int_0^\t \grad^2
u\right]
\\
=& \Exp \left[ - \int_0^\t 1
\right]
\\
=-&\Exp [\t]
\end{align*}
Finally, invoke $u$'s BCs, $u|_{\di E} = 0$, and $X$'s ICs, $X_0=0 = x$ to
conclude that $$ u(x) = \Exp [\t]$$
The solution to the PDE evaluated at $x$ is the expected exit time from
$E$ for an $X_t$  starting at $x$.
\end{ex}

\section{Parameter Estimation for SDEs}
\label{sec:estimation}
Discuss MAx Likelihood, Fisher information, Fortet equation. 

Parameter estimation for SDEs has a rich body of theory, \cite{??} and there are
many mathematical techniques available. Let us rewrite the SDE
in \cref{eq:generic_Ito_SDE_Rn} in order to explicitly that the 
functions $F, G$ are parametrized by the some parameter set, $\th$.
\begin{equation}
dX_t = F(X,t;\th) dt + G(X,t;\th) dW,
\label{eq:generic_Ito_SDE_Rn_parameterized}
\end{equation}
For example in the case of the O-U process, \cref{eq:OU_equation_generic},
where $F(X,t;\th) = (\m - {X_t})/{\tc}$ and $G(X,t;\th) = \b$, the
parameter set is $\th = \{\m, \tc, \b\}$.

In the standard problem formulation of SDE parameter formulation, one has exact
observations, $\{x_n\}_{n=0}^N$ at times $\{t_n\}_{n=0}^N$ from a process $X_t$
satisfying \cref{eq:generic_Ito_SDE_Rn_parameterized} and one seeks to find the
values of the parameter set $\th$.

\subsection{Maximum Likelihood Estimation}
A fundamental method, both
practically and theoretically, for estimating parameters in an SDE is the {\sl Maximum
Likelihoood} (ML) method, which proceeds by seeking those parameters which
maximize the likelihood of the observed data, $\{x_n\}_{n=0}^N$. In particular
let $$L(\{x_n\}; \th) = \Prob[ X_0 = x_0\ldots X_n = x_n\ldots X_N = x_N |
\th]$$ be the joint probability of observing the data $\{x_n\}$ given the
parameter set $\th$, also known as the likelihood. Then the ML method seeks to
maximize $L$.

In the case of independent observations, the
likelihood is just the product of the individual probabilities of each observation. In the case of SDEs is only
slightly more complicated, due to the Markov nature of the stochastic process. In particular,
the likelihood becomes the product of the transition probabilities. Recall that
earlier we defined the transition probability, $\f(x,t| y,s)$ in
\cref{eq:transition_prob_defn}. We shall also this as $\f_\th(x,t|
y,s)$ is we need to explicitly be reminded of $\f$'s dependence on the
parameter set, $\th$. With that we the likelihood of the observed $x_n$
becomes
\begin{equation}
L(\{x_n\}; \th) = \prod_{n=1}^{N} f_\th(x_n, t_n| x_{n-1}, t_{n-1})
\label{eq:SDE_discrete_likelihood}
\end{equation}
Here, we assume that $x_0$ is fixed, otherwise we would have to add a term
specifying the probability distribution of $X_0$.

In general the transition density for a generic SDE is impossible to find
analytically. There are several ways to approximate it numerically. The most
generic relies on the numeric solution of the Fokker-Planck PDE in
\cref{eq:FP_pde}, but that is quite expensive and suffers from the
'curse-of-dimensionality' for SDEs of higher dimension. 

In some simple cases, the transition density {\sl can} be calculated. The OU
example is computed above is one such a case, where the transition density is
\begin{align*}
f(x_n, t_n| x_{n-1} t_{n-1}) &=
 f(x_n, \Delta| x_{n-1} 0)\\& =
 \frac{1}{ \b \sqrt{\tc 2\pi(1 -  e^{-2 \Delta/\tc}})}
 	\cdot \exp\left(\frac{\left( x - \mu)  - (x_{0} - \mu) \cdot
 	 e^{-\Delta/\tc} \right)^2  } {\t \s^2  (1-e^{-2 \Delta/\tc})}
 	\right) 
\end{align*}
assuming that $\Delta_n = t_n-t_{n-1} = \Delta$ is constant for all $n$.
With this one can form the likelihood and solve analytically for the
maximizers, $\{\hat\m , \hat\tc, \hat\b \}$.
\begin{eqnarray} 
\hat{ \mu} &=& 
\frac{  \sum_{n=1}^{N } 
     \left( X_n - e^{-\frac{\Delta} {\hat \tc}} X_{n-1} \right)} 
	 { N( 1-e^{-\frac {\Delta} {\hat \tc}}) }
\\
e^{-\frac {\Delta}{\hat{\tc}} } &=& 
\frac { \sum_{n=1}^{N} 
			( X_n -  \hat \mu)(X_{n-1} -  \hat \mu) }
    {   \sum_{n=1}^{N } \left( X_{n-1} - \hat \mu
    \right)^2 }
\\
\hat\beta^2 &=&  
\frac{ 2  \sum_{n=1}^{N_k}  \left( X_n - \hat \mu - (X_{n-1} -
\hat \mu) e^{-\frac {\Delta} {\hat \tc}} \right)^2 } 
	  { N (1-e^{-2\frac {\Delta} {\hat \tc}}) \hat \tc}
\end{eqnarray}
The solution for the ML estimates is almost explicit. It requires one numerical
single-dimensional root-finding, which is an easy numerical task.

\subsection{Numerical Simulation of SDEs}
TODO: Euler-Maruyama scheme nothing fancy here. \cite{Higham2001}.


\section{Deterministic Optimal Control}
\label{sec:deterministic_control}
Optimal Control theory has three main components - a state, $x$, a control, $u$
and an objective $J$ which is a functional of $x,u$ and which we try to either
minimize or maximize. Here, we will be minimizing.

We introduce the theory for a finite dimensional state, $x(t) \in \R^{n_x}$, but
we will not address state constraints. Our control will also be
finite-dimensional $u(t) \in \R^{n_u}$ and it will be allowed to take values in
some closed domain, $\Udomain$, of allowed controls, e.g.\ $\Udomain = \{ u_t
\in \R^{n_u} \, \st |u| \leq \Umax \}$ for a control constrained to a sphere.
The objective $J$ will be given by a time-integral-plus-terminal cost:
\begin{equation}
J[u] = \int_0^\tf L(x_s, u_s) \intd{s} + M(x_\tf)
\label{eq:generic_objective_function_deterministic}
\end{equation}
Where $\tf$ could be either a variable or fixed. 

The dynamics of $x$ are given by some controlled ordinary differential
equation (ODE) given some initial conditions:
\begin{equation}
\dot{x} = f(x,t); \quad x(0) = x_0
\label{eq:generic_dynamics_deterministic}
\end{equation}

Then we seek the optimal control $u^*$, such that:
\begin{equation}
u^* = \argmin_{u \in \Udomain} J[u]  
\label{eq:generic_objective_argmin}
\end{equation}

Note that $x$ may also have final conditions, which we will deal with later,
and/or path constraints. If there are path-constraints, we talk about a
state-constrained problem. However, we will not face path constraints in the
sequel and so will not say more about how to handle them.

There are essentially three methods for solving equation
\cref{eq:generic_objective_function_deterministic}. Two, Pontryagin's Minimum
Principle and Dynamic Programing, are analytic in nature providing a set of
equations that characterize the minimum and then solving these equations,
numerically if necessary. The third, the direct method, simply discretizes the
problem and solves the resulting nonlinear programing (NLP) problem via standard
NLP techniques. We will describe both Pontryagin's Minimum Principle and Dynamic
Programing below and indeed we will use both later on.

\subsection{Pontryagin's Maximum Principle}
Pontryagin's Principle is derived from on a variational argument, similar to the
Euler-Lagrange equations from the Calculus of Variations, and it characterizes
the optimal trajectory/control pair. It can be thought of as a generalization of
the zero-tangent rule (Fermat's Rule) for finding optima in single-variable
calculus and like both the Euler-Lagrange equations and Fermat's Rule it
provides necessary, but not sufficient conditions for a minimum.

Pontryagin's Principle also holds a similarity to Hamiltonian Mechanics, which
should not be surprising, since Hamiltonian Mechanics is founded on
the minimization of action.

There are several versions of Pontryagin's Principle, depending on whether $x$ has
terminal conditions or not, whether the final time, $\tf$ is specified or a
variable in itself and most importantly whether the state has constraints or
not. We will state the versions that are relevant for our needs, namely, $x$ has
no constraints and $\tf$ is specified.

We first introduce the Pontryagin Hamiltonian, $\H$, also called the control
theory Hamiltonian and an adjoint state, $p \in \R^{n_x}$:
\begin{defn} [$\H$] The control theory Hamiltonian is the function:
$$
\H(x,p, u) := f(x,u) \cdot p + L(x,u) \quad (x,p \in R^{n_x}, u \in \Udomain)
$$ 
\end{defn}

\begin{thm}[Pontryagin's Fixed-Time, Free-End-Point] Assume $u^*(t)$ is
optimal for
\cref{eq:generic_objective_function_deterministic,eq:generic_dynamics_deterministic}
and that $x^*$ is the corresponding trajectory. 

Then there exists a function $p^*:[0,\tf]:\ra \R^{n_x}$ st.
\begin{equation}
\begin{cases}
\dot{x}^*(t) &=  \grad_p \H(x^*,p^*, u^*) = f(x^*, u^*)
\\
x(0) &= x_0
\end{cases}
\label{eq:pontryagin_state_ode}
\end{equation}
\begin{equation}
\begin{cases}
\dot{p}^*(t) &= -\grad_x \H(x^*,p^*, u^*) = -\grad_xL - p \grad_x f
\\
p^*(\tf) &= \grad_x M(x)
\end{cases}
\label{eq:pontryagin_adjoint_ode} 
\end{equation}
and
\begin{equation}
\H(x^*(t), p^*(t), u^*(t)) = \min_{u(t) \in \Udomain(t)}  \H(x^*(t), p^*(t),
u(t))
\label{eq:pontryaginH_optimality_condition} 
\end{equation}
\end{thm}
We will not give a proof of this theorem (see A.2 in \cite{Evansb}),
but we do provide a heuristic derivation in 
\cref{sec:Pontryagin_heuristic_derivation}, which is very helpful in deriving
the necessary conditions for PDE dynamics and it explains the otherwise
mysterious origins of the optimal control Hamiltonian, $\H$, and the adjoint
state, $p$.

At each time, $t$, we obtain the optimal control $u(t)$, by solving
\cref{eq:pontryaginH_optimality_condition}. This can be easy, hard or useless.
In the easy case, the minimum is analytically obvious, e.g.\ if the Hamiltonian
is quadratic in $u$. If it is linear in $u$ then the minimum will occur on the
boundaries of the control feasible region and we talk about a bang-bang
control. Conversely, solving the minimum of $\H$ with respect to $u$ may be impossible,
since $x,p$ are unknown. In this case we need to have an iterative scheme which
alternately solves
\cref{eq:pontryaginH_optimality_condition} 
and then \cref{eq:pontryagin_state_ode,eq:pontryagin_adjoint_ode}. Basic
numerical techniques using this idea are discussed in the appendix in
\cite{Kirk2004}. However, if we are going to be solving the problem numerically,
then the direct method might be superior, see \cite{Ross2005} for a discussion.
 
Finally, it is possible that \cref{eq:pontryaginH_optimality_condition} does not
depend on $u$ explicitly (it will still depend  on $u$ implicitly through $x,p$)
and so we cannot find $u$ at all by minimizing $\H$ wrt.\ $u$. This happens, for
example, if the cost, $L$, does not depend on $u$, the dynamics, $f$, are linear
in $u$ and $p=0$ over an interval of positive length. Such an optimal control
problem is called singular and it is discussed briefly in \cite{Kirk2004}, and
the theory can still be useful in finding a solution, but many complications can
and do arise. We face a singular control problem in our own work later, however
instead of tackling it head on, we merely regularize the problem - thus we will
not say much more on Singular Control Problems.

The version of Pontryagin's Principle for Fixed End Point, meaning $x$ has some
terminal conditions, is very similar to the one with free end-point. Suppose 
\begin{equation}
\dot{x} = f(x,t); 
\quad
\begin{cases}
x(0) = x_0
\\
x(\tf) = x_f
\end{cases}
\label{eq:generic_dynamics_deterministic_fixed_end_point}
\end{equation}

\begin{thm}[Pontryagin's Fixed-Time, Fixed-End-Point] Assume $u^*(t)$ is
optimal for
\cref{eq:generic_objective_function_deterministic,eq:generic_dynamics_deterministic_fixed_end_point}
and that $x^*$ is the corresponding trajectory. 

Then there exists a function $p^*:[0,\tf]:\ra \R^{n_x}$ st.
\begin{equation}
\begin{cases}
\dot{x}^*(t) &=  \grad_p \H(x^*,p^*, u^*) = f(x^*, u^*)
\\
x(0) &= x_0
\\
x(\tf) &= x_f
\end{cases}
\label{eq:pontryagin_state_ode}
\end{equation}
\begin{equation}
\dot{p}^*(t) &= -\grad_x \H(x^*,p^*, u^*) = -\grad_xL - p \grad_x f
\label{eq:pontryagin_adjoint_ode} 
\end{equation}
and
\begin{equation}
\H(x^*(t), p^*(t), u^*(t)) = \min_{u(t) \in \Udomain(t)}  \H(x^*(t), p^*(t),
u(t))
\label{eq:pontryaginH_optimality_condition} 
\end{equation}
\end{thm}
I.e. $p$ loses its final conditions because $x$ has them. This relates to the
variational nature of Pontryagin's Principle and the idea that $p$ represents
the sensitivity of the objective wrt. small changes in $x$. Thus, if $x$ is
fixed, there are no variations and $p$ can be anything.
% To illustrate Pontryagin's Principle we will solve the moon
% lander problem (Ex. 4.4.4 \cite{Evansb}), with the twist of holding the final
% time fixed. 
% \begin{ex}
% 
% \end{ex}

\subsection{Dynamic Programing - the Hamilton-Jacobi-Bellman equation} 
Dynamic Programing uses backwards recursion to tabulate the optimal control
starting from the terminal time. The basic object in dynamic programing is the value function, $v$. In order to
introduce it, we first extend our definition for the objective, $J$, to include later
times, with different initial conditions. 

The running cost-to-go corresponding to
\cref{eq:generic_objective_function_deterministic} is:
\begin{equation}
J[u; x, t] = \int_t^\tf L(x_s, u_s) \intd{s} + M(x_\tf), \quad x_t = x
\label{eq:generic_cost_to_go_deterministic} 
\end{equation}
So that $J[u; x_0, 0] = J[u]$. 

The dynamic programming value function, $v$ is then defined by:
$$
v(x,t) = \inf_{u \in \Udomain} J[u; x,t] 
$$
it is also called the optimal cost-to-go. 

We immediately note the terminal conditions on $v$:
\begin{equation}
v(x,\tf ) = M(x)
\end{equation}


The Dynamic Programing approach is then to solve for $v$ backwards and in the
process infer the control $u$.
\begin{thm}[Hamilton-Jacobi-Bellman Equation] Assume that $v$ is $C^{1,1}$ in
$x,t$. Then $v$ solves:
\begin{equation}
\begin{cases}
-\di_t v(x,t) &= \max_{u(t) \in \Udomain(t)} \left\{ f(x,t) \cdot \di_x v +
L(x,u) \right\}
\\
v(x,\tf) &= M(x)
\end{cases}
\end{equation}
\end{thm}
We will not derive the HJB equation here, but we will derive a similar version
later for the stochastic case, when the dynamics are given by SDEs. 

% \subsection{Model Predictive Control}
% Solving the HJB equations and obtaining a feedback control is usually
% impossible. Thus a practical alternative is to solve a simplified problem on a
% limited horizon and use the simplified solution now and then after stepping
% forward in time and observing the system in a new position, to solve a revised
% limited horizon problem and use the new solution.
% Let us make this more precise.

\section{Stochastic Optimal Control}
\label{sec:stochastic_control}
Here we extend the Dynamic Programing theory to the case where the dynamics are
governed by an SDE:
\begin{equation}
dX = F(X,t, u) dt + G(X,t) dW
\end{equation} 
i.e.\ we assume control affects only the drift term. This 
assumption is not necessary, but it corresponds to our problems.

For stochastic dynamics one can only optimize probabilistically and so
$J$ becomes:
\begin{equation}
J[u] = \Exp \left[ \int_0^\tf L(x_s, u_s) \intd{s} + M(X_\tf)
\right] 
\label{eq:generic_objective_function_stochastic}
\end{equation}
while the running cost-to-go is:
\begin{equation}
J[u; x, t] = \Exp \left[ \int_t^\tf L(X_s, u_s) \intd{s} + M(X_\tf) \right], 
\quad X_t = x
\label{eq:generic_cost_to_go_stochastic}
\end{equation}
the value function is:
\begin{equation}
v(x,t) := \inf_{u \in \Udomain} J[u; x,t]
\end{equation}

Finally, we again have that $v$ satisfies a certain PDE:
\begin{thm}[Stochastic HJB] 
\label{thm:stochastic_hjb}
\begin{equation}
\begin{cases}
-\di_t v(x,t) &=  \max_{u(t) \in \Udomain(t)} \big\{ L(x,u)  +
F(x,t,u) \cdot \di_x v
\big\} + \frac{G^2(x,t)}{2} \di_x^2v
\\
v(x,\tf) &= M(x)
\end{cases}
\end{equation}
\begin{proof}[Heuristic Derivation (from \cite{Evansb})] Suppose we are at time
$t$ and $X_t = x$.

Take a time increment $[t, t+h]$ and assume that during that time we apply a
control $u$ and subsequently we apply the the optimal control $u^*$. The
running-cost will then break down as: 
$$
J[u; x,t] = \Exp \left[ \int_t^{t+h} L(X_s, u_s) \intd{s}  + v(X_{t+h},
t+h) \right] $$

Since $v(x,t) = \inf J(u; x,t)$, we must have that:
$$
v(x,t) \leq  \Exp \left[ \int_t^{t+h} L(X_s, u_s) \intd{s}  + v(X_{t+h},
t+h) \right] $$
or, rearranging, that
\begin{align*}
0 \leq^&  \Exp \left[ \int_t^{t+h} L(X_s, u_s) \intd{s} \right]  
+ \underbrace{\Exp \left[ v(X_{t+h}, t+h) - v(x,t) \right]}_{\Exp\left[ dv
\right]}
\end{align*}
Now, $\Exp\left[ dv \right]$ can be expressed using Dynkin's Formula, as:
$$
\Exp[dv] = \int_t^{t+h} \di_t v +  F\cdot \di_x v + \frac{G^2 }{2}\cdot \di_x^2
v \intd{s} $$
Plugging that back, we get
$$
0 \leq \int_t^{t+h} L(X,u) +  \di_tv +  F\cdot \di_x v + \frac{G^2 }{2}\cdot
\di_x^2 v \intd{s} $$
Taking $h \ra 0$ we get 
$$
0 \leq  L(x,u) +  \di_tv(x,t) +  F\cdot \di_x v(x,t) + \frac{G^2 }{2}\cdot
\di_x^2 v(x,t) $$
and we conjecture that for the actual optimal control, the inequality becomes an
equality:
\begin{equation}
\begin{cases}
0 &=   L(x,u^*)+ \di_t v(x,t) +
 F(x,t,u^*)\cdot \di_x v(x,t) + \frac{G^2(x,t)}{2}\cdot \di_x^2 v(x,t)
\\
u^*(t) &= \argmax_{u \in \Udomain(t)}  
\big\{L(x,u) + F(x,t,u)\cdot \di_x v \big\}
\end{cases}
\end{equation}
\end{proof}
\end{thm}
% A rigorous proof can be found in \cite{Krylov2008}, sec???ch???

\section{Deterministic Optimal Control in Infinite Dimensions}
Now we discuss Optimal Control when the dynamics are deterministic but of
infinite-dimension, i.e. the state evolution is a PDE instead of an ODE.  The
reason we discuss this, is that optimization of ordinary SDEs in an open-loop
control is actually a deterministic optimal control problem in infinite
dimensions.

Why?

Since the objective is probabilistic and the probability density evolution is
given by the (deterministic) Fokker-Planck PDE, we need to optimize a
deterministic system whose dynamics are given by the Fokker-Planck equation.

There is no all-encompassing generalization of Pontryagin's Principle to
infinite dimensions, but a Maximum Principle can be derived for certain systems.
To illustrate the idea we will discuss a simple case as in \cite{Palmer2011}.
Our controlled SDE will be given by: 
$$ dX_t = F(X,t, u) dt + G(X,t) dW, $$
The corresponding probability density is given by: $$ \di_t \f=
-\sum_i \di_{x_i} \left[ F_i(x,t,u) \f(x,t) \right] + \sum_{i,j}  \di^2_{x_i
x_j} \left[ D_{ij}(x,t) \f(x,t) \right] $$ which we will write as $$ \dot{\f} =
\L_{u} [\f] $$ where $\L_{u}$ is the differential operator corresponding to the
RHS of the Fokker-Planck equation parametrized by the control $u$. For now we
will assume that there are no BCs, and the domain of $X$ is all of $\R^n$.

A natural inner-product, $ \langle.,. \rangle$ on functions in $L^2[\R^n]$ is
given by: $$
 \langle\psi,\phi  \rangle = \int_{\R^n} \psi(x) \phi(x) \intd{x}
$$
Our objective will be to minimize:
\begin{equation}
J[u] = \int_0^\tf  \langle L(\cdot,s,u), \f_u(s) \rangle \intd{s} +  
\langle M(\cdot),\f(\tf) \rangle
\label{eq:generic_objective_function_probabilistic}  
\end{equation}.

\Cref{eq:generic_objective_function_probabilistic} is a very natural cost
function in the stochastic dynamics case, because it can also be written as: 
$$
J[u] = \Exp \left[ \int_0^\tf  L(X_s, s,u) \intd{s} + M(X_\tf) \right] 
$$ which
is just \cref{eq:generic_objective_function_stochastic}. We use
\cref{eq:generic_objective_function_stochastic} if we have some feedback and
want to use dynamic programming. We use
\cref{eq:generic_objective_function_probabilistic} if we have no observations
and need to run an open-loop control.

Since, \cref{eq:generic_objective_function_probabilistic} is the objective
function for deterministic dynamics, we can use Pontryagin's Principle adapted
to the case of infinite dimensional dynamics (PDEs). 

Introduce an adjoint co-state, $\l \in \R^n$ and let the Hamiltonian, $\H$ be:
$$
\H(\f, \l, u)= \langle \phi, L + \Lstar[\l] \rangle$$
where $\Lstar$ is the adjoint operator to $\L$ defined via:
$$
\langle \L [\phi], \l \rangle =\langle \phi, \Lstar[\l] \rangle
$$

Let $\l$ evolve (backwards) according to:
\begin{equation*}
\begin{cases}
-\di_t \l =& \Lstar[\l] + L
\\
\l(\tf) =& M(x)  
\end{cases}
\end{equation*}
then the optimal control, $u$, is given via
$$
u = \argmin_u {\H(\f, \l, u)}
$$   

An important fact, useful for computations, is that the total derivative of the
objective with respect to the control is given by the partial derivative of the
Hamiltonian with respect to the control:
\begin{equation}
\grad_u \big[J[u] \big] = \di_u \H(\f, \l, u)
\label{eq:objective_gradient_wrt_control}
\end{equation}

So for a given $u$, we can compute $\f,\l$, calculate the objective gradient
using \cref{eq:objective_gradient_wrt_control} and then iterate in the direction
of descent (if we are minimizing).

All that remains is to compute $\Lstar$. It turns out that if we do not have any
BCs, then
$$
\Lstar[\l] = \sum_i F_i(x,t) \cdot \di_{x_i} \l + \sum_{i,j} D_{ij}(x,t)
\di^2_{x_i x_j}\l 
$$ 
which is just the generator of $X$! 

We will see later, that things become more complicated when we need to
consider a PDE's BCs.

 
\subsection{Heuristic Derivation of the Optimality Conditions for PDEs}
\label{sec:Pontryagin_heuristic_derivation}
TODO: \cite{Borzi2012,Lenhart2007}

\section{Optimal Design}
\label{sec:optimal_design}
Optimal design is approach to design of statistical experiments that is guided
by the 
desire to optimize some formal measure of the parameter estimates of the
experiments. TODO: cite Pukelheim. Common criteria are minimizing the
determinant or the trace of the covariance matrix of the parameter estimators. 
The covariance matrix, also known as the Fisher Information, however often
depends on the very parameters that one seeks to estimate. 

An alternative to the Fisher Information as an
objective for the formal design of experiments is to use of concepts from
Information Theory,  \cite{MacKay2003}. 

We now define and discuss the {\sl Mutual Information} between
two random variables $X, \Th$. For reference we will follow
\cite{MacKay2003}. 

\begin{defn}Mutual Information:
Given two random variables $X,\Th$ with joint probability density
$p(x,\th)$ and marginal densities $p(x), p(\th)$, the Mutual Information between
$X$ and $\Th$ is given by
\begin{equation}
I(X,\Theta) = \int_\Theta \int_X p(x,\th) \cdot \log \left(
\frac{p(x,\th)}{p(x)p(\th)}\right) \intd{x} \intd{\th}
\label{eq:mutual_info_defn}
\end{equation}
\end{defn}

It is obvious that if $X,\Th$ are independent than $I(X,\Theta) = 0$ and it can
be verified that $I\geq0$ and that it is maximized if $\Theta$ is a function of
$X$; that is, if the entropy of $\Th$ conditional on $X$ is zero. 
The mutual information, $I(X,\Th)$ represents the 'average
reduction in uncertainty about $\Th$ that results from learning the value of
$X$' \cite{MacKay2003}. This statement is formally correct if one takes
 'uncertainty' to mean the entropy of a random
 variable.
 
In order to make use of the Mutual Information in a parameter estimation
context, we need to specify a prior belief distribution over the  parameters.  
$$
\rho(\th) = \Prob(\Theta = \th)
$$
A prior distribution is standard and central concept in Bayesian statistics.
 
With that we are treating the unknown parameter, $\th$ like a random variable,
which we write as $\Th$. It is then natural to seek an experiment which
maximizes the Mutual Information between, $\Th$ and the observations $X$

To obtain the joint and margincal distributions of $\Th$ and $X$, we
need to recall Bayes' formula which relates the posterior of the parameters,
$p(\th|x)$ to the observations and the parameters' prior. 

\begin{equation}
p(\th|x) = \frac{L(x|\th)\rho(\th)}{\int_\Th L(x|\th)\rho(\th)}
\label{eq:bayes_formula}
\end{equation}
where $L$ is the likelihood of the observations.

Moreover, the marginal distribution of the parameters, is just its prior,
$p(\th) = \rho(\th)$, while the joint distribution of the parameters and the
observations is $$p(x,y) = L(x|\th)\rho(\th)$$.

while the $x$ marginal is $$p(x) = \int_\Theta L(x|\th)\rho(\th) \intd{\th}$$
Plugging the three expressions for $p(x,\th), p(x)$ and $p(\th)$ in the
definition, \cref{eq:mutual_info_defn}, gives:
\begin{equation}
I(X,\Th) = \int_\Theta \int_X L(x|\th)\rho(\th) \cdot 
\log \left( \frac{L(x|\th) }
				{\int_\Theta L(x|\th)\rho(\th) \intd{\th}  } \right)
\intd{x}\intd{\th}.
\label{eq:mutual_info_posterior_vs_observations} 
\end{equation} 
\Cref{eq:mutual_info_posterior_vs_observations}  is used in
\cref{ch:optimal_design}.




\section{Mathematical Models in Neuroscience}
\label{sec:math_models_in_neuroscience}
We now describe in detail the basic mathematical model of a neuron that will be
used in the sequel. 

An introduction to mathematical models in neuroscience is given in Gerstner and
Kistler, \cite{Gerstner2002}, also available online, while the book Stochastic
Methods in Neuroscience, \cite{Laing2009} gives a nice overview of several
current research applications of stochastic techniques to neuroscience.


Neurons relay information by means of voltage spikes - sudden sharp increases in
voltage. Although many details remain unclear, the information content is
thought to be contained in the length of the time-interval between these spikes.
In the simplest case, this can be thought of as a rate - the average number of
spikes per time interval, but more complicated coding schemes are hypothesized to
exist. 

Many experiments allow for manipulating an individual neural cell. A natural
goal then is to make a cell produce a given spike train. This may arise, for
example, in brain-machine interfaces or in artificial prosthetics. 


TODO: Go from HH $\ra$ FN/ML  $\rightarrow$ LIF

Then add noise\ldots

Thus the final simplification of the basic spike-generation mechanism is the
noisy leaky-integrate-and-fire model: 
%TODO: tau_c -> tau
\begin{equation}
\begin{gathered}
dX_t = \left(\a(t) - \frac{(X_t - \m)}{\tc} \right) \intd{t} + \b \intd{W_t},
\\
X(0) = 0,
\\
X(\ts) = \xth \implies  
\begin{cases}
X(\ts^+) = 0 &  
\end{cases}
\end{gathered}
\label{eq:X_evolution_uo}
\end{equation}
That is $X$ follows an OU process, but upon reaching a pre-determined threshold,
$\xth$, a 'spike' is deemed to have occurred and the process is 'reset' to an
initial value, here $0$.

Alternatives to the hard-threshold integrate-and-fire model are so called
'soft-threshold' models which use a {\sl hazard} function which is akin to the
intensity of a Poisson Process to determine the spike time. The hazard function
is increasing in voltage, thus higher voltages imply higher likelihood of
spiking. The hard-threshold model can be seen as a special case of the
soft-threshold model, given the hazard function which is 0 below the threshold
and is infinite above the threshold. We will not address hazard-function based
spiking models in the rest of the thesis. 

\cleardoublepage
\include{chapters/ch4_SinEstimate}
\cleardoublepage
\include{chapters/ch5_OptSpike}
\cleardoublepage
\chapter{Optimal Design for Estimation in SDEs}
\label{ch:optimal_design}
\graphicspath{{../OptEstimate/}}
\input{../OptEstimate/local_style.sty}

 %TODO: change to consistent notation:
 
 \section{Problem Introduction and LIterature Review (.0)}
 

\subsection{Mutual Information as the formal criterion for Optimal Design}
The use of Mutual Information as a design criterion is newer than the use of
Fisher Information. A tutorial paper summarizing its use in Mathematical
Psychology is given in \cite{Myung2013}, while its use in neuroscience
experiments is discussed in a series of papers by the group of Liam Paninsky in
Columbia \cite{Paninski2003,Paninski2005,Lewi2009}. 




We start with a tutorial paper from the Journal of Math. Psychology (2013),
which begins as follows: { \sl ``Imagine an experiment in which each and every
stimulus was custom tailored to be maximally informative about the question of
interest, so that there were no wasted trials, participants, or redundant data
points.''} 

Their work focuses on both parameter estimation and the larger task of model
selection, let's just discuss the parameter estimation bit. 

In their terminology, the task of the {\sl experimenter}
 is to find a { \sl design, $d$ } (this is the stimulation $\a(t)$ in our
 context) that will best facilitate the estimation of the model parameters.
 After introducing a prior on the parameters $\rho(\th)$, and denoting the
 {\sl outcome} of the experiment as $t$, they state that the design selection
 can be formalized by optimizing the following expression: $$
d^* =  \argmax_d \int \int u(\th, t; d) L(t| \th, d) \rho (\th) \intd{t}
\intd{\th}
$$
where $u$ is some utility function. For example one could maximize the inverse
sum of CVs:
$$
u() = \sum_i \frac{\Exp[\th_i]}{STD[\th_i]}
$$
where 
$\Exp[\th_i], STD[\th_i]$ are the posterior mean/std. dev of the estimates, e.g.

$$
\Exp[\th_i] = \int_{\th_i} \th_i p(\th_i| t, d) \intd{\th_i} = 
\int_{\th_i} \th_i \frac{ L(t| \th, d) \rho (\th) }
					    {\int_\th L(t| \th, d) \rho (\th)\intd{\th_i} } \intd{\th_i} 
$$
(An aside, if the parameter estimate is on average zero, or negative I think
this will not work at all, but I think they just suggest it as an obvious, but
problematic utility function)

They then state that the most common utility function, $u$, in the
literature is the Mutual Information b/w the random variables $\Th$ and $T$. 

Again, quote, {\sl (The Mutual Information)  measures the reduction in
uncertainty about the values of the parameters that would be provided by the
observation of an experimental outcome under design $d$. In other words, the
optimal design is the one that extracts the maximum information about the
model's parameters.}

So as far as I can tell, the fact that Mutual Information maximization equates
to reduction of parameter uncertainty is here taken dogmatically!

They then go on to talk about { \sl sequential } optimal design, which is the
same as above, but then you Bayes-update the prior of $\th$ while the experiment
continues and then roll forward. 

In our context this could mean that after $N_{s,1}$ spikes, we update the prior
of $\b$ and redo the calculation of the optimal perturbation $\a(t)$, then
observe for $N_{s_2}$ spikes, then recompute $\a(t)$ and so on. At this point,
one can get creative - instead of optimizing $\a(t)$ we can just take 1(or 2 or
whatever) steps in the gradient descent. In practice, this is probably
not very realistic due to computational costs, but in principle very cool.

\subsection{Further Literature Review}	
In principle OPtimal Design for Dynamic Systems is a subset of System
Identification, which is a very researched topic in the control community.
Here's a review paper \cite{Gevers2011} from the Control Theory Community. 
They talk about things like Transfer Functions etc. usually things are in
Discrete time. There is no notion of SDEs although there is a=often a white
noise error term, so in a sense these are discretized SDEs, although without ay
of the formalism of Ito Calculus etc. 

The papers on Adaptive Optimal Design from the Myung et al group in Ohio is well
cited (the main 2010 \cite{Cavagnaro2010} article has 45 citations) 

Оh, cool,  \cite{Cavagnaro2010} says in its lit. review leading into the paper
that 
\begin{quote}
the desirability and usefulness of (using Mutual Information as the objective
functional) was formally justified by Paninski (2005) who proved that under
acceptably weak modeling conditions, the adaptive approach with a utility function
based on mutual information leads to consistent and efficient parameter
estimates. 
\end{quote}

Ok, so we just need to dig up Paninsky2005 and insert here:) Here it is:
\cite{Paninski2005}. Here is its abridged abstract:
\begin{quote}
... on any given trial,
we want to adaptively choose the input in such a way that the mutual information
between the (unknown) state of the system and the (stochastic)
output is maximal, given any prior information (including data collected
on any previous trials). We prove a theorem that quantifies the effectiveness
of this strategy\ldots and demonstrate that
this method is in a well-defined sense never less efficient—and is generically
more efficient—than the non-adaptive strategy\ldots  
\end{quote}

Here's another good quote from 
\begin{quote}
\ldots several attempts have been made to devise algorithms
to find the “optimal stimulus” of a neuron, where optimality is
defined in terms of firing rate (Tzanakou, Michalak, & Harth, 1979; Nelken,
Prut, Vaadia, & Abeles, 1994; Foldiak, 2001), but we should emphasize that
the two concepts of optimality are not related in general and turn out to be
typically at odds (maximizing the firing rate of a cell does not maximize—
and in fact often minimizes—the amount we can expect to learn about the
cell; see sections 3 and 4).
\end{quote}
Again the punch line in \cite{Paninski2005} is:
\begin{quote}
Our main result (in section 2) states that under
acceptably weak conditions on the models $p(t| \a, \th)$ (our notation, not
his) the information maximization strategy leads to consistent and
efficient estimates of the true underlying model, in a natural sense. 
In particular, the information maximization
strategy is never less efficient, in a well-defined sense—and
is generically more efficient—than the simpler, non-adaptive, i.i.d. x strategy.
\end{quote}



The experimental/computational Lewi, Buttera et Paninsky paper from 2009 is
very popular, `` Sequential
optimal design of neurophysiology experiments'', \cite{Lewi2009}. It has 64
citations, although its main citing authors are Paninsky himself
(self-references) + the Myung et Cavagnaro group above :). Another very popular
paper from Paninsky (117 cites) is \cite{Paninski2006a}, which deals with
several things including 'Optimal Stimulus'. Reading Paninsky can be pleasant or
unpleasant, the 2009 \cite{Lewi2009}paper is 73 pages of dense Convex
Optimization, which hurts my eyes:) but the 2006 paper is much more fluent, at
only 14 pages. 

Let's review it and in paritcular let's review the part about 'Optimal
Stimulus', this starts in Sec. 3 (p.11)

\begin{quote}
If we use the entropy of the posterior
distribution on the model parameters ( what we call $L(t|\th)
\rho(\th)$) to quantify this uncertainty, we arrive at the [\ldots] 
mutual information between the response (for us this is $t_s$) and the model
parameters $\th$ given the stimulus and past data.  
\end{quote}

They then acknowledge that optimizing $I$ is problematic, and even just
computing it quickly becomes hard for multi-dimensional params\ldots

They then suggest to approximate the posterior of $\th$ with a Gaussian. They
self-quote (\cite{Paninski2005}, which seems to be the main theoretical
justification paper for using MI).

With that the Mutual Information calculation is reduced to determinig the
log-Determinant of the Hessian matrix (the Gaussian Covariance)\ldots

They then do a very smart observation that for online optimization this Hessian
recuses very nicely with itself over observations and so updating it is very
cheap (computationally). Thus one can optimize-stimulate-update online, meaning;

\begin{enumerate}
  \item Optimize the MI for a given param prior
  \item stimulate the system with the MI-optimal stimulus
  \item update the prior given the new observation
  \item Reoptimize the MI using the new prior (the posterior)
  \item roll on\ldots
\end{enumerate}

OK! Now I understand the rough idea of the 76 page 2009 paper\cite{Lewi2009}.

\subsection{Cool (Waterloo) Physics paper}
\cite{Granade2012} describe something that is in principle very close to what we
aim at. They are trying to estimate the 'Hamiltonian' of a quantum experiment
(its parameters) and they talk about all the things we talk about - optimal
stimulus, belief distn over the params. In Algos, 1-7 of their paper they give a
very readable summary of particle filtering! (which in turn sources
\cite{Liu2001})

\section{Problem Formulation}
The basic goal is to perturb a dynamical system in an 'optimal' way such as to
estimate its structural parameters. 

As such the problem is a blend of optimal control and estimation, where the
objective of the optimal control is to improve the estimation, e.g.\ by
minimizing the variance of the estimators. 

For illustration sake we return to our favourite LIF model
Given a noisy LIF neuronal model:
\begin{equation}
\begin{gathered}
dX_s = (\a(t) + \m + \g \sin(\o t) - \frac{X_s}{\tc} ) \intd{s} + \b \intd{W_s},
\\
X(0) = .0,
\\
X(\ts) = \xth \implies  
\begin{cases}
X(\ts^+) &= .0   
\\
t_k &=  \ts
\\
k  &= k+1
\end{cases}
\end{gathered}
\label{eq:X_evolution_uo}
\end{equation}
With the parameter set $\th = \{\m, \g, \tc, \b\}$ unknown. i.e\ assume that
$\o$ is known. Actually, to keep things pedagogical, let us say only $\tc$ is
unknown.

Our goal is to choose $\a(t)$ as to estimate $\tc$. Again, we consider two
different data scenarios
\begin{itemize}
  \item $X_t$ is continuously observed. 
	\item Only the spike times $\{t_k\}$ are observed
\end{itemize}
The first scenario is addressed (for the MOrris-Lecar model) in a submitted
paper, \cite{Lin} where I got this idea, the second scenario is likely
unaddressed in the literature. (I say likely as it is a 'harder' problem than
the first scenario and the first scenario is only now being addressed in 
\cite{Lin} which claims to be one of the first optimal design papers using
stochastic control)

In addition to addressing the second scenario, which is a highly non-trivial
extension of the first scenario, we might also consider several smaller ways to
differentiate ourselves from \cite{Lin} - for example their numerical method for
the optimal control problem is relatively simple, and also their optimization
criteria is slightly self-inconsistent in a way that I will explain later, in
particular I feel there might be an alternative optimization criteria phrased
in terms of posterior distributions (from Bayesian theory).


 

As such the Fisher Matrix also known as the FIsher INformation matrix
plays a main role. 

While Optimal Design of statistical experiments is a fairly well-developed
research topic, optimal design for estimation of parameters in dynamic models
such as ordinary differential equations has only recently begun to be addressed.
For stochastic differential equations, there is very little done. The only
reference to my knowledge is the work of Lin et al. \cite{Lin}, who also
attempt to minimize the Fisher Information of parameters in an SDE model
using the fact that for one unknown parameter the Fisher Information  of the
continuous observations form an SDE yields a tractable dynamic optimization
criterion.

Let us briefly sketch their basic idea. We write the dynamics as
\begin{equation}
dX = \underbrace{F(X,\th, \a)}_{\textrm{controlled drift}}dt
+ G dW
\end{equation}
where $X$ is the state, $\th$ are the unknown parameters, in this case just one,
and $\a$ is some dynamic control. Assume that $G$ is constant. Then given an
observed path $\{x_t\}_0^\tf$, the log-likelihood, $l$ wrt.\ the parameter set $\th$ is
\begin{align}
l(\th | x_t) =&  \frac 12 \int_0^\tf \frac{F^2(x_t,\th, \a)}{G^2} \intd{t}
\notag
\\
&- \int_0^\tf  \frac{F(x_t,\th, \a)}{G^2} \intd{W}
\label{eq:log_likelihood_cts_time}
\end{align}
The suggestion in \cite{Lin} then is to choose $\a$ by maximizing the Fisher Information
\begin{equation}
\FI(\th, \a) = \Exp_X \left[ \int_0^\tf \frac{ \left( \di_\th F(X_t,\th, \a)
\right)^2}{G^2}
\intd{t}
\right]
\label{eq:Fisher_Information}
\end{equation}

Note that there are two optimizations intertwined. One, to maximize
the likelihood $l$ in order to obtain the actual estimate $\th$, the other - to
maximize the Fisher Information evaluated at the (a priori unknown!) estimator $\th$.

The authors in Lin et al. \cite{Lin} acknowledge that clearly one cannot form
the Fisher Information directly since its evaluation requires the very parameter
being sought! To remedy this, they apply a prior of $\th$. I still need to
understand exactly what they do, but as far as I understand, they augment $\FI$
by an outer expectation over the prior for $\th$, i.e.\ (I think!) the objective
determining the control $\a$ becomes
\begin{equation}
\tilde{I}(\th, \a) = \underbrace{\Exp_\th \left [
\underbrace{\Exp_{X} \left[ \int_0^\tf
\frac{ \left( \di_\th f(x_t,\th, \a) \right)^2}{\b^2}
\intd{t}
\right]}_{\textrm{average over trajectories}}
\right]}_{\textrm{average over prior}}
\label{eq:Fisher_Information}
\end{equation}
and then they show that the estimator so obtained, i.e.\ the one which uses the
optimal $\a$, is still better than a naive estimator (without any control)
   


\section{SDE Parameter Estimation and Optimal Design}
The basic goal of 'Optimal Design' is to perturb a dynamical system in an
'optimal' way such as to 'best' estimate its structural parameters.

Consider the generic parametrized and controlled SDE system 
\begin{equation}
dX = U(x,\a(x,t); \th) \intd{t} + \sqrt{2D} dW
\label{eq:SDE_evolution} 
\end{equation}
and, for illustration case, specialize \cref{eq:SDE_evolution} to the OU
system:
\begin{equation}
dX = \underbrace{(\a + \b( \mu - X)}_{U(x,\a; \th)} \intd{t} +
\underbrace{\s}_{\sqrt{2D}} dW
\label{eq:OU_evolution} 
\end{equation}
Whose parameter set is:
$$
\th = \{\m, \b, \s\}
$$
or a subset thereof.

We want to estimate $\th$  from observations $\{ X_t \}$, given that we have
latitude in choosing a perturbation (control) policy $\a(\cdot)$. We will assume
that we have cts. observations (i.e.\ very-high-frequency observations). 

\subsection{Some basic concepts/notation}
We will call $f$ the probability transition density associated with $X_t$, 
\begin{equation}
f(x,t| x_0 ;\th, \a(\cdot)) dx = \Prob[ X_t \in x + dx | x_0, \a, \th]
\end{equation} 

For a fixed parameter set, $\th$, and control policy, $\a(\cdot)$, $f$ is the
solution to a Fokker-Planck equation:
\begin{equation}
\di_t f = \L_{\th, \a(\cdot)}[f]
\label{eq:fokker_planck_forward_density}
\end{equation}
where the differential operator $\L$ is given by:
$$
\L_{\th, \a(\cdot)}[f] = D \cdot \di_x^2 [f] - \di_x[U(x,\a(x,t); \th) \cdot f]
$$
We will call $\L$ the forward operator and its adjoint, $\Lstar$, the backward
operator.


\section{Optimal Design using Mutual Information}
Our main point-of-departure in obtaining $\a(\cdot)$ is the concept of Mutual
Information (here are two books on information theory -
\cite{Cover2006,MacKay2003} :), see \cref{sec:mutual_info_defn} for
definitions).

In particular, two references suggest that one should use the Mutual
Information \cite{Myung2013,Lewi2009} as the criterion for designing a parameter
estimation experiment. (Designing the experiment is synonymous to
choosing the control policy $\a(\cdot)$.)

In order to apply the concept of mutual information, we need to have a prior on
the parameter set, $\th$. We will call this prior: 
$$\rho(\th).$$ 
Naturally a optimally-designed experiment is one which reveals most the actual
value of $\th$. That is we want to choose $\a(\cdot)$ such as to maximize the
Information from the random variable $\th$, given observations of the random
variable $X_t$.

First we need to consider the posterior of $\th$, $p(\th|X_t)$ given an
observation $X_t$ for $t$ fixed
\begin{subequations}
\begin{equation}
p(\th| X_t; \a) =
\frac{f(X_t,t|\th; \a)\cdot \rho(\th)}
{\int_\Theta f(X_t,t|\th; \a)\cdot \rho(\th)
\intd{\th}}
\label{eq:parameter_posterior_single_observation}
\end{equation} 
That is the transition density is the likelihood of the parameters.
\begin{equation}
L(X_t|\th ) = f(X_t, t| x_0, 0; \th, \a(\cdot))
\label{eq:paramter_likelihood_single_observation}
\end{equation}
\end{subequations}
However, if we have many observations $\{X_{t_n}\}$, the likelihood is, of
course, more complicated. Then 
\begin{subequations}
\begin{equation}
L(\{X_n\} | \th) = 
\prod_n (f(X_{t_{n+1}},t_{n+1}|X_{t_{n }},t_{n} ;\th)
\end{equation}
and the parameter posterior is then
\begin{equation}
p(\th| \{X_{t_n}\}; \a) =
\frac{L(\{X_{t_n}\},t|\th; \a)\cdot \rho(\th)}
{\int_\Theta L(\{X_{t_n}\},t|\th; \a)\cdot \rho(\th)
\intd{\th}}
\label{eq:parameter_posterior_multi_observations}
\end{equation}  
\end{subequations}
Once we have the concepts of the prior, posterior and likelihood, $\rho, p, L$,
then the mutual information, $I$, between the two random variables, $\th$, $X_t$
is: 
\begin{equation}
I(\a) = \int_\Theta \int_X  \log\left[\frac{p(\th| x; \a)}{\rho(\th)}\right]
\cdot L(x| \th;\a) \cdot \rho(\th) \intd{\th} \intd{x}
\label{eq:mutual_info_with_posterior}
\end{equation}
This is straight from equations 6,8,9 of \cite{Myung2013} (That paper is on the
OptEstimate Mendeley group). Also see \cref{sec:mutual_info_defn}.
 
Replacing the posterior in \cref{eq:mutual_info_with_posterior}, with the
bayesian formula from \cref{eq:parameter_posterior_single_observation} or
\ref{eq:parameter_posterior_multi_observations}, we get:
\begin{equation}
I(\a) = \int_\Theta \int_X  \log\left[\frac{L(x|\th; \a)}
										{\int_\Theta L(x|\th;\a)\cdot \rho(\th) \intd{\th}} \right] \cdot L(x|
										\th;\a) \cdot \rho(\th) \intd{\th} \intd{x}
\label{eq:mutual_info_with_likelihood}
\end{equation}

If $I$ is deemed to encode the information in $\th$ given $X_t$, it is
natural to seek the policy $\a(\cdot)$ that maximizes $I$.

However \cref{eq:mutual_info_with_likelihood} can appear deceptively
simple. The integral over $X$, $\int_X \intd{x}$, is really a multi-dimensional integral
over all observations. Thus just the evaluation of $I$ is only possible through
Monte Carlo methods, unless we are dealing with very few, say $N<5$ observations.
However if you have only one observation, as with the posterior given
\cref{eq:parameter_posterior_single_observation}, then we have a very simple
single integral over $x$, which can be easily evaluated approximately using any
quadrature rule.

A crucial assumption in the sequel is that we will NOT be updating the prior on
the fly, that is we will always have the same $\rho(\th)$ in
\cref{eq:mutual_info_with_likelihood}, even though new observations can
obviously be used to update the prior, $\rho$, through Bayes' formula.


\section{Simplest Example without any reference to Optimal Control Concepts}

Reconsider the system in \cref{eq:OU_evolution} 
$$
dX = \underbrace{(\a + \b( \mu - X)}_{U(x,\a)} \intd{t} +
\underbrace{\s}_{\sqrt{2D}} dW
$$

We will rederive the standard Maximum Likelihood formulas for $\th = \{\m, \b,
\s\}$ given the non-zero, but known evolution of $\a(\cdot)$.

$X_t$ can be solved explicitly for as:
\begin{align*}
dX =& (\a + \b( \mu - X) dt + \s dW
% \\
% (dX + \b X_t) dt =& (\a + \b \m) dt + \s dW
\\
(e^{\b t} dX + e^{\b t}\b X_t) dt
=&
e^{ \b t}( \a + \b \m) dt + \s e^{\b t} dW
\\
Xe^{\b t} - X_0 =&
\int e^{ \b t}( \a + \b \m) dt +  \int \s e^{\b t} dW
\\
\implies
X_t =& e^{-\b t} X_0 
		+ \frac{(\a + \b \m)}\b \cdot ( 1- e^{-\b t}) +
	       \s  \cdot \sqrt\frac{{ 1 - e^{-2\b t}}}{2\b} \cdot \xi 
\end{align*}
where $\xi$ is a standard normal RV. Crucially, we have assumed that $\a$ is
constant, o/w the integral $\int e^{ \b t}  \a(t) dt$ cannot be solved
explicitly.

Then if we have known initial conditions (ICs), i.e. if $X_0$ is constant 
$$
X_t \sim N \left(e^{-\b t} x_0 + \frac{(\a + \b \m)}\b \cdot ( 1- e^{-\b t}),
\quad  \s  \cdot \sqrt{ \frac{ 1 - e^{-2\b t}}{2\b}} \right)
$$

Consider discrete observations ${X_n, t_n}$ obtained at uniform
$t_n$, such that $\a$ is constant in between observations.
Then the transition probabilities $p_n(X_n|X_{n-1})$ are given by:
\begin{align*}
p_n(X_n|X_{n-1}; \m,\b, \s; \Delta_n) \propto &
\frac{\sqrt \b}{\s \sqrt{1 -  e^{-2\b \Delta_n}}}
\\ &\cdot 
\exp\left(\frac{\left( X_n - (\frac \a\b + \mu)  - (X_{n-1} - \frac \a\b - \mu) \cdot
e^{-\b \Delta_n} \right)^2 \cdot \b}
			{ \s^2  (1-e^{-2\b\Delta_n})} \right)
\end{align*}
TODO: Change symbol $p \ra f$ to keep consistent with FP notation.

The likelihood is simply the product:
\begin{equation}
L( \{X_n\}| ; \m,\b, \s; \Delta_n; \a) = \prod_n p_n(X_n|X_{n-1}; \m,\b, \s;
\Delta_n)
\label{eq:Likelihood_OU}
\end{equation}
And the log-likelihood of  ${X_n, t_n}$ is
\begin{align*}
l(\b, \m, \s | )=& \sum \log p_n(X_n|X_{n-1})
\\
=& \frac{N}{2} \log \frac{\b}{\s^2(1-e^{-2\b\Delta_n})}
\\ & -\sum_n
{\left( X_n - (\frac \a\b + \mu)  - 
		(X_{n-1} - \frac \a\b - \mu) \cdot e^{-\b \Delta_n} \right)^2 } \cdot
				\frac{\b}{\s^2  (1-e^{-2\b\Delta_n})}
\end{align*}

\def \Xn {{ X_n }}
\def \Xm {{ X_{n-1} }}
\def \deltan {{ \Delta_n }}
% SAGE Derivation:
% \\
% In order to use classical formulas for ML estimates of OU  processes, we will
% rewrite $l$ in terms of the variable $$ m = \frac{\a + \b \m}{\b}$$
% which in the case of $\a=0$ gives $m=\m$.

ML estimators for $\m,\b,\s$ are obtained via setting $\di_{\th} l$ to zero for
each parameter $\th$. (ignore for now that $\Delta_n$ is not the same
throughout). However, it turns out that it is easier to first compute the ML estimate for
$\s$ and then plug it back into the likelihood, $l$, to simplify things:
\begin{align}
\di_{\s} l()=& -\frac N\s + 2\sum_n
\frac{ \left( X_n - e^{-\b \Delta_n} X_{n-1} -
				 (\frac \a\b + \m) \cdot ( 1-e^{-\b \Delta_n}) \right)^2 \b}
				 {\s^3 \cdot (1-e^{-2\b\Delta_n})}
				 \notag
				 \\ 
\implies \hat\s^2 =& 2\sum_n \frac{ \left( X_n - e^{-\hat\b \Delta_n} X_{n-1} -
				 (\frac {\a}{\hat{\b}} + \m) \cdot ( 1-e^{-\hat\b \Delta_n}) \right)^2
				 \hat\b} {N \cdot (1-e^{-2\hat\b\Delta_n})}
				 \label{eq:sigma_root}
\end{align}
With that the likelihood becomes:
\begin{align*}
l(\b, \m |\, X_n )=& 
 -N\log \left( 
 \frac{ \sum_n\left( X_n - e^{-\b \Delta_n} X_{n-1} -
  (\frac {{\a}}{{\b}} + \m) \cdot ( 1-e^{-\b \Delta_n})
  \right)^2}{N} \right) - 1
\end{align*}
Now maximizing the negative of a log plus a const is the same as minimizing the
argument of the log. So we need to {\itshape minimize}
\begin{align*}
l(\b, \m |\, X_n )\equiv& 
\sum_n \left( X_n - e^{-\b \Delta_n} X_{n-1} -
  (\frac {{\a}}{{\b}} + \m) \cdot ( 1-e^{-\b \Delta_n})
  \right)^2 
\end{align*}
% That is a non-linear least squares problem! And in fact it is bi-linear in the
% variable $$ b = \exp(-\b \Delta)$$ Billinear because of the term $\m b$.
 
Let's differentiate
\begin{subequations}
\begin{align}
\di_{\m} l()=& 2 \sum_n \left( X_n - e^{-\b \Delta_n} X_{n-1} -
  (\frac {{\a}}{{\b}} + \m) \cdot ( 1-e^{-\b \Delta_n})  \right) 
  \cdot   ( 1-e^{-\b \Delta_n})
\\
\di_{\b} l()=& 2 \sum_n \left( X_n - e^{-\b \Delta_n} X_{n-1} -
  (\frac {{\a}}{{\b}} + \m) \cdot ( 1-e^{-\b \Delta_n})  \right) 
  \\ \notag 
  & \times \left( \Delta_n e^{- \b \Delta_n}X_{n-1} 
  							  + \frac{ \a } {\b^2} (1-e^{-\b \Delta_n}) 
  					 		  - (\frac {{\a}}{{\b}} + \m) ( \Delta_n e^{- \b \Delta_n})          
  					 		  \right)
\end{align}
\end{subequations}
The $\m$ equation is straight-forward to solve

\begin{align}
\hat{\m} =&  \frac{ \sum_n \left( X_n - e^{-\b \Delta_n} X_{n-1} -
	\frac{\a}{\b} ( 1-e^{-\b \Delta_n}) \right)}
	{N ( 1-e^{-\b \Delta_n})}
	\label{eq:mu_root}
\end{align}

This means that we only need to solve one equation in one unknown
\begin{align}
\label{eq:beta_root}
0=& \sum_n \left( X_n - e^{-\b \Delta_n} X_{n-1} -
  (\frac {{\a}}{{\b}} + \m) \cdot ( 1-e^{-\b \Delta_n})  \right) 
  \\ \notag 
  & \times \left( \Delta_n e^{- \b \Delta_n}X_{n-1} 
  							  + \frac{ \a } {\b^2} (1-e^{-\b \Delta_n}) 
  					 		  - (\frac {{\a}}{{\b}} + \m) ( \Delta_n e^{- \b \Delta_n})          
  					 		  \right)
\end{align}
for $\b$ and then plug into \cref{eq:mu_root,eq:sigma_root}

(I have checked that in the case of $\a=0$ this reduces to well-known
expressions!)

Note: we've been quite cavalier about the constancy of $\Delta_n$, mostly
treating it as a constant, in order to focus on the impact of $\alpha$. Later we
can go back and be a little more rigorous, treating $\Delta_n$ as a function of
$n$.
%  
% On the contrary, while not being explicit in the notation, we have never assumed
% $\a$ is constant and everything above can be rewritten in terms of $\a_n$.

\subsection{Optimal Design}
Now we proceed to apply the Mutual Information criterion in order to choose the
controls $\a(t) = \a(t_{n-1})$, assumed piecewise constant over $\Delta_n$, such
as to facilitate the estimation of the parameters $\m,\b,\s$?
   
Let us call the prior over the parameters $\th = \{\m, \b, \s\}$:
$$\rho(\th)$$
and the posterior over $\th$ given $X$:
\begin{equation}
p(\th| X; \a) =
\frac{L(x|\th; \a)\cdot \rho(\th)}{\int_\Theta L(x|\th; \a)\cdot \rho(\th)
\intd{\th}}
\label{eq:parameter_posterior_defn}
\end{equation} 
Where $ L(X|\th; \a)$ is the likelihood of $X$ given in
\cref{eq:Likelihood_OU}. $X$ could represent only one observation, $X_n$ or a
set of observations $\{X_k\}_n^{n+K}$.

Then we seek to find the $\a$ that maximizes the
mutual information:
\begin{equation}
I(\a) = \int_\Theta \int_X  \log\left[\frac{p(\th| x; \a)}{\rho(\th)}\right]
\cdot L(x| \th;\a) \cdot \rho(\th) \intd{\th} \intd{x}
\label{eq:mutual_info_objective}
\end{equation}
This is straight from equations 6,8,9 of \cite{Myung2013} (That paper is on
Mendeley)
 
Replacing the posterior in \cref{eq:mutual_info_objective}, with the Bayesian
formula from \cref{eq:parameter_posterior_defn}, we get:
\begin{equation}
I(\a) = \int_\Theta \int_X  \log\left[\frac{L(x|\th; \a)}
										{\int_\Theta L(x|\th;\a)\cdot \rho(\th) \intd{\th}} \right] \cdot L(x|
										\th;\a) \cdot \rho(\th) \intd{\th} \intd{x}
\label{eq:mutual_info_objective_posteriored}
\end{equation}

Again, the whole magic is to find the $\a$ that gives the highest value of $I$.

Now we need to consider how are we going to represent/choose the parameter prior
distribution $\rho(\th) = \rho(\m,\b,\s)$? 

One very simplistic way to proceed is to let the system roll on unperturbed,
then to obtain ML estimates for $\th$ using a small initial segment of $X_t$
and then to take a Gaussian approximations centred at the so-obtained estimates. 
 
So  
\begin{equation}
\rho(\th) \propto 
\exp\left( (\th - \hat\th) \cdot \hat\Xi^{-1} \cdot (\th-\hat\th)\right))
\label{eq:prior_gaussian_approxmn}
\end{equation}

It is not entirely clear how to compute the covariance matrix, $\Xi$, perhaps
something related to the Fisher Information of the ML estimates might be viable. 

 \subsubsection{Curse of Dimensionality given mulit-observations}

As we already mentioned in \cref{eq:mutual_info_objective} we have integration
wrt.\ the RV $x$. If, this is only one observation of the process, then $x$ is
just $X_n$, but if we take this to be $K$ observations: then 'x' is
$\{X_k\}_n^{n+K}$ and we have a $K$ dimensional integral\ldots However it does
factor as we can integrate backwards, first wrt $X_{n+K}$ then $x_{n+K-1}$ and
so on all the way to $X_{n}$.
%  I must say this is all quite painful.

One thing to do is for fixed $\th$ to sample a few, say $M$, $X$ paths. Then the
objective will look like:
\begin{equation}
I(\a) = \int_\Theta \sum_{X_i|\th}  \log\left[\frac{L(X_i|\th; \a)}
										{\int_\Theta L(X_i|\th;\a)\cdot \rho(\th)\intd{\th}} \right]
										 \cdot \rho(\th) \intd{\th}
\label{eq:mutual_info_objective_particlized}
\end{equation}

Now suppose that $\th$ is chosen using some kind of a Gauss-Hermite quadrature
scheme in 3-d. Say that requires 125 points (that is only five points in each
direction, so quite conservative, but also should give reasonable accuracy).
That means that to evaluate $J$ we need to sample $M \times 125$ paths. And then do
the summation. But wait, there's more. For each $x,\th$ pair we also need to do
the integral for the normalizing constant\ldots so now we have $M \times 125^2$
process samples in order to naively evaluate $I$. 
Thus we might first consider, what if we only considered 1-slice ($X_n$)
observations.

\subsection{1-Slice Illustration}
Let us make a simple proof-of-concept.

We will take for the true parameters
$$
\b = .05; \m = -60; \s = .1;
$$
Now since the values of $\m, \s$ are determined by $\b$, let us reduce the prior
to a one-dimensional Gaussian
$\rho(\b) \propto \exp( -(\b -\hat \b)^2 / {\s_\b}^2) $
and then for a given $\b$ we will obtain $\m, \s$ from
the formulas in \cref{eq:mu_root,eq:sigma_root}.

We will only focus on the next observation, so the likelihood is also a Gaussian
distribution (conditional on $\m,\b,\s$) in 1-d. 

Let us write it out using $x_0$ as the current value and $x$ as the future
value and $\Delta$ as the time interval until the next observation:
then the terms $L,\rho$ in the mutual information
\begin{align*}
I(\a) =& \int_\Theta \int_X  \log\left[\frac{L(x|\th; \a)}
										{\int_\Theta L(x|\th;\a)\cdot \rho(\th) \intd{\th}} \right] \cdot L(x|
										\th;\a) \cdot \rho(\th) \intd{\th} \intd{x}
										\end{align*}
										are given by:
\begin{align*}
L(x|\th; \a) =& \frac{\b}{\s \sqrt{2\pi(1 -  e^{-2\b \Delta}})}
 	\cdot \exp\left(\frac{\left( x - (\frac \a\b + \mu)  - (x_{0} - \frac \a\b
 	- \mu) \cdot e^{-\b \Delta} \right)^2 \cdot \b} { \s^2  (1-e^{-2\b\Delta})}
 	\right) 
 	\\
 	\rho(\th) =&  
 	\frac{1}{\s_\b \sqrt{2\pi} } \exp(-\frac{(\b-\hat\b)^2}{\s_\b^2}) 
		\end{align*}
Thus for each $x$ we need to form a Gaussian integral for the normalizing
constant. And on top of that we need to make a two dimensional independent
gaussian integral for the $x,\b$.

Let us illustrate this. Start with a sample of duration 50 ms sampled at .1 ms ($500$ observed pts.). Then if we
take a coarser sampling of 1s to create our boot-strap of estimates for $\beta$
(50 observations in each sub-sample) we get:
\begin{equation*}
\begin{gathered}
\hat {\hat \beta} \quad \s_{\hat\b} \\
0.0881, 0.0182\\
0.0974, 0.0514\\
\boxed{0.0902, 0.0562} // \textrm{used in subsequent simulations}\\
0.3060, 0.2968\\
0.0696, 0.0386\\
0.1947, 0.0790\\
0.1702, 0.0624\\
0.1706, 0.0481  // \textrm{way off including too small std}\\
 0.1332, 0.0879\\
\end{gathered}
\end{equation*}
That is actually ok. In 1 case, we couldn't actually estimate $\b$ (the data
implies there is no solution to \cref{eq:beta_root}). In the other 9 cases, 8
times the true value (.05) is within 2 standard deviations of the mean estimate
and in 5 cases it is within 1 standard deviation of the mean.

Note that calculating $\s_\b$ as $$ \s_\b^2 = \sum(\hat{ \hat \b} - \hat \b)^2
$$ where $\hat{ \hat \b}$ are bootstrap estimates from a reduced data set helps
increase the value of $\s_\b^2$. If instead of $\hat \b$, we use the mean of the
bootstrap values themselves to calculate $\s_\b$, we will get smaller values for
$\s_\b$, which will usually put the true value more than two standard deviations
from the short-time estimate.

Let's use the values of $\hat \beta, \s_{\hat\b} =  0.0902, 0.0562$ and
visualize the resulting distributions for $\b$ and the functions $\m(\b),
\s(\b, \m)$, see \cref{fig:prior_mu_sigma}. Essentially what we see is that $\m$
depends on $\b$ inversely while $\s$ does not really depend on $\b$ given the
data and a $\m$ estimate. These two facts seem reasonable given
\cref{eq:mu_root,eq:sigma_root}, where $\mu$ is inversely proportional to
$\b$ and $\s$ is proportional to the product of $\mu \beta$ once the term
$\exp(-\beta \Delta)$ is ignored, and ignoring it is likely justified as long as
$\Delta$ is small enough\ldots

\begin{figure}[h]
\begin{center}
\subfloat[prior]
{
\label{fig:prior}
\includegraphics[width=0.48\textwidth]{Figs/MI/prior_example.pdf}
}
\subfloat[underlying $X$ path]
{
\label{fig:prior_path}
\includegraphics[width=0.48\textwidth]
{Figs/MI/prior_example_path.pdf}
}
\caption[labelInTOC]{An example of a path, a prior over $\b$ built based on
the path and the resulting functional relations between $\b$, $\m$ and $\s$,
which are from \cref{eq:mu_root,eq:sigma_root}. The prior for $\b$ and the
resulting $\m,\s$ are obtained using the path in (b)}
\label{fig:prior_mu_sigma}
\end{center}
\end{figure}
At this point we might start to think a normal prior for an a priori positive
random variable, like $\b$, is a bad idea, especially if we are near zero and the std.
dev. of the $\b$ belief distribution is non-negligible. Perhaps, we should use
a log-normal prior or a gamma distribution \ldots


Now let's calculate the integral of the likelihood wrt.\ the prior for a fixed
$x$, ie. the marginal forward distribution of $x$. $$ p(x) = \int_\Theta
L(x|\th;\a)\cdot \rho(\th) \intd{\th} $$ This is also the normalizing constant
in bayes rule. We will use a forward horizon of  $\Delta_f = 5$ (ms) (Recall the
data used to generate the prior distributions had $\Delta = 0.1$ and has length
$T = 50$ ms). $p(x)$ is shown in \cref{fig:marginal_px}. Essentially,
\cref{fig:marginal_px} is telling us that the current value of $x$ is lower than
the estimate for the long-term mean, all things considered. Our current
estimates for $\b, \m, \s$ are letting us believe that it should move up
towards approximately $59.5$ provided that we continue with the
current value of $\a=0$. 

The two opposing proposed values for $\a = \pm 0.25$ result in much
more spread distributions for $X_{t+\Delta_f}$. In particular, one might argue that
$\a=-0.25$ is most informative, since the forward distribution has the most
spread. (if we intuitively equate spread with information). This is consistent
with the notion that the most informative experiments are the ones that lead $X$
furthest from its current equilibrium (given $\a=0$). In this case the negative
is better than positive, since the current value of $X_t$ is already negative in
relation to the current equilibrium. Basically this is consistent with the
following selection mechanism of $\a$: If $X_t > \m$ choose $\amax$ otherwise
if $X_t < \m$ chose $\amin$. This is illustrated in
\cref{fig:marginal_px_shifted}, where we artificially move the value of $X_t$ to
the right of the (current) equilibrium, which makes the forward distribution
arising from $\a=0.25$ more informative (higher spread), then the one with
$\a=-0.25$.

\begin{figure}[htp]
\begin{center}
\subfloat[Left of equilibrium]{
  \includegraphics[width=.5\textwidth]{Figs/MI/x_marginal_example.pdf}
  \label{fig:marginal_px}
}
\subfloat[Right of equilibrium]{
  \includegraphics[width=.5\textwidth]{Figs/MI/x_marginal_example_x0_shifted_right.pdf}
    \label{fig:marginal_px_shifted}
  }
  \caption[labelInTOC]{
  Marginal of $x$ (normalizing constant), the tall red stem
  is the current value of $X_t$. This is the last observed value based on
  which we compute transition densities. The lower, blue stems are all the
  other data previously observed. The blue and red curves are forward densities
  given different applied forward values for $\a$. The green curve is the forward density given the hitherto used value of $\a=0$. It mostly coincides
  with the so-far observed data(the blue stems).
  In float (b) we artificially move the starting point to the right. Note that
  the green curves in a), b) are not the same, since they depend on the value
  of $X_t$ as well as the observed data ($X_t$ is different in the two
  panels, but all other data points, the blue stems, are the same)}
\end{center}
\end{figure}

Let us verify this intuition formally, by calculating $I(\a)$, for $\a =[-0.25 , 0, 0.25]$.

Aside: Unfortunately, naively calculating the double integral (in Python using
'quad' or 'romberg') is not numerically efficient. Calculating $I(\a)$ for a
single $\a$ takes on the order of 10 secs, once you relax the quadrature
tolerances without incurring any significant error\ldots Since we are dealing
with Gaussian-type integrals, a Gauss-Hermite Integration scheme might be very
effective, but we will leave that for now.

Crushing through the integration with brute force we get the result
in \cref{tab:MI_3alphas_basic_quad}.
\begin{table}
\begin{centering}
\begin{tabular}{cc}
$\a$& $I(\a)$ \\
-0.25  &
 0.688 
\\
0.00 &
   0.100 
\\
   0.25 &
   0.419 
\end{tabular}
\caption{values for the mutual information, $I(\a)$, for various values of
$\a$, starting from the value of $X_t$, (the red stem in
\cref{fig:marginal_px})}
\label{tab:MI_3alphas_basic_quad}
\end{centering}
\end{table}
We have used the same starting (current) value of $X_t$ to form the forward
likelihoods as in \cref{fig:marginal_px} and indeed we get the result.
$$I(-0.25) >I (.25) > I(.0)$$
which is consistent with our expectations after calculating the corresponding
$p(x)|\a$ in \cref{fig:marginal_px}. This basically says that it should be most
informative to stimulate down ($\a<0$), and it should be least informative to
do nothing $\a=0$.

 \subsection{Follow-up Estimation}
 Let's now see what that means in practice. Pretend that we have done the
 analysis instantaneously and let the process unroll further from $X_t$ for
some time. Then we will check if there is any advantage to
 using the mutually most informative $\a$, i.e.\ $\a = \argmax I(\a)$.
First we generate 10 forward trajectories of duration 2$\Delta_f = 10$ (ms).
They are shown in \cref{fig:perturbed_trajectories}. 
\begin{figure}[htp]
\begin{center}
  \includegraphics[width=1\textwidth]{Figs/MIML/forward_sims.pdf}
  \caption[labelInTOC]{Different Trajectories perturbed by different values of
  $\a$ after the MI calculation}
  \label{fig:perturbed_trajectories}
\end{center}
\end{figure}
% 
Now what we would like to see is that the estimates corresponding to $\a=-.25$
are 'better' than the ones corresponding to $\a=.25$ and that they are much
better than the ones corresponding to $\a=.0$.

Let's see: The resulting estimates for the 10 trajectories are shown in
\cref{fig:perturbed_estimates}. Well, what do you know, visually, it is clear
that indeed $\a=-.25$ is  'better' than $\a=.25$, which in turn is better than $\a=.0$.
\begin{figure}
\begin{center}
\includegraphics[width=1\textwidth]{Figs/MIML/perturbed_estimates.pdf}
\caption[]{The estimates for $\b$ given the three perturbed
trajectories (one is actually un-perturbed ($\a=.0$)). The solid black line
indicates the true value of $\b$}
\label{fig:perturbed_estimates}
\end{center}
\end{figure}

\subsection{Bang-Bang?}
We expect that it is actually best to apply maximum inhibition or maximum
excitation. That is we expect that $I(\a_2) > I(\a_1)$ for $|\a_2| > |\a_1|$. We
verify this in \cref{tab:MI_bang_bang_alphas}, where we see the general tendency
that bigger is more informative than smaller and negative is more informative
than positive.
\begin{table}
\begin{centering}
\begin{tabular}{cc}
$\a$& $I(\a)$ \\
-2&  2.31 \\
-1.00 & 1.714 \\
-0.50 & 1.182 \\
0.50 & 0.885 \\
1.00 & 1.563 \\
2.00 &  2.20 \\
\end{tabular}
\caption{values of MI}
\label{tab:MI_bang_bang_alphas}
\end{centering}
\end{table}
\Cref{tab:MI_bang_bang_alphas} and our intuition suggest then that the choice
for which is always between the two extremes st.\ $\a_{\textrm{most
informative}} \in \{\amin, \amax\}$.

Now we wonder what happens to the calculated value of $I$ as we move $\Delta_f$,
see \cref{fig:MI_delta_f_variation}. Now this is interesting. As $\Delta_f$
increases, we have a raise in the mutual information. Intuitively this is
obvious, since bigger $\Delta_f$ means more data. However, recall that we are
only considering the information contained in the final value (at $t=\Delta_f$).
That is also why $I$ levels off eventually, if we were considering the full path
and not just the final value, then it should continue increasing monotonically,
although perhaps with a decreasing slope. However! It is also clear that while
in the current context and for the current example at this time, inhibition is
best $I(-) > I(+)$. At some point in the future excitation will be better! That
is as the $X$ variable settles into its new, lower, equilibrium, $I(+) > I(-)$.

The question becomes how to formulate this problem as to decide on when to
switch. 
%\usepackage{graphics} is needed for \includegraphics
\begin{figure}[htp]
\begin{center}
  \includegraphics[width=1\textwidth]{Figs/MI/forward_deltas_vs_MIs.pdf}
\caption{values of MI while varying $\Delta_f$. Note that the right-most value
is a little iffy, as the integration routines warn about possible problems with
the various integrals}
	\label{fig:MI_delta_f_variation}
\end{center}
\end{figure} 
Let me explain why, potentially, this is an interesting problem, what is a
possible solution and why that solution is possibly very difficult to enact:

\subsection{Optimal Switching for Optimal Design}
Let us recap where (we think) we are:
There is an observed OU process $X_t$. We estimate it on the fly and have
estimates for $\b, \m, \s$ or more precisely we have a distribution for $\b$ and
a one-to-one relation between $\b$ and $\m,\s$. 

We have a control of $\a$ with which we can stimulate $X$. We want to use $\a$
to improve the observations of $\b, \m, \s$. We use the Mutual Information
criterion, $I(\a)$ to select $\a$. From the structure of the OU process, it is
conjectured and empirically observed that the bigger in magnitude $\a$ the more
informative it will be. Thus, in the absence of an energy cost, we are left only
to select between the two extreme values of $\a$, $[\amin, \amax]$, which
we can assume to be just $[\pm \amax]$ for some $\amax>0$. Thus at any time, $t$,
we can calculate $I(\amin), I(\amax)$ and choose the $\a$ associated with the larger $I$.

However, in a sense this is greedy and thus not necessarily optimal.

Here is a simple way to think about it:

Imagine that $\amin, \amax$ correspond to two equilibria
$\xmin, \xmax$. Our intuition is that there is a mid-point $\xmid$ between
$\xmin, \xmax$ st. if $X_t < \xmid$, we should apply $\amax$ and conversely. 
It is now clear that this can easily result in chattering - being below $\xmid$
we stimulate which sends us above $\xmid$ and then we inhibit, since above
$\xmid$ it is most informative to inhibit and so on. 

When you add the noise, it is hard to see if that is even a better idea than
doing nothing. 

Of course, things are not so simple as the distribution on the parameters,
$\rho$ may make $\xmid$ itself move as $\rho$  shifts and shrinks, but let's
ignore that for now. 

At this point it becomes clear that we need a way to choose the switching time
to switch between  $\amax, \amin$ using something more sophisticated than just
the instantaneous value of $I(\a)$. This is related to the problem of
how to choose the observation time $\Delta_f$ in the formulation of $I(\a)$. 

Basically we need to select what we are going to do now in part based on what we
can do later. And that is Dynamic Optimization!

The main difference here from standard Dynamic Optimization is
that our state is not so much the value of $X_t$ but the value of $\hat \b_t,
\hat \m_t, \hat \s_t$, the estimates at time $t$. Once we realize this, we also
realize our main challenge - the updates for $\b, \m, \s$ using the ML
formulas are non-Markovian! That is we look back on all the old
data when taking the new data into account to form $\hat \b_t, \hat \m_t, \hat
\s_t$. 

We could consider further one of two things
\begin{enumerate} 
  \item come up with a heuristic way of choosing $\Delta_f$ before which to
  consider switching (if $\Delta_f$ is large enough, we will always switch (I
  think))
  \item Come up with an incremental form for updating $\b$
\end{enumerate}

In the next section we do NEITHER:) Instead we consider the Optimal Design
problem, ie. the selection of $\a(\cdot)$ from the point-of-view of
Stochastic Optimal Control Theory.



\section{Finding the optimal design $\a(\cdot)$ using Stochastic
Optimal Control}

In principle now, we have a criterion, the mutual information in
\cref{eq:mutual_info_with_likelihood}, using which to select the most
informative perturbation $\a(\cdot)$. Intutitively, the 'optimal'  $\a(t)$
depends on the up-to $t$ realization of $X_s ; s\leq t$. This is clearly a
problem from optimal stochastic control. However, the objective in
\cref{eq:mutual_info_with_likelihood} is not in the form needed to apply dynamic
programing (or the maximum principle for that matter). The multi-dimensional
integral in $X$ makes things 'too' complicated.

On the other hand, a possible simplification is to take the mutual information
criterion in \cref{eq:mutual_info_with_likelihood} using the single observation
likelihood in \cref{eq:paramter_likelihood_single_observation} and just
integrate it over time.
\begin{equation}
J(\a)  = \int_\th \int_0^T\int_{X_t}
 \log\left(\frac{f(x,t|\th ; \a(\cdot) )}
 			{\int_\th f(x,t|\th; \a(\cdot)) \rho(\th)\intd{\th}}\right) 
 f(x,t|\th; \a(\cdot)) \cdot \rho(\th) \intd{x}\intd{t}\intd{\th}
\label{eq:mutual_info_time_integrated}
\end{equation}
where the $dx$ integral has the same dimension as the dimension of the SDE not
the dimension of the SDE times the number of observations. I should repeat that
$J$ is NOT the mutual information between a realization of the process, $\{
X_t\}_0^T$ and the parameter set, $\th$. It is the time-integral of the
individual mutual informations between each $X_t$ at time $t$ and the parameter
set $\th$. 

The reason we would like to consider this simplification is that now
$J$ can be written as
\begin{equation}
J( \a) = \Exp_{\th} \Bigg[
\Exp_{X_t|\th; \a(\cdot)}
\left[ \int_0^T \log\left(\frac{f(X_t,t|\th; \a(\cdot))}
{\int_\th f(X_t,t|\th; \a(\cdot)) \rho(\th)\intd{\th}}\right) \intd{t} \right]
\Bigg] 
\label{eq:mutual_info_time_integrated_as_expectations}
\end{equation}
which almost has the classical form of a stochastic optimal control
problem, except for two non-standard features:
\begin{enumerate}
  \item We have an extra outer expectation, this is the integration wrt. to the
  parameter prior.
\item There is a forward-backward coupling in the determinations of the optimal
control, meaning we can't just back out, the optimal control with a backwards
solution to an HJB equation
\end{enumerate}
Of these two features, we discuss pt. 1 first.

\subsection{Stochastic Optimal Control with a Prior}
We would like to apply Dynamic Programing to the problem of maximizing
\cref{eq:mutual_info_time_integrated_as_expectations}. However
\cref{sec:DP_with_a_prior} shows why this is impossible (or at least why the
first thing one thinks of does not work).

Instead, in \cref{sec:MP_with_a_prior}, we use the Maximum Principle in order to
find the optimal $\a(\cdot)$.

\subsubsection{Dynamic Programing with a Prior}
\label{sec:DP_with_a_prior}

WARNING: THIS SECTION ULTIMATLY EXPLAINS WHY YOU \underline{CANNOT!} USE
\cref{eq:HJB_equation_with_prior}. I.E. WHY DYNAMIC PROGRAMING CANNOT! BE
USED TO FIND THE CONTROL, $\a(\cdot)$. 

\vskip 15pt

In order to apply Dynamic Programing, i.e. in order to set up an HJB PDE, let us
write our objective as:
\begin{equation}
J(x_0, 0; \a) = \Exp_\th \Bigg[ \Exp_{X_0^T|\th, \a} \bigg[ \int_0^T r(X_t|
\th)\intd{t} \bigg]\Bigg]
\label{eq:stochastic_objective_with_prior_generic} 
\end{equation}
where, in our case, the reward $r$ is the mutual information between $X_t$ and
$\th$ 
\begin{equation}
r(x|\th) = \log\left(\frac{f(x,t|\th; \a(\cdot))}
{\int_\th f(x,t|\th; \a(\cdot)) \rho(\th)}\right)
\label{eq:reward_funciont_mutual_information}
\end{equation}
But we can consider $r(\cdot)$ as a generic function of $X_t$.

Now we try to follow the standard dynamic programing approach to obtain an
HJB-type equation. 

Call $w$ the value function, i.e.\ the optimal reward-to-go.
\begin{equation}
w(x_t, t) = \sup_{\a(\cdot)} 
\Exp_\th \Bigg[ \Exp_{X_t^T|\th, \a} \bigg[ \int_t^T r(X_t| \th) \intd{t}
\bigg]\Bigg]
\label{eq:value_function_with_prior_defn}
\end{equation}
Or in particular, starting from $x_0$ at $t=0$ 
\begin{equation}
w(x_0, 0) = \sup_{\a(\cdot)} 
\Exp_\th \Bigg[ \Exp_{X_0^T|\th, \a} \bigg[ \int_0^T r(X_t| \th) \intd{t}
\bigg]\Bigg]
\end{equation} 

$$
\Exp_{X_0^T|\th, \a} [\cdot] 
$$
means the expectation over the full trajectory of $X$ from $0$ to $\T$, $X_0$
held fixed, while evolving under the parameter set $\th$ and the control
policy $\a$.
$$
\Exp_{\Delta X|\th, \a} [\cdot] 
$$
Is the expectation over the single point realization of $\Delta X = X(\Delta t)
= X_{\Delta t}$, again under a given parameter set and policy, $\th, \a$.

The Markovian nature of $X_t|\th$ implies that for $x_0$ fixed.
$$
\Exp_{X_0^T|\th, \a} [\cdot ] =
\Exp_{\Delta X|\th} \Big[ \Exp_{X_{\Delta t}^T|\th, \a} [ \cdot  | \Delta X ]
\Big] $$

With that we can start deriving an equation for $w$, starting from
\begin{align*}
w(x_0, 0) =& \sup_{\a(\cdot)} 
\Exp_\th \Bigg[ \Exp_{X_0^T|\th, \a} \bigg[ \int_0^T r(X_t| \th) \intd{t}
\bigg] \Bigg]
\\
=&  \sup_{\a(\cdot)} 
\Exp_\th \bigg[ r(x_0) \Delta t \bigg] + 
\Exp_\th \Bigg[ \Exp_{X_0^T|\th, \a} \bigg[ \int_{\Delta t}^T r(X_t| \th)
\intd{t} \bigg] \Bigg]
\end{align*}
All we've done so far is split the time integral into an incremental initial
part which is approximately equal to $r(x_0)\dt$ and the rest $\smallint_\dt^T$.
Now let's focus on the second term, $\Exp_\th \Big[ \Exp_{X_0^T|\th, \a} \big[ \int_{\Delta t}^T r(X_t| \th)
\intd{t} \big] \bigg]$ and condition on $x_0 + \Delta X$:
\begin{align*}
\Exp_\th \Bigg[ \Exp_{X_0^T|\th, \a} \bigg[ \int_{\Delta t}^T r(X_t| \th)
\intd{t} \bigg] \Bigg]
=&  
\Exp_\th \Bigg[ \Exp_{\Delta X |\th, \a} \Big[ \Exp_{X_{\dt}^T|\th, \a}
\int_\dt^T r(X_t| \th) \intd{t} | X_\dt \Big] \bigg] \Bigg]
\end{align*}

Now here is the main problem! We WOULD LIKE to say that
\begin{multline}
\Exp_\th \Bigg[ \Exp_{\Delta X |\th, \a} \Big[ \Exp_{X_{\dt}^T|\th, \a}
\int_\dt^T r(X_t| \th) \intd{t} | X_\dt \Big] \bigg] \Bigg] 
= \\
\Exp_\th \Bigg[ \Exp_{\Delta X |\th, \a} \Big[ 
\underbrace{\boldsymbol{\Exp_\th}}_{\uparrow\textrm{add this?} \uparrow} \Big[
\Exp_{X_{\dt}^T|\th, \a} \int_\dt^T r(X_t| \th) \intd{t} | X_\dt \Big]\Big]
\quad \bigg] \Bigg]
\label{eq:incremental_bellman_with_falsely_added_prior}
\end{multline}
Because then we could plug in $w(x_0 + \Delta X, \Delta t)$ in:
$$
\Exp_\th \Bigg[
\Exp_{X_{\dt}^T|\th, \a} \int_\dt^T r(X_t| \th) \intd{t} | X_\dt \Big]\Bigg] =
 w(x_0
+ \Delta X, \Delta t) $$
And then the rest rolls off easily to get the PDE:
\begin{equation}
\di_t w(x,t) + \sup_{\a(x,t)} \bigg\{  \Exp_\th \big[\Lstar_\th [w] +
r(x|\th)\big] \bigg\} = 0
\label{eq:HJB_equation_with_prior}
\end{equation}
Where $\Lstar_\th$ is the generator (backward Kolmogorov operator) corresponding
to the SDE \cref{eq:SDE_evolution} for fixed parameters, $\th$,
$$
\Lstar_\th[\cdot] = U(x,\a; \th) \di_x[\cdot] + D \di_x^2[\cdot]
$$

HOWEVER! Can we just put the extra
$\Exp_\th$ in \cref{eq:incremental_bellman_with_falsely_added_prior}? 

It comes down to what exactly is the meaning of the prior on
SDE parameters. Is it that:
\begin{enumerate}
  \item You choose $\th$ at each time-step (infinitesimally going to 0) let $X$
  evolve accordingly for an increment $dt$ and then choose $\th$ again.
  \\
  or
  \item You choose $\th$ at time $0$ and then let $X$ evolve accordingly to this
  once-and-for-all fixed $\th$
\end{enumerate}

If it is the former, then we can indeed add the extra expectation wrt.\ $\th$
in \cref{eq:incremental_bellman_with_falsely_added_prior} and then use
\cref{eq:HJB_equation_with_prior} to compute $w$. If it is the latter, then we
cannot.

HOWEVER! Assuming pt.1, i.e.\ that we re-choose $\th$ at each incremenent,
fundamentally violates the basic point of the parameter estimation.

Let me explain.

We suppose that the underlying process $X$ is governed by a single value of
$\th$, we just don't know which. We would like to observe the
trajectory of $X$ so as to determine which is the underlying value of $\th$, but
if we re-choose $\th$ at each time-increment of $X$'s evolution, then there is
NO single $\th$ and indeed the whole estimation problem is moot.

So adding the inner expectation wrt.\ $\th$ in
\cref{eq:incremental_bellman_with_falsely_added_prior} is wrong! And solving
\cref{eq:HJB_equation_with_prior} will not at all help in finding the maximally
informative stimulus $\a(\cdot)$.

More mundanely, I actually, computed the solution to
\cref{eq:HJB_equation_with_prior} for the Double-Well Potential and it gives
non-sense results (for $\a$ AND $w$\ldots)

We must try something else!


\subsubsection{Maximum Principle with a Prior}
\label{sec:MP_with_a_prior}
Let us go back to the original objective,
\cref{eq:mutual_info_time_integrated} or equivalently
\cref{eq:mutual_info_time_integrated_as_expectations}
$$
J(\a)  = \int_\th \int_0^T\int_X
 \log\left(\frac{f(x,t|\th ; \a(\cdot) )}
 			{\int_\th f(x,t|\th; \a(\cdot)) \rho(\th)\intd{\th}}\right) 
 f(x,t|\th; \a(\cdot)) \cdot \rho(\th) \intd{x}\intd{t}\intd{\th}
$$

$J$ then is a functional of a family of distributions $f(|\th)$ parametrized by
$\th$.

We will write this as:
$$
J(\a)  = \Exp_\th \left[ \int_0^T\int_{\O_X}
r(f(x,t|\th; \a(\cdot))) \cdot   
 f(x,t|\th; \a(\cdot)) \intd{x}\intd{t} \right]
$$ 

Then optimizing $J$ looks a lot like the a generic problem in optimizing over
PDEs with the added complexity of the outer expectation (the one wrt.\ $\th$).

We now attempt to set up a Pontryagin-Type equation for the optimal value of
$\a$: We start by augmenting the objective with the dynamics:
\begin{equation}
J =  \Exp_\th
\left[ \int_0^T\int_{\O_X} r(f) \cdot f - p \cdot (\di_t f - \L_{\th;\a}[f])
\intd{x}\intd{t}\right] 
\label{eq:objective_augmented}
\end{equation} 
where $p =  p(x,t|\th; \a(\cdot))$ is the adjoint co-state and $\di_t f -
\L_{\th;\a}[f]$ is the Fokker-Planck equation,
\cref{eq:fokker_planck_forward_density}.

What we are going to do is calculate the differential of $J$ wrt.\ $\a(\cdot)$
and then use this in a gradient ascent procedure. First what we would like to do
is transfer all the differentials from $f$ to $p$. That is we will integrate $p
\cdot (\di_t f - \L[f])$ by parts so that only $f$ appears in the expression
without any of its derivatives. This is a standard exercise, we show it in
detail:
\begin{align*}
&-\int_0^T \int_{\O_X} p \cdot (\di_t f - \L[f]) \intd{x} \intd{s}=
\\
=&-\int_0^T \int_{\O_X} p \cdot 
(\di_t f_0 - D \cdot \di_x^2 f + \di_x [U \cdot f]
\intd{x} \intd{t} \quad \textrm{// what is }\L
\\
=&
 \int_0^T\int_{\O_X} \di_t p  f \intd{x} \intd{t} +
  \int_{\O_X} p f\intd{x}  \Big|^{T}_{t=0} \quad \textrm{// the time-derivative pieces }
  \\
  &+ \int_0^T \int_{\O_X}
	    (D \di^2_x p + U \di_x p)\cdot f 
	  \intd{t}\intd{x}  \textrm{// the space-derivative pieces }
	  \\
	  &+ \int_0^T 
	   \Big( p U f - p D \di_xf + \di_x p D f \Big|_{x=\xmin}^{\xmax} 
	  \intd{t}
	   \textrm{// the BC terms in 1-d}
\end{align*}

Thus the terminal and boundary conditions of $p$ are chosen given those of $f$
and the objective $J$. Since there are no boundary or terminal terms
contributing to our $J$, only the boundary conditions for $f$ impact the
choice of boundary conditions for $p$. In particular the following has to be
true:
\begin{align*}
pf&\Big|_{t=T}  = 0 & \textrm{Null TCs} \\
p U f - p D \di_xf + \di_x p D f &\Big|_{x=\xmin, \xmax} = 0 & \textrm{Null BCs}
\end{align*}

This usually implies that $p(T) \equiv 0$, since there are usually no a priori
restrictions on $f$ at the terminal time. For the boundary terms, 
if the forward density has reflecting boundaries such that:
$$
U f - D\di_x f\Big|_{x=\xmin, \xmax}  = 0
$$
then the BC terms for the adjoint are just the simple Neumann BCs:
$$
\di_x p \Big|_{x=\xmin, \xmax} = 0
$$

Once this integration-by-parts is done and the appropriate BCs applied, we
can return to the augmented objective, \cref{eq:objective_augmented} which now
looks like:
\begin{equation}
J =  \Exp_\th
\left[ \int_0^T\int_{\O_X} \log\left(\frac{f_\th}
 					{\int_\th f_\th \cdot \rho(\th)\intd{\th}}\right) 
 			 \cdot f_\th 
 			 + 
 			 (\di_t p_\th + \Lstar_{\th;\a}[p_\th] ) \cdot f_\th
\intd{x}
\intd{t} \right]
\label{eq:objective_augmented_adjoined}
\end{equation}
with $\Lstar$ the adjoint operator to $\L$.

The next step is to take the differential of $J$ wrt.\ the control
$\a(x,t)$

% Conceptually this is best imagined for a one-dimensional $X$. Basically,
% $\a(x,t)$ is like a sheet in $x-t$ space and the differential of $J$ is telling
% us how to move this sheet at any given point in order to improve the overall
% $J$. 

In order to make things simpler, we will write the integral wrt.\
$\th$ as a sum, i.e.:
$$
\int_\Theta f(\th) \rho(\th) \intd{\th} = \Exp_\th [f(\th) ] = 
\sum_\th w_\th (f(\th))
$$ where the weights $w_\th$ approximate the density $\rho(\th)$.
If one assumes a discrete prior this is just a different way of writing the
integral, if the prior is assumed continuous, then this is an approximation.
Then $J$ reads like:

$$
J =  \Exp_\th
\left[ \int_0^T\int_{\O_X} \log\left(f_\th\right)\cdot \ft - 
\log(\sum_\th \wt \ft) \cdot f_\th 
 			 + 
 			 (\di_t p_\th + \Lstar_{\th;\a} [p_\th] ) \cdot f_\th
\intd{x}
\intd{t} \right]
$$
and its differential wrt.\ $\a$ for given $x,t$ is:
\begin{align*}
\delta J|_{x,t} =& \sum_\th \Big[
\frac{\dft}{\ft} \cdot \ft - \frac{  w_\th \dft }{\sum_\th w_\th\ft}\cdot \ft 
+ \log\left(\frac{f_\th} {\sum_\th \wt \ft }\right) \cdot \dft  
\\&
+  (\di_t p_\th + \Lstar_{\th;\a} p_\th ) \cdot \dft
+ \delta \a \cdot (\di_x \pt \cdot \ft)
\Big]
\\
=& \sum_\th \Bigg[\Big(
1 - \frac{  w_\th \ft }{\sum_\th w_\th\ft} 
+ \log\left(\frac{f_\th} {\sum_\th \wt \ft }\right)   
+  (\di_t p_\th + \Lstar_{\th;\a} p_\th )
\Big)  \cdot \dft 
\\
&+ \delta \a \cdot (\di_x \pt \cdot \ft)
\Big]
\end{align*}
Now we need to knock out the $\dft$ terms so that we are left with only $\delta
\a$ terms. Thus we set the coeffiecent of $\dft$ to zero, which completes the 
evolution equation for a given $p_\th$
\begin{equation}
-\di_t \pt = 
\Lstar_{\th;\a} [p_\th] +  
1 - \frac{  w_\th \ft }{\sum_\th w_\th\ft} 
+ \log\left(\frac{f_\th} {\sum_\th \wt \ft }\right)   
\end{equation}
And once $\pt, \ft$ are solved for, the differential wrt.\ $\a$ comes
out to:
\begin{equation}
\frac {\delta J}{\delta \a} \Big|_{x,t} = \sum_\th (\di_x \pt \cdot \ft)
\label{eq:differential_objective_wrt_control_final}
\end{equation}
\Cref{eq:differential_objective_wrt_control_final} forms the key ingredient
in our gradient search for the most informative perturbation $\a^*(x,t) =
\argmax J(\a)$.


\subsubsection{Stationary Maximum Principle with a Prior}
\label{sec:StatMP_with_a_prior}
Let us now consider a variation on the Maximum Principle approach, when we just
maximize the Mutual Information between the stationary distribution and the
prior.

This amounts to changing the objective in
\cref{eq:mutual_info_time_integrated} to  
$$
J(\a)  = \int_\th \int_X
 \log\left(\frac{f(x|\th ; \a(\cdot) )}
 			{\int_\th f(x|\th; \a(\cdot)) \rho(\th)\intd{\th}}\right) 
 f(x|\th; \a(\cdot)) \cdot \rho(\th) \intd{x}\intd{\th}
$$

In effect this is a simplified version of \cref{sec:MP_with_a_prior}, where we
ignore the time evolution of $f$ and focus on its long-term equilibrium.

The calculations are very similar with the exception of $\di_t f = 0$. We start
by augmenting the objective with the dynamics:
\begin{equation}
J =  \Exp_\th
\left[ \int_{\O_X} r(f) \cdot f + p \cdot \L_{\th;\a}[f]
\intd{x}\right] 
\label{eq:objective_stat_augmented}
\end{equation} 
where $p =  p(x|\th; \a(\cdot))$ is the statinoary adjoint co-state and 
$\L_{\th;\a}[f]$ is the right hand of the Fokker-Planck equation,
\cref{eq:fokker_planck_forward_density}.

What we are going to do is calculate the differential of $J$ wrt.\ $\a(\cdot)$
and then use this in a gradient ascent procedure. First what we would like to do
is transfer all the differentials from $f$ to $p$. That is we will integrate $p
\cdot (\L[f])$ by parts so that only $f$ appears in the expression
without any of its derivatives. This is done exactly as before:
\begin{align*}
& \int_{\O_X} p \cdot (\L[f]) \intd{x}=
\\
=& \int_{\O_X}
	    (\Lstar[p])\cdot f 
	 \intd{x}  \textrm{// the space-derivative pieces }
	  \\
	  &+ \int_0^T 
	   \Big( p U f - p D \di_xf + \di_x p D f \Big|_{x=\xmin}^{\xmax} 
	   \textrm{// the BC terms in 1-d}
\end{align*}
Thus we keep the BCs as in the time-dependent section
\cref{sec:MP_with_a_prior}, and ditch the TCs:
\begin{align*}
p U f - p D \di_xf + \di_x p D f &\Big|_{x=\xmin, \xmax} = 0 & \textrm{Null BCs}
\end{align*}
Once this integration-by-parts is done and the appropriate BCs applied, we
can return to the augmented objective, \cref{eq:objective_augmented} which now
looks like:
\begin{equation}
J =  \Exp_\th
\left[ \int_{\O_X} \log\left(\frac{f_\th}
 					{\int_\th f_\th \cdot \rho(\th)\intd{\th}}\right) 
 			 \cdot f_\th 
 			 + 
 			 (\Lstar_{\th;\a}[p_\th] ) \cdot f_\th
\intd{x}
\right]
\label{eq:objective_stat_augmented_adjoined}
\end{equation}

Taking the differential of $J$ wrt.\ the control $\a(x,t)$ gives
\begin{align*}
\delta J|_{x} =&
\sum_\th \Bigg[\Big(
1 - \frac{  w_\th \ft }{\sum_\th w_\th\ft} 
+ \log\left(\frac{f_\th} {\sum_\th \wt \ft }\right)   
+  \Lstar_{\th;\a} p_\th
\Big)  \cdot \dft 
\\
&+ \delta \a \cdot (\di_x \pt \cdot \ft)
\Big]
\end{align*}
This gives us the differential equation for the adjoint $p$
\begin{equation}
0 = 
\Lstar_{\th;\a} [p_\th] +  
1 - \frac{  w_\th \ft }{\sum_\th w_\th\ft} 
+ \log\left(\frac{f_\th} {\sum_\th \wt \ft }\right)   
\end{equation}
And once $\pt, \ft$ are solved for, the differential wrt.\ $\a$ comes
out to:
\begin{equation}
\frac {\delta J}{\delta \a} \Big|_{x} = \sum_\th (\di_x \pt \cdot \ft)
\label{eq:differential_objective_stationary_wrt_control_final}
\end{equation}
\Cref{eq:differential_objective_stationary_wrt_control_final} forms the key ingredient
in our gradient search for the most informative {\sl stationary} perturbation
$\a^*(x) = \argmax J(\a)$. It is exactly the same as
\cref{eq:differential_objective_wrt_control_final} and it is imagine that
\Cref{eq:differential_objective_stationary_wrt_control_final} can be derived
from \cref{eq:differential_objective_wrt_control_final} without going to all
the trouble above\ldots
 

\section{Illustrative Example - Double Well Potential}
\subsection{Maximum Principle Approach - Time-Dependent Case}
\label{sec:MP_Doublewell_TimeDependent}
We now follow up the theoretical developments from \cref{sec:MP_with_a_prior}
with a concrete example.

Our first test problem will be the problem on estimating the double-well
potential barrier height as in Sec. 4 of the latest draft of Hooper et al.
\cite{Lin} on arXiv. (From June 7th, 2013). We shall use exactly the same parameter values
etc. as in sec. 4 in \cite{Lin}.

We would now like to compute the forward density and the adjoint functions
$\{\ft, \pt\}$ for the double-well problem.

Let's explicitly state the evolution equations for $\ft,\pt$
\begin{equation}
\begin{gathered}
\di_t \ft(x,t; \th, \a(\cdot)) = -\di_x [ U(x;A, \a) \cdot \ft(x,t)] + D \di_x^2
\ft(x,t)
\\
\begin{array}{ll}
	&
	\left\{ \begin{array}{lcll}
	 \ft(x,0) &=& \delta(x-x_0)  &\textrm{delta function at some } x_0
	\\
	U \ft - D \di_x \ft \big|_{x=\xmin,\xmax} &\equiv& 0 & \textrm{reflecting BCs
	at some } \xmin, \xmax \end{array} \right.
\end{array}
\label{eq:forward_density_double_well}
\end{gathered}
\end{equation}

\begin{equation}
\begin{gathered}
-\di_t \pt(x,t) =
D \di_x^2 \pt(x,t) +
U(x;A, \a(x,t))\cdot \di_x \pt(x,t) \\
+ 1 - \frac{  w_\th \ft }{\sum_\th \pt_\th\ft} 
+ \log\left(\frac{f_\th} {\sum_\th \wt \ft }\right)
\\
\begin{array}{ll}
	&
	\left\{ \begin{array}{lcll}
	\di_x \pt(x, t)|_{x = \xmin, \xmax}  &=& 0  \quad &\textrm{BCs}
	\\
	\pt(x,\T)  &=& 0 \,& \textrm{TCs}
\end{array} \right.
\end{array}
\label{eq:backward_adjoint_double_well}
\end{gathered}
\end{equation}
where,  
\begin{eqnarray*}
U(x; A, \a) &= -\left( 4x^3 - 4x -A \frac xc e^{-(x/c)^2/2} \right) + \a(x,t)
\\
&= -\grad_x \left( x^4  - 2x^2 + A e^{-(x/c)^2/2} \right)  + \a(x,t)
\\
&= -\grad_x \left(\mathcal V(x) + \mathcal{A}(x) \right)
\end{eqnarray*}

Having computed $\ft, \pt$, we compute the gradient $\delta J / \delta \a$
as
$$
\frac {\delta J}{\delta \a} |_{x,t} = \sum_\th \wt (\di_x \pt \cdot \ft)
$$
This just restates \cref{eq:differential_objective_wrt_control_final}.

Following \cite{Lin} (with some deviation in the exact values) we set the
parameters as $\s = 1. \implies D = 0.5$, (actually it is a little smaller in
\cite{Lin}, but this eases the numerics) and $c = 0.3$.

We will use $A = 4$ as the value of $A$ under which the actual process evolves,
but that is not important in the computation of the optimal control, $\a^*$. For
the prior on $A$ we use a uniform over $[2,5]$, which we represent with
only $N_\th = 2$ uniform points - $[2.0, 5.0]$.  In general it is
not clear how to start the forward density (what its ICs should be). 
My first attempt - to use a delta mass at $x_0 = 0$, i.e.\ to assume the
process starts at the crest of the barrier, resulted in poor convergence
properties for the gradient ascent procedure (by delta mass, we mean a very
narrow Gaussian, of course). Instead, using a broad Gaussian distribution
centred at the crest gave better results. 

The control is constrained to lie in the set $[-10, 10]$, i.e.\ $\amax = 10$.
The space is constrained to $x \in [-2.,2.]$, i.e.\ $\xmin, \xmax = -2.,2.$,
using  It is further discretized using $\Delta x = .1$, i.e.\ with 101 uniform
points, $[-5, -4.9\ldots 5.]$.

Let's first see what happens when we run the solver until $T = 5$ (Similar to
the $T$ value in \cite{Lin}.)

The experiment proceeds as follows: For our initial guess we take $$\a_0(x,t)
\equiv 0$$ Then we would like to see that 
$$ \sgn \left(\frac {\delta J}{\delta \a}\right) \Big|_{x,t} = - \sgn(x)$$
That is that for negative $x$ we want to drive to the right, $\a > 0$ and for
positive $x$ we want to drive to the right $\a < 0$. 
Let's see.  

The results are visualized in \cref{fig:FBSoln_doublewell_alpha_null}. Let's
discuss \cref{fig:FBSoln_doublewell_alpha_null}. The most important plots are on
the right, $\delta J$ which indicate how we are supposed to be changing $\a$. 
It is clear that except for the very beginning $t \approx 0$, and the end $t
\approx T$, $\delta J$ is essentially constant in time.  

Let's focus then on what happens in the bulk of time in the middle. Indeed we  
have that 
$$
\sgn \left(\frac {\delta J}{\delta \a}\right) \Big|_{x,t} = - \sgn(x)
$$
which implies that we should push the particle to the right (resp. left)
depending on whether we are to the left (resp. right) of the barrier at $x=0$.
Everything looks good, except for two points.
\begin{enumerate}
  \item The magnitude of $\delta J$ is very small. If we were to take steps of
  size $s=1.$, it would take us thousands of iteratons to get to what we expect to be
the right solution, i.e. bang-bang at $\amax = 10$.
\item The behaviour of $\delta J$ is 'wrong' or 'surprising near the end, $t
\approx T$, which appear around $t>4.75$ and there is also some discrepancies
near the beginning, the negative wiggles for $t \approx 0$, which vanish by the
time $t > 0.5$
\end{enumerate}

Both issues are minor, but we shall keep them in mind when we go through a full
iteration of the gradient descent. 
%\usepackage{graphics} is needed for \includegraphics
\begin{figure}[htp] 
\begin{center}  
  \includegraphics[width=.9\textwidth]{Figs/DoublewellFBSolver/FB_alpha_null_solution_2.pdf}
  \caption[labelInTOC]{Solution to the test Double Well potential problem using
  $\a \equiv 0$. }
  \label{fig:FBSoln_doublewell_alpha_null}
\end{center}
\end{figure} 

\subsubsection{Going through a full gradient descent iteration}
See
\cref{fig:FBSoln_doublewell_alpha_iterations,fig:FBSoln_doublewell_J_iterations},
basically, we are (almost) able to converge to the bang-bang control and from
\cref{fig:FBSoln_doublewell_J_iterations} we are led to believe that the
bang-bang control is indeed optimal. 

\begin{figure}[htp]
\begin{center} 
  \includegraphics[width=.9\textwidth]{Figs/DoublewellFBSolver/FB_alpha_iterates_uICs_4.pdf}
  \caption[labelInTOC]{Solution to the test Double Well potential problem
  starting from $\a \equiv 0$ until convergence. Note that except at the very
  beginning, $t \approx 0$ and very end of the interval $t\approx T$, we have
  converged to the bang-bang solution. }
  \label{fig:FBSoln_doublewell_alpha_iterations}
\end{center}
\end{figure}

\begin{figure}[htp]
\begin{center}
  \includegraphics[width=.9\textwidth]{Figs/DoublewellFBSolver/FB_J_iterates_uICs.pdf}
  \caption[labelInTOC]{The evolution of $J$ during the Gradient Ascent, we see
  it goes up and eventually asymptotes with the value of $J$ corresponding to
  the bang-bang control. }
  \label{fig:FBSoln_doublewell_J_iterations}
\end{center}
\end{figure}
 
\clearpage

\section{Second Illustrative Example - the time-constant for the OU model}
\label{sec:MP_OU_TimeDependent}

In \cref{sec:MP_Doublewell_TimeDependent}, we got decent results for the
Double-Well potential problem. Let's now apply the same techniques, (from
\cref{sec:MP_with_a_prior}) on the OU process.

We have seen this a few times, now (\cref{eq:OU_evolution})
$$
dX = \underbrace{(\a + \b( \mu - X)}_{U(x,\a; \th)} \intd{t} +
\underbrace{\s}_{\sqrt{2D}} dW
$$

Since, we've seen the details in \cref{sec:MP_Doublewell_TimeDependent}, we'll
just skip to the chase:

(Known) Parameter values are given by: 

For the fixed parameters we assume $\s = 1. \implies D = 0.5$, $\m = 0$.

For the prior on $\tc = 1/\b$ we use a uniform over $[.5,2]$, which we represent
with only $N_\th = 2$ uniform points located at $[.5,2]$.  

For ICs for the forward density, we take a very broad Gaussian centred at $0$
(the equilibrium).

The control is constrained to lie in the set $[-1 , 1 ]$, i.e.\ $\amax
= 1$. The space is constrained to $x \in [-2.,2.]$, i.e.\ $\xmin, \xmax =
-2.,2.$. 

The $\a_k$ iterates and the gradient-ascent progress of $J$ are shown
in \cref{fig:FBSoln_OU_alpha_iterations,fig:FBSoln_OU_J_iterations}. We see that
again, we converge close to the bang-bang control (which in this case does the
exact opposite than it did in the Double-Well test case). We also see that
the bang-bang control is likely MI-optimal (the red curve in
\cref{fig:FBSoln_OU_J_iterations}). 
 
 
\begin{figure}[htp]
\begin{center} 
  \includegraphics[width=.9\textwidth]{Figs/OUFBSolver/FB_J_iterates_uICs.pdf}
  \caption[labelInTOC]{The evolution of $J$ during the Gradient Ascent
  Ornstein-Uhlenbeck, we see it goes up and eventually asymptotes to the value
  of $J$ corresponding to the bang-bang control.}
  \label{fig:FBSoln_OU_J_iterations}
\end{center}
\end{figure}

\begin{figure}[htp]
\begin{center} 
  \includegraphics[width=.9\textwidth]{Figs/OUFBSolver/FB_alpha_iterates_uICs_5.pdf}
  \caption[labelInTOC]{Solution to the test Ornstein-Uhlenbeck
  problem starting from $\a \equiv 0$ until convergence. Note that except at the very
  beginning, $t \approx 0$ and very end of the interval $t\approx T$, we have
  converged to the bang-bang solution.}
  \label{fig:FBSoln_OU_alpha_iterations}
\end{center}
\end{figure}


\vskip15pt
\begin{center}

\end{center}

\clearpage

\subsection{Is bang-bang (in space, i.e. feedback) optimal?}
There is some speculation (Susanne!) whether a bang-bang in space control is
optimal. Perhaps a bang-bang control in time is better? In particular a bang-bang control in
space is a feedback control which pulls away from the equilibrium ($\mu=0$,
known) always in the direction depending on the current position of $X_t$. On
the other hand, it might be better to alternatively pull to the left and then to
the right. ('Better' in the sense that one gets better estimates). Let's then
run a simulation comparison. 

On one hand we will take the feedback, MI-optimal bang-bang solution which we
obtained in the beginning of this section (\cref{sec:MP_OU_TimeDependent}). On
the other hand we will divide the time interval in $N$ equal length segments and
alternate Up/Down on each adjacent segment, for illustration see  
\cref{fig:bang_bang_time_vs_space}. When we apply the controls (using identical
streams of random numbers) we get the trajectories shown in
\cref{fig:trajectories_bang_bang_time_vs_space}. We also show the reference
trajectory, using no control at all (labeled 'placebo'). There is obviously
quite a big similarity between all trajectories, since we are using a fairly 
small value of $\amax= 1$.

%\usepackage{graphics} is needed for \includegraphics
\begin{figure}[htp]
\begin{center}
  \includegraphics[width=1\textwidth]{Figs/OU_MIControlSimulator/Fb_vs_det_control_illustrate.pdf}
  \caption[labelInTOC]{Illustration of the two competing bang-bang controls,
  space and time}
  \label{fig:bang_bang_time_vs_space}
\end{center}
\end{figure}
%\usepackage{graphics} is needed for \includegraphics
\begin{figure}[htp]
\begin{center}
  \includegraphics[width=1\textwidth]{Figs/OU_MIControlSimulator/det_vs_fb_amax1.pdf}
  \caption[labelInTOC]{Illustration of the trajectories resulting from the two 
  competing  controls, bang-bang in space and time}
  \label{fig:trajectories_bang_bang_time_vs_space}
\end{center}
\end{figure}



\subsubsection{Estimating $\b$}
We now need a MaxLikelihood formula for $\b$ given the known $\m =0 $ and $\s =
1.$. 
Consider discrete observations ${X_n, t_n}$ obtained at uniform
$t_n$.
Then the transition probabilities $p_n(X_n|X_{n-1})$ are given by:
\begin{align*}
p_n(X_n|X_{n-1}; \b; \a_n, \Delta_n) \propto &
\sqrt{ \frac{\b}{ 1 -  e^{-2\b \Delta_n}}}
\\ &\cdot 
\exp\left(-\frac{\left( X_n - (\frac {\a_n}\b )  - (X_{n-1} -
	 \frac {\a_n}\b) \cdot e^{-\b \Delta_n} \right)^2 \cdot \b}
			{ (1-e^{-2\b\Delta_n})} \right)
\end{align*} 


which makes the log-likelihood look like:

\begin{align*}
l( \b|\ldots) = & \sum \log p_n(X_n|X_{n-1})
\\
=& \frac{N}{2} (\log \b - \log{ (1-e^{-2\b\Delta_n})}
\\ & -\sum_n
{\left(\frac{\left( X_n - (\frac {\a_n}\b )  - (X_{n-1} -
	 \frac {\a_n}\b) \cdot e^{-\b \Delta_n} \right)^2 \cdot \b}
			{ (1-e^{-2\b\Delta_n})} \right)}	 + \const		
\end{align*}

ML estimators for $\b$ are obtained via setting $\di_{\b} l$ to zero.

i.e. we must solve for 
\begin{align}
\di_\b l( \b|\ldots)  
=& \frac{N}{2} \left(\frac 1 \b - 
\frac{2 \Delta_n e^{-2\b\Delta_n}}{(1-e^{-2\b\Delta_n})} \right) \nonumber \\ 
& -\sum_n \textrm{a horrendous mess}  \nonumber
\\
=& 0 
\label{eq:OUML_estimate_beta}
\end{align}


% $$
% \frac{N}{2} \, {\left(\frac{2 \, \Delta e^{\left(-2 \, \Delta
% \beta\right)}}{e^{\left(-2 \, \Delta \beta\right)} - 1} +
% \frac{1}{\beta}\right)}
% $$
where 'a horrendous mess' is actually (using SAGE - python's CAS):
\begin{multline}
-\frac{2 \, {\left(X_{n-1} e^{\left(-\Delta \beta\right)} -
\frac{{\left(e^{\left(-\Delta \beta\right)} - 1\right)}
\a_{n}}{\beta} - X_{n}\right)}^{2} \Delta \beta e^{\left(-2 \,
\Delta \beta\right)}}{{\left(e^{\left(-2 \, \Delta \beta\right)} -
1\right)}^{2}} 
+
\\
 \frac{2 \, {\left(X_{n-1} e^{\left(-\Delta
\beta\right)} - \frac{{\left(e^{\left(-\Delta \beta\right)} - 1\right)}
\a_{n}}{\beta} - X_{n}\right)} {\left(\Delta X_{n-1}
e^{\left(-\Delta \beta\right)} - \frac{\Delta \a_{n} e^{\left(-\Delta
\beta\right)}}{\beta} - \frac{{\left(e^{\left(-\Delta \beta\right)} -
1\right)} \a_{n}}{\beta^{2}}\right)} \beta}{e^{\left(-2 \, \Delta
\beta\right)} - 1} -
\\ \frac{{\left(X_{n-1} e^{\left(-\Delta
\beta\right)} - \frac{{\left(e^{\left(-\Delta \beta\right)} - 1\right)}
\a_{n}}{\beta} - X_{n}\right)}^{2}}{e^{\left(-2 \, \Delta
\beta\right)} - 1}
\end{multline}

!!!
Once that is all done, we can compute some estimates for the trajectories
from \cref{fig:trajectories_bang_bang_time_vs_space} and tabulate those in
\cref{tab:OU_control_estimates_bang_bang_time_vs_space}. Although not
substantially, the feedback bang-bang control is clearly superior as it has a 
lower bias and a lower variance than the estimates obtained when using the
deterministic, bang-bang in time, control. I've tried with other values for the
switching frequency of the deterministic control (switch every .5, 1, 2
time units) and the results do NOT change. I've also tried with tweaking the
value of $\alpha_{max}$, again, same results. 

In \cref{tab:OU_control_estimates_bang_bang_time_vs_space}, we also show the
results for the base-case (called 'placebo') where $\a(x,t) \equiv 0$.

\begin{table}
\begin{tabular}{l|ccc}
$T_f$:
 & 8
 & 16
 & 32
\\
placebo :
 & (1.41, 0.66)
 & (1.19, 0.41)
 & (1.09, 0.27)
\\
det :
 & (1.37, 0.60)
 & (1.16, 0.37)
 & (1.08, 0.24)
\\
feedback :
 & (1.29, 0.39)
 & (1.14, 0.24)
 & (1.06, 0.15)
\\
\end{tabular}
\caption{(Mean/st.deviation) of the $\b$- ML Estimates obtained using the three
controls, given different value for $T_f$. Although not substantially, the
bang-bang feedback control is clearly superior as it has a lower bias and a
lower variance. We have used $N_{traj} = 1000$ to form the statistics for each
$T_f$ and control. The 'true' value is $\b = 1.$}
\label{tab:OU_control_estimates_bang_bang_time_vs_space}
\end{table}


Finally, in \cref{fig:ML_beta_root}, we show something that might be of
interest. It is the graph of the score function of the $\b$ ML estimates,
$\di_\b l$, \cref{eq:OUML_estimate_beta} as a function of $\b$ for several
different trajectories ${X_n}$. Recall that an estimate is obtained by finding
the solution to $\di_\b l(\b) = 0$. What we see is that the
root function corresponding to the feedback control, tends to be much steeper. Intuitively, I associate this
with a more robust estimation process, since small vertical perturbations will
have small effects on the root $\b$, while for the flatter curves (the ones
corresponding to the placebo and the deterministic control), small vertical
perturbations will much more drastically change the $\b$ estimate.
 
%\usepackage{graphics} is needed for \includegraphics
\begin{figure}[htp]
\begin{center}
  \includegraphics[width=1\textwidth]{Figs/OU_MIControlSimulator/BetaRoot_Tf=16.pdf}
  \caption[labelInTOC]{THe root function \cref{eq:OUML_estimate_beta} for a few
  trajectories}
  \label{fig:ML_beta_root}
\end{center}
\end{figure}



\subsubsection{Sweep through $\b_{true}, \sigma $}

For completeness we repeat the exercise above, i.e. we recalculate the results
from \cref{tab:OU_control_estimates_bang_bang_time_vs_space} for several
different values of the 'true' value of $\b$ and of $\sigma$. See
\cref{tab:betasigma_sweep_for_ML_estimation_of_OU}

\begin{table}
\begin{tabular}{l|cccccc}
\input{../OptEstimate/Figs/OU_MIControlSimulator/betasigmasweep_tabulate.txt}
% $(\beta_{true}, \sigma)$: & (5.00,0.25) & (5.00,4.00) & (1.00,0.25) &
% (1.00,4.00) & (0.20,0.25) & (0.20,4.00)\\ \hline placebo: & (68.67, 9.09, 0) &
% (0.34, 0.08, 0) & (17.90, 6.15, 0) & (0.10, 0.05, 0) & (6.00, 3.77, 0) & (0.06, 0.05, 2)\\det: & (19.62, 2.04, 0) & (0.41, 0.11, 0) & (5.40, 0.81, 0) & (0.11, 0.07, 3) & (2.51, 0.88, 0) & (0.06, 0.05, 4)\\feedback: & (14.83, 1.32, 0) & (0.89, 0.10, 0) & (1.59, 0.13, 0) & (0.34, 0.08, 0) & (0.34, 0.02, 0) & (0.18, 0.06, 0)\\
\end{tabular}
\caption{Sweep through the $\b_{true}, \sigma $ parameters and the effect on
the estimates. $Tf =16$. We've used $N=100$ to form the statistics. In
brackets are displayed the mean estimate (out $N$) and the estimates standard
deviation. We have set the initial value to $X_0=2.$, if we set it to $X_0 =
.0$ the feedback-based estimator will be much more dominant, especially for low
noise, presumably because for low noise  the other systems spend too much
time near the equilibrium if they already start there, and being near
equilibrium there is no restoring force and thus $\b$ is harder to estimate...}
\label{tab:betasigma_sweep_for_ML_estimation_of_OU}
\end{table}


\clearpage



\subsection{Multi-Parameter Case}
So far, we have worked in the simplest possible context - with only one unknown
parameter. Let's try to ramp it up a bit to TWO unknown parameters. In the OU
case this means, $\{\b, \m\}$. 

WAIT! First let's consider if we only had $\m$ uncertainty? Suppose we
knew $\b = 1$ and only cared about $\mu$, which could be one of ${-1,1}$. Then, after some
crunching it turns out (Results NOT shown) that the best thing to do is do
nothing, $\a \equiv 0$. I don't know why, but that is what the numerics
suggest\ldots. Again, by 'best thing' we mean the control, $\a(x,t)$,
which maximizes $J[\a]$ in \cref{eq:mutual_info_time_integrated}. 

So as a first guess, the optimal control for when both $\b,\m$ are uncertain
should be some combination of 'bang-bang' and 'null'. Bang-bang is best for $\b$
and 'null' (do nothing/ $\a=0$) is (we think) best for $\m$.
We let the prior be equally weighted between the following (cartesian
-product ) possibilities:
$$
(\b, \m) =\{.5, 2\} \times \{-1,1\} = (.5,-1), \ldots (2,1)
$$
i.e.\ in the prior, each of the 4 possibilities has probability
0.25. The calculations after that are exactly as in the earlier section when we only had
an uncertainty over $\b$. We show the distributions and optimal control in
\cref{fig:fpalpha_iterates_mubeta} and the objective value in
\cref{fig:OU_mubeta_Jiterates}. 

Strictly speaking, what we see is quite novel and interesting! The optimal
control is a strange combination of 'bang-bang' outside for  $x \notin [-1,1]$.
But null ($\a=0$.), for $x in [-1,1]$. This is slightly cheating, as we take
that that to be the case as initial guess (for $\a_k$) and if we started from
different initial guess for $\a_k$ we will not be able to converge to this,
although, that is likely more a fault of the gradient algorithm\ldots

However, we see that the difference is really very small. WHereas in the $\b$
uncertainty only,   \cref{fig:FBSoln_OU_J_iterations}, we could triple the value
of the objective, $J$ by going from the initial guess ($\a=0$) to the optimal
control, here the difference between bang-bang, do-nothing, and optimal control
is very, very slight!

Uf!


%\usepackage{graphics} is needed for \includegraphics
\begin{figure}[htp]
\begin{center}
  \includegraphics[width=1\textwidth]{Figs/OUFBSolver_BetaMu/FB_J_iterates_uICs_Tf=2.pdf}
  \caption[tableofCs]{The iterations for $J$ when both $\tc, \m$ are
  uncertain.}
  \label{fig:OU_mubeta_Jiterates}
\end{center}
\end{figure}

\begin{figure}[htp]
\begin{center}
  \includegraphics[width=1\textwidth]{Figs/OUFBSolver_BetaMu/FB_alpha_iterates_uICs_5_Tf=2.pdf}
  \caption[tableofCs]{The gradient ascent iterations for $f,p,\a$ when both
  $\tc, \m$ are uncertain. The initial guess for $\a_0$ is in 'blue' on the
  right-most }
  \label{fig:fpalpha_iterates_mubeta}
\end{center}
\end{figure}






\section{Problem Formulation}
The basic goal of 'Optimal Design' is to perturb a dynamical system in an
'optimal' way such as to 'best' estimate its structural parameters. 

As such the problem is a blend of optimal control and estimation, where the
objective of the optimal control is to improve the estimation, for example by
minimizing the variance of the estimators. 

For illustration sake we return to our favourite LIF model
Given a noisy LIF neuronal model:
\begin{equation}
\begin{gathered}
dX_s = (\underbrace{\a(t)}_{\textrm{control}} + \b(\m %\g \sin(\o t)
 - {X_s} ) \intd{s} + \s\intd{W_s},
\\
X(0) = .0,
\\
X(\ts) = \xth \implies  
\begin{cases}
X(\ts^+) &= .0   
\\
t_k &=  \ts
\\
k  &= k+1
\end{cases}
\end{gathered}
\label{eq:X_evolution_uo}
\end{equation}
where (a subset of) the parameter set $\th = \{\m, \b, \s\}$ is unknown.

Our goal is to choose $\a(t)$ as to estimate $\b$ 'best' given only that the
spike times $\{t_k\}$ are observed 
 
\section{Notation}
The probability density of the $n$th interval,
conditional on some applied control $\a$:
\begin{equation} 
\begin{array}{rcll} 
g_n(\t) \intd{\t} &:=& \Prob(I_{n} \in [\t, \t + \intd{\t})  \,|\,
 \a(t)) &
 \textrm{(probability density)} 
\\ 
G_{n}(t) &:=& \Prob \left[I_{n} \leq t  \,|\,
 \a(t) \right] = \int_0^t g_{\phi}(\t) \intd{\t} &
 \textrm{(cumulative distribution)}
\\
\G_{n}(t) &:= & \Prob(I_{n}>t \,|\, \a(t) ) = 1 - G_{\phi}(t)
&
 \textrm{(survivor distribution)}
\end{array}
\label{eq:ISI_distribution_functions}
\end{equation}
We'll drop the $n$ subscript when there is no confusion. 
There is also the transition distribution for $X_t$ for $t \in [0,
I_{n})$:
\begin{equation}
f(x,t) := \Prob \left[X_{t} \in x+ \intd{x}  \,|\,
 X_0 = 0, X_{s < t} < 1  \right]  \quad
 \textrm{(transition distribution)}
 \label{eq:transition_distribution}
\end{equation} 
\begin{equation}
\begin{gathered}
\begin{array}{lcl}
	\di_t f (x,t) &=&
					\underbrace{\frac{\b^2 }{2}}_{D}\cdot \di^2_x f 
					+ \di_x \Bigg(  
					\underbrace{\Big( \b (x-\m) - \a(t) \Big)}_{U(x,t)}  \cdot  f \Bigg)
					\\
					&=&
					D \cdot \di^2_x f +
					\di_x  \Big( U(x,t) \cdot f \Big)
					\\
					&=&
					- \di_x \phi(x,t)
					\\
					&=&
					\L[f] 
					\end{array}
	\\
	\left\{ \begin{array}{lcl}
	 f (x,0) &=& \delta(x)
	\\
	D \di_xf + U f |_{x=\xmin} &\equiv& 0 
	\\
	f |_{x=\xth} &\equiv& 0.
	\end{array} \right.
\label{eq:FP_pde_OU_absorbBC_CDF}
\end{gathered}
\end{equation}

The probability flux-out at the threshold boundary $$\phi(\xth, s) = D
\di_xf |_{x=\xth}$$ is very important as it is related to the spike-time density
via $$g( t)  = \phi(\xth, t) = D\cdot \di_x f|_{x=\xth}$$
 
% 
% In a typical (maximum likelihood) estimation experiment, we will see a lot of
% spikes and form the likelihood as
% $$
% L(\th| t_n ) = \prod_n g_n(t_n)
% $$
% We will then take logs and proceed as usual:
% $$
% l(\th| t_n ) = \sum_n \log (g_n(t_n)) =  \sum_n \log ( -\di_t F(1,t_n)) 
% $$
% and then maximize $l$ over the parameters $\th$. 
% 
% The associated {\sl score} function is
% $$
% S(\th | \ts ) = \grad_\th l(\th | \ts)
% $$
% The score function is a vector\footnote{We write $\grad$ for the vector
% differential and $\di$ for its scalar components, i.e.\ $\grad_\th =
% [\di_{\th_1},\ldots\di_{\th_i}],\ldots$}.
% 
% The typical Maximum Likelihood process is to 
% maximize the likelihood, $l$ which, if one uses a gradient-based approach
% amounts to finding the roots of the score, $S$.

In a Bayesian approach, we have some {\sl a priori} belief over the possible
values of $\th$.

Let us call the prior over the parameters $\th = \{\m, \b, \s\}$:
$$\rho(\th)$$.

Given a single observation $\ts$, the posterior of the parameter belief dist'n
is 
\begin{equation}
p(\th| \ts; \a) =
\frac{g(\ts |\th; \a)\cdot \rho(\th)}{\int_\Theta g(\ts|\th; \a)\cdot \rho(\th)
\intd{\th}}
\label{eq:parameter_posterior_defn}
\end{equation} 
Where $ g( \ts |\th; \a)$ is the likelihood of $X$ given in
\cref{eq:ISI_distribution_functions}.

The idea now, is to choose $\a$ such that the mutual information $I$ between the
two random variables is maximized. Here the Mutual Information is given by
\begin{equation}
I[\a]= 
\int_\Theta \int_{[0, \infty]} g(t|\th)\rho(\th) \cdot 
\log \left( \frac{g(t|\th)}
{\int_\Theta g(t|\th)\rho(\th) \intd{\th}   } \right)
\intd{t}\intd{\th}.
\label{eq:J_mutual_info_objective}
\end{equation}

See \cref{sec:mutual_info_defn} for why. 
Naturally for different controls, $\a(\cdot)$, the mutual info, $I$, will
be different since $g$, the hitting time density depends on the shape of $\a$.
($\rho$ does NOT).

However, there is an added complication b/c we actually will observe many
hitting times, and having less informative hitting times happen more often might
be better than hitting times which are informative but happen less often. 

The rigorous way to deal with this is to consider the mutual information between
the parameters and the set of hitting times $\{t_n\}$, however this seems
incredibly complicated, so instead we will maximize the 'Mutual Information
rate, $J$, which we define as 
\begin{align}
J[\a]= & \Exp[\ts]^{-1} \cdot I[\a]
\\
= & \frac{
\int_\Theta \int_{[0, \infty]} g(t|\th)  \rho(\th) \cdot 
\log \left( \frac{g(t|\th)}{\int_\Theta g(t|\th)\rho(\th) \intd{\th} } \right)
\intd{t}\intd{\th}}
{ \int_\Theta \int_{[0, \infty]} \t g(t|\th)\rho(\th) \intd{t}\intd{\th}}
\label{eq:J_mutual_info_rate_objective}
\end{align}

Well, this does NOT look any less complicated\ldots, Note that $g$ (and thus
implicitly $\a$) appears 4 times in this expression. Recall that $g$ is related
to $\a$ via the solution of the Fokker-Planck equation and thus we can also
write

\begin{align}
J[\a] 
= & \frac{
\int_\Theta \int_{[0, \infty]} \di_xf(1, t)  \rho(\th) \cdot 
\log \left( \frac{\di_xf(1, t)}{\int_\Theta \di_xf(1, t)\rho(\th) \intd{\th} } \right)
\intd{t}\intd{\th}}
{ \int_\Theta \int_{[0, \infty]} t \di_xf(1, t)\rho(\th) \intd{t}\intd{\th}}
\label{eq:J_mutual_info_rate_objective_in_terms_of_dixf} 
\end{align}
  
We want to find the control input $\a(t)$, which maximizes $J$ in
\cref{eq:J_mutual_info_rate_objective_in_terms_of_dixf}. 

\section{Gradient Ascent using Maximum Principle in order to maximize 
\cref{eq:J_mutual_info_rate_objective_in_terms_of_dixf} or rather the simpler 
\cref{eq:J_mutual_info_objective}} 

Let us discuss the optimization problem

$$
\a(\cdot) = \argmax_{\a \sim \textrm{admissible}} J[\a]
$$ 
   

\subsection{The nitty-grity of calculating the (infinite-dimensional) gradient
$\grad_{\a} J$ } We would like to maximize
\cref{eq:J_mutual_info_rate_objective_in_terms_of_dixf}, but now we realize that
doing so is very difficult, b/c we have a ratio of integrals. (The standard
theory always works with just one integral).

Let's then drop the denominator integral and focus on the numerator. (i.e. we
just go back to \cref{eq:J_mutual_info_objective})
\begin{equation}
I[\a] 
=  -
\int_\Theta \int_{[0, \infty]} \di_xf(1, t)  \rho(\th) \cdot 
\log \left( \frac{ \di_xf(1, t)}{\int_\Theta \di_xf(1, t)\rho(\th) \intd{\th}
} \right)
\intd{t}\intd{\th}
\label{eq:I_mutual_info_objective_in_terms_of_dixf} 
\end{equation}

To proceed, we apply a Maximum Principle type derivation, in which we first seek
the differential of the objective $I$ in
\cref{eq:I_mutual_info_objective_in_terms_of_dixf} wrt.\
$\a(\cdot)$ and proceed from there.   

As always, we start by augmenting our objective functional with the
dynamics:
\begin{align}
I=&  
\int_\Theta \int_{[0, \infty]} \di_xf_\th(1, t)  \rho(\th) \cdot 
\log \left( \frac{\di_xf_\th(1, t)}{\int_\Theta \di_xf_\th(1, t)\rho(\th)
\intd{\th} } \right)
\intd{t}\intd{\th} 
\\
	  &- \int_\Theta \int_0^\infty <p_\th, (\di_t f_\th - \L[f_\th])> \intd{s} 
\end{align}
where the inner product, $<f, g>$ is just the space integral $\int f\cdot g
\intd{x}$ and we write $\L$ for the spatial differential operator in 
\cref{eq:FP_pde_OU_absorbBC_CDF}.

This is exactly the same problem as we faced in the spike-time optimal control,
except there the integrand looked something like $(t-t^*)D\di_xf$. Thus the
equations for the adjoint look exactly the same as there, with the exception of the Terminal
Conditions (here assumed 0) and the BCs at the threshold:



In short, the equation for the adjoint function, $p$, is
\begin{equation}
\begin{gathered}
\begin{aligned}
\di_t p =& - \Lstar[p]
\\
		=&
			- \Big[ D\cdot \di^2_x p +
			 U(x,t,\th)   \cdot \di_x p \Big].
\end{aligned}
\\
\begin{cases}
	p_\th \big|_{x=\xth} &=  \log\left(\frac{\di_xf_\th}{\int_\Th
	\di_xf_\th \rho(\th)\intd{\th})}\right) +
	 1 -
	  \frac{\di_xf_\th}{\int_\Th
	\di_xf_\th \rho(\th)\intd{\th})}
	\\
	\di_x p_\th  \big|_{x=\xmin} &= 0
	\\
	p_\th(x,\infty) &= 0
\end{cases}
\label{eq:adjoint_pde_OU}
\end{gathered}
\end{equation}

In practice of-course, we will set the terminal conditions for $p_\th$ at some
finite value of $t$

\vskip10pt The whole goal of this exercise is to calculate the differential of
$I$ in \cref{eq:I_mutual_info_objective_in_terms_of_dixf},
wrt.\ the control $\a(t)$, i.e.\ to calculate $\delta I / \delta \a$. After the
introduction of the adjoint state, $p_\th$, that is just:

 
\begin{align*}
\delta I =&   
\int_\Theta  \rho(\th) \cdot \Bigg(  
- \int_\xmin^{1} \di_x p_\th f_\th \intd{x} + 
   p_\th f_\th \Big|_\xmin 
    \Bigg) \intd{\th}
\end{align*}

I.e for a given $\alpha(t)$, we solve for $p,f$ and a few values of $\th$ from
the current belief distribution $\rho(\th)$ and their corresponding
probabilities/weights. Compute the above expression and increment $\alpha$ in
the direction of increasing $\delta I$.

In practice, we usually take a very simple prior, something like three values
with equal probability, something like:.
\begin{equation}
\rho(\b) = 
\begin{cases}
	\tfrac 13 & \textrm{if } \b= \in \{.5,    1 ,  2 \}\\
	0   &\textrm{o/w }
\end{cases}
\label{eq:basic_prior_over_tau}
\end{equation} 

\subsection{Effect of the Prior}

Here we show that the optimal control is sensitive to the {\sl spread} of the
prior, for example if we have a tightly clustered vs. loosely spread prior, both
centred at roughly the same mean (the log-prior has the same mean). 

Very interestingly, we see that while for a wide prior, the optimal control has
its characteristic double hump shape, that we have seen already, for a tight
prior, that is no longer the case

Thus we see that the shape of the prior {\sl has!}
an effect on the optimal control.

\begin{figure}[h]
\begin{center}
\subfloat[Wide Prior]
{
\label{fig:prior_spread_wide}
\includegraphics[width=0.48\textwidth]
{Figs/FP_Adjoint/PriorBox_wide_prior.pdf}
}
\subfloat[Tight Prior]
{
\label{fig:prior_spread_tight}
\includegraphics[width=0.48\textwidth]
{Figs/FP_Adjoint/PriorBox_concentrated_prior.pdf}
}
\caption[labelInTOC]{The effect of the spread (variance) of the prior on the
resulting optimal control}
\label{fig:prior_spread}
\end{center}
\end{figure}

Let's look at it another way, we will consider our basic prior as a function of
$w$
\begin{equation}
\rho(\b) = 
\begin{cases}
	\tfrac 12 & \textrm{if } \b= \in \{1- w, 1/(1-w) \}\\
	0   &\textrm{o/w }
\end{cases} 
\end{equation} 
and sweep for $w = .1:.1:.9$ (in matlab notation).

The results are in \cref{fig:effect_of_prior_width}. Looking at
\cref{fig:effect_of_prior_width}, we might be optimistic to hypothesize that we
should be doing this online and as the uncertainty (roughly speaking $w$) of the
parameter decreases, we should be changing the applied control\ldots This
brings us to {\sl adaptive } versions of our scheme which is NOT something we
have yet implemented. 
 
\begin{figure}[h]
\begin{center} 
\includegraphics[width=\textwidth]
{Figs/FP_Adjoint/Effect_of_prior_spread.pdf} 
\caption[labelInTOC]{The effect of the width ($w$, a measure
of uncertainty) of the prior on the resulting optimal control}
\label{fig:effect_of_prior_width}
\end{center}
\end{figure}


\clearpage
\section{Basic Estimation Experiment}

We will run the following test:

Assume the true parameters
$$
 \m = 0; \b = 1; \s = 1.;
$$
We will assume we know $\s, \m$ and don't know $\b$ so we are trying to
maximize the Mutual Information between $\ts$ and $\b$.
 
Let's assume a very simple uniform prior on $\b$, $\b_i = \{.5, 1. 2\},$
each with probability 1/3. 

Then running the gradient ascent (details of the gradient ascent are omitted) we
get the controls, objective and hitting time densities shown in 
\cref{fig:hitting_time_density_g_aopt_bprior}. The optimal control seems to be
independent of the # of pts in the prior (i.e. instead of 3 we could use 5 pts
with weight 1/5 and get the same opt. control as in
\cref{fig:hitting_time_density_g_aopt_bprior}).



\begin{figure}[htp] 
\begin{center}
  \includegraphics[width=1\textwidth]{Figs/FP_Adjoint/ExampleOptControl_MI_HT.pdf}
  \caption[labelInTOC]{The gradient ascent for the optimization of $I$ in
  \cref{eq:I_mutual_info_objective_in_terms_of_dixf}. Top panel, the initial and
  the optimal optimal controls, $\a_{0}(t), \a_{opt}(t)$
   true' density $
  g_{\a}(s|\b = 1.)$, ) given various controls (top panel) and the resulting Mutual Information ($I$ ) i.e using \cref{eq:J_mutual_info_objective} NOT \cref{eq:J_mutual_info_rate_objective}.
  The bottom plot shows the hitting times $g(t| \tc)$ corresponding to the 3
  distinct values of $\tc$ in the prior $\rho(\tc)$}
  \label{fig:hitting_time_density_g_aopt_bprior}  
\end{center}
\end{figure} 

\subsubsection{Aside: the nitty-gritty of the estimation procedure}
We have posed a fairly-simple estimation objective, in that it amounts to single
variable optimization. The negative log-likelihood of an observed hitting-time
set $\{t_n\}$ is
\begin{equation}
l(\b) = - \sum_n \log ( g(t_n | \b) ) =  - \sum_n \log ( -D \di_x f(t_n |
\b) |_{\xth} )
\end{equation}

The distributions are exemplified in
\cref{fig:log_likelihood_beta_examples_1000,fig:log_likelihood_beta_examples_10000,fig:log_likelihood_beta_examples_100000},
for three different values of $N_s = 1e3, 1e4, 1e5$ respectively. We see that in
principle it is very hard to distinguish between different values of $\b$. In
\cref{fig:log_likelihood_beta_examples_10000}, we see the first indications that
the Opt Control, might have some superiority over the 'Crit' Control (for
example) as it seems to estimate a $\b$ closer to 1 (the 'true' value). However,
on average, the different shapes of $\a(t)$ seems to have a very limited impact
on the estimates for $\b$ (even though it has a very obvious impact on the shape
of the hitting time dist'n $g(t)$).


\begin{figure}[h]
\begin{center} 
\subfloat[opt]  
{
\includegraphics[width=.75\textwidth]
{Figs/HitTime_MI_TauChar_Adjoint_Estimate/Adjoint_TauChar_Estimator_estimatorWorkbench_b=0x1000_a0.pdf}
}
\\   
\subfloat[crit]
{
\includegraphics[width=.75\textwidth]
{Figs/HitTime_MI_TauChar_Adjoint_Estimate/Adjoint_TauChar_Estimator_estimatorWorkbench_b=0x1000_a1.pdf}
}
\\
\subfloat[max] 
{
\includegraphics[width=.75\textwidth]
{Figs/HitTime_MI_TauChar_Adjoint_Estimate/Adjoint_TauChar_Estimator_estimatorWorkbench_b=0x1000_a2.pdf}
}
\caption[labelInTOC]{Example of Empirical vs. Analytical Hitting time
distributions and the associated log-likelihoods. $N_s = 1e3$}
\label{fig:log_likelihood_beta_examples_1000}
\end{center}
\end{figure} 


\begin{figure}[h]
\begin{center}
\subfloat[opt]
{
\includegraphics[width=.75\textwidth]
{Figs/HitTime_MI_TauChar_Adjoint_Estimate/Adjoint_TauChar_Estimator_estimatorWorkbench_b=0x10000_a0.pdf}
}
\\
\subfloat[crit]
{
\includegraphics[width=.75\textwidth]
{Figs/HitTime_MI_TauChar_Adjoint_Estimate/Adjoint_TauChar_Estimator_estimatorWorkbench_b=0x10000_a1.pdf}
}
\\ 
\subfloat[max]
{
\includegraphics[width=.75\textwidth]
{Figs/HitTime_MI_TauChar_Adjoint_Estimate/Adjoint_TauChar_Estimator_estimatorWorkbench_b=0x10000_a2.pdf}
}
\caption[labelInTOC]{Same as \cref{fig:log_likelihood_beta_examples_1000}, but
with  $N_s = 1e4$. Notice that ( but only slightly ) the 'opt' control tilts in
the right direction for the $\b$ estimate (i.e. towards 1). }
\label{fig:log_likelihood_beta_examples_10000}  
\end{center}
\end{figure}

\begin{figure}[h] 
\begin{center}
\subfloat[opt]
{
\includegraphics[width=.75\textwidth]
{Figs/HitTime_MI_TauChar_Adjoint_Estimate/Adjoint_TauChar_Estimator_estimatorWorkbench_b=0x100000_a0.pdf}
}
\\
\subfloat[crit]
{
\includegraphics[width=.75\textwidth]
{Figs/HitTime_MI_TauChar_Adjoint_Estimate/Adjoint_TauChar_Estimator_estimatorWorkbench_b=0x100000_a1.pdf}
}
\\
\subfloat[max]
{
\includegraphics[width=.75\textwidth]
{Figs/HitTime_MI_TauChar_Adjoint_Estimate/Adjoint_TauChar_Estimator_estimatorWorkbench_b=0x100000_a2.pdf}
}
\caption[labelInTOC]{Same as  
\cref{fig:log_likelihood_beta_examples_1000,fig:log_likelihood_beta_examples_10000}
but with $N_s = 1e5$ hits}
\label{fig:log_likelihood_beta_examples_100000}
\end{center}
\end{figure}  
 -
\clearpage

\subsubsection{Batch Performance of the perturbations over the estimators.}
As is we have 3 candidates for perturbing the hitting times: 
\begin{enumerate}
  \item 
the optimal gradient-ascent-based  control $\a_{opt}$ (see
\cref{fig:hitting_time_density_g_aopt_bprior} top panel)
\item   the 'critical' constant control
$\a_{crit}$, ($\a_{crit}(t) = 1/\b$)
\item  the max constant control, $\amax$ ($=2$)
\end{enumerate} 

We now simulate $N_b = 100$ blocks of $N_s=1000$ hitting times each for the
3 alphas and then estimate $\b$ over each set using MaxLikelihood over our
computed expression for the density, $g(t|\tc; \a(t) )$). Examples for
differnt $N_s$ of the hitting time empirical
distribution are shown in \cref{fig:log_likelihood_beta_examples_1000} etc..
Naturally, for each control, we use the same gaussian random draws per block of $N_s$ Hitting of times).
%\usepackage{graphics} is needed for \includegraphics
% \begin{figure}[htp]
% \begin{center}
%   \includegraphics[width=\textwidth]{Figs/HitTime_MI_TauChar_Adjoint_Estimate/three_pt_prior_thits_distn.pdf}
%   \caption[labelInTOC]{Empirical Hitting-TIme distributions for the different
%   choices of $\a$}
%   \label{fig:empirical_hitting_times_3alphas}
% \end{center}
% \end{figure}

The estimation results are tabulated in in
\cref{tab:beta_estimates_from_hitting_times_different_alphas}.

\begin{table}
\subfloat[$N_b=1000, N_s = 1e2$]{
\begin{tabular}{ccc}
\input{../OptEstimate/Figs/HitTime_MI_TauChar_Adjoint_Estimate/beta_hit_time_100.txt}
\end{tabular}
}
\subfloat[$N_b=100, N_s = 1e3$]{
\begin{tabular}{ccc}
\input{../OptEstimate/Figs/HitTime_MI_TauChar_Adjoint_Estimate/beta_hit_time_1000.txt}
\end{tabular}
}\\
\subfloat[$N_b=10, N_s = 1e4$]{
\begin{tabular}{ccc}
\input{../OptEstimate/Figs/HitTime_MI_TauChar_Adjoint_Estimate/beta_hit_time_10000.txt}
\end{tabular}
} 
\subfloat[$N_b=1, N_s = 1e5$]{
\begin{tabular}{ccc}
\input{../OptEstimate/Figs/HitTime_MI_TauChar_Adjoint_Estimate/beta_hit_time_100000.txt}
\end{tabular}
}
\caption{Results for the estimates arising from simulations using various values
of $\a$ (opt, crit, max). In each sub-table there are $N_b$
parameter estimates for each distinct $\a$, with $N_s$ hitting times used to form an $\b-$estimate.
The 'true' value of $\b$ is $\b=1$. }
\label{tab:beta_estimates_from_hitting_times_different_alphas}
\end{table}     
% \begin{table} 
% \caption{$N_b=10$, $N_s = 1e4$  }
% \label{tab:beta_estimates_from_hitting_times_different_alphas_Nhits10000}
% \end{table}   
% \begin{table}
% \begin{tabular}{ccc}
% \input{../OptEstimate/Figs/HitTime_MI_TauChar_Adjoint_Estimate/beta_hit_time_100000.txt}
% \end{tabular}   
% \caption{$N_b=1$, $N_s = 1e5$ (estimator variance is irrelevant here as there is only 1 estimate per alpha)}
% \label{tab:beta_estimates_from_hitting_times_different_alphas_Nhits100000}
% \end{table}  

Comments: It looks like there is a marginal advantage to using the 
Optimal Control, $\a_{opt}$ over the simpler, constant controls. In particular
the bias of the estimates seems to be reduced. The variance of the estimates
seems to be independent of the perturbation\ldots 
 
\clearpage


\section{Online MI Optimization}
Hеre we outline a tentative approach to {\sl online} optimization of the MI,
which means

\begin{enumerate}
  \item Find $\aopt$ using the gradient ascent, for the prior $\rho$
  \item Apply $\aopt$ and measure several $1,2\ldots,N_{s,1}$ hitting times
  $t_k$
  \item Update the $\rho$ into a posterior conditional on the observed $\{t_k\}$
  \item Recalibrate $\aopt$ using the new $\rho$, i.e. go back to 1. 
\end{enumerate}

 Efficiency considerations aside, we have all the tools to do pts. 1,2,4, it
 is only the prior update that needs to be discussed. 

Of course we start by restating Bayes' formula

$$
\rho(\th| \{t_k\} ) = 
\frac{  \rho(\th) \cdot \prod_k g(t_k|\th ; \a) }
	 { \int_\Th  \rho(\th) \cdot \prod_k g(t_k|\th ; \a)  \intd{\th}}
$$

In practice, exact calculation of $\rho(\th|\t_k)$ would not be possible in our
context, so an approximation approach needs to be made.

I know that Susanne has done some work on a Bayesian approach to param.
estimation, so I shall first discuss with her. 

\clearpage
\cleardoublepage

\chapter{Conclusion and Outlook}
\label{ch:conclusion}
Ah, la conclusion:-) 
TODO:

In the first task, of estimating, one can see that there remains one interesting
problem, and that is why the estimation quality degrades with higher sinusoidal
frequency in particular for one of the two methods (the
Fokker-Planck / Likelihood based method).

In the second task, the possible avenues for progress are much broader\ldots 

The third task\ldots . 

Hopefully, this can be applicable to more complex areas of neuroscience and
physiology, where the systems under study are quite amenable to direct external
perturbation.

\cleardoublepage
% \bibliographystyle{plain}
%TODO: Use JNE biblio-style???
\bibliographystyle{bmc_article}
%TODO: Check for ?? (include local.bib from the various sub-folders) 
\bibliography{library,local}

\end{document}
