\chapter{Estimation in the Sinusoidaly-driven Leaky-Integrate and Fire Model}
% \chapter[Estimation in the Sinusoidaly-driven LIF Model]{Estimation in the
% Sinusoidaly-driven Leaky-Integrate and Fire Model}
% \chaptermark{Estimation in the Sinusoidaly-driven LIF Model}
\label{ch:estimate}
\graphicspath{{../LIFEPaper/}}
<<<<<<< HEAD
=======

\usepackage{amsfonts}
\usepackage{mathrsfs}

>>>>>>> 6d1ee3c9eb52b6bed66343a6488d0f9a4ca3aef0
% DEFINITIONS:
\def \Prob 	  {{ \mathbbmtt{P}  }} %the expectation operator
\def \th 	  {{ \theta}}
\def \FI 		{{\Phi}}
\def \KL     {{ K\!L}}

\def \adot {{ \dot{\alpha} }}
\def \Udot {{ \dot{U}}}
\def \In	{{ i_n}}
\def \vt 	{{ v_{\textrm{th} } }}
\def \Ihat  {{ \hat{I}  }}
\def \p 	{{ \phi  }}

\def \G		{{ \bar{G}  }} %{{ \reflectbox{G} }}
\def \Gest  {{ \hat{\G} }}
\def \Gtilde		{{ \tilde{\G} }}
\def \D		{{ \reflectbox{D} }}
\def \dphi {{ \delta \phi }}

\def \L {{ \Lambda }}
\def \P {{ \Phi }}
\def \n {{\nu}}
\def \Fx {{ F_x}}
\def \Fxx {{ F_{xx} }}
\def \Ft {{ F_t }}
\def \xest {{ \hat{x}_t}}
\def \muncond {{ {m}_x}}
\def \mcond {{ {m}_x^c}}

\def \f {{\rho}}
\def \F {{\Phi}}
\def \Fn {{\mathcal{F}}}

\def \abg		{{\a,\b,\g }} 
\def \aest      {{ \hat{\a} }}
\def \best      {{ \hat{\b} }}
\def \gest      {{ \hat{\g} }}
\def \estabg	{{\aest, \best, \gest}}
\def \abgest	{{\estabg }}

\def \sAlg {{ \mathcal{A} }}
\def \N {{ \mathcal{N} }}
\def \Udomain {{ \mathcal{U} }}
\def \Umax {{ u_{\textrm{max}} }}
\def \umax	{{ u_{\textrm{max}} }}
\def \amin	{{ \a_{\textrm{min}} }}
\def \amax	{{ \a_{\textrm{max}} }}
\def \astar {{ \a^* }}
\def \xth	{{ x_{th} }} 
\def \yth	{{ y_{th} }}
\def \xmin	{{ x_{-} }}
\def \xmax	{{ x_{+} }}
\def \xmid  {{ x_{mid} }}

% \def \x {{ \boldsymbol{x} }}
% \def \u {{ \boldsymbol{u} }}
% \def \p {{ \boldsymbol{p} }}
% \def \q {{ \boldsymbol{Q} }}
% \def \f {{ \boldsymbol{f} }}
\def \tf {{ t_f }}
\def \tc {{ \tau_{c} }}
\def \lc {{ \lambda_{c} }}
\def \ts {{ t_{\textrm{sp} } }}
\def \tn {{ t_n }}
\def \tns {{ \{t_n \} }}
\def \T {{ T^*  }}

\def \free {{\textrm{free} }}


\def \Ttwo {{ \hat{T}_{(2)} }}
\def \Ttwol {{ \hat{T}^{\lambda}_{(2)} }}
\def \Tone {{ \hat{T}_{(1)} }}
\def \Ti {{ \hat{T}_{(i)} }}

\def \Normal {{ \mathcal{N} }}
\def \L {{ \mathcal{L} }}
\def \Lstar {{ \L^* }}
\def \H {{ \mathcal{H} }}
\def \dx {{ \delta\! x}}
\def \da {{ \delta\! \a}}
\def \df {{ \delta\! f}}

% COMP EXAM:
\def \Lonepm {{\mathbb{L}^1}}
\def \Ltwopm {{\mathbb{L}^2}}
\def \Fil 	{{\mathcal{F}}}

\def \x {{ \vec{x} }}
\def \X {{ \vec{X} }}
<<<<<<< HEAD
% parameterized densities / adjoints:
\def \ft {{ f_\th}}
\def \pt {{ p_\th}}
\def \dft {{ \delta f_\th}}
\def \wt {{ w_\th }}

\def \aopt {{\a_{opt} }}
=======

>>>>>>> 6d1ee3c9eb52b6bed66343a6488d0f9a4ca3aef0

   
\section{Thesis Context}
 In the first of three main chapters of the thesis, we focus on the problem of
 estimating model parameters in a sinusoidally-perturbed LIF model from the
 observation of spike timings only. We discuss two approaches - one based on
 solving an integral equation and one based on solving for the forward density.
 In both cases we need to make an approximation in order to account for the
 non-renewal property of the hitting-time process. 
Both estimation algorithms are iterative, therefore we propose a non-trivial
 method for initializing them with observations-driven initial guesses.  
 
\section{Problem Introduction} 
Information processing in the nervous system is carried out by spike timings in
neurons. To study the neural code in such a complicated system, a first step is
to understand signal processing and transmission in single neurons. Stochastic
leaky integrate-and-fire (LIF) neuronal models are a good compromise between
biophysical realism and mathematical tractability, and are commonly applied as
theoretical tools to study properties of real neuronal systems. A central issue
is then to perform statistical inference from experimental data and estimate
model parameters. Many electrophysiological experiments on neurons, namely
extra-cellular recordings, are only capable of detecting the time of the spike
and not the detailed voltage trajectory leading up to the spike. Estimating the
parameters of the LIF model from this type of data is equivalent to estimating
the parameters of a stochastic model from the statistics of the first-passage
times only. A common assumption is that the data are well described by a renewal
process, thus basing the statistical inference on the interspike intervals
(ISIs), assuming these are realizations of independent and identically
distributed random variables. Since only partial information about the process
is available, the statistical problem becomes more difficult, and no explicit
expression for the likelihood is available. 

Different methods have been
proposed. In the seminal paper \cite{Brillinger1988}, a point process approach
is proposed. The spike trains of a collection of neurons are 
represented as counting processes. Time is discretized and the point processes
approximated by 0-1 time series. Then the probability of firing in the
next time interval is modeled as a function of the
spike history. In this way maximum likelihood
estimation is feasible. External stimuli are not considered. 
In \cite{Inoue1995} a numerically involved moment method is
developed. It uses the first two moments of the first-passage times of
the Ornstein-Uhlenbeck process to a constant threshold, which are
given as series expressions, and equates them to their empirical
counterparts. In
\cite{DitlevsenLansky2005,DitlevsenLansky2006} certain explicit 
moment relations derived from the Laplace transform of the first-passage time
distribution are applied, but these are only valid under 
stimulation (supra-threshold regime). In
\cite{MullowneyIyengar2008} inference is based on numerical inversion of the Laplace transform. In \cite{Zhangetal2009}, a
functional of a 3-dimensional Bessel bridge is applied to obtain a maximum
likelihood estimator. None of these methods
are feasible to extend to the
non-timehomogenous case, which is of our interest. In \cite{Ditlevsen2008,Ditlevsen2007} an
integral equation is used to derive an estimator in the
time-homogenous setting. This approach is readily extended to time
varying input, which we will explore in this paper. Some of the above methods are compared in
\cite{Ditlevsen2008a}. Finally, a review of estimation methods is provided in
\cite{Lansky2008}. 
%We provide more details on previous methodologies in
%\cref{sec:estimation_algos} below.

Many sensory stimuli, like sound, contain an oscillatory component
\cite{Braunetal1994,Chacron2000}. Such inputs will cause oscillating membrane
potentials in the neuron, generating rhythmic spiking patterns. The oscillation
frequency determines the basic rhythm of spiking, and is considered to be
significant for neuronal information processing. The dynamics of periodically
forced neuron models have been extensively studied, see
\cite{Bulsaraetal1996,Burkitt2006b,Lansky1997,Longtingetal1994,SacerdoteGiraudo2013,Shimokawa2000}
and references therein. Even so, attempts to solve the estimation problem in
these non-stationary settings have been rare. One problem is that the ISIs are
no longer independent nor identically distributed. In
\cite{Paninski2004} a more complicated model with linear filters is considered, allowing
also for the spike history to influence the membrane potential
dynamics. The estimation problem is solved through numerical solutions to the
Fokker-Planck equation, and it is shown that the log-likelihood is
concave, thus ensuring a global maximum, see also
\cite{Dong2011,Sirovich2011a}. Because their model is more involved,
some approximations to the solution of the Fokker-Planck equation is
applied, to ensure acceptable computing times. We will apply the full 
Fokker-Planck equation to solve our estimation problem, since the
computing time is always lower than 2 seconds for a sample size of
1000 spikes.   
  

In this paper, we thus describe and discuss two methods to estimate parameters
of LIF models with the added complexity of a time-varying input current. We
assume that the time-varying current is a sinusoidal wave, but we believe that
the approaches generalize to an arbitrary periodic forcing with known frequency.
One approach relies on the Fortet integral equation, which is readily extended
to the time non-homogeneous case. An advantage of this approach is that if the
transition density of the diffusion in the LIF model is known, as is the case
for the Ornstein-Uhlenbeck and the Feller model, the computational burden is
limited. A second approach involves numerical solution of the Fokker-Planck
equation, where the time-dependence is explicitly accounted for. After a
numerical differentiation, the likelihood function can be calculated providing
the maximum likelihood estimator. Nevertheless, we chose an alternative loss
function which seem marginally more robust, directly comparing the survival
function provided by the solution of the Fokker-Planck equation with its
empirical counterpart. The two approaches give similar results and they are more
carefully compared in the supplementary online material.

Both methods need sensible starting values for the optimization
algorithms, and we provide an easy-to-implement initializer. The estimation
procedures are compared on simulated data and we find that both algorithms are
able to find estimates close to the true values for several different dynamical
regimes. We find that for small sample sizes the Fokker-Planck algorithm can be
considered marginally preferable, whereas for larger sample sizes the Fortet
algorithm becomes marginally superior. Moreover, at high frequencies of the
sinusoidal forcing, the Fortet is better at identifying the parameters, though in general
there is less information in the data to distinguish between a constant input
and the amplitude of the periodic forcing.

\section{Model}
The time evolution of the voltage of a spiking neuron is modelled by a
stochastic process, $V$, given as solution to the following stochastic
differential equation (SDE)
\begin{equation}
\begin{gathered}
\intd{V}(t) = \left(\m - \frac{V(t)}{\t} +  A \sin(\o t ) \right) \intd{t} + \s
\intd{W}(t),
\\
t_0 = 0; \quad V(t_0) = v_0,
\\
t_{n } = \inf \{ t > t_{n-1} : V(t) = \vt \} \quad \text{for } n \geq 1
\\
\begin{cases}
V(t_n^+) = v_0 &  
\\
J_n = t_n-t_{n-1} .
% \\
% \phi_{n+1} = t_n \mod 2 \pi
\end{cases}
\end{gathered}
\label{eq:v_evolution_uo}
\end{equation}
Here, $\mu$ is a bias current acting on the cell, $\t$ is the decay time, $A$
and $\o$ are the amplitude and (angular) frequency of the sinusoidal current
acting on the cell, $\s$ is the strength of the stochastic fluctuations, $W =
\{W_t\}_{t\geq0}$ is a standard Wiener process, and $t_n^+$ denotes the right
limit taken at $t_n$. A spike occurs when the membrane voltage $V(t)$ crosses a
voltage threshold, $\vt$, and then $V(t)$ is instantaneously reset to the
resting potential $v_0$. The difference between subsequent spike times, $J_n =
t_{n} - t_{n-1}$, is called the interspike interval (ISI).

We will assume that $\t$ is known (but see \cref{sec:optimal_design} for a discussion
of the alternative) and non-dimensionalize \cref{eq:v_evolution_uo} as follows
\begin{align*}
s &= \frac{t}{\t}	& 
X_s &= \frac{V(t) - v_0}{\vt - v_0}&
W_s &= \frac{W(t)}{\sqrt \t}	&
\xth&= 1
\\
\a &= \frac{\m \t}{\vt - v_0}	&
\b &= \frac{\s\sqrt{\t}}{\vt-v_0}	&
\g &= \frac{A \t}{ \vt - v_0 }	&
\th &= \o \t \notag
% \label{defn:nond_params}
\end{align*}
to obtain
\begin{equation}
\begin{gathered}
dX_s = (\a - X_s + \g \sin( \th s ) ) \intd{s} + \b \intd{W_s},
\\
s_0 = 0; \quad X_{s_0} = 0,
\\
s_{n } = \inf \{ s > s_{n-1} : X_s = \xth =1 \} \quad \text{for } n \geq 1 
\\
\begin{cases}
X_{s_n^+} = 0 &  
\\
I_n = s_n-s_{n-1} ,
% \\
% \phi_{n+1} = t_n \mod 2 \pi
\end{cases}
\end{gathered}
\label{eq:X_evolution_uo}
\end{equation}
where we have defined $I_n = J_n / \t$. We can also write the dynamics between two spike times $s_n$ and $s_{n+1}$ in terms of elapsed time since
the last spike, $s' = s- s_n$, $s' < I_{n+1}$,
\begin{equation}
\begin{gathered}
dX_{s'} = (\a - X_{s'} + \g \sin( \th (s' + \phi_n ) ) \intd{s'} + \b
\intd{W_{s'}},
\\
\begin{cases}
s' &= s - s_n
\\
 \phi_n &= s_n \mod \tfrac{2 \pi}{\th}
\end{cases}
\end{gathered}
\label{eq:X_evolution_uo_renewal}
\end{equation}
This form of the dynamics highlights that this is not a renewal process since
different trajectories between spikes have different phase shifts $\phi_n = s_n$
modulo ${2 \pi}/{\th}$. This will be important in the following discussion. The
shape of the ISI distribution depends on the model parameters, and it is natural
to divide the parameter space in different regimes characterized by their
qualitative behaviour. Four distinct parameter regimes will be considered;
supra-threshold, critical, sub-threshold and super-sinusoidal. 
To understand the reasoning behind the regime names, observe that in the absence
of noise, $\b=0$, the deterministic model will produce spikes if and only if $$
\a + \frac{\g}{\sqrt{1 + \th^2} } > 1, $$ see the discussion in
\cite{Burkitt2006b}, which can be directly inferred from the solution in
\cref{eq:LIF_deterministic_solution} below. In both the supra-threshold and
super-sinusoidal regimes, $ \a + \g/\sqrt{1 + \th^2}  > 1$. The
difference between the two is that in the supra-threshold regime the constant
bias current alone is sufficient for spikes to occur, also in absence of noise,
that is,  $\a > 1$. In the super-sinusoidal regime the sinusoidal current is
necessary for spikes to occur in absence of noise, that is, $\alpha +
\gamma /
\sqrt{1+\Omega^2} > 1$ and $\alpha \leq 1$. In the critical regime, the sum of
the two terms is just barely enough to guarantee deterministic spiking, that is
$\a + \g / \sqrt{1 + \th^2}  \approx 1$. Finally, in the sub-threshold regime,
there would be no spikes without the noise, $ \a + \g / \sqrt{1 + \th^2} < 1$.

\Cref{tab:regimes} tabulates examples of corresponding parameter values for each
regime, while \cref{fig:trajectory_examples} shows examples of individual
voltage trajectories and their associated spike trains.
\Cref{fig:4regimes_illustrated_SDF,fig:4regimes_illustrated_PDF} illustrate
 how each regime behaves for selected $\phi$'s by plotting the survivor
 distribution, $\G_{\phi}(t)$, and the probability density, $g_{\phi}(t)$, both
 defined in \cref{eq:ISI_distribution_functions} below.
 
With regards to
\cref{fig:4regimes_illustrated_SDF,fig:4regimes_illustrated_PDF}, it is worth
noting explicitly, that combinations of noise and sinusoidal forcing can cause
firing patterns in which spikes are phase locked, but skip a certain number of
cycles. This leads to multi-modal ISI densities. There are many different
dynamical mechanisms that can yield such patterns, and the particular
correlations between the ISIs will depend on the underlying voltage dynamics
(which, in our case, we assume to be given by \cref{eq:v_evolution_uo}); in
particular, it may be difficult to distinguish whether the dynamics are
sub-threshold or supra-threshold, since both can show similar ISI densities,
see \cite{Longtin1995}.


 
% \begin{table}[ht] \begin{center} \begin{tabular}{l|ccc}
% Regime Name & $\a$ & $\b$ & $\g$ \\
% \hline Supra threshold &
% 1.5 & 0.3 & 1.0 \\ 
% Critical &
% 0.5 & 0.3 & 1.12 \\ 
% Sub threshold &
% 0.4 & 0.3 & 0.4 \\ 
% Super-sinusoidal & 0.1 & 0.3 & 2.5 \end{tabular} \caption{Example $\abg$
% parameters for the different regimes, given $\th = 2.0$} \label{tab:regimes}
% \end{center} \end{table}
\begin{table}[ht]
\begin{center}
\begin{tabular}{l|ccc}
Regime Name & $\a$ & $\b$ & $\g$ \\ \hline
Supra-threshold&1.40&0.30&0.14 \\
Super-sinusoidal&0.10&0.30&1.98 \\
Critical&0.50&0.30&0.71 \\
Sub-threshold&0.40&0.30&0.57 \\
\end{tabular}
\caption[Parameter values for model regimes]{Example of $\abg$ parameter values
for the different regimes, given $\th = 1$.}
\label{tab:regimes}
\end{center} 
\end{table}
% \begin{figure}[ht] \begin{center} \subfloat {
% \includegraphics[width=0.49\textwidth] {Figs/FP/Illustrate_refinedsuperT.png}
% } \subfloat { \includegraphics[width=0.49\textwidth]
% {Figs/FP/Illustrate_refinedcrit.png} }
% \\
% \subfloat{ \includegraphics[width=0.49\textwidth]
% {Figs/FP/Illustrate_refinedsubT.png} } \subfloat{
% \includegraphics[width=0.49\textwidth]
% {Figs/FP/Illustrate_refinedsuperSin.png} } \caption{the Four Regimes -
% illustrated are the empirical and analytical SDF, $\G(t)$, for the 4 most
% populated bins. Recall that the analytical $\G(t)$ is derived using the
% representative $\phi_m$ for each bin, while the empirical $\G$ is obtained
% directly from the spike data, each spike having an associated $\phi_n \neq
% \phi_m$.} \label{fig:4regimes_illustrated} \end{center} \end{figure} \SDF:
\begin{figure}[ht]
\begin{center}
% \subfloat[Supra-threshold Regime]
% {
% \includegraphics[width =0.48\textwidth,height=.25\textheight]
% {Figs/Trajectories/path_T=24_TrajExample_superT.png}
% }
% \subfloat[Critical Regime]
% {
% \includegraphics[width =0.48\textwidth,height=.25\textheight]
% {Figs/Trajectories/path_T=24_TrajExample_crit.png}
% }
% \\
% \subfloat[Sub-threshold Regime]
% {
% \includegraphics[width =0.48\textwidth,height=.25\textheight]
% {Figs/Trajectories/path_T=24_TrajExample_subT.png}
% }
% \subfloat[Super-sinusoidal Regime]
% {
% \includegraphics[width=0.48\textwidth,height=.25\textheight]
% {Figs/Trajectories/path_T=24_TrajExample_superSin.png}
% }
\includegraphics[width=0.95\textwidth]{Figs/Trajectories/path_T=24_combined.pdf}
% \includegraphics[width=0.95\textwidth]{Figs/Trajectories/path_T=24_combined.png}1
\end{center}
\caption[Example model trajectories]{Example trajectories from
\cref{eq:X_evolution_uo} for the four different parameter regimes using the parameter values given in
\cref{tab:regimes}. A) supra-threshold, B) super-sinusoidal, C) critical, D)
sub-threshold. In the supra-threshold regime
spikes occur regularly and often; in the super-sinusoidal regime
spikes cluster near the peak of the sine wave; in the critical regime
they occur less often; and in the sub-threshold regime, spikes occur
rarely. For all regimes, $\th = 1$.} 
\label{fig:trajectory_examples}    
\end{figure}
% #PDF: 
\begin{figure}[ht]
\begin{center}
% \subfloat
% {
% \includegraphics[width=0.48\textwidth]
% {Figs/FP/Illustrate4superT_pdf.png}
% }
% \subfloat
% {
% \includegraphics[width=0.48\textwidth]
% {Figs/FP/Illustrate4crit_pdf.png}
% }
% \\
% \subfloat{
% \includegraphics[width=0.48\textwidth]
% {Figs/FP/Illustrate4subT_pdf.png}
% }
% \subfloat{
% \includegraphics[width=0.48\textwidth]
% {Figs/FP/Illustrate4superSin_pdf.png}
% }
\includegraphics[width=0.99\textwidth]{Figs/FP/regimes_pdf_combined.pdf}
\caption[Model Regimes hitting-time densities]{The four different parameter
regimes using the parameter values given in \cref{tab:regimes}. Illustrated are the
probability density functions, $g_{\phi_m}(t)$, for representative $\phi_m =
2\pi/\th \times \{0,0.25,0.5,0.75\}$. 
Varying $\phi_m$ has,
 for the most part, the effect of shifting the curves laterally, 
 while varying $\abg$ changes their characteristic form. For all regimes, $\th
 = 1$.
 A) supra-threshold, B) super-sinusoidal, C) critical, D) sub-threshold}
\label{fig:4regimes_illustrated_PDF}  
\end{center}      
\end{figure}             
\begin{figure}[ht]    
\begin{center} 
% \subfloat  
% {
% \includegraphics[width=0.48\textwidth]
% {Figs/FP/Illustrate4superT.png}
% } 
% \subfloat
% {
% \includegraphics[width=0.48\textwidth]
% {Figs/FP/Illustrate4crit.png}
% }
% \\
% \subfloat{
% \includegraphics[width=0.48\textwidth]
% {Figs/FP/Illustrate4subT.png}
% }
% \subfloat{
% \includegraphics[width=0.48\textwidth]
% {Figs/FP/Illustrate4superSin.png}
% }
\includegraphics[width=0.99\textwidth]{Figs/FP/regimes_sdf_combined.pdf}
\caption[Model Regimes hitting-time survivor distributions]{The four different
parameter regimes using the parameter values given in \cref{tab:regimes}. Illustrated are the survivor
distribution functions, $\G_{\phi_m}(t)$, for representative $\phi_m = 2\pi/\th
\times \{0,0.25,0.5,0.75\}$. Varying $\phi_m$ has, for the most part, the effect
of shifting the curves laterally, while varying $\abg$ changes their
characteristic form.
A) supra-threshold, B) super-sinusoidal, C) critical, D) sub-threshold.}
\label{fig:4regimes_illustrated_SDF}    
\end{center}
\end{figure}     

\clearpage

\subsection{Basic ISI probability density functions}
Here we introduce the notation for the  probability density, distribution and
survival functions of $I_n$, an ISI arising from a trajectory
produced by \cref{eq:X_evolution_uo_renewal},
\begin{equation} 
\begin{array}{rcll}
g_{\phi}(\t) \intd{\t} &:=& \Prob(I_{n+1} \in [\t, \t + \intd{\t})  | \phi_n =
\phi) &
 \textrm{(probability density)} 
\\ 
G_{\phi}(t) &:=& \Prob(I_{n+1} \leq t  |\phi_n = \phi ) = \int_0^t g_{\phi}(\t)
\intd{\t} &
 \textrm{(cumulative distribution)}
\\
\G_{\phi}(t) &:= & \Prob(I_{n+1}>t | \phi_n = \phi ) = 1 - G_{\phi}(t)
&
 \textrm{(survivor distribution)}
\end{array}
\label{eq:ISI_distribution_functions}
\end{equation}
The subscript $\phi$ is to stress that $g, G$ and $\G$ depend on the value of
$\phi_n$ in \cref{eq:X_evolution_uo_renewal}. This is the formal statement that
in a sinusoidally-driven neuron, the interspike intervals are not identically
distributed, and are only independent conditioned on the sinusoidal phase at an
interval's onset. Knowing these distributions would provide the likelihood function,
offering estimation by the preferred method of choice, the maximum likelihood
estimator. Unfortunately, explicit expressions for the ISI distribution are not
available except for the special case of $\g = 0$ and $\a=1$ , see
\cite{DitlevsenLansky2005}. Different representations of the likelihood function
are available though, see \cite{Alili2005}, one of which we will use below.

\subsection{Fokker-Planck Equation with Absorbing Boundaries}
\label{sec:fp_estimation}
The Fokker-Planck equation is a partial differential equation (PDE) describing
the evolution of the probability density, $f(x,t)$, of $X_t$. 
For the sinusoidally-forced Ornstein-Uhlenbeck process,
\cref{eq:X_evolution_uo_renewal}, with the threshold $x_{th} = 1$, the PDE is
\begin{equation}
\di_t f^{(\phi)}(x,t) = -\di_x[(\a - x + \g \sin(\th (t + \phi))\cdot
f^{(\phi)}] +\di^2_x[ \tfrac{\b^2}{2}f^{(\phi)}], \quad x \in (-\infty,
1).
\label{eq:FP_pde_OU_absorbBC}
\end{equation}
Due to the reset, we have that at time $t=0$, $X_t=0$  and so for the initial
conditions we can write
\begin{equation}
f^{(\phi)}(x,t=0) = \delta(x) ,
\label{eq:PDF_ICs}
\end{equation}
where $\delta(\cdot)$ is the Dirac delta function. The spike is represented as
a zero boundary condition for $f$ at $x = 1$ $$
f(1, t) =0.
$$

The natural way of using the Fokker-Planck equation in first-hitting-times
problems is as follows. Denote the integral of $f^{(\phi)}$ by $F^{(\phi)}(x,t)
= \int_{\xi \leq x} f^{(\phi)}(\xi, t) \intd{\xi}$. $F^{(\phi)}(x,t)$ can be
related to the ISI's survivor distribution function, $\G_{\phi}(t)$, by
\begin{equation}
\G_{\phi}(t) = F^{(\phi)}(1,t).
\label{eq:SDF_vs_F_at_thresh}
\end{equation}
\Cref{eq:SDF_vs_F_at_thresh} forms the
basis of one of the methods below for estimating the structural parameters from
the observed data.

Since \cref{eq:FP_pde_OU_absorbBC} has to be solved numerically, we will need to
truncate its domain from below. The most natural way to do this, given the
dynamics, is to impose reflecting boundary conditions at some $x=\xmin \ll
(\a-\g/\sqrt{1+\th^2})$ where the probability mass is very small. For the left
(lower) limit of the computational domain, we use the formula $$ \xmin =
\min(\underbrace{\a -\g/\sqrt{1+\th^2}}_{\textrm{mean}} - 2 \underbrace{\b/
\sqrt{2}}_{\textrm{std. dev}}, -0.25).$$ This choice requires some explanation.
In the $t\ra \infty$ limit, the distribution of $X_t$ in
\cref{eq:X_evolution_uo_renewal} {\sl without} thresholding is Gaussian with
mean given by \cref{eq:LIF_deterministic_solution} (below) and variance equal to
$\beta^2/2$. Thus to truncate the computational
domain for the thresholded process from below, we take the lowest value of
the asymptotic mean, $\alpha - \gamma / \sqrt{1 + \th^2}$, then from
this we subtract two standard deviations, $2\b/\sqrt{2}$ and set the result to be the lower
bound, $\xmin$. Finally, if this value for $\xmin$ happens to be larger than
$-.25$, we enforce that  $\xmin \leq -0.25$. 

% The reflecting BCs look like:
% \begin{equation}
% \big[ -(\a - x + \g \sin(\th (t + \phi) ) \cdot f + 
% 	\frac{\b^2}{2} \cdot \di_x f \Big] \Big|_{x=c} = 0.
% \label{eq:reflecting_BCs}
% \end{equation}

Numerical considerations lead us to solve for $F$, instead of $f$, since delta
functions are difficult to represent in floating point, while the initial
conditions for $F$, the Heaviside step function, $H(x)$, faces no such
difficulties \cite{Hurn2005}. The Heaviside step function is defined to be
equal to $0$ for $x<0$ and to be equal to $1$ for $x \geq 0$. At this point we
need to derive the PDE for the distribution $F$, starting from the PDE for the
density, $f$, \cref{eq:FP_pde_OU_absorbBC}.

First, at the lower boundary, it is intuitive that the distribution should be
zero, $ F(\xmin,t) = 0 $, while $f(1,t) = 0$ implies that at the upper boundary
$ \di_xF(1, t) = 0 $. Inside the domain, the PDE itself reformulates as
\begin{eqnarray*}
\di_t f(x,t) &=&  \di_x \left[\frac{1}{2}\di_x [\b^2 f] -  (\a - x + \g \sin(\th
(t - \phi)) f \right]
\end{eqnarray*}
so that
\begin{eqnarray*}
\di_x \di_t F(x,t) &=& \di_x \left[
\frac{\b^2 }{2}\cdot \di^2_x F -  
						(\a- x + \g \sin(\th (t + \phi))  \cdot \di_xF \right].
\end{eqnarray*}
Integrating with respect to $x$ then gives
$$
\di_t F(x,t) =
\frac{\b^2 }{2}\cdot \di^2_x F -  
						(\a- x + \g \sin(\th (t + \phi))  \cdot \di_xF + C(t)
$$
where $C(t)$ is a constant of integration depending on $t$. Now consider
the lower boundary condition, $x =
\xmin$. Here $F(\xmin,t) = 0$ implies that $\di_tF = 0$ and so 
\begin{equation}
C(t) = - \left[ \frac{\b^2 }{2}\cdot \di^2_x F -  
			(\a- x + \g \sin(\th (t + \phi))  \cdot \di_xF \right].	
\label{eq:const_of_integration}
\end{equation}
The right-hand side in eq.\ \eqref{eq:const_of_integration} is precisely the
reflecting boundary condition on $f$ once we recall that $\di_x F = f$. Therefore $C(t) \equiv 0$.

Thus, the fully specified PDE for $F$, which we will be solving frequently in what
follows, is
\begin{equation}
\begin{gathered}
	\di_t F^{(\phi)}(x,t) =
					\frac{\b^2 }{2}\cdot \di^2_x F^{(\phi)} -  
					\Big(\a- x + \g \sin(\th (t + \phi))\big)  \cdot \di_x F^{(\phi)},
	\\
	\\
	\left\{ \begin{array}{lcl}
	 F^{(\phi)}(x,0) &=& H(x)
	\\
	F^{(\phi)}(x,t) |_{x=\xmin} &\equiv& 0 
	\\
	\di_x F^{(\phi)}(x,t) |_{x=\vt} &\equiv& 0.
	\end{array} \right.
\label{eq:FP_pde_OU_absorbBC_CDF}
\end{gathered}
\end{equation}
Numerical solutions for \cref{eq:FP_pde_OU_absorbBC_CDF} are shown in
\cref{fig:FP_pde_OU_absorbBC_CDF}. We have used the standard Crank--Nicholson
finite-difference algorithm (central-differences in space with equally weighted
implicit-explicit terms in time, see \cite{Karniadakis2003}). 

\begin{figure}[h]
\begin{center}
% \subfloat[]{
% \includegraphics[width=0.45\textwidth]
% {Figs/FP/surf_phi_105_az=-50.png}
% } 
% \subfloat[] { 
% \includegraphics[width=0.45\textwidth]
% {Figs/FP/surf_phi_105_az=-10.png}
% }
% \\
% \subfloat[] {
% \includegraphics[width=0.5\textwidth]
% {Figs/FP/SDF_phi_105.png}
% }
\includegraphics[width=0.95\textwidth]{Figs/FP/SDF3D_combined.pdf} 
\caption[Transition distribution example solution]{Example solution to
\cref{eq:FP_pde_OU_absorbBC_CDF} for $(\a,\b,\g) = (0.5, 0.3, 0.5\sqrt{2})$; $\th= 1, \phi = \pi / 2$.
In A,B,C, we show the full solution in space-time $F(x,t)$. In (d) we show
the time solution at the upper boundary, $F(1,t)$.} 
\label{fig:FP_pde_OU_absorbBC_CDF} 
\end{center}
\end{figure}

\subsection{Fortet Equation}
\label{sec:fortet_estimation}
Consider a general form of \cref{eq:X_evolution_uo_renewal}, $$ dY_t =
b(t,Y_t)dt + \sigma(t,Y_t) dW_t. $$ Let $\F(y,t| y_0, t_0) :=   \Prob[Y_t \leq
y| Y_{t_0} = y_0]$ be the transition cumulative distribution of $Y$. Note that
this is the distribution of $Y_t$ in absence of a threshold, different from the
distribution given in \cref{eq:SDF_vs_F_at_thresh},
 which is the distribution 
of the process constrained to be below the threshold. 
Now consider an arbitrary time-dependent threshold $\vt(t)$. The Fortet
equation, see \cite{Fortet1943}, convolves the first-hitting time probabilities,
$g(t)$, with the transition density of the process. Integrating
over $(-\infty, \vt(t))$, we obtain
\begin{equation}
1 - \F(\vt(t), t|v_0, 0) =
\int_0^t g(\tau) [1-\F (\vt(t),  t| \vt(\t), \t)] \intd{\t}.
\label{eq:Fortet_moving_vth}
\end{equation}
The left hand side is simply the probability of exceeding $\vt$ at time
$t$ starting at $v_0$ at time $0$. This can also be written as the probability
of hitting $\vt$ for the first time at time $\t < t$ and then exceeding
$\vt$ at time $t$ starting at $\vt$ at time $\t$, integrated over all $\t$.
% Strictly speaking, \cref{eq:Fortet_moving_vth} is not the Fortet equation, but
% the integral of the Fortet equation, while the original Fortet equation uses
% $\phi(x,t) = \di_x \F(x,t)$; we use the integrated version for numerical
% convenience.

The Fortet equation is particularly appealing to use when we have an analytical
expression for $\F$. For the problem at hand, $\Phi$ is complicated
due to the time-dependent forcing. However, the following transformation yields
a time-homogeneous $Y$ for which $\Phi$ will be tractable, along with an associated
moving threshold, $\vt(t)$. This makes feasible
the use of the Fortet equation. To obtain this transformation, cf.\
\cite{Shimokawa1999}, consider the deterministic version of the SDE in
\cref{eq:X_evolution_uo_renewal} 
\begin{equation}
dv(t) = \left( \a - v + \g \sin(\th (t + \phi) ) \right) \intd{t},
\label{eq:LIF_deterministic}
\end{equation}
$$v(0) = 0
$$
with solution
\begin{equation}
v(t) = \a( 1- \exp(-t)) 
  + \frac{\g}{\sqrt{1+\th^2}} 
\left[  \sin(\th(t + \phi) - \psi )
 - \exp(-t)\sin(\phi\th - \psi ) \right]; \quad  
\psi = \arctan(\th).
\label{eq:LIF_deterministic_solution}
\end{equation}

Now take $X_t$, the solution to \cref{eq:X_evolution_uo_renewal} and
$v(t)$, \cref{eq:LIF_deterministic_solution}, and 
let $Y_t = X_t - v(t)$. Then
\begin{equation}
dY_t = -Y_t dt + \b \intd{W},
\label{eq:Y_evolution_uo_transformed}
\end{equation}
which has the time and parameter dependent threshold
\begin{equation}
\vt_{\{\a,\g;\phi\}}(t) = \vt - v(t).
\end{equation}
That is, $X_t$ hits the constant threshold $\vt$ if and only if $Y_t$ hits the
moving threshold $\vt_{\{\a,\g;\phi\}}(t)$, where the subindex
indicates the dependence on $\a,\g$ and $\phi$. Therefore the ISIs produced by $X$ and $Y$
are the same and so are their distributions. Thus, $g_\phi(\t)$ satisfies
\begin{equation}
1 - \F_{\{\b\}}(\vt_{\{\a,\g;\phi\}} (t), t|0, 0) =
\int_0^t g_{\phi} (\tau)
\left[1-\F_{\{\b\}} \left(\vt_{\{\a,\g;\phi\}} (t),  t|
						  \vt_{\{\a,\g;\phi\}} (\t), \t ) \right)
      \right] \intd{\t},
\label{eq:Fortet_moving_vth}
\end{equation}
where 
$$
\F_{\{\b\}}(y,t| y_0, t_0) = \frac{1}{ \sqrt{\pi \b^2(1-e^{-2(t-t_0)}) }}
\int_{-\infty}^{y} \exp \left( -\frac{(x - y_0e^{-(t-t_0)})^2}
							         {\b^2(1-e^{-2(t-t_0)})} \right) \intd{x}
$$
is the conditional cumulative distribution function of $Y_t$ defined in
\cref{eq:Y_evolution_uo_transformed}.

\section{Parameter Estimation Algorithms}
\label{sec:estimation_algos}
The unknown parameters in \cref{eq:X_evolution_uo_renewal} are $\a,\b$ and $\g$,
while we assume $\th$ known. The reason why the amplitude, $\g$, is often
unknown while the frequency, $\th$, is known is that one can usually observe the
sinusoidal input and thus its frequency. Further, the encoding of the input into
neuronal firing patterns often involves phase locking to the sinusoidal
component. However, the actual forcing amplitude at the level of the neuron is
usually modified by various synaptic and other filtering processes, unless the
cell receives direct sinusoidal current injection.

Our goal is to estimate the structural parameters $(\abg)$ from a sample of
spike time data, $\{i_1,\ldots,i_N\}$. There are several algorithms for
estimating the parameters for the simpler and more common case of $\g = 0$. One
such algorithm relies on the Fortet equation, see
\cite{Ditlevsen2008,Ditlevsen2007}, which we extend to the presence of a
time-varying current. A more basic approach is to directly solve the
Fokker-Planck equation for the probability density of $X_t$,
\cite{Sirovich2011a,Paninski2004,Dong2011}, from which one can derive the
survival distribution of $I_n$ and use this to compare against the empirical
survival distribution of $I_n$ obtained from data. An approximate maximum
likelihood approach is also possible by numerical differentiation. The relation
between Fokker-Planck equations and the first-passage time problem is discussed
in most introductory books on stochastic analysis, see, for example,
\cite{Jacobs}. A recent review of this approach for the simple $\g=0$ case in
neuronal modeling can be found in \cite{Sirovich2011a}, wherein the first
passage problem is discussed at great lengths in the context of spiking neurons.
We will use this in \cref{sec:fp_estimation}. A more elaborate approach using
the Fokker-Planck equation to approximate the hitting time distribution is given
in \cite{Lo2006}. The techniques in \cite{Lo2006} avoid the need to compute the
Fokker-Planck PDE numerically, instead approximating it with analytically known
solutions. This approach might offer significant computational savings, but
since this would at most amount to a computational speed-up of our algorithm, we
have left this unexplored for now.


The immediate problem in generalizing the aforementioned approaches to the case
of $\g \neq 0$ is that the $I_n$'s are no longer identically distributed since
the phase $\phi_{n-1}$ of the $n$th interval $I_n$ depends on $t_{n-1}$, the
time the previous spike occurred. The $I_n$'s are also dependent, but conditionally
independent given $\phi_{n-1}$. So the trajectories in each interval are
parametrized by the value of $\phi_{n-1}$ at the time of the last spike/reset.
We overcome this obstacle by splitting the $I_n$'s in groups, and
approximating the $I_n$'s within groups as coming from identically
distributed trajectories in a sense to be specified below. This approximation
which solves the challenge of dependent and non-identically distributed ISIs is
the primary contribution of this paper.

\subsection{$\phi$ - binning}
Before we can use \cref{eq:FP_pde_OU_absorbBC_CDF} or
\eqref{eq:Fortet_moving_vth}, we need to deal with the fact that $\phi$ is not
fixed, but instead each $I_n$ starts with a distinct $\phi_n$. Our approach is
to partition the interval $[0, 2 \pi/\th]$ into $M$ bins, where $M \ll N$, and
represent each bin by the midpoint of the bin, $\phi_m$. Then we approximate the
$N$ observed $\phi_n$'s by the closest $\phi_m$ and pretend that any observed
$I_n$ was not produced by a trajectory of the form in
\cref{eq:X_evolution_uo_renewal} with $\phi = \phi_n$, but with $\phi = \phi_m$.
Our hope is that for a judicious choice of $M$, we can balance the error of
$\phi_n \neq \phi_m$ with having enough data points in each bin in order to
obtain a useful estimate from \cref{eq:FP_pde_OU_absorbBC_CDF} or
\eqref{eq:Fortet_moving_vth}.

There is clearly much freedom in how one sets up these bins, but we will do the
simplest thing and make them all of equal width, $\dphi = 2 \pi / {(\th M)}$.
Each $\phi_n$ will belong to one and only one of the bins $ [\phi_m - \dphi/2,
\phi_m + \dphi/2)_{m=1}^M, $ with centre points $ \phi_m = \dphi / 2 + (m-1)
\dphi$, for $m = 1,\ldots, M$. Thus, given an empirically observed $I_n$ with
associated $\phi_n$, we will pretend that it was produced by the process $$ dX_s
= (\a - X_s) \intd{s}  + \g \sin(\th (s+\phi_m(n) ))
\intd{s}
+ \b \intd{W_s}, $$ where $$ \phi_m(n)  = \argmin_{\phi_m} {| \phi_n - \phi_m|}.
$$ This binning is illustrated in \cref{fig:binning_visualized}.
\begin{figure}[ht]
\begin{center}
%     \subfloat[Raw ISIs]{
% 	\includegraphics[width=0.45\textwidth]{Figs/Bins/Example_raw.png}
% 	}
%     \subfloat[Binned ISIs]{
% 	\includegraphics[width=0.45\textwidth]{Figs/Bins/Example_binned.png}
%   	}  
\includegraphics[width=0.9\textwidth]{Figs/Bins/Example_composite.pdf}
  \end{center}
\caption[Phase binning illustration]{The raw $(\In, \phi_n)$ pairs (left) are
binned into a set of $M$ bins with a representative $\phi_m$ (right) and the ISIs within each bin are treated
as a renewal process. In this illustration, $M=8$, $\th =
1$ while the parameters $\abg$ are taken from the
supra-threshold regime. }
\label{fig:binning_visualized} 
\end{figure}

While we have no rigorous approach to determine the value of $M$, our limited
experience suggests that given $N=1000$ ISIs, $M=10$ or $M=20$ gives satisfactory
results for very different parameter regimes. In general, choosing $M$ is a
balancing act. For $M$ too high, the resulting bins will have too few data
points to approximate $\G(I)$ accurately. Therefore $M$ is forced to be small
when sample size is not large. For $M$ too low, the approximation of the phase
shifts will be poor, leading to a biased estimate of $\G(I)$. We illustrate the
effect of increasing $M$ in \cref{fig:effect_of_M}. Generally, as long as there
are sufficient data points, as $M$ increases, the approximation of using the
survival distribution with $\phi_m$ instead of $\phi_n$ improves since
$\phi_m(n) \ra \phi_n$ as $M \ra \infty$. In the sequel, we will use $M=20$ for
sample sizes of $N=1000$ and $M=8$ for sample sizes of $N=100$.

\begin{figure}[h]
\begin{center}
% \subfloat[M=5]
% {
% \includegraphics[width=0.25\textwidth]
% {Figs/FP/SuperSin_M=5.png}
% }
% \subfloat[M=10] 
% {
% \includegraphics[width=0.25\textwidth]
% {Figs/FP/SuperSin_M=10.png}
% } 
% \subfloat[M=20]
% {
% \includegraphics[width=0.25\textwidth]
% {Figs/FP/SuperSin_M=20.png}
% }
% \subfloat[M=40]
% {
% \includegraphics[width=0.25\textwidth]
% {Figs/FP/SuperSin_M=40.png}
% }
% \includegraphics[width=0.99\textwidth]{Figs/FP/EffectOfM.pdf}
\includegraphics[width=\textwidth]{Figs/FP/EffectOfM_Referees.pdf}
\caption[Effect of Bin-size]{Effect of $M$, the number of bins, on the
approximate survival distribution. The full-drawn blue curve is the true survivor distribution
given in \cref{eq:FP_pde_OU_absorbBC_CDF}, the red points are the approximation
given in \cref{eq:SDF_estimate_per_bin}.  
In the figures, the least populous (above) and most populous (below) bin for
each $M$ is shown. The width of the bins is $\dphi = {2\pi}/{(\th M)}$.
We have used A,E) $M=5$, B,F) $M=10$, C,G) $M=20$,
D,I) $M=40$. As $M$ increases, the approximation of using the survival
distribution using $\phi_m$ instead of $\phi_n$ improves since $\phi_m(n) \ra
\phi_n$ as $M \ra \infty$. The data is generated using parameter
values from the super-sinusoidal regime and $N=1000$. For this
particular data set the largest generated ISI was 6.55 time units.}
\label{fig:effect_of_M}
\end{center}
\end{figure}

\subsection{Fokker-Planck Algorithm}
Within each bin it is clear how to apply \cref{eq:SDF_vs_F_at_thresh}. In the
$m$th bin, for a given $\phi_m$, we approximate $\G_{\phi}(t)$  by
\begin{equation}
\Gest_{\phi_m}(t) =
 \frac{\#[i_n > t \, \big| \, \phi_{n-1} \in [\phi_m - \dphi/2,
\phi_m + \dphi/2) ]}{N_m},
\label{eq:SDF_estimate_per_bin}
\end{equation}
where $N_m$ is the number of ISIs in bin $m$. Using \cref{eq:SDF_vs_F_at_thresh}
we define the loss function
\begin{equation}
L(\a,\b,\g) = 
\sum_{\phi_m} N_m \Big\{ 
% \int_0^T  \left( \G_{\phi_m}(t) - F^{\phi_m}_{\a,\b,\g}(\vt, t) \right)^2
% \intd{t} \Big\}.
\sup_{t>0} \left| \Gest_{\phi_m}(t) - F^{\phi_m}_{\a,\b,\g}(\xth,
t) \right| \Big \}.
\label{eq:loss_function_absorbingBC}
\end{equation}
The weight $N_m$ is included so that bins with larger sample sizes have a
larger influence on the estimates. 
 
To evaluate the supremum in \cref{eq:loss_function_absorbingBC}, we
spline interpolate the empirically discrete $\Gest$ for each $\phi_m$, sample at
the time nodes of the PDE discretization and 
finally take the maximum amongst the sampled values.
We then minimize $L$ using an optimization algorithm (see below,
\cref{sec:method_performance}) and take our estimates $\abgest$ to be
$$
\abgest = \argmin_{\abg} L(\abg).
$$

Note that the relation between the spike time survival density, $\G_{\phi}$ and
the transition distribution, $F_{\phi}$, in \cref{eq:SDF_vs_F_at_thresh} could
also allow for an approximate maximum likelihood estimator (MLE), based on
maximizing 
$$
L^{\textrm{MLE}}(\a,\b,\g) = \sum_n  \log ( g_{\phi_{n-1}}(i_n) )
= \sum_n \log \big[ -\di_t F^{\phi_{n-1}}_{\a,\b,\g}(\xth, t) \big] \Big|_{t =
i_n},
 $$ where the derivative has to be approximated by finite differences. We
can then again use binning to avoid having to compute the PDE separately for
each $(i_n, \phi_{n-1})$. Our experience with the MLE approach has been that the
quality of the estimates provided are similar to those obtained by minimizing
\cref{eq:loss_function_absorbingBC} and that the associated computing times are
on the same order. Due to this similarity and in order to keep the paper
concise, we include details of the MLE estimates only in the supplementary
online material.
\subsection{Fortet Algorithm}
An alternative approach is to form a loss function from
\cref{eq:Fortet_moving_vth}. This is similar to what is done in
\cite{Ditlevsen2008,Ditlevsen2007} for the simpler case of a constant threshold.
Noting that $\int_0^t g (\t) [1-\F ] \intd{\t} = \Exp[ (1 - \F ) \charf_{I \leq
t} ]$ where the expectation is taken with respect to the distribution of the
random variable $I$, we can use the fact that the ISIs are approximately
independent and invoke the law of large numbers to estimate the integral as
\begin{multline*}
\int_0^t g_{\phi_m}(\t) 
\big[1-\F^{(\phi)}_{\{\b\}}(\vt_{\{\a,\g;\phi\}}(t),  t \,|\,
\vt_{\{\a,\g;\phi\}}(\t), \t) \big] \intd{\t} \approx 
\\
1/N_m \sum_{\In < t} 
\big[1
- \F^{(\phi)}_{\{\b\}}(\vt_{\{\a,\g;\phi\}}(t), t \,|\,
\vt_{\{\a,\g;\phi\}}(\In),\In )\big].
\end{multline*}


We then define the loss function
\begin{align}
L(\a,\b,\g) = 
\sum_{\phi_m} N_m \Bigg\{ 
% \sum_\In \Bigg[
% \int_{\e}^{I_{\text{max}}}
% 		 \Big[& 1 - \F^{(\phi_m)}_{\abg}(\vt(s), s|v_0) ] -
% 		 \notag
% \\
% 		 &-  \tfrac{1}{N_m}\sum_{\In' < s} 
% 		 [1 - \F^{(\phi_m)}_{\abg}(\vt(s),s-\In'| \vt(\In') ) ]
% 		   \Big]^2	\intd{s}
\sup_{s > 0}
		 \Big|& 1 - \F^{(\phi_m)}_{\{\b\}}(\vt_{\{\a,\g;\phi\}}(s), s \,|\,0,0) ] 
		 \notag
\\
		 &-  \tfrac{1}{N_m}\sum_{\In < s}
		 \big[1 - \F^{(\phi_m)}_{\{\b\}}(\vt_{\{\a,\g;\phi\}}(s),s  \,|\,
		 \vt_{\{\a,\g;\phi\}}(\In), \In) \big] \Big| / \omega(\phi_m; \a,\b,\g)
		   \Bigg\}.
\label{eq:loss_function_fortet}
\end{align}
We divide each inner term by $\omega(\phi_m; \a,\b,\g) =
\sup_{s > 0} |1 - \F^{(\phi_m)}_{\abg}(\vt(s), s|v_0) |$,
following the suggestion in
\cite{Ditlevsen2007}. This scaling ensures that \cref{eq:Fortet_moving_vth}
divided by $\omega(\a,\b,\g)$ will vary between $0$ and $1$ for all parameter
values thus giving sense to the measure defined by the loss
function. Since we can solve in closed form for $\F$, we have all we
need given an observed spike train of 
$\In$'s. We evaluate the $\sup$ by sampling at 
$K=500$ uniformly spaced points in $(0, I_{\max} + \e]$ and taking the maximum
of the sampled values. 


\subsection{Initialization of the algorithms}
The parameter search can be initialized in a simple way using the fact that
the Fokker-Planck PDE is almost an 'advection-diffusion' equation whose solution is
almost a Gaussian. Then $\G(t)$ can be approximated by the
amount of probability mass of a Gaussian to the left of the threshold at time $t$. The
idea is as follows. Suppose we are solving the following PDE
\begin{equation}
\di_t \f = -U \di_x [\f] + \frac{\b^2}{2} \di^2_x [\f].
\label{eq:FP_pde}
\end{equation}
Its solution given an initial condition $\f(x,0) = \delta(x)$ will be a 
Gaussian bell moving to the right with speed $U$ and standard deviation $\s =
\b\sqrt{t}$.

The survivor function $\G(t)$ can be thought of as the amount of area that has
passed the threshold (from the left moving to the right). We can then invert the
information about $\G$ to estimate $U$ and $\b$. In particular, a Gaussian bell
has $\approx 0.158$ of its mass more than one standard deviation to the right of
its mean. Thus, at time $t_1$ such that $\G(t_1) = 0.842$, the right tail of
more than one standard deviation of the Gaussian bell has crossed the threshold.
The threshold is at $x = 1$ and we obtain the following equation
\begin{equation}
U t_1 + \b \sqrt{t_1} = 1.
\label{eq:right_1std}
\end{equation}
Similarly, at time $t_2$ such that $\G(t_2) = 1 - 0.842$, the Gaussian bell has
crossed the threshold except for the left tail and we have
\begin{equation}
U t_2 - \b \sqrt{t_2} = 1.
\label{eq:left_1std}
\end{equation}
If $U$ and $\b$ were constant, then \cref{eq:right_1std,eq:left_1std} provide
two equations in two unknowns.
% \begin{figure}[htp]
% \begin{center}
%   \includegraphics[width=.75\textwidth]{Figs/Diagrams/Normal_quantiles_from_wp.png}
%   \caption[labelInTOC]{Illustration of Gaussian bell -
%   from: http://en.wikipedia.org/wiki/File:Standard\_deviation\_diagram.svg }
%   \label{fig:gaussian_bell}
% \end{center}
% \end{figure}
However, $U = U(x,t) = (\a - x + \g \sin(\th (t + \phi)) )$ is not constant and
we approximate $U$ as
\begin{equation}
U(x,t ) \approx \a -0.5 + \g \frac{1}{t}\int_0^t \sin(\th (\t+\phi)) \intd{\t}, 
\end{equation}
i.e.\ we approximate the space-dependent term, $x$, with the mid-point between
the reset value, $v_0 = 0$, and the threshold, $\vt = 1$, and we approximate
the time-dependent term, $\sin(\th (\t+\phi))$, by its time-average value
between $0$ and $t$. If we use the \nth{0}, \nth{1} and \nth{2} standard 
deviation points, we can form 5 equations in 3 unknowns as follows
\begin{eqnarray*}
\a t_1 + \g s(t_1) + 2\b \sqrt{t_1}
&=& 1 + 0.5 t_1
\\
\a t_2 + \g s(t_2) + \b \sqrt{t_2}
&=& 1 + 0.5 t_2 
\\
\a t_3 + \g s(t_3) + 0\b
&=& 1 + 0.5 t_3
\\
\a t_4 + \g s(t_4)-1 \b \sqrt{t_4}
&=& 1 + 0.5 t_4 
\\
\a t_5 + \g s(t_5) - 2\b \sqrt{t_5}
&=& 1 + 0.5 t_5 
\end{eqnarray*}
with the time-average weighting function $s(t) = (\cos(\th \phi)
-\cos(\th(t+\phi)))/{\th} $. However, the  approximation is best for earlier
times, when the solution is closer to a Gaussian bell that is approaching the
threshold, but less correct for later times, since it neglects the loss of
probability mass and thus overestimates the backward probability current.
Indeed, we have found it to be best to use only $t_1$ and $t_2$. In the
following we use only these equations
\begin{eqnarray*}
\a t_1 + \g s(t_1) + 2\b \sqrt{t_1}
&=& 1 + 0.5 t_1
\\
\a t_2 + \g s(t_2) + \b \sqrt{t_2}
&=& 1 + 0.5 t_2 
\end{eqnarray*}
for the initializer.
We can form these equations separately for each $\phi_m$ bin, thus
resulting in $M \times 2$ equations for the unknowns $\a, \b$ and $\g$. Since
we have more equations than unknowns, we use least-squares estimates in a
regression to pick out unique $\a,\b$ and $\g$ estimates. 

The proposed initialization procedure has two advantages. First, it is
automatic, i.e.\ it requires only the data and no input or  guidance from the
user. Second, it is extremely fast. While the precise effect of the initializer
is shown in \cref{sec:method_performance}, it is intuitively clear that it will
work best in the supra-threshold parameter regime when the bell curve is truly
moving past the threshold as a whole and less so for sub-threshold regimes, when
only the diffusive force serves to propel the process to reach $\vt$. The
behaviour of the initializer in the different regimes is illustrated in
\cref{fig:sdf_real_vs_init_estimated}. What we show in
\cref{fig:sdf_real_vs_init_estimated} is the following: First we show the
survival distribution for a given regime and $\phi_m$ fixed. Then using data
generated from such a regime and with $\phi_n$ in the $m$th bin, the initializer
tries to find the best approximation by the motion of a Gaussian bell which will
fit this data, in the sense of solving for $\a,\b,\g$ as previously described.
Once this is done, we then show in red the amount of area under this Gaussian
bell to the left of the threshold. Of course the interpretation of the survival
distribution for an ISI as a fraction of the area under a moving bell with
conserved total area is wrong, but the assumption is useful in automatically
generating initial values for the more appropriate approximations to start their
work.

\begin{figure}[htp]
\begin{center}
\includegraphics[width=.99\textwidth]{Figs/FP/sdf_init_vs_exact.pdf}
\caption[Estimation initiation]{The blue curves are the numerically
obtained survivor distributions $\G_\phi$ for the exact parameters in the four regimes (as in
  \cref{tab:regimes}) and $\th=1$. The red curves are obtained in the following
  manner: Simulations using the true parameters were used to generate sample spikes.
  Using these samples, the initializer algorithm was used to generate estimates
  for $\a,\b,\g$. Using these estimates, the bell curve discussed in sec.\ 3.4
  was formed and evolved in time. 
  Thus, the red curve drawn in the figures measures the area under
  this bell that is to the left of the threshold at time $t$. 
 A) supra-threshold, B) super-sinusoidal, C) critical, D) sub-threshold.}
  \label{fig:sdf_real_vs_init_estimated}
\end{center}
\end{figure}


\section{Method Comparison on Simulated Data}
\label{sec:method_performance}
We will now use our algorithms on spike trains simulated from the four different
regimes; the supra-threshold, the critical, the super-sinusoidal and the
sub-threshold. We have used 100 sample spike trains per regime, with $N=100$ as
well as $N=1000$ spikes per train. In order to perform the numerical
minimization of \cref{eq:loss_function_absorbingBC,eq:loss_function_fortet}, we
have used an implementation of the Nelder-Mead algorithm from the SciPy library
\cite{scipy}. The Nelder-Mead algorithm is a non-linear minimization routine
which uses a bounding-polygon method to zero-in on the minimum and thus avoids
the need to provide the gradient of the loss function. It is the standard
non-gradient minimization algorithm. 

The estimation results are shown in
\cref{fig:comprehensive_test_SuperT_relerrors,fig:comprehensive_test_SubT_relerrors,fig:comprehensive_test_crit_relerrors,fig:comprehensive_test_SuperSin_relerrors},
where we plot box plots for the estimates, $\abgest$ in the four regimes. We
also tabulate the average and the empirical 95\% confidence intervals of the
estimates in \cref{tab:est_quantiles_100,tab:est_quantiles_1000}. Conclusions
that can be drawn from these results are as follows. The initializer method is
effective for the supra-threshold regime and gives reasonable ballpark estimates
for all regimes, though the error can be substantial for the super-sinusoidal
regime. In general, both the Fortet and Fokker-Planck algorithm estimate the
parameters well in the supra-threshold, critical and super-sinusoidal regimes.
The estimators variance is especially low in the supra-threshold regime, while
it is higher for the critical and super-sinusoidal regimes. In the
super-sinusoidal regime the two algorithms give accurate estimates even though
the initializer can be quite off. On the other hand, in the sub-threshold regime
the initializer has a performance comparable to that of the two more involved
methods. It seems that distinguishing between the constant bias and the
sinusoidal current is difficult if their sum is not sufficient to generate
spikes without noise.


The Fokker-Planck method has a larger bias but a smaller spread than the Fortet
method for $N=100$, \cref{tab:est_quantiles_100}. However for
$N=1000$, the two methods have comparable 
spreads, while the Fortet method retains a smaller bias, see
\cref{tab:est_quantiles_1000}. More precisely, for $N=1000$, the Fokker-Planck
method has a smaller spread in the sub-threshold regime, while the Fortet method
has a smaller spread in the super-sinusoidal regime. As such, at least in the
super-sinusoidal regime, the Fortet method seems superior.

The two algorithms are numerically intensive. For $N=100$ and $N=1000$ spikes,
we show the times taken for the estimation in \cref{tab:walltimes}. While we
have done most of our numerical work in Python/SciPy\cite{scipy}, we have
implemented the critical components of both algorithms in C. That is we solve
the inner part of \cref{eq:loss_function_fortet} and the Fokker-Planck PDE,
\cref{eq:FP_pde_OU_absorbBC_CDF}, in C using the GSL libraries\cite{gsl}. From
\cref{tab:walltimes}, we can verify that the computing time for the Fortet
algorithm scales proportionally with the number of spikes. This is to be
expected, since the Fortet equation has a term of the form $\sum_{i_n}$ which in
turn has $N$ terms and this forms the bulk of the computing time for the
Fortet equation. The Fokker-Planck algorithm, on the other hand, scales
less-than-linearly with $N$, since the dependency on $N$ is in forming the
approximation, $\Gest$ to the survivor function and that is not computationally
intensive (solving the PDE is).


\begin{table}
\begin{center}
%\subfloat[Supra-threshold]
{\begin{tabular}{|c|ccc|} 
Parameter
& Initializer
& Fokker-Planck
& Fortet
\\ \hline
\multicolumn{4}{|c|}{Supra-threshold regime} \\[1mm]
$\alpha=1.40$
& $1.43 : [1.29, 1.56]$
& $1.34 : [1.24, 1.43]$
& $1.41 : [1.33, 1.49]$
\\
$\beta=0.30$
& $0.17 : [0.10, 0.24]$
& $0.29 : [0.21, 0.39]$
& $0.29 : [0.22, 0.36]$
\\
$\gamma=0.14$
& $0.16 : [0.02, 0.33]$
& $0.12 : [0.02, 0.23]$
& $0.12 : [0.01, 0.24]$
\\
\hline \hline
\multicolumn{4}{|c|}{Super-sinusoidal regime} \\[1mm]
$\alpha=0.10$
& $0.92 : [0.83, 1.01]$
& $0.28 : [0.02, 0.59]$
& $0.24 : [-0.22, 0.42]$
\\
$\beta=0.30$
& $0.15 : [0.10, 0.25]$
& $0.31 : [0.14, 0.53]$
& $0.32 : [0.14, 0.46]$
\\
$\gamma=1.98$
& $1.35 : [1.13, 1.57]$
& $1.67 : [1.33, 2.05]$
& $1.77 : [1.44, 2.38]$
\\
\hline \hline
\multicolumn{4}{|c|}{Critical regime} \\[1mm]
$\alpha=0.50$
& $0.72 : [0.66, 0.80]$
& $0.57 : [0.32, 0.73]$
& $0.57 : [0.36, 0.73]$
\\
$\beta=0.30$
& $0.19 : [0.10, 0.26]$
& $0.27 : [0.17, 0.40]$
& $0.25 : [0.15, 0.40]$
\\
$\gamma=0.71$
& $0.57 : [0.44, 0.73]$
& $0.55 : [0.30, 0.83]$
& $0.62 : [0.38, 0.93]$
\\
\hline \hline
\multicolumn{4}{|c|}{Sub-threshold regime} \\[1mm]
$\alpha=0.40$
& $0.62 : [0.57, 0.67]$
& $0.63 : [0.33, 0.84]$
& $0.58 : [0.03, 1.00]$
\\
$\beta=0.30$
& $0.17 : [0.10, 0.29]$
& $0.20 : [0.10, 0.37]$
& $0.19 : [0.00, 0.41]$
\\
$\gamma=0.57$
& $0.32 : [0.00, 0.53]$
& $0.29 : [0.00, 0.62]$
& $0.46 : [0.00, 1.19]$
\\
\hline
 \end{tabular}}\\
\end{center}
\caption[Estimator Performance given 100 spikes]{Averages and empirical 95\%
confidence intervals of the estimates for $N=100$ spikes per train.}
\label{tab:est_quantiles_100}
\end{table}

%\end{document}

\begin{table}
\begin{center}
{\begin{tabular}{|c|ccc|} 
Parameter
& Initializer
& Fokker-Planck
& Fortet
\\ \hline
\multicolumn{4}{|c|}{Supra-threshold regime} \\[1mm]
$\alpha=1.40$
& $1.44 : [1.40, 1.50]$
& $1.36 : [1.33, 1.40]$
& $1.40 : [1.37, 1.42]$
\\
$\beta=0.30$
& $0.25 : [0.22, 0.28]$
& $0.29 : [0.26, 0.32]$
& $0.30 : [0.27, 0.32]$
\\
$\gamma=0.14$
& $0.14 : [0.10, 0.19]$
& $0.14 : [0.10, 0.17]$
& $0.14 : [0.10, 0.18]$
\\
\hline \hline
\multicolumn{4}{|c|}{Super-sinusoidal regime} \\[1mm]
$\alpha=0.10$
& $0.90 : [0.85, 0.92]$
& $0.11 : [0.03, 0.29]$
& $0.10 : [0.03, 0.16]$
\\
$\beta=0.30$
& $0.18 : [0.14, 0.23]$
& $0.30 : [0.21, 0.34]$
& $0.31 : [0.22, 0.34]$
\\
$\gamma=1.98$
& $1.26 : [1.16, 1.34]$
& $1.92 : [1.49, 2.05]$
& $1.96 : [1.86, 2.07]$
\\
\hline \hline
\multicolumn{4}{|c|}{Critical regime} \\[1mm]
$\alpha=0.50$
& $0.73 : [0.70, 0.75]$
& $0.51 : [0.43, 0.63]$
& $0.53 : [0.45, 0.64]$
\\
$\beta=0.30$
& $0.20 : [0.17, 0.24]$
& $0.29 : [0.24, 0.32]$
& $0.28 : [0.19, 0.33]$
\\
$\gamma=0.71$
& $0.54 : [0.44, 0.61]$
& $0.66 : [0.52, 0.76]$
& $0.67 : [0.54, 0.77]$
\\
\hline \hline
\multicolumn{4}{|c|}{Sub-threshold regime} \\[1mm]
$\alpha=0.40$
& $0.62 : [0.55, 0.65]$
& $0.57 : [0.45, 0.66]$
& $0.56 : [0.26, 0.71]$
\\
$\beta=0.30$
& $0.20 : [0.17, 0.26]$
& $0.22 : [0.18, 0.29]$
& $0.21 : [0.13, 0.35]$
\\
$\gamma=0.57$
& $0.36 : [0.18, 0.44]$
& $0.36 : [0.25, 0.50]$
& $0.43 : [0.28, 0.72]$
\\
\hline
 \end{tabular}}\\
\end{center}
\caption[Estimator Performance given 1000 spikes]{Averages and empirical 95\%
confidence intervals of the estimates for $N=1000$ spikes per train. }
\label{tab:est_quantiles_1000}
\end{table}


% \begin{table}
% \begin{center}
% \subfloat[N=100]{
% \begin{tabular}{c|ccc|ccc|ccc|}
% Regime & Initializer && & Fortet &&& FP && \\
% \hline
%  & $\aest$ &$\best$&$\gest$& &&&&& \\
% \hline
%  superT
% & $1.61 \pm [1.34, 1.76]$
% & $0.16 \pm [0.10, 0.32]$
% & $1.15 \pm [0.88, 1.39]$
% & $1.44 \pm [1.31, 1.56]$
% & $0.30 \pm [0.13, 0.42]$
% & $0.95 \pm [0.75, 1.09]$
% & $1.53 \pm [1.41, 1.73]$
% & $0.33 \pm [0.13, 0.45]$
% & $0.91 \pm [0.61, 1.08]$
% \\
% subT
% & $0.60 \pm [0.53, 0.67]$
% & $0.15 \pm [0.10, 0.23]$
% & $0.16 \pm [0.00, 0.42]$
% & $0.68 \pm [0.43, 0.83]$
% & $0.17 \pm [0.10, 0.29]$
% & $0.15 \pm [-0.00, 0.50]$
% & $0.74 \pm [0.07, 1.00]$
% & $0.11 \pm [0.00, 0.40]$
% & $0.28 \pm [-0.05, 1.24]$
% \\
% crit
% & $0.87 \pm [0.78, 0.95]$
% & $0.17 \pm [0.10, 0.27]$
% & $0.91 \pm [0.73, 1.07]$
% & $0.53 \pm [0.34, 0.76]$
% & $0.32 \pm [0.13, 0.44]$
% & $0.98 \pm [0.73, 1.27]$
% & $0.54 \pm [0.40, 0.68]$
% & $0.30 \pm [0.12, 0.41]$
% & $1.03 \pm [0.83, 1.25]$
% \\
% superSin
% & $1.06 \pm [0.92, 1.19]$
% & $0.17 \pm [0.10, 0.32]$
% & $1.75 \pm [1.51, 1.97]$
% & $0.19 \pm [-0.16, 0.95]$
% & $0.30 \pm [0.14, 0.50]$
% & $2.33 \pm [1.64, 2.69]$
% & $0.29 \pm [-0.16, 0.59]$
% & $0.32 \pm [0.13, 0.43]$
% & $2.29 \pm [1.90, 2.78]$
% \\
% \end{tabular}
% }
% \end{center}
% \label{tab:est_quantiles}
% \end{table}

% #ABSOLUTE ERRORS
% \begin{figure}[h]
% \begin{center}
% \subfloat[SUPER THRESHOLD]
% {
% \label{fig:comp_test_superT}
% \includegraphics[width=0.48\textwidth]
% {Figs/Estimates/FP_vs_Fortet_100x1000superT_est_errors.png}
% }
% \subfloat[CRITICAL]
% {
% \label{fig:comp_test_critical}
% \includegraphics[width=0.48\textwidth]
% {Figs/Estimates/FP_vs_Fortet_100x1000crit_est_errors.png} 
% }
% \\
% \subfloat[SUB THRESHOLD]
% {
% \label{fig:comp_test_subT}
% \includegraphics[width=0.48\textwidth]
% {Figs/Estimates/FP_vs_Fortet_100x1000subT_est_errors.png}
% }
% \subfloat[SUPER SINUSOID]
% {
% \label{fig:comp_test_superSin}
% \includegraphics[width=0.48\textwidth]
% {Figs/Estimates/FP_vs_Fortet_100x1000superSin_est_errors.png}
% }
% \caption{Absolute Errors of the parameter estimation routines for the 4
% different spike regimes. For each regime, we plot the difference, e.g.\
% $\aest - \a$, between the simulation parameters and each estimated
% parameter for each of the three estimation routines (upper plots) and in the
% lower plot, we show the sum of the absolute values of the errors,
% $|\aest - \a| + |\best - \b| +|\gest - \g|$. Note that figures
% for different regimes have different scales.}
% \label{fig:comprehensive_tests_abserrors}
% \end{center}
% \end{figure}
% % #RELATIVE ERRORS:
% \begin{figure}[h]
% \begin{center}
% \subfloat[SUPER THRESHOLD]
% {
% \label{fig:comp_test_superT}
% \includegraphics[width=0.48\textwidth]
% {Figs/Estimates/FP_vs_Fortet_100x1000superT_est_rel_errors.png}
% }
% \subfloat[CRITICAL]
% {
% \label{fig:comp_test_critical}
% \includegraphics[width=0.48\textwidth]
% {Figs/Estimates/FP_vs_Fortet_100x1000crit_est_rel_errors.png} 
% }
% \\
% \subfloat[SUB THRESHOLD]
% {
% \label{fig:comp_test_subT}
% \includegraphics[width=0.48\textwidth]
% {Figs/Estimates/FP_vs_Fortet_100x1000subT_est_rel_errors.png}
% }
% \subfloat[SUPER SINUSOID]
% { 
% \label{fig:comp_test_superSin}
% \includegraphics[width=0.48\textwidth]
% {Figs/Estimates/FP_vs_Fortet_100x1000superSin_est_rel_errors.png}
% }
% \caption{Relative Errors of the parameter estimation routines for the 4
% different spike regimes. For each regime, we plot the difference, e.g.\
% $\tfrac{\aest - \a}{\a}$, between the simulation parameters and each
% estimated parameter for each of the three estimation routines (upper plots) and in the
% lower plot, we show the sum of the absolute values of the errors,
% $|\tfrac{\aest - \a}{\a}|+
%  |\tfrac{\best - \b}{\b}|+
%  |\tfrac{\gest  - \g}{\g}|$. Note that figures for different regimes
%  have different scales!}
% \label{fig:comprehensive_tests_relerrors}
% \end{center}
% \end{figure}

\begin{figure}[p]
\begin{center}
% \subfloat[N=100]
% {
% \label{fig:comp_test_superT_100}
% \includegraphics[width=0.48\textwidth]
% {Figs/Estimates/FP_vs_Fortet_100x100superT_est_rel_errors.png}
% }
% \subfloat[N=1000]
% {
% \label{fig:comp_test_superT_1000}
% \includegraphics[width=0.48\textwidth]
% {Figs/Estimates/FP_vs_Fortet_100x1000superT_est_rel_errors.png}
% }
\includegraphics[width=0.99\textwidth]
{Figs/Estimates/FP_vs_Fortet_100x100_x1000superT_est_rel_errors_joint.pdf}
\caption[Estimates box-plots for supra-threshold regime]{Boxplots of parameter
estimates for the supra-threshold regime. The upper plots (A,B,C) show estimates using $N=100$ sample spikes per
estimation, while the lower plots (D,E,F) use $N=1000$. The dashed line
indicates the true parameter value, while the red line inside the boxes
indicates the median of the estimates.
\\
The boxes contain the central 50\% of the estimates. The bars indicate
the range of the estimates, except for outliers given by the points
outside the bars, and defined to be more than 1.5 times the
interquantile range (the height of the box) from the box.}
\label{fig:comprehensive_test_SuperT_relerrors}
\end{center}
\end{figure}
\begin{figure}[p]
\begin{center}
% \subfloat[N=100]
% {
% \label{fig:comp_test_superT_100}
% \includegraphics[width=0.48\textwidth]
% {Figs/Estimates/FP_vs_Fortet_100x100superSin_est_rel_errors.png}
% }
% \subfloat[N=1000]
% {
% \label{fig:comp_test_superT_1000}
% \includegraphics[width=0.48\textwidth]
% {Figs/Estimates/FP_vs_Fortet_100x1000superSin_est_rel_errors.png}
% }
\includegraphics[width=0.99\textwidth]{Figs/Estimates/FP_vs_Fortet_100x100_x1000superSin_est_rel_errors_joint.pdf}
\caption[Estimates box-plots for super-sinusoidal regime]{Boxplots of parameter
estimates for the super-sinusoidal regime. The upper plots (A,B,C) show estimates using $N=100$
sample spikes per estimation, while the lower plots (D,E,F) use $N=1000$. The dashed line
indicates the true parameter value, while the red line inside the boxes
indicates the median of the estimates.\\
The boxes contain the central 50\% of the estimates. The bars indicate
the range of the estimates, except for outliers given by the points
outside the bars, and defined to be more than 1.5 times the
interquantile range (the height of the box) from the box.}
\label{fig:comprehensive_test_SuperSin_relerrors}
\end{center}
\end{figure}
\begin{figure}[p]
\begin{center}
% \subfloat[N=100]
% {
% \label{fig:comp_test_superT_100}
% \includegraphics[width=0.48\textwidth]
% {Figs/Estimates/FP_vs_Fortet_100x100crit_est_rel_errors.png}
% }
% \subfloat[N=1000]
% {
% \label{fig:comp_test_superT_1000}
% \includegraphics[width=0.48\textwidth]
% {Figs/Estimates/FP_vs_Fortet_100x1000crit_est_rel_errors.png}
% }
\includegraphics[width=0.99\textwidth]{Figs/Estimates/FP_vs_Fortet_100x100_x1000crit_est_rel_errors_joint.pdf}
\caption[Estimates box-plots for critical regime]{Boxplots of parameter
estimates for the critical regime.
The upper plots (A,B,C) show estimates using $N=100$ sample spikes per
estimation, while the lower plots (D,E,F) use $N=1000$. The dashed line
indicates the true parameter value, while the red line inside the boxes
indicates the median of the estimates.
\\
The boxes contain the central 50\% of the estimates. The bars indicate
the range of the estimates, except for outliers given by the points
outside the bars, and defined to be more than 1.5 times the
interquantile range (the height of the box) from the box.}  
\label{fig:comprehensive_test_crit_relerrors}
\end{center}    
\end{figure} 
\begin{figure}[p]  
\begin{center}
% \subfloat[N=100]
% {
% \label{fig:comp_test_superT_100}
% \includegraphics[width=0.48\textwidth]
% {Figs/Estimates/FP_vs_Fortet_100x100subT_est_rel_errors.png} 
% }
% \subfloat[N=1000]
% {
% \label{fig:comp_test_superT_1000}
% \includegraphics[width=0.48\textwidth]
% {Figs/Estimates/FP_vs_Fortet_100x1000subT_est_rel_errors.png}
% }
\includegraphics[width=\textwidth]{Figs/Estimates/FP_vs_Fortet_100x100_x1000subT_est_rel_errors_joint.pdf}
\caption[Estimates box-plots for sub-threshold regime]{Boxplots of parameter
estimates for the sub-threshold regime.
The upper plots (A,B,C) show estimates using $N=100$ sample spikes per
estimation, while the lower plots (D,E,F) use $N=1000$. The dashed line
indicates the true parameter value, while the red line inside the boxes
indicates the median of the estimates.
\\
The boxes contain the central 50\% of the estimates. The bars indicate
the range of the estimates, except for outliers given by the points
outside the bars, and defined to be more than 1.5 times the
interquantile range (the height of the box) from the box.}
\label{fig:comprehensive_test_SubT_relerrors}
\end{center}
\end{figure}
% #CROSS ERROS:
\begin{figure}[htp]
\begin{center}
% \subfloat[SUPRA THRESHOLD]
% {
% \label{fig:comp_test_superT}
% \includegraphics[width=0.90\textwidth]
% {Figs/Estimates/FP_vs_Fortet_100x100superT_cross_compare.png} 
% }
% \\ 
% \subfloat[CRITICAL] 
% {
% \label{fig:comp_test_critical}
% \includegraphics[width=0.90\textwidth] 
% {Figs/Estimates/FP_vs_Fortet_100x100crit_cross_compare.png}  
% }
% \\
% \subfloat[SUB THRESHOLD] 
% {
% \label{fig:comp_test_subT}
% \includegraphics[width=0.90\textwidth]
% {Figs/Estimates/FP_vs_Fortet_100x100subT_cross_compare.png}
% } 
% \\
% \subfloat[SUPER SINUSOID] 
% {
% \label{fig:comp_test_superSin} 
% \includegraphics[width=0.90\textwidth]
% {Figs/Estimates/FP_vs_Fortet_100x100superSin_cross_compare.png}
% }     
\includegraphics[width=0.99\textwidth]
{Figs/Estimates/FP_vs_Fortet_100x100_cross_compare_joint.pdf}
\caption[Fortet-based vs. Fokker-Planck-based algorithm performance
with 100 spikes]{Estimates based on samples of $N = 100$ spikes obtained from
the Fokker-Planck algorithm against the Fortet algorithm for the four different parameter regimes, with parameter values given in table
\cref{tab:regimes}, fixing $\th=1$. Each row corresponds to one regime
and one set of simulations. Each column corresponds to a parameter,
with the specific value indicated above each plot.  
A,B,C) Supra-threshold; D,E,F) Super-sinusoidal; G,H,I) 
Critical; J,K,L) Sub-threshold.}
\label{fig:comprehensive_tests_cross_comparison}
\end{center}
\end{figure}
\begin{figure}[htp]
\begin{center}
% \subfloat[SUPRA THRESHOLD]
% {
% \label{fig:comp_test_superT}
% \includegraphics[width=0.90\textwidth]
% {Figs/Estimates/FP_vs_Fortet_100x1000superT_cross_compare.png} 
% }
% \\ 
% \subfloat[CRITICAL] 
% {
% \label{fig:comp_test_critical}
% \includegraphics[width=0.90\textwidth] 
% {Figs/Estimates/FP_vs_Fortet_100x1000crit_cross_compare.png}  
% }
% \\
% \subfloat[SUB THRESHOLD] 
% {
% \label{fig:comp_test_subT}
% \includegraphics[width=0.90\textwidth]
% {Figs/Estimates/FP_vs_Fortet_100x1000subT_cross_compare.png}
% } 
% \\
% \subfloat[SUPER SINUSOID] 
% {
% \label{fig:comp_test_superSin}
% \includegraphics[width=0.90\textwidth]
% {Figs/Estimates/FP_vs_Fortet_100x1000superSin_cross_compare.png}
% }  
\includegraphics[width=0.99\textwidth]    
{Figs/Estimates/FP_vs_Fortet_100x1000_cross_compare_joint.pdf}
\caption[Fortet-based vs. Fokker-Planck-based algorithm performance
with 1000 spikes]{Estimates based on samples of $N = 1000$ spikes obtained from
the Fokker-Planck algorithm against the Fortet algorithm for the four different
parameter regimes, with parameter values given in table
\cref{tab:regimes}, fixing $\th=1$. Each row corresponds to one regime
and one set of simulations. Each column corresponds to a parameter,
with the specific value indicated above each plot.  
A,B,C) Supra-threshold; D,E,F) Super-sinusoidal; G,H,I) 
Critical; J,K,L) Sub-threshold.}
\label{fig:comprehensive_tests_cross_comparison}
\end{center}
\end{figure}
\begin{table}
\begin{center}
\subfloat[N=100]{
\begin{tabular}{c|cc|}
Regime & Fortet & Fokker-Planck \\
\hline
Sub-threshold
& 1.29 $\pm$ 0.72
& 0.52 $\pm$ 0.21
\\
Supra-threshold
& 0.83 $\pm$ 0.28
& 0.18 $\pm$ 0.20
\\
Critical
& 0.94 $\pm$ 0.42
& 0.36 $\pm$ 0.16
\\
Super-sinusoidal
& 1.36 $\pm$ 0.46
& 0.43 $\pm$ 0.17
\\
\end{tabular}
}
\subfloat[N=1000]{
\begin{tabular}{|c|cc}
Regime & Fortet & Fokker-Planck \\
\hline
Sub-threshold
& 9.68 $\pm$ 4.98
& 1.69 $\pm$ 0.91
\\
Supra-threshold
& 3.90 $\pm$ 1.05
& 0.21 $\pm$ 0.06
\\
Critical
& 10.03 $\pm$ 2.88
& 1.28 $\pm$ 0.41
\\
Super-sinusoidal
& 10.13 $\pm$ 2.24
& 1.06 $\pm$ 0.33
\\
\end{tabular}
}
\end{center}
\caption{Estimator Algorithm Computational Time}
\label{tab:walltimes}
\end{table} 

\section{The effect of $\th$}
So far we have held $\th$ constant and equal to $1$. We now investigate the
effect of varying $\th$ on the quality of estimates. To narrow the scope, we
focus on increasing $\th$ while keeping the parameters in the critical regime
such that $\a + \g/\sqrt{1+\th^2} = 1$ and $\a=0.5$. This amounts to increasing
$\g$ with $\th$. We do the estimations for four values of $\th=[1,5,10,20]$.
Similarly to the previous section, we use 100 sample spike trains per 
parameter set, with each spike train consisting of $N=1000$ ISIs.

We show box plots of the estimates for each $\th$ in
\cref{fig:comprehensive_test_thetas_relerrors}. We then directly compare the two
algorithms, Fortet vs.\ Fokker-Planck, in
\cref{fig:comprehensive_test_thetas_cross_compare}. The immediate observation is
that the Fokker-Planck algorithm fails to keep up at the higher frequencies and
consistently underestimates $\g$. The Fortet algorithm does better, but still
underestimates $\g$. In general, this underestimation of $\g$ is accompanied by
an over-estimation of $\a$. This is exacerbated at higher $\th$. We illustrate
the relation between estimates for $\a$ vs. $\g$ in
\cref{fig:comprehensive_test_thetas_alpha_vs_gamma}, where it is quite clear
that an underestimation of $\g$ is proportional to the overestimation of $\a$.
\begin{figure}[htp]
\begin{center}
% \subfloat[$\th=1$]
% {
% \includegraphics[width=0.48\textwidth]
% {Figs/Estimates/thetas_100x1000theta1_est_rel_errors.png}
% }
% \subfloat[$\th=5$]
% {
% \includegraphics[width=0.48\textwidth]
% {Figs/Estimates/thetas_100x1000theta5_est_rel_errors.png}
% }
% \\
% \subfloat[$\th=10$]
% {
% \includegraphics[width=0.48\textwidth]
% {Figs/Estimates/thetas_100x1000theta10_est_rel_errors.png}
% }
% \subfloat[$\th=20$]
% {
% \includegraphics[width=0.48\textwidth]
% {Figs/Estimates/thetas_100x1000theta20_est_rel_errors.png}
% }
\includegraphics[width=0.99\textwidth]  
{Figs/Estimates/thetas_100x1000thetas_est_rel_errors.pdf}
\caption[Estimates' box-plots for varying sinusoidal frequency]{Boxplots of
parameter estimates for varying $\th$ across $[1, 5, 10, 20]$ while holding $\g / \sqrt{1+\th^2}$ constant as to keep the parameters in the critical
regime.
A-C) $\th=1$,  D-F) $\th=5$,        
G-I) $\th=10$, J-L) $\th=20$.
\\
The boxes contain the central 50\% of the estimates. The bars indicate
the range of the estimates, except for outliers given by the points
outside the bars, and defined to be more than 1.5 times the
interquantile range (the height of the box) from the box.}  
\label{fig:comprehensive_test_thetas_relerrors}    
\end{center}
\end{figure}   
\begin{figure}[htp]
\begin{center}
% \subfloat[$\th=1$]
% {
% \includegraphics[width=0.48\textwidth]
% {Figs/Estimates/FP_vs_Fortet_thetastheta1_cross_compare.png}
% }
% \subfloat[$\th=5$]
% {
% \includegraphics[width=0.48\textwidth]
% {Figs/Estimates/FP_vs_Fortet_thetastheta5_cross_compare.png}
% }
% \\
% \subfloat[$\th=10$]
% {
% \includegraphics[width=0.48\textwidth]
% {Figs/Estimates/FP_vs_Fortet_thetastheta10_cross_compare.png}
% }
% \subfloat[$\th=20$]
% {
% \includegraphics[width=0.48\textwidth]
% {Figs/Estimates/FP_vs_Fortet_thetastheta20_cross_compare.png}
% } 
\includegraphics[width=0.99\textwidth]
{Figs/Estimates/FP_vs_Fortet_thetas_cross_compare_joint.pdf} 
\caption[Fortet-based vs Fokker-Planck-based algorithms for varying
sinusoidal frequency]{Estimates based on samples of $N = 1000$ spikes obtained
from the Fokker-Planck algorithm against the Fortet algorithm for a parameter set in the critical regime, while varying $\th$ across $[1, 5, 10, 20]$ and holding 
$\g / \sqrt{1+\th^2}$ and $\a$ constant.
A,B,C) $\th=1$; D,E,F) $\th=5$; G,H,I)
$\th=10$; J,K,L) $\th=20$. } 
\label{fig:comprehensive_test_thetas_cross_compare}
\end{center}
\end{figure}
\begin{figure}[htp]
\begin{center}
% \subfloat[$\th=1$]
% {
% \includegraphics[width=0.48\textwidth]
% {Figs/Estimates/thetavariation_100x1000theta1_alphagamma_compare.png}
% }
% \subfloat[$\th=5$]
% {
% \includegraphics[width=0.48\textwidth]
% {Figs/Estimates/thetavariation_100x1000theta5_alphagamma_compare.png}
% }
% \\
% \subfloat[$\th=10$]
% {
% \includegraphics[width=0.48\textwidth]
% {Figs/Estimates/thetavariation_100x1000theta10_alphagamma_compare.png}
% }
% \subfloat[$\th=20$] 
% {
% \includegraphics[width=0.48\textwidth]
% {Figs/Estimates/thetavariation_100x1000theta20_alphagamma_compare.png}
% }
\includegraphics[width=\textwidth]  
{Figs/Estimates/thetavariation_100x1000_alphagamma_compare_joint.pdf}
\caption[Fortet-based vs. Fokker-Planck-based algorithm performance
for varying sinusoidal frequency]{Comparison of $\aest$ vs.
$\gest$ parameter estimates while varying $\th$ across $[1, 5, 10, 20]$,  holding $\g / \sqrt{1+\th^2}$ constant as to
keep the parameters in the critical regime.
A,B,C) $\th=1$; D,E,F) $\th=5$; G,H,I)
$\th=10$; J,K,L) $\th=20$.   
}
\label{fig:comprehensive_test_thetas_alpha_vs_gamma}
\end{center}
\end{figure}


For completeness we also include the estimates' average and empirical 95\%
confidence intervals in \cref{tab:thetas_est_quantiles_1000}.
\begin{table}[htp]
\begin{center}
{\begin{tabular}{|c|ccc|} 
Parameter
& Initializer
& Fokker-Planck
& Fortet
\\ 
\hline \hline
\multicolumn{4}{|c|}{$\Omega=1$} \\[1mm]
$\alpha=0.50$
& $0.73 : [0.69, 0.75]$
& $0.52 : [0.45, 0.61]$
& $0.52 : [0.44, 0.62]$
\\
$\beta=0.30$
& $0.20 : [0.17, 0.25]$
& $0.29 : [0.24, 0.33]$
& $0.29 : [0.22, 0.34]$
\\
$\gamma=0.71$
& $0.54 : [0.44, 0.62]$
& $0.64 : [0.53, 0.75]$
& $0.68 : [0.55, 0.81]$
\\
\hline \hline
\multicolumn{4}{|c|}{$\Omega=5$} \\[1mm]
$\alpha=0.50$
& $0.88 : [0.76, 0.99]$
& $0.78 : [0.61, 0.89]$
& $0.64 : [0.39, 0.99]$
\\
$\beta=0.30$
& $0.24 : [0.17, 0.31]$
& $0.26 : [0.20, 0.34]$
& $0.27 : [0.12, 0.34]$
\\
$\gamma=2.55$
& $0.85 : [0.00, 1.65]$
& $0.92 : [0.00, 1.68]$
& $1.86 : [0.00, 3.10]$
\\
\hline \hline
\multicolumn{4}{|c|}{$\Omega=10$} \\[1mm]
$\alpha=0.50$
& $0.90 : [0.78, 0.99]$
& $0.71 : [0.52, 0.88]$
& $0.58 : [0.37, 0.86]$
\\
$\beta=0.30$
& $0.25 : [0.18, 0.33]$
& $0.26 : [0.20, 0.35]$
& $0.28 : [0.23, 0.32]$
\\
$\gamma=5.02$
& $2.82 : [0.92, 4.38]$
& $2.72 : [0.95, 3.88]$
& $4.32 : [1.20, 6.49]$
\\
\hline \hline
\multicolumn{4}{|c|}{$\Omega=20$} \\[1mm]
$\alpha=0.50$
& $0.93 : [0.76, 1.02]$
& $0.75 : [0.50, 0.92]$
& $0.62 : [0.31, 0.97]$
\\
$\beta=0.30$
& $0.27 : [0.20, 0.33]$
& $0.29 : [0.20, 0.43]$
& $0.29 : [0.25, 0.33]$
\\
$\gamma=10.01$
& $5.35 : [0.00, 12.29]$
& $3.98 : [0.00, 6.83]$
& $7.48 : [0.00, 13.96]$
\\
\hline
\end{tabular}}\\
\end{center}
\caption[Impact of sinusoidal frequency on estimators]{Averages and empirical
95\% confidence intervals of estimates for $N=1000$ spikes per train in the critical regime for varying $\th$ across
[1,5,10,20]. Note that the upper subtable corresponds to the third
subtable in \cref{tab:est_quantiles_1000}. Numbers differ slightly due to statistical
fluctuations in the simulations. }
\label{tab:thetas_est_quantiles_1000}
\end{table}


\section{Maximum Likelihood Estimation Procedure}
We detail the maximum likelihood (ML) approach to estimating the parameters
$\abg$ in the sinusoidal, noisy LIF model as discussed in the main text
\cite{Iolov2013}. We assume the reader is familiar with the notions and notation
in \cite{Iolov2013}, most pertinently the material in sec.\ 3.2.

We start by recalling the relationship between the ISI survival density, $g$,
the survival distribution $\G$ and the Fokker-Planck transition distribution,
$F$. $$\G_{\phi}(t) = F^{(\phi)}(1,t)$$ and $$g_{\phi}( t)  = -\di_t
\G_{\phi}(t). $$

Given a set of $N$ observed spikes and phase angles, $(i_n,
\phi_{n-1})_{n=1}^N$, the ML estimates are obtained by considering the
log-likelihood function, $L$,
\begin{equation}
 L(\a,\b,\g) = \sum_n  \log (
g_{\phi_{n-1}}(i_n) ) = \sum_n \log \big[ -\di_t F^{\phi_{n-1}}_{\a,\b,\g}(\xth, t) \big] \Big|_{t =
i_n}. 
\label{eq:loss_function_MLE}
\tag{A1}
\end{equation}
and maximizing it,$\abgest = \argmax L.$
Theoretically, one could stop there and start the number crunching. However, for
each evaluation of $L$ we would have to solve $N$ PDEs for $F$. That might
become computationally burdensome. Instead, in keeping with the spirit of the
paper, we choose a set of $M$ representative $\phi$'s $\{\phi_m\}$, chosen as
the same phase bin midpoints as in the main text, and approximate $\G_\phi$ as
a weighted average of $\{F^{\phi_m} \}$.



In particular given $\phi \in [\phi_m, \phi_{m+1}]$, we take a simple
linear average:
$$\G_\phi(t) \approx
\frac{\phi_{m+1} - \phi}{\phi_{m+1} - \phi_{m}} \cdot F^{\phi_m}(t)
+ 
\frac{\phi - \phi_{m}}{\phi_{m+1} - \phi_{m}} \cdot F^{\phi_{m+1}}(t).
$$
The basic idea is that if $\phi \approx \phi_m$ we use $F^{\phi_m}$ and
conversely if $\phi \approx \phi_{m+1}$ we use $F^{\phi_{m+1}}$.
Note that this assures that $0 \leq \G \leq 1$.

Finally, recall that $F$ is solved numerically, using a simple finite difference
to estimate the derivative in \cref{eq:loss_function_MLE}. If a spike time $i_n$
falls between two time slices $t_k, t_{k+1}$, $i_n \in [t_k, t_{k+1})$ then we approximate $-\di_t F(\xth, i_n)$ as $$ -\di_t F(\xth, i_n)
\approx -\frac{F(\xth, t_{k+1}) - F(\xth, t_{k})}{t_{k+1} - t_{k}}. $$


Putting it all together, the estimates are obtained as the maximizers of 
\begin{align}
\tag{A2}
\label{eq:loss_function_ML}
L(\a,\b,\g) = \sum_n  \log \Bigg(
&-\frac{\phi_{m+1} - \phi}{\phi_{m+1} - \phi_{m}} \cdot 
\frac{F_{\a,\b,\g}^{\phi_m}(\xth, t_{k+1}) - F_{\a,\b,\g}^{\phi_m}(\xth,
t_{k})}{t_{k+1} - t_{k}}
\\&-
\frac{\phi - \phi_{m}}{\phi_{m+1} - \phi_{m}} \cdot 
\frac{F_{\a,\b,\g}^{\phi_{m+1}}(\xth, t_{k+1}) - F_{\a,\b,\g}^{\phi_{m+1}}(\xth,
t_{k})}{t_{k+1} - t_{k}} 
 \Bigg)
 \Bigg|_{\substack{i_n \in [t_k, t_{k+1}] \\
 			 \phi_{n-1} \in [\phi_m, \phi_{m+1}]}} .
 			 \notag
\end{align} 
after numerically solving the PDE for $F$, eq. 9 in \cite{Iolov2013}. 

To maximize $L$ we again use the Nelder-Mead optimizer from the SciPy library,
\cite{scipy}. We also tried the gradient-based optimizers in SciPy (BFGS,
Truncated Newton, Sequential Least Squares), but they turned out slower and
provided less accurate estimates.
  
\section{ML Estimates}
We label estimates based on maximizing \cref{eq:loss_function_ML} as ML
estimates, while estimates obtained by minimizing eq.\ 17 in the main text, \cite{Iolov2013},
are called FP estimates. Of course, both of them rely on solving the
Focker-Planck equation, but we do this in order to be consistent with the main text
where estimates based on eq. 17 are labeled with 'FP'. We summarize the two
estimators in \cref{tab:estimators}. \begin{table} \begin{tabular}{ccp{4cm}}
Name & Loss function $L$ & Short description
\\
\hline FP &$\sum_{m} N_m \Big\{ \sup_{t} \left| \Gest_{\phi_m}(t) -
F^{\phi_m}_{\a,\b,\g}(\xth,
t) \right| \Big \}$ & Minimize the $sup$ of the  differences between a
numerically calculated survivor distribution and the empirically observed
survivor function
\\
ML & $-\sum_n  \log \Big( g_\phi(i_n)
 \Big)$
& Maximize the sum of the log-likelihoods of the observed ISIs
 		\\
\hline  \end{tabular}
\caption[Comparison of Maximum Likelihood vs.\ Fokker-Planck distributional
algorithm]{The two estimators we compare: the ML estimator and  the FP
estimator, which has already been discussed at length in the main text,
\cite{Iolov2013}} \label{tab:estimators} \end{table}

To compare the FP vs.\ ML estimators, we draw cross-plots between
the two in the same style as figs. 12,13 in the main text,
\cite{Iolov2013}. As always, we use 100 spike trains to obtain 100
estimates for $\abg$ using each of the two methods. In all simulations, we set
$\th = 1$.

We use the same regimes, super-threshold, super-sinusoidal, critical,
sub-threshold, with the same parameters as in the main text,

We start by using $N=100$ spikes per spike train - see
\cref{fig:MLCompare_N100}. We use 16 bins for the ML estimate and 8 for the FP.
The differences between the two methods appear minor. Neither method seems to be
consistently better than the other. In some cases, the ML estimators are
noticeably worse - especially for the super-sinusoidal regime, where the ML
method shows a bimodal distribution of the $\a$ parameter. In other cases, the
ML estimator has a smaller variance - see $\g$ in the critical regime and maybe
$\a$ in the supra-threshold regime, but that is already estimated quite
accurately.

Next, we try using the larger sample size, i.e.\ $N=1000$ spikes per spike train
- see \cref{fig:MLCompare_N1000}. Again the differences are minor and similar to
what was seen in the smaller sample size, $N=100$. Most notably the ML is now
better at estimating $\a$ in the super-sinusoidal regime, except for one rare
case.

At this point, we would conclude that the performance of the ML and FP
methods is comparable. However the ML method has the potential advantage that
one can use much smaller bins (more $\phi_m$s) no matter how big or small the
data size. This is because the FP method relies on there being a sufficient
amount of spikes in each bin in order to approximate the survivor distribution,
while the ML method does not have this requirement.

So as a final comparison we redo the estimation for $N=100$, but now
we use $M=32$ bins for the ML method while still using only $M=8$ bins for
FP. The results are in \cref{fig:MLCompare_N100_32bins}. It is evident
that the differences between using \cref{fig:MLCompare_N100} using 16 bins for
the ML method and \cref{fig:MLCompare_N100_32bins} using 32 bins are not
significant.

As a final note, we mention that the computing times for both methods are
roughly the same, modulo that one uses the same number of bins.

As such we conclude that the FP estimates and the ML estimates behave comparably.
\begin{figure}[htp]
\begin{center}
  \includegraphics[width=1\textwidth]{Figs/Estimates/MLE_N100_cross_compare_joint.pdf}
  \caption[Maximum-Likelihood vs. 'Fokker-Planck' algorithm performance with 100 spikes]
  {Comparison between the FP vs.\ ML
  estimators in the four regimes, using $N=100$ spikes per sample. We used 8
  bins for the FP and 16 bins for the ML. In all simulations $\th = 1$.}
  \label{fig:MLCompare_N100}
\end{center}
\end{figure}

\begin{figure}[htp]
\begin{center}
  \includegraphics[width=1\textwidth]{Figs/Estimates/MLE_N1000_cross_compare_joint.pdf}
  \caption[Maximum-Likelihood vs. 'Fokker-Planck' algorithm performance with
  1000 spikes] {Comparison between the FP vs.\ ML
  estimators in the four regimes, using $N=1000$ spikes per sample. We used 8
  bins for the FP and 16 bins for the ML. In all simulations $\th = 1$.}
  \label{fig:MLCompare_N1000}
\end{center}
\end{figure} 

\begin{figure}[htp]
\begin{center}
  \includegraphics[width=1\textwidth]
  {Figs/Estimates/MLE_N100_32bins_cross_compare_joint.pdf}
  \caption[Effect of bin-refinement on Maximum-Likelihood estimator]
  {Same as \cref{fig:MLCompare_N100}, but using 32
  bins for ML instead of 16. $N=100$. In all simulations $\th = 1$.}
  \label{fig:MLCompare_N100_32bins}
\end{center}
\end{figure}



\section{Discussion and Outlook}
\label{sec:sin_estimate_discusion}
We have shown two methods to estimate parameters in
\cref{eq:X_evolution_uo} from ISI data. Our methods are based
on binning the spikes into bins with representative phase shifts. We have
devised a constructive procedure to automatically initialize the methods from
the data.

Our computational results suggest that for low frequencies the Fortet algorithm
is superior for large sample sizes, especially in the super-sinusoidal regime, while the
Fokker-Planck algorithm has a comparable accuracy and a lower variance for small
sample sizes. Both algorithms find sensible estimates most of the time, although
they seem less effective in the sub-threshold regime. Their performance can
be partially attributed to the ability of the initializer algorithm to supply
good guesses for starting the optimization iterations.

The Fokker-Planck equation allows for approximate maximum likelihood
estimation. We chose an alternative loss function, though, because it
marginally appeared more robust, possibly because a numerical derivation
step is avoided. This is further investigated by simulations in the
supplementary online material. The simulations suggest that the
distribution of the maximum likelihood estimates in the super-sinusoidal
regime appears bimodal, which is not occurring for the alternative loss
function, \cref{eq:loss_function_absorbingBC}. %eq. \eqref{}. (((Eq. 17))) 

We have also made a preliminary exploration of the effect of $\th$ on the
quality of the estimates. Our results show that an increase in $\th$ makes the
parameters $\a$ and $\g$ more difficult to estimate accurately and at high
$\th$, $\g$ is underestimated, while $\a$ is over-estimated. We find that in
this scenario, the Fortet algorithm does a markedly more accurate job then the
Fokker-Planck algorithm.
 
We have assumed  the time-constant $\t$ of the leak term known. In most
experiments that is not realistic, and it would be preferable to estimate $\t$
alongside the other parameters. However, it is difficult to estimate
\cite{DitlevsenLansky212}. When we tried to estimate it together with the other
parameters, we usually obtained results which were not accurate. The obtained
estimates resulted in ISIs that very well matched the data, no worse than the
ISIs obtained from the true parameters. This leads us to believe that the
simultaneous estimation of $\t$ along with $\abg$ using only ISI data suffers
from identifiability problems. In \cite{MullowneyIyengar2008}, they were able to
estimate $\t$ in the simpler non-sinusoidally-driven model, but concluded that
adding $\t$ as an unknown dramatically reduced the accuracy in the estimation of
the other unknown parameters. The reason is that if $\t$ is also estimated from
a single dataset alongside the other parameters, then a reasonable fit can be
found to the data for various combinations of $\abg$ and $\t$, but the so-obtained
parameter values can be far from the true values.

Our model is relatively simple and ignores neurophysiological realism, such as
the fact that the spiking threshold is likely non-constant, with a
time-dependent functional form that would involve further unknown parameters. A
recent paper attempting the parameter estimation in such a model, but without
sinusoidal forcing, is \cite{Dong2011}. Furthermore, intra-cellular recordings
suggest that a hard threshold is a rough approximation and an exponential
voltage-dependent spiking intensity is more realistic \cite{Jahn2011}.

While our work has used a very specific form of the periodic forcing term,
namely $\g \sin(\th t)$,  it is clear how to apply the approach to an arbitrary
periodic function. This can be done as long as one knows where in the period of
oscillation a spike has occurred. If that is the case then the binning procedure
can be applied and the estimation methods proposed can be attempted. 


\section*{Implementation Details}
All the code used in this chapter including code to generate the figures
can be found on
\href{https://github.com/aviolov/SinSpikePython}{github @
https://github.com/aviolov/SinSpikePython}. Please see the README.md file for
for guide to the codes.
